\documentclass[german,12pt]{article}  
\usepackage{babel}            % Sprachunterstuetzung

\parindent0pt
\begin{document}

\textbf{Ausgew\"ahlte Kapitel der Modernen Physik}
\vspace{1cm}


Dozent: apl.\ Prof.\ Dr.\ Thomas Filk\\
Zeit: 2+2 st. (Do 16-18; \"Ubungen n.V.)\\
5 ECTS\\
Beginn: 28.\ April 2022

%\chapter*{Vorwort}
%\addcontentsline{toc}{chapter}{Vorwort}
\thispagestyle{empty}
\vspace{1cm}

Diese Vorlesung richtet sich in erster Linie an Lehramtsstudierende der Physik. Sie kann als
Wahlpflichtvorlesung im Master of Education geh\"ort werden oder auch im Bachelor als
Spezialvorlesung, falls Mathematik das zweite Hauptfach ist. 
Inhaltlich deckt diese Vorlesung einige Themen ab, die im normalen Curriculum nicht
oder nur am Rande behandelt werden, die aber f\"ur zuk\"unftige Lehrer*Innen relevant
sind, entweder weil sie im Rahmenlehrplan der Oberstufe vorgesehen sind oder aber
zur Motivation der Sch\"uler*Innen beitragen k\"onnen. 
\vspace{1cm}

\textbf{Programm (Auswahl):}
\begin{itemize}
\item[-]
Standardmodell der Kosmologie
\item[-]
elementare Einf\"uhrung in die Allgemeine Relativit\"atstheorie
\item[-]
elementare Einf\"uhrung in die Astrophysik
\item[-]
Halbleiter und Photovoltaik, etc.
\item[-]
Bildgebende Verfahren in der Medizin    
\item[-]
Physik des Klimas
\end{itemize}   
\vspace{1cm}

\textbf{Vorkenntnisse:}\\[0.2cm]
Man sollte die Exp I--III sowie die Theo I und II erfolgreich geh\"ort haben. 
Literatur wird in den jeweiligen Vorlesungsstunden bekannt gegeben. Es soll
parallel zur Vorlesung ein Skript erstellt werden.


\end{document}
