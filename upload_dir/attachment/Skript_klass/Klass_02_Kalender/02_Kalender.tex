%\documentclass[german,10pt]{book}      
\usepackage{makeidx}
\usepackage{babel}            % Sprachunterstuetzung
\usepackage{amsmath}          % AMS "Grundpaket"
\usepackage{amssymb,amsfonts,amsthm,amscd} 
\usepackage{mathrsfs}
\usepackage{rotating}
\usepackage{sidecap}
\usepackage{graphicx}
\usepackage{color}
\usepackage{fancybox}
\usepackage{tikz}
\usetikzlibrary{arrows,snakes,backgrounds}
\usepackage{hyperref}
\hypersetup{colorlinks=true,
                    linkcolor=blue,
                    filecolor=magenta,
                    urlcolor=cyan,
                    pdftitle={Overleaf Example},
                    pdfpagemode=FullScreen,}
%\newcommand{\hyperref}[1]{\ref{#1}}
%
\definecolor{Gray}{gray}{0.80}
\DeclareMathSymbol{,}{\mathord}{letters}{"3B}
%
\newcounter{num}
\renewcommand{\thenum}{\arabic{num}}
\newenvironment{anmerkungen}
   {\begin{list}{(\thenum)}{%
   \usecounter{num}%
   \leftmargin0pt
   \itemindent5pt
   \topsep0pt
   \labelwidth0pt}%
   }{\end{list}}
%
\renewcommand{\arraystretch}{1.15}                % in Formeln und Tabellen   
\renewcommand{\baselinestretch}{1.15}                 % 1.15 facher
                                                      % Zeilenabst.
\newcommand{\Anmerkung}[1]{{\begin{footnotesize}#1 \end{footnotesize}}\\[0.2cm]}
\newcommand{\comment}[1]{}
\setlength{\parindent}{0em}           % Nicht einruecken am Anfang der Zeile 

\setlength{\textwidth}{15.4cm}
\setlength{\textheight}{23.0cm}
\setlength{\oddsidemargin}{1.0mm} 
\setlength{\evensidemargin}{-6.5mm}
\setlength{\topmargin}{-10mm} 
\setlength{\headheight}{0mm}
\newcommand{\identity}{{\bf 1}}
%
\newcommand{\vs}{\vspace{0.3cm}}
\newcommand{\noi}{\noindent}
\newcommand{\leer}{}

\newcommand{\engl}[1]{[\textit{#1}]}
\parindent 1.2cm
\sloppy

         \begin{document}  \setcounter{chapter}{1}

\chapter{Kalendersysteme}
% Kap 2

\info{Thomas Filk}{28.03.2024}%
Der (Sonnen-)Tag als die Zeitdauer\index{Kalender|(} 
zwischen zwei aufeinanderfolgenden Sonnenh\"ochstst\"anden, der 
(synodische) Monat als die Zeitdauer zwischen zwei aufeinanderfolgenden Voll- oder Neumonden und das 
(tropische) Jahr als die Zeitdauer zwischen zwei Sonnendurchg\"angen durch den Fr\"uhlingspunkt
gaben insbesondere in der Antike die wichtigen Perioden der Zeitrechnung vor. Dabei hat
die Tatsache, dass weder die Mondphasen noch ein ganzzahliges Vielfaches eines Sonnentages 
\"uber einen l\"angeren Zeitraum mit den Jahreszeiten und den damit zusammenh\"angenden 
Erscheinungen (z.B.\ die regelm\"a\ss ig im Juli bis September auftretenden Nilschwemmen in \"Agypten)
\"ubereinstimmen, zu teilweise sehr komplizierten Kalendersystemen gef\"uhrt. 
Tabelle \ref{tab_Kal} enth\"alt die wichtigsten Zahlen, die in diesem Kapitel ben\"otigt werden.

\begin{table}[htb]
\begin{tabular}{r|l}
tropisches Jahr in Tagen &  $365, 24219$\,d   \\
Julianisches Jahr in Tagen &  $365,25$\,d \\
Gregorianisches Jahr in Tagen & $ 365,2425$\,d \\
synodischer Monat &  $29,5306$\,d \\
\end{tabular}
\caption{\label{tab_Kal}%
Die wichtigsten physikalischen Gr\"o\ss en im Zusammenhang mit den
Kalendersystemen.}
\end{table}

Au\ss erdem ben\"otigen wir noch die folgenden Begriffe, die ausf\"uhrlicher in
Kapitel \ref{chap_Nachthimmel} erl\"autert werden:

Unter der \textit{Ekliptik}\index{Ekliptik}\index{Ekliptikebene} 
versteht man einen Gro\ss kreis am Nachthimmel, den man erh\"alt,
wenn man die Sonne von der Erde aus an den Himmel projiziert. Umgekehrt kann man auch
die Projektion der Erde vom Mittelpunkt der Sonne aus an den Nachthimmel als Ekliptik 
definieren. (Der Mittelpunkt der Erde und der Mittelpunkt der Sonne legen eine Gerade
fest; diese Gerade \"uberstreicht im Laufe eines Jahres eine Ebene -- die Ekliptik.)
Die Umlaufbahn der Erde um die Sonne liegt dann in der Ekliptikebene.
Der \textit{Himmels\"aquator} ist die Projektion des Erd\"aquators an den Himmel vom Mittelpunkt 
der Erde aus betrachtet.\index{Himmels\"aquator}
Man kann ihn auch als den Gro\ss kreis am Himmel definieren, der senkrecht zum
Himmelsnordpol (der Projektion der Erdachse an den Himmel) steht. Diese beiden
Gro\ss kreise (Ekliptik und Himmels\"aquator) schneiden sich in zwei Punkten:
dem \textit{Fr\"uhlingspunkt}\index{Fruehlingspunkt@Fr\"uhlingspunkt}
und dem \textit{Herbstpunkt}.\index{Herbstpunkt}
Befindet sich die Sonne von der Erde aus betrachtet im Fr\"uhlings- oder Herbstpunkt, sind Nacht und Tag
gleich lang, daher spricht man auch von den \textit{\"Aquinoktien}.\index{Aequinoktien@\"Aquinoktien}

\section{Tage, Monate und Jahre}

Wir alle wissen, was im Alltag gemeint ist, wenn wir von \glqq Tag\grqq,  \glqq Monat\grqq, oder
\glqq Jahr\grqq\ sprechen, und doch erweisen sich diese Begriffe als recht komplex und vieldeutig,
wenn man versucht, sie pr\"aziser zu definieren. 

\subsection{Tage}

Unter einem Tag\index{Tag} 
verstehen wir im Allgemeinen den Zeitraum zwischen zwei gleichen 
Sonnenst\"anden, z.B.\ zwei Sonnenh\"ochstst\"anden,
also von Mittag bis zum Mittag des n\"achsten Tages. Bei Sonnenh\"ochststand steht die Sonne
f\"ur jeden Beobachter n\"ordlich des n\"ordlichen Wendekreises (dem Breitengrad bei $23,4^\circ$)
exakt im S\"uden, f\"ur einen Beobachter s\"udlich des s\"udlichen Wendekreises exakt im Norden.
F\"ur Beobachter zwischen diesen beiden Wendekreisen h\"angt der Sonnenh\"ochststand von der
Jahreszeit ab. In jedem Fall befindet sich die Sonne zum Zeitpunkt \glqq 12 Uhr mittags (wahre Zeit)\grqq\
auf einer gedachten Linie, die den L\"angengrad des Beobachters
(also den Nord-S\"ud-Meridian, der durch den Ort des Beobachters verl\"auft) vom Erdmittelpunkt
aus an den Himmel projiziert. Ein solcher Tag hei\ss t \textit{wahrer Sonnentag}.\index{Sonnentag!wahrer} 

Da der wahre Sonnentag aus verschiedenen Gr\"unden im Laufe eines Jahres in seiner L\"ange
schwanken kann (siehe den Abschnitt zu \glqq Zeitgleichung\grqq, Kap.\ \ref{chap_Zeitgleichung}), 
definiert man einen sogenannten \textit{mittleren\index{Sonnentag!mittlerer}
Sonnentag}, das ist ein \"uber das Jahr genommenes Mittel der wahren Sonnentage. Dieser
mittlere Sonnentag wird in 24 Stunden bzw.\ 86\,400\,Sekunden unterteilt. 

Versteht man unter einem Tag, dass sich die Erde einmal um ihre Achse gedreht hat, muss man
einen Bezugspunkt angeben, der \glqq einmal rum\grqq\ spezifiziert. Ist dieser Bezugspunkt die
Sonne, erhalten wir den oben beschriebenen Sonnentag.\index{Sonnentag} 
Handelt es sich bei diesem Bezugspunkt
aber um den Sternenhimmel, also z.B.\ einen bestimmten Fixstern, dessen Eigenbewegung wir
vernachl\"assigen k\"onnen,\index{Sternentag}\index{Tag!siderischer} 
erhalten wir einen \textit{Sternentag} oder \textit{siderischen Tag}. 

\begin{SCfigure}[30][htb]
\begin{picture}(220,100)(-20,0)
\put(90,80){\circle{40}}
\put(5,80){\circle{20}}
\put(30,20){\circle{20}}
\put(120,80){\vector(1,0){60}}
\put(120,50){\vector(1,0){60}}
\put(120,20){\vector(1,0){60}}
\put(15,80){\line(1,0){73}}
\put(37,27){\line(1,1){52}}
\put(180,40){\makebox(0,0){Fixstern}}
\put(90,90){\makebox(0,0){Sonne}}
\put(90,80){\makebox(0,0){$\bullet$}}
\put(78,75){\makebox(0,0){$\alpha$}}
\put(45,27){\makebox(0,0){$\alpha$}}
\qbezier(5,70)(5,45)(23,27)
\thicklines
\put(15,80){\vector(1,0){20}}
\put(40,20){\vector(1,0){20}}
\put(37,27){\vector(1,1){14}}
\end{picture}
\caption{\label{fig_SiderischerTag}%
Siderischer Tag und Sonnentag. Da sich die Erde im Verlauf eines Tages um den Winkel
$\alpha$ weiterbewegt hat (hier \"ubertrieben dargestellt), muss sie sich relativ zum Fixsternhimmel
um diesen Winkel weiter drehen, damit ein bestimmter Punkt wieder in Richtung Sonne zeigt.}
\end{SCfigure}

Siderischer Tag und (mittlerer) Sonnentag unterscheiden sich um ein paar Minuten. 
Grund ist, dass sich die Erde im Laufe eines Tages etwas weiter um die Sonne bewegt hat
und daher die Sonne nach einem Tag nicht mehr exakt unter derselben Richtung steht
(im Vergleich zum Fixsternhimmel) wie vorher. Ganz grob kann man den Unterschied folgenderma\ss en
absch\"atzen: Die Erde bewegt sich in rund 365 Tagen einmal um die Sonne, d.h.\ an einem Tag
bewegt sie sich um den Winkel $\alpha = 365/360\approx 1$\,Grad weiter. Da der Tag $24\times 60$
Minuten hat, dreht sich die Erde in 4\,Minuten um ein Grad weiter. Der Sonnentag (24h) ist somit im
Mittel um rund 4 Minuten l\"anger als der siderische Tag (23h56m). Eine genauere Rechnung
ergibt als Differenz 3 Minuten und 56,6 Sekunden. 

\subsection{Monate}

F\"ur den Monat gibt es gleich mehrere Definitionen (die Zahlenangaben beziehen sich auf
die Bewegung des Mondes am 1.\ Januar des Jahres 2000):\index{Monat}
\begin{enumerate}
\item
\textit{Synodischer Monat}:\index{Monat!synodischer} 
Der synodische Monat ist die Zeitspanne zwischen zwei
gleichen Stellungen des Monds relativ zu Erde und Sonne, also beispielsweise die Zeitspanne
zwischen zwei Vollmonden oder Neumonden. Bei Vollmond spricht man auch von 
Opposition,\index{Opposition}\index{Konjunktion}
bei Neumond von Konjunktion. Dieser Monat ist am l\"angsten, da sich im Verlauf eines Monats
das Erde-Mond-System weiter um die Sonne bewegt hat. Ein synodischer Monat dauert rund
29,5306 Tage oder 29 Tage, 12 Stunden, 44 Minuten und 3 Sekunden. 
\item
\textit{Siderischer Monat}:\index{Monat!siderischer}
Beim siderischen Monat hat sich der Mond relativ zum Fixsternhimmel einmal um die
Erde gedreht. Der siderische Monat ist deutlich k\"urzer als der synodische Monat: In einem Monat
dreht sich das Erde-Mond-System um etwas weniger als 30 Grad um die Sonne. Das Verh\"altnis
von synodischem zu siderischem Monat ist somit ungef\"ahr $(360+30)/360=13/12$. Genauer
erh\"alt man f\"ur den siderischen Monat 27,3217 Tage oder 27 Tage, 7 Stunden, 43 Minuten und
12 Sekunden. 
\item
\textit{Tropischer Monat}:\index{Monat!tropischer}
Beim tropischen Monat bezieht sich \glqq einmal rum\grqq\ nicht auf den Fixsternhimmel
sondern auf den Fr\"uhlingspunkt der Erde. Wegen der\index{Praezession@Pr\"azession} 
Pr\"azession - der langsamen Drehung der Rotationsachse der Erde - verschiebt sich dieser
Fr\"uhlingspunkt im Vergleich zum Fixsternhimmel um rund 7 Sekunden im Monat. Um diese
7 Sekunden ist ein tropischer Monat k\"urzer als ein siderischer Monat.
\item
\textit{Anomalistischer Monat}:\index{Monat!anomalistischer}
Die Mondbahn um die Erde (genauer um den gemeinsamen Schwerpunkt) ist eine Ellipse.
Bei einem idealen gravitativen Zwei-K\"orper-Problem (ohne relativistische Korrekturen)
w\"are diese Ellipse stabil, d.h., das Perig\"aum (der erdn\"achste Punkt dieser Bahn) w\"are
relativ zum Fixsternhimmel immer derselbe. Durch verschiedene St\"orfaktoren (insbesondere
den Einfluss der Sonne aber auch relativistische Korrekturen) verschiebt sich dieser Punkt
jedoch im Laufe der Zeit relativ zum Fixsternhimmel. Als anomalistischen Monat bezeichnet man
die Zeitdauer zwischen zwei aufeinanderfolgenden Durchl\"aufen des Mondes durch das
Perig\"aum. Diese Definition bezieht sich somit ausschlie\ss lich auf die Bahnperiode der
Mondbahn um die Erde und bedarf keines \"au\ss eren Fixpunkts. 
Ein anomalistischer Monat dauert 27,55455 Tage oder 27 Tage, 13 Stunden, 18
Minuten und 33 Sekunden. 
\item
\textit{Drakonitischer Monat:}\index{Monat!drakonitischer}
Die Mondbahn liegt in einer Bahnebene, die relativ zur Ekliptik (also der Bahnebene der
Erde um die Sonne) um ungef\"ahr 5 Grad geneigt ist. Als Mondknoten bezeichnet man
die beiden Punkte der Mondbahn, die in der Ebene der Ekliptik liegen. Die Zeitspanne
zwischen zwei Durchg\"angen (von S\"ud nach Nord) des Mondes durch einen Mondknoten
bezeichnet man als drakonitischen Monat. Eine Sonnen- bzw.\ Mondfinsternis kann nur
dann von einem Punkt der Erde aus beobachtet werden, wenn diese drei Punkte -- der Punkt 
auf der Erde, der Mittelpunkt des Mondes und der Mittelpunkt der\index{Sonnenfinsternis}\index{Mondfinsternis}
Sonne -- auf einer Linie liegen. Dazu muss sich der Mond in diesem Augenblick in der
N\"ahe eines Mondknotens befinden, da sonst der Mond ober- bzw.\ unterhalb der Sonne
(bzw.\ des Erdschattens der Sonne bei einer Mondfinsternis) vorbeizieht. Der 
drakonitische Monat ist somit f\"ur die Berechnung von Sonnen- und Mondfinsternissen von
Bedeutung. Ein drakonitischer Monat dauert 27,21222 Tage oder 27 Tage, 5 Stunden, 5
Minuten und 36 Sekunden. 
\end{enumerate}

\subsection{Jahre}

Auch beim Jahr kann man wieder mehrere Definitionen unterscheiden (auch hier beziehen sich
die Zahlenangaben auf die Bewegung der Erde um die Sonne am 1.\ Januar 2000; die Dauer
eines Jahres ist aus dieser Bewegung rechnerisch extrapoliert):
\begin{enumerate}
\item
\textit{Siderisches Jahr}:\index{Jahr!siderisches}
Ein siderisches Jahr bezeichnet die Zeitdauer, in der sich die Erde relativ zum Fixsternhimmel
einmal um die Sonne bewegt hat. Es dauert 365 Tage, 6 Stunden, 9 Minuten und 9,54
Sekunden oder 365,2563604167 Tage. 
\item
\textit{Tropisches Jahr}:\index{Jahr!tropisches}
F\"ur das tropische Jahr gibt es zwei Definitionen. Die \"altere Definition bezieht sich auf den
Durchgang der Erde durch den Fr\"uhlingspunkt. Der Fr\"uhlingspunkt ist
dabei der Zeitpunkt, bei dem die Sonne vom Erdmittelpunkt aus betrachtet genau \"uber
dem \"Aquator steht.\footnote{Es gibt zwei solche Punkte im Jahr: der Fr\"uhlingspunkt und
der Herbstpunkt. Man sollte daher spezifizieren, dass die Sonne vom Erdmittelpunkt aus
betrachtet in diesem Augenblick den \"Aquator von S\"ud nach Nord durchl\"auft.} 
Zu diesem Zeitpunkt sind an einem idealisierten Tag die Sonnenstunden
(der helle Tag) und die Nachtstunden gleich lang. Daher spricht man auch von der
Tagundnachtgleiche bzw.\ dem \"Aquinoktium. Eine zweite Interpretation 
(dieser ersten Definition) ist: Zu diesem Zeitpunkt befindet sich die
Erde in einem der beiden Schnittpunkte ihrer Bahnebene (der Ekliptik) mit ihrer \"Aquatorebene,
d.h.\ die Rotationsachse der Erde steht senkrecht zur Verbindungslinie Erde-Sonne.  
Die L\"ange des so definierten tropischen Jahres kann in verschiedenen Jahren um mehrere Minuten
(bis zu einer Viertelstunde) schwanken. Die Einfl\"usse anderer Planeten auf die Pr\"azession der
Erde oder auch die Tatsache, dass der Fr\"uhlingspunkt (wegen der Pr\"azession) immer an einem
anderen Punkt der elliptischen Bahn der Erde ist, spielen hier eine wesentliche Rolle.

Die Internationale Astronomische Union (IAU) hat daher 1955 beschlossen, diese zwar sehr
anschauliche aber durch keinen pr\"azisen Wert angebbare Definition der L\"ange eines tropischen
Jahres durch eine zweite Definition zu ersetzen. Danach bestimmt man die L\"ange eines tropischen
Jahres in Bezug auf einen bestimmten Augenblick. In diesem Augenblick wird die Winkelgeschwindigkeit
einer mittleren Sonne -- die periodische Schwankung der Winkelgeschwindigkeit der Sonne aufgrund
der elliptischen Bahn der Erde wird hierbei durch den Bezug auf eine mittlere Sonne ausgeglichen -- bestimmt
und extrapoliert, wie lange es dauert, bis bei dieser Winkelgeschwindigkeit $360^\circ$ zur\"uckgelegt
werden. Dies bezeichnet man dann als momentanes tropisches Jahr. Diese Winkelgeschwindigkeit 
der mittleren Sonne wird auf die Drehachse der Erde, ein sogenanntes \glqq mittleres \"Aquinoktium
des Datums\grqq, bezogen. Diese Definition ist zwar unanschaulich, hat aber den Vorteil, dass man
einem tropischen Jahr zu jedem Augenblick einen pr\"azisen Wert zuordnen kann.   

Am 1.\ Januar 2000 dauerte ein tropisches Jahr nach dieser Definition $365,24219052$ SI-Tage. 
\item
\textit{Anomalistisches Jahr}:\index{Jahr!anomalistisches}
\"Ahnlich wie beim anomalistischen Monat bezeichnet ein anomalistisches Jahr die Zeitdauer
zwischen zwei Periheldurchg\"angen der Erde, wobei das Perihel der sonnenn\"achste Punkt der
Erdbahn ist. Aufgrund des Einflusses anderer Planeten sowie relativistischer Korrekturen
verschiebt sich das Perihel jedes Jahr um ungef\"ahr 5 Minuten relativ zum siderischen Jahr.
Das anomalistische Jahr (1.\ Januar 2000) dauert $365,259635864$ Tage oder 365 Tage,
6 Stunden, 13 Minuten und 52,54 Sekunden.
\end{enumerate}

\section{Der Menton-Zyklus und die Struktur von Mondkalendern}

Viele antike Kalendersysteme sind\index{Mondkalender} 
Mondkalender, d.h.\ bei ihnen ist der Monat die nat\"urliche
Einheit und zur ungef\"ahren Anpassung an die Jahresl\"ange wurden gelegentlich zus\"atzliche Tage oder gar
zus\"atzliche Monate eingef\"ugt. Schon im antiken Babylon war bekannt, dass 19 Jahre ziemlich genau
235 Monaten entsprechen. W\"ahlt man die Julianische Schaltjahrregelung, nach der ein Jahr
$365,25$ Tage hat, entsprechen 19 Jahren $6939,75$ Tage. Andererseits entsprechen 235 synodische
Monate $6939,691$ Tage. Auf 19 Jahre ein Fehler von $0,059$ Tagen (oder 1 Stunde, 24 Minuten und
58 Sekunden) bedeutet, dass sich in $19 \times 1/0,059 = 322$ 
(Julianischen) Jahren die Beziehung zwischen Mondphasen und Sonnenjahr um einen Tag verschiebt. 
Definiert man andererseits die L\"ange eines Jahres als 1/19.tel von 235 Monaten (das entspricht $365,2469$ Tagen), 
so liegt diese Zeitdauer zwischen dem tropischen Jahr und dem Jahr nach dem 
Julianischen Kalender.\index{Julianischer Kalender}\index{Kalender!Julianischer} 
Die Dauer von 235 (synodischen) Monaten 
bezeichnet man auch als Menton-Zyklus.\index{Menton-Zyklus} 

Im Folgenden geht es meist nur um die gr\"obsten Regeln eines Kalenders, sodass Jahre in Monate und Monate in 
Tage unterteilt werden k\"onnen. Die meisten Kalender enthalten weitere Ausnahmeregelungen, auf die hier
nicht eingegangen wird. 

\subsection{Der J\"udische Kalender}

Der J\"udische Kalender\index{Kalender!j\"udischer}
ist ein reiner Mondkalender. Monate mit 29 und 30 Tagen wechseln sich im
Wesentlichen ab. Ein Jahr besteht aus 12 Monaten. Damit hat ein Jahr rund 354 Tage. Da dies etwas zu
kurz ist, werden gelegentlich Schaltmonate\index{Schaltmonat} 
mit meist 30 Tagen eingef\"ugt. Insgesamt richtet sich diese
Einteilung nach dem Menton-Zyklus, d.h.\ 19 Jahre bestehen aus 235 Monaten. Da 19 Jahre mit 12 Monaten
nur 228 Monaten entsprechen, werden in den 19 Jahren insgesamt 7 Monate als Schaltmonate eingef\"ugt, 
und zwar in den Jahren 3, 6, 8, 11, 14, 17 und 19. Auf diese Weise erreicht man, dass der Jahresanfang
mehr oder weniger gleich bleibt; im J\"udischen Kalender im September oder Oktober. 

\subsection{Der Islamische Kalender}

Es gibt mehrere\index{Kalender!islamischer} 
verschiedene islamische Kalender, aber der Kalender, nachdem sich auch heute noch
die Festtage (oder beispielsweise der Beginn des Monats Ramadan) bestimmen, umfasst 12 Monate
mit jeweils 29 oder 30 Tagen. W\"ahrend im altarabischen Kalender alle zwei oder drei Jahre ein Schaltmonat
eingef\"ugt wurde (dieser Kalender also dem J\"udischen Kalender \"ahnelte), wurde im Islam dieser Schaltmonat
abgeschafft. Ein Jahr hat nun also rund 354 Tage. Damit verschieben sich der Jahresanfang und auch die
wichtigsten Feiertage j\"ahrlich um rund 10-12 Tage nach vorne und wandern im Verlauf der Zeit durch das
ganze Jahr. 

\section{Die Wochentage}

Schon in der Sch\"opfungsgeschichte (Genesis) des alten Testaments, die vermutlich auf das 5.\ bis 6.\ Jahrhundert
vor Christus zur\"uckgeht, ist davon die Rede, dass Gott die Welt in sechs Tagen erschuf und am siebten
Tage ruhte.\index{Woche} 
Die Zeiteinheit \glqq Woche\grqq\ als sieben Tage war schon in Babylon in Gebrauch und
vermutlich hat die j\"udische Tradition diese Einheit w\"ahrend des babylonischen Exils
\"ubernommen. 

Die Zahl sieben f\"ur die Anzahl der Tage in einer Woche geht vermutlich auf astronomische 
Beobachtungen zur\"uck: Die damals bekannten sieben beweglichen Himmelsk\"orper waren
(in aufsteigender Reihenfolge ihrer Umlaufzeiten): Mond (1 Monat), Merkur (3 Monate), Venus (7 Monate),
Sonne (1 Jahr), Mars (2 Jahre), Jupiter (12 Jahre) und Saturn (30 Jahre). Wie man heute noch an
den Bezeichnungen in einigen europ\"aischen Sprachen ablesen kann, wurden die 
Wochentage\index{Wochentage, Bezeichnungen}
urspr\"unglich nach diesen sieben Himmelsk\"orpern benannt (Tab.\ \ref{tab_Wochentage})

\begin{table}[htb]
\begin{tabular}{l|l|l|l|l}
Deutsch      &  Englisch      & Franz\"osisch & Lateinisch & Himmelsk\"orper   \\ \hline
Sonntag      &  Sunday       &  dimanche  & Solis dies &  Sonne      \\
Montag       &  Monday       &  lundi         & Lunae dies &  Mond       \\
Dienstag     &  Tuesday      &  mardi        & Martis dies &  Mars      \\
Mittwoch     &  Wednesday &  mercredi    & Mercurii dies &  Merkur    \\
Donnerstag &  Thursday     &  jeudi         & Iovis dies & Jupiter    \\
Freitag        &  Friday          &  vendredi   & Veneris dies  & Venus  \\
Samstag     &  Saturday     &  samedi   & Saturni dies  &  Saturn   \\
\end{tabular}
\caption{\label{tab_Wochentage}%
Die Wochentage in verschiedenen europ\"aischen Sprachen und die zugeh\"origen Planeten.}
\end{table}

Auch in den germanischen Sprachen sind diese Urspr\"unge teilweise erkennbar.
Der Donnerstag ist der Tag des Donnergottes Thor (im Englischen Thursday erkennbar),
der wiederum dem r\"omischen Gott Jupiter entsprach. Und der Freitag ist vermutlich
der Freyastag, der Tag der G\"otting Freya, die wiederum der r\"omischen Gottheit
Venus entsprach. Hier gibt es aber unterschiedliche Theorien.

\section{Die Kalenderreform von 1582}
% Kap 1
\label{sec_Kalender1582}

Einen relativ genauen\index{Kalenderreform}\index{Kalender!Julianischer} 
Sonnenkalender hat Julius C\"asar um 45 v.\ Chr.\ eingef\"uhrt:
Er sah vor, dass ein Jahr 365 Tage hat und dass alle vier Jahre ein sogenanntes Schaltjahr
eingef\"ugt wird, bei dem ein Jahr 366 Tage hat. Der zus\"atzliche Tag ist der 29.\ Februar.
Diese Schaltjahrregelung war schon vorher in \"Agypten in Gebrauch, und C\"asar hat sie
f\"ur das r\"omische Reich \"ubernommen und angepasst.

Damit ergibt sich f\"ur die Jahresl\"ange im Julianischen Kalender
\begin{equation}
          (4 \cdot 365 + 1)/4 = 365,25  \,  {\rm Tage}\, .
\end{equation}
Diese Formel folgt aus folgender \"Uberlegung: Ein Zeitraum von 4 Jahren bildet eine
Periode des Julianischen Kalenders, d.h.\ nach vier Jahren wiederholt sich das Schema.
Diese vier Jahre haben 4$\times$365 Tage plus einen weiteren Tag wegen des Schaltjahrs.
Teilt man diese Anzahl von Tagen wieder durch die Anzahl der Jahre einer Periode (also 4)
so erh\"alt man die mittlere Dauer eines Jahres in Tagen.

Unsere Jahreszeiten werden durch das tropische Jahr bestimmt, d.h.\ durch den
Zeitraum zwischen zwei aufeinanderfolgenden Fr\"ulingspunkten.\index{Jahr!tropisches}
Ein tropisches Jahr dauert $365,24219$ (Sonnen)\-Tage (genau gilt dies f\"ur ein von der Bewegung
der Sonne am 1.\ Januar 2000 extrapoliertes Jahr). Die Differenz zum Julianischen Kalender
sind $0,00781$\,Tage (oder 11 Minuten und 14,8 Sekunden) pro tropischem Jahr, bzw.\ alle 128 Jahre verschiebt sich
im Julianischen Kalender der Fr\"uhlingspunkt um einen Tag (nach vorne). Diese Differenz war
bereits im 9.\ Jahrhundert bekannt.\footnote{Bereits Mitte des 8.\ Jahrhunderts wusste man, dass
sich die Mondphasen relativ zu den theoretischen \"Uberlegungen (nach denen sich die
Mondphasen alle 235 Jahre wiederholen -- dies wusste man schon im alten Babylonien) 
verschoben hatten.} 
Da der Fr\"uhlingspunkt, d.h.\ der Beginn des Fr\"uhjahrs, 
im christlichen Jahr eine besondere Bedeutung hat -- Ostern (und damit viele bewegliche Feiertage
im Kirchenjahr) bestimmt sich aus dem ersten\index{Ostern}
Vollmond im Fr\"uhjahr -- wollte man nat\"urlich nicht, dass beispielsweise Ostern irgendwann
im Herbst stattfindet und Weihnachten im Hochsommer. Daher gab Papst Gregor XIII.\ die Ausarbeitung einer 
Kalenderreform in Auftrag, die er dann im Jahre 1582 per p\"apstlichem Dekret verk\"undete.

Die Kalenderreform bestand aus zwei Anteilen:\index{Kalender!Gregorianischer}
\begin{enumerate}
\item
Die 10 Tage zwischen dem 4.\ Oktober und dem 15.\ Oktober 1582 wurden ausgelassen, d.h.\ Donnerstag, 
dem 4.\ Oktober 1582, folgte Freitag, der 15.\ Oktober 1582. Die dazwischenliegenden Tage gibt es im 
Gregorianischen Kalender nicht. Damit wurde der Fr\"uhlingsanfang wieder auf den 21.\ M\"arz gelegt, 
wie im Jahr 325, als auf dem Konzil von Nic\"aa die Bestimmung des Osterfestes in Abh\"angigkeit vom 
Fr\"uhlingsanfang festgelegt wurde.  
\item
Die Schaltjahrregelung\index{Schaltjahr} 
wurde leicht abge\"andert: Alle vier Jahre (in den Jahren, die glatt durch 4
teilbar sind) gibt es ein Schaltjahr mit 366 Tagen, allerdings gibt es alle hundert Jahre (in den Jahren, die
glatt durch 100 teilbar sind) kein Schaltjahr, jedoch gibt es alle 400 Jahre (in den Jahren, die glatt
durch 400 teilbar sind) trotzdem ein Schaltjahr. Somit gab es im Jahr 1600 und 2000 ein Schaltjahr,
in den Jahren 1700, 1800 und 1900 jedoch nicht.   
\end{enumerate}
Damit ergibt sich f\"ur die Jahresl\"ange im Gregorianischen Kalender:
\begin{equation}
          (400 \cdot 365 + 100 - 4 + 1)/400 =  365,2425  \, {\rm Tage} \, .
\end{equation}
Hier gilt wieder: Eine Periode des Gregorianischen Kalenders dauert 400 Jahre, in diesen 400 Jahren
finden $100 - 4 + 1=97$ Schaltjahre mit 366 Tagen statt, alle anderen Jahre haben 365 Tage. Die 
Differenz zwischen gregorianischem Jahr und tropischem Jahr betr\"agt nun $0,00031$ Tage (oder
26,8 Sekunden). Das hei\ss t,
alle 3226 Jahre verschiebt sich dieser Kalender relativ zum tropischen Jahr (des Jahres 2000) um einen
Tag. Damit kann man zun\"achst einmal leben. Weitere Kalenderreformen sind in die ferne Zukunft verschoben.

Vom Kalendersystem unabh\"angig ist die Festlegung des Jahresbeginns. Obwohl schon der Kalender
von Julius Caesar den Jahresbeginn auf den 1.\ Januar festgelegt hatte, waren unterschiedliche
Konventionen in Gebrauch. Erst im 16.\ und 17.\ Jahrhundert setzte sich der 1.\ Januar als Jahresbeginn
weitgehend durch.


\section{Kuriosit\"aten der Kalenderreform}

Die Kalenderreform von Papst Gregor XIII.\ wurde nicht \"uberall gleich angenommen. Selbst katholische
Gebiete str\"aubten sich teilweise, und protestantische bzw.\ reformierte Gebiete verweigerten sich der
Reform schon alleine deshalb, weil sie vom Papst ausging. So wurde der Kalender in England erst 1752 
\"ubernommen (zu diesem Zeitpunkt mussten bereits 11 Tage ausfallen), in Sowjetrussland erst 1918
(hier wurden 13 Tage gestrichen) und beispielsweise in Griechenland erst 1923. In manchen 
orthodoxen Ostkirchen gilt heute noch der Julianische Kalender. Mit diesen unterschiedlichen
Daten der \"Ubernahme des Gregorianischen Kalenders sind einige Kurosit\"aten verbunden.
\begin{enumerate}
\item
William Shakespeare\index{Shakespeare, William}\index{Cervantes, Miguel de} 
und Miguel de Cervantes starben offiziell beide am 23.\ April 1616. 
Allerdings liegt der Todestag der beiden Dichter 10 Tage auseinander: In England verwendete man Anfang
des 17.\ Jahrhunderts noch den Julianischen Kalender, in Spanien bereits den Gregorianischen. 
Das bedeutet, Cervantes starb tats\"achlich 10 Tage vor Shakespeare. 
\item
F\"ur die Lebensdaten von Newton\index{Newton, Isaac} 
findet man zwei Versionen: 25.\,12.\,1642 bis 20.\ 3.\,1726
und 4.\,01.\,1643 bis 31.\,03.\,1727. Der Unterschied im Geburtsdatum ist offensichtlich: Das eine
Geburtsdatum (Dezember 1642) bezieht sich auf den Julianischen Kalender, der zu diesem Zeitpunkt in England
noch g\"ultig war, das andere Datum (Januar 1643) auf den Gregorianischen Kalender. \"Uberraschend am
Todesdatum ist weniger der Tag (20.3.\ bzw.\ 31.3.), der sich durch die Differenz zwischen Julianischem
und Gregorianischem Kalender ergibt, als vielmehr das Jahr: 1726 bzw.\ 1727. In England war Neujahr 1727 
(der Tag es Jahreswechsels) der 25.\ M\"arz und nicht der 1.\ Januar (diese Konvention bezeichnet man
als \textit{Annuntiationsstil}, weil der 25.\ M\"arz als die Verk\"undigung (Annuntiation) der
Empf\"angnis von Maria galt). Das bedeutet, nach dem damals in England g\"ultigen
Kalender starb Newton noch vor dem Jahreswechsel (also im Jahr 1726), nach
dem Gregorianischen Kalender nach dem Jahreswechsel (also im Jahr 1727).  
\item
Ebenfalls kurios ist Newtons Beerdigung, die nach dem Julianischen Kalender am 28.\ M\"arz 1727
stattfand. Das bedeutet, Newton ist nach dem Julianischen Kalender am 20.\ M\"arz 1726 gestorben
und am 28.\ M\"arz 1727 beerdigt worden. Das klingt zun\"achst so, als ob zwischen dem Sterbedatum
und der Beerdigung \"uber ein Jahr liegt. Doch wie schon erw\"ahnt fand in England damals der 
Jahreswechsel am 25.\ M\"arz statt. 
\item
Die sogenannte Oktoberrevolution\index{Oktoberrevolution} 
in Russland fand nach dem Julianischen Kalender am 25.\ Oktober
1917 statt, nach dem Gregorianischen Kalender am 7.\ November 1917. Im Jahre 1917 galt in Russland
jedoch noch der Julianische Kalender, daher \glqq Oktober-\grqq Revolution. Nach dem heutigen in Russland 
(bzw.\ der fr\"uheren Sowjetunion) g\"ultigen Kalender
handelte es sich eigentlich um eine Novemberrevolution. In der Sowjetunion wurde der
Jahrestag der Oktoberrevolution immer am 7.\ November gefeiert. 
\end{enumerate}  
\index{Kalender|)}

%\end{document}

