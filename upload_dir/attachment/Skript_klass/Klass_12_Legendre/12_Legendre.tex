\documentclass[german,10pt]{book}      
\usepackage{makeidx}
\usepackage{babel}            % Sprachunterstuetzung
\usepackage{amsmath}          % AMS "Grundpaket"
\usepackage{amssymb,amsfonts,amsthm,amscd} 
\usepackage{mathrsfs}
\usepackage{rotating}
\usepackage{sidecap}
\usepackage{graphicx}
\usepackage{color}
\usepackage{fancybox}
\usepackage{tikz}
\usetikzlibrary{arrows,snakes,backgrounds}
\usepackage{hyperref}
\hypersetup{colorlinks=true,
                    linkcolor=blue,
                    filecolor=magenta,
                    urlcolor=cyan,
                    pdftitle={Overleaf Example},
                    pdfpagemode=FullScreen,}
%\newcommand{\hyperref}[1]{\ref{#1}}
%
\definecolor{Gray}{gray}{0.80}
\DeclareMathSymbol{,}{\mathord}{letters}{"3B}
%
\newcounter{num}
\renewcommand{\thenum}{\arabic{num}}
\newenvironment{anmerkungen}
   {\begin{list}{(\thenum)}{%
   \usecounter{num}%
   \leftmargin0pt
   \itemindent5pt
   \topsep0pt
   \labelwidth0pt}%
   }{\end{list}}
%
\renewcommand{\arraystretch}{1.15}                % in Formeln und Tabellen   
\renewcommand{\baselinestretch}{1.15}                 % 1.15 facher
                                                      % Zeilenabst.
\newcommand{\Anmerkung}[1]{{\begin{footnotesize}#1 \end{footnotesize}}\\[0.2cm]}
\newcommand{\comment}[1]{}
\setlength{\parindent}{0em}           % Nicht einruecken am Anfang der Zeile 

\setlength{\textwidth}{15.4cm}
\setlength{\textheight}{23.0cm}
\setlength{\oddsidemargin}{1.0mm} 
\setlength{\evensidemargin}{-6.5mm}
\setlength{\topmargin}{-10mm} 
\setlength{\headheight}{0mm}
\newcommand{\identity}{{\bf 1}}
%
\newcommand{\vs}{\vspace{0.3cm}}
\newcommand{\noi}{\noindent}
\newcommand{\leer}{}

\newcommand{\engl}[1]{[\textit{#1}]}
\parindent 1.2cm
\sloppy

         \begin{document}  \setcounter{chapter}{11}

\chapter{Die Legendre-Transformation}
% Kap 12
\label{chap_Legendre}

\info{Thomas Filk}{11.05.2024}%
Legendre-Transformationen treten in der Physik in verschiedenen Zusammenh\"angen auf.
Die bekanntesten Beispiele sind die Transformation zwischen 
der Lagrange-Funktion (als Funktion von Ort und Geschwindigkeit)
und der Hamilton-Funktion (als Funktion von Ort und Impuls) in
der klassischen Mechanik, sowie die Transformationen zwischen
verschiedenen thermodynamischen Potentialen (freie Energie,
Enthalpie, gro\ss kanonische Freie Energie, etc.) in der
Thermodynamik bzw.\ statistischen Mechanik. Trotz ihrer gro\ss en Bedeutung
in der Physik, bleibt die Legendre-Transformation oft etwas R\"atselhaftes,
von dem man nicht so ganz versteht, worin ihre besondere Bedeutung liegt.

In der Physik beschreibt die Legendre-Transformation h\"aufig den 
Zusammenhang zwischen zwei durch verschiedene Funktionen ausgedr\"uckten 
Extremalbedingungen, die sich allerdings auf dasselbe Sys\-tem beziehen.
In der Mechanik beschreiben die L\"osungen der
Euler-Lagrange-Gleichungen, ausgedr\"uckt durch die Lagrange-Funktion, dieselben
Bahnkurven wie die Hamilton'schen Bewegungsgleichungen. In der
Thermodynamik soll in Abh\"angigkeit von den kontrollierten Parametern eines Ensembles
ein thermodynamisches Potential extremal werden: die Entropie im mikrokanonischen --
vollkommen abgeschlossenen --
Ensemble, die freie Energie im kanonischen Ensemble, bei dem ein W\"armeaustausch
mit der Umgebung erlaubt ist, oder auch die Enthalpie bzw.\
die freie Enthalpie, falls statt des Volumens der Druck kontrolliert wird. In allen
F\"allen wird die Ableitung der einen Funktion zur
Variablen der transformierten Funktion. 

Was oftmals verloren geht, ist die geometrische
Bedeutung der Legendre-Transformation als einer
\glqq Ber\"uhrungstransformation\grqq\ oder
\glqq Transformation von Linienelementen\grqq. Dieser
Zugang stand historisch im Vordergrund (wobei es
immer schon darum ging, Bedingungen f\"ur die L\"osbarkeit
bestimmter Differentialgleichungen zu formulieren).
In diesem Abschnitt soll dieser Aspekt n\"aher
beleuchtet werden. Viele \"Uberlegungen stammen aus
dem klassischen Buch von Sophus Lie 
\glqq Geometrie der Ber\"uhrungstransformationen\grqq\ \cite{Lie},
das zwar in etwas veralteter Sprache geschrieben ist,
aber trotzdem interessante geometrische Einblicke
liefert.


\section{Konvexe und konkave Funktionen}

Wir bezeichnen eine Teilmenge $K$ eines Vektorraums $V$ als konvex, wenn
die gerade Verbindungslinie zwischen zwei beliebigen Punkten 
$\pmb{x}_1, \pmb{x}_2 \in K$, d.h.\ die Menge der Punkte
\begin{equation}
      \pmb{y} = \pmb{x}_1 + \alpha (\pmb{x}_2-\pmb{x}_1) \hspace{1cm} 0 \leq \alpha \leq 1
\end{equation}
ganz in $K$ liegt. Insbesondere hei\ss t ein Gebiet in der Ebene konvex, wenn die
geradlinige Verbindungslinie zwischen je zwei Punkten dieses Gebiets ganz in dem
Gebiet verl\"auft. 

Das Konzept des Vektorraums l\"asst sich in diesem Fall auch auf
einen affinen Raum verallgemeinern. Auch auf Riemann'schen Mannigfaltigkeiten
kann man von konvexen Mengen sprechen, sofern die Bedingung 
\glqq gerade Verbindungslinie\grqq\ durch \glqq k\"urzeste Verbindungslinie\grqq\ ersetzt wird. 

Der Graph einer Funktion $f:U\in \mathbb{R}^n \rightarrow \mathbb{R}$ bezeichnet die
Menge der Punkte  $(\pmb{x},f(\pmb{x}))\in \mathbb{R}^{n+1}; x\in U$.
Eine Funktion $f:U\in \mathbb{R}^n \rightarrow \mathbb{R}$ hei\ss t konvex, 
wenn die Punkte oberhalb des Graphen dieser Funktion,
das sind die Punkte $(\pmb{x},y)$, f\"ur
die gilt $y\geq f(\pmb{x})$, eine konvexe Menge bilden (siehe Abb.\ \ref{fig_Leg_0} (links)). 
Die Menge der Punkte oberhalb des Graphen einer Funktion bezeichnet man auch 
als ihren Epigraph. 

Eine Funktion $f(\pmb{x})$ hei\ss t konkav, wenn die Punktemenge unterhalb des Graphen
dieser Funktion -- der sogenannte Hypograph, also die Menge der Punkte
$(\pmb{x},y)$ mit $y\leq f(\pmb{x})$ -- eine konvexe Menge bildet (siehe Abb.\ \ref{fig_Leg_0} (rechts)).

\begin{figure}[htb]
\begin{picture}(200,150)(0,0)
\put(10,20){\vector(1,0){180}}
\put(20,10){\vector(0,1){130}}
\qbezier(30,140)(50,30)(100,30)
\qbezier(100,30)(150,30)(170,140)
\put(70,60){\makebox(0,0){$\bullet$}}
\put(145,85){\makebox(0,0){$\bullet$}}
\put(70,60){\line(3,1){75}}
\put(190,13){\makebox(0,0){$x$}}
\put(30,147){\makebox(0,0){$f(x)$}}
\put(70,67){\makebox(0,0){$\pmb{x}_1$}}
\put(145,92){\makebox(0,0){$\pmb{x}_2$}}
\end{picture}
\hfill
\begin{picture}(200,150)(0,0)
\put(10,20){\vector(1,0){180}}
\put(20,10){\vector(0,1){130}}
\qbezier(40,30)(60,130)(110,130)
\qbezier(110,130)(160,130)(180,30)
\put(70,60){\makebox(0,0){$\bullet$}}
\put(145,85){\makebox(0,0){$\bullet$}}
\put(70,60){\line(3,1){75}}
\put(190,13){\makebox(0,0){$x$}}
\put(30,147){\makebox(0,0){$f(x)$}}
\put(70,67){\makebox(0,0){$\pmb{x}_1$}}
\put(145,92){\makebox(0,0){$\pmb{x}_2$}}
\end{picture}
\caption{\label{fig_Leg_0}%
Konvexe und konkave Funktionen. (links) Bei einer konvexen Funktion bildet die Menge oberhalb
des Graphen der Funktion eine konvexe Menge, d.h., die Verbindungslinie zweier beliebiger
Punkte aus dieser Menge liegt ebenfalls in dieser Menge. (rechts) Bei einer konkaven Funktion
ist die Menge unterhalb des Graphen dieser Funktion konvex.}
\end{figure} 

Eine \"aquivalente Definition von konvexer Funktion $f$ lautet, dass f\"ur zwei beliebige
Punkte $\pmb{x}_1$ und $\pmb{x}_2$ aus dem Definitionsbereich die Ungleichung 
\begin{equation}
          f ( \alpha \pmb{x}_1 + (1-\alpha) \pmb{x}_2) \leq  \alpha f (\pmb{x}_1) + (1-\alpha) f(\pmb{x}_2)
\end{equation}
gilt. Bei konvexen Funktionen handelt es sich somit um Verallgemeinerungen von linearen
Funktionen, f\"ur die das Gleichheitszeichen gilt. Bei streng konvexen Funktionen gilt immer
die Ungleichheit. Falls die Funktionen zweimal stetig differenzierbar sind, ist in diesem F\"allen 
die Hesse-Matrix der zweiten Ableitungen $\partial^2 f(x)/\partial x_i \partial x_j$ 
immer positiv definit.  

Im Folgenden beschr\"anken wir uns meist auf den Fall $n=1$, d.h., wir betrachten Funktionen
$f:U\in \mathbb{R} \rightarrow \mathbb{R}$. Au\ss erdem beschr\"anken wir uns auf streng konvexe
Funktionen. Die Verallgemeinerungen auf $n>1$ und konkave Funktionen ist in den meisten F\"allen
offensichtlich. Bei konkaven Funktionen vertauschen h\"aufig die Beziehungen zwischen Maxima und Minima.  

\section{Verschiedene Definitionen der Legendre-Transformation}

In den folgenden Abschnitten gehen wir auf unterschiedliche
Definitionen der Legendre-Transformation ein, vermeiden jedoch
vorl\"aufig noch die geometrische Bedeutung als Ber\"uhrungstransformation.

\subsection{Definition aus einem Extremalprinzip}

{\em Definition}:
Gegeben sei eine konvexe Funktion $f:U\subset \mathbb{R} \rightarrow \mathbb{R}$. 
Die Legendre-Transformierte von $f$ ist die Funktion
\begin{equation}
\label{eq_Leg_1}
     g(p) = {\rm max}_x \{  p \cdot x - f(x)\} \, . 
\end{equation}

Ist $f(x)$ einmal stetig differenzierbar, wird der gesuchte Extremalwert bei dem Punkt $x$
angenommen, der sich aus folgender Bedingung ergibt:
\begin{equation}
   \frac{{\rm d}}{{\rm d}x} (p \cdot x - f(x) )\Big|_{x(p)} = 0  ~~~{\rm bzw.}~~~
         p = \frac{{\rm d}f(x)}{{\rm d}x}\Big|_{x(p)} \, . 
\end{equation}
$x(p)$ kann auch als Umkehrabbildung zu
\begin{equation}
               p(x) =  \frac{{\rm d}f(x)}{{\rm d}x}
\end{equation}
aufgefasst werden. Der gesuchte Maximalwert ist dann
\begin{equation}
      g(p) = p \cdot x(p) - f(x(p)) \, .
\end{equation}     

\begin{figure}[htb]
\begin{picture}(200,200)(0,0)
\put(10,60){\vector(1,0){180}}
\put(50,0){\vector(0,1){200}}
\put(125,60){\line(0,1){76}}
\put(50,119){\line(1,0){75}}
\put(125,50){\makebox(0,0){$x(p)$}}
\put(155,190){\makebox(0,0){$f(x)$}}
\put(145,128){\makebox(0,0){$g(p)$}}
\put(128,128){\makebox(0,0){$\big\}$}}
\put(30,119){\makebox(0,0){$f(x(p))$}}
\put(30,22){\makebox(0,0){$y=px$}}
\thicklines
\put(20,30){\line(1,1){170}}
\qbezier(10,80)(120,70)(170,190)
\end{picture}
%
\hfill
\begin{picture}(200,200)(0,0)
\put(10,60){\vector(1,0){180}}
\put(50,0){\vector(0,1){200}}
\put(125,60){\line(0,1){60}}
\put(50,119){\line(1,0){75}}
\put(125,50){\makebox(0,0){$x(p)$}}
\put(150,180){\makebox(0,0){$f(x)$}}
\put(30,52){\makebox(0,0){$g(p)$}}
\put(46,52){\makebox(0,0){$\big\{$}}
\put(30,119){\makebox(0,0){$f(x(p))$}}
\put(180,150){\makebox(0,0){$y(x)=$}}
\put(175,135){\makebox(0,0){$px-g(p)$}}
\thicklines
\put(20,14){\line(1,1){170}}
\qbezier(10,80)(120,70)(170,190)
\end{picture}
\caption{\label{fig_Leg_1}%
Legendre-Transformation. (links) Wir k\"onnen durch den Ursprung eine Gerade
mit der Steigung $p$ legen, also $y=px$. Die Legendre-Transformierte $g(p)$ ist dann der
extremale Abstand zwischen dieser Geraden und der Funktion $f(x)$. Dieser befindet sich 
an der Stelle $x$, die durch die Bedingung $p=\frac{{\rm d}f(x)}{{\rm d}x}$ gegeben ist. 
(rechts) Zu einer gegebenen Funktion $f(x)$ k\"onnen wir an einen Punkt
$x$ eine Tangente mit der Steigung $p=\frac{{\rm d}f(x)}{{\rm d}x}$ legen. Der Schnittpunkt
dieser Tangente mit der $y$-Achse ist $-g(p)$.}
\end{figure}         

Anschaulich handelt es sich bei der Extremalbedingung f\"ur
$px-f(x)$ um die Differenz zwischen der Funktion $f(x)$ und 
der Funktion $px$, also einer Geraden durch den Ursprung 
mit Steigung $p$ (siehe Abb.\ \ref{fig_Leg_1} (rechts)). 
Diese Differenz wird an dem Punkt $x(p)$
extremal, wo die Steigung von $f$ gleich der Steigung
$p$ der Geraden ist, also $f'(x(p))=p$. Die Differenz zwischen beiden
Kurven an dieser Stelle ist gerade der Wert $g(p)$, um den
die Gerade nach oben oder unten verschoben werden muss, damit
sie zu einer Tangente an $f(x)$ wird.

\subsubsection{Anmerkungen}
\begin{enumerate}
\item
In mehr als einer Dimension wird aus Gl.\ \ref{eq_Leg_1}
\begin{equation}
           g(\pmb{p}) = {\rm max}_{\pmb{x}} \{  \langle \pmb{p}, \pmb{x} \rangle - f(\pmb{x}) \}  \, ,
\end{equation}
wobei $\langle \cdot , \cdot \rangle$ die nat\"urliche Paarung zwischen einem Element des
dualen Vektorraums, $\pmb{p} \in \mathbb{R}^{n\,*}$, und einem Element des Vektorraums
$\pmb{x} \in \mathbb{R}^n$ ist. Manchmal schreibt man auch 
$ \langle \pmb{p}, \pmb{x} \rangle = \pmb{p}^T \cdot \pmb{x}$. Streng genommen ist $g(\pmb{p})$
daher eine Funktion auf dem dualen Vektorraum. Dies wird in der Mechanik deutlich, wo
die Geschwindigkeit
$\pmb{v}=\dot{\pmb{x}}$ ein Element des Tangentialraums an die Mannigfaltigkeit des
Konfigurationsraums ist, wohingegen der Impuls $\pmb{p}$ ein Element des Kotangentialraums ist.
\item
Das Vorzeichen f\"ur die Legendre-Transformation ist Konvention. Manchmal definiert man auch
\begin{equation}
     g(p) = {\rm min}_x \{ f(x) -  p \cdot x \} \, . 
\end{equation}
\item
Es handelt sich bei dieser Definition um eine Verallgemeinerung der Suche nach
Extremalpunkten. Ein Extremalpunkt $x_0$ einer Funktion $f$ liegt vor, wenn der Abstand 
eines Funktionswertes von $f$ von der $x$-Achse, also der Geraden $y=0$, extremal 
(maximal oder minimal) wird. Dieser Abstand ist gleich $f(x_0)$. 
 
Dies wird nun verallgemeinert zu der Fragestellung: Gesucht ist ein Punkt $x(p)$,
f\"ur den der Abstand einer Funktion $f$ von der Geraden $y=px$ extremal wird.
Dieser Abstand ist dann durch $g(p)=p x(p)-f(x(p))$, also der Legendre-Transformation von 
$f(x)$ gegeben.  
\end{enumerate}

\subsection{Analytische Definition der Legendre-Transformation}

{\em Definition}:
Gegeben sei eine konvexe Funktion $f:U \subset \mathbb{R} \rightarrow \mathbb{R}$
mit $x \mapsto f(x)$. Wir definieren eine zweite
Funktion $g:\mathbb{R} \rightarrow \mathbb{R}$ mit
$p \mapsto g(p)$ nach der Vorschrift:
\begin{equation}
  g(p) = p \cdot  x(p)  - f(x(p))  \, . 
\end{equation}
Hierbei ist $x(p)$ die Umkehrabbildung zu
\begin{equation}
     p(x) := \frac{{\rm d}f(x)}{{\rm d}x} \, . 
\end{equation}     
Die geometrische Konstruktionsvorschrift dieser Funktion 
$g(p)$ zu einer gegebenen Funktion $f(x)$ ist die
Folgende:

Gegeben $p\in \mathbb{R}$. Wir suchen zun\"achst den Punkt
$x\in \mathbb{R}$, bei dem die Steigung der Funktion $f$
gleich $p$ ist: 
\begin{equation}
              p=\left. \frac{{\rm d} f(x)}{{\rm d} x} \right|_{x=x(p)} \, .
\end{equation}      
Nun legen wir an diesen Punkt
$(x(p),f(x(p)))\in \mathbb{R}^2$ die Tangente: 
\begin{equation}
                    y(x)= p (x-x(p)) +f(x(p)) \,  ,
\end{equation}                    
(siehe Abb.\ \ref{fig_Leg_1}(rechts)). 
Der Schnittpunkt dieser Tangente mit der $x=0$-Achse
ergibt $-g(p)$:
\begin{equation}
      - g(p) = y(x)\Big|_{x=0} = - p x(p)  + f(x(p)) \, .
\end{equation}     

\subsection{Bezug zur Laplace-Transformation}

Der Bezug der Legendre-Transformation zur Laplace-Transformation wird aus
folgender \"Uberlegung deutlich. Sei
\begin{equation}
      F(x) = e^{-f(x)}    
\end{equation}     
eine (positive) Funktion, wobei $f$ konvex sein soll (d.h.\ $f''(x)>0$) und sich
f\"ur $x\rightarrow \pm \infty$ so verhalten soll, dass die folgenden Integrale existieren. 
Dann ist die Laplace-Transformierte:
\begin{equation}
      G(p) = \int_0^{\infty} F(x) \, e^{-px}\,{\rm d}x
    = \int_0^{\infty} e^{-f(x)-px}\, {\rm d}x \, . 
\end{equation}     
Wenn nun, wie es in der statistischen Mechanik der
Fall ist, die Funktion $e^{-f(x)-px}$ ein sehr scharfes
Maximum an einer Stelle $x_0$ hat, kann man 
um dieses Maximum entwickeln und erh\"alt
$y=x-x_0$ als neue Integrationsvarialbe:
\begin{equation}
    G(p) = \int e^{-f(x_0)-px_0 - f''(x_0)y^2 \pm ...}
     \, {\rm d}y = e^{-f(x_0)-px_0} \int e^{-f''(x_0)y^2 + ...}
         {\rm d}y \, . 
\end{equation}     
F\"ur den Logarithmus von $G(p)$ folgt:
\begin{equation}
  g(p) = \log G(u) = - f(x_0) - p x_0 + ... 
\end{equation}     
Von den h\"oheren Termen kann man zeigen, dass sie
im thermodynamischen Grenzfall verschwinden,
sodass n\"aherungsweise:
\begin{equation}
    g(p) \approx - f(x_0) - p x_0  
\end{equation}     
gilt. $x_0$ ist dabei als L\"osung aus der Extremalbedingung
\begin{equation}
   (-f(x)-px)'=0 
\end{equation}     
zu bestimmen. Das abge\"anderte Vorzeichen im Vergleich zur \glqq normalen\grqq\
Legendre-Transformation ist in der Thermodynamik \"ublich.    

\subsection{Definition \"uber die Ableitungen als Umkehrfunktionen}
\label{sec_Leg_Umkehr}

Wir k\"onnen die Legendre-Transformation auch \"uber
folgende Bedingung einf\"uhren:

Gesucht ist die Beziehung zwischen zwei Funktionen $f(x)$  
und $g(p)$, so dass die ersten Ableitungen dieser
beiden Funktionen jeweils Umkehrabbildungen voneinander
sind. Etwas anders ausgedr\"uckt: Gegeben sei eine Funktion $f(x)$ mit
$f'(x)=p(x)$. Die Umkehrabbildung von $p(x)$ sei $x(p)$. Gesucht ist die
Funktion $g(p)$, sodass $g'(p)=x(p)$. 

Wir zeigen, dass die Legendre-Transformierte $g(p)=px(p) -f(x(p))$ von 
$f(x)$ diese Bedingung erf\"ullt. Es gilt:
\begin{eqnarray*}
   \frac{{\rm d}g}{{\rm d}p} &=& 
         \frac{{\rm d}}{{\rm d}p} (px(p) - f(x(p))) \\[0.3cm]
   &=&  x(p) + p \frac{{\rm d}x(p)}{{\rm d}p} -
        \frac{{\rm d}f(x)}{{\rm d} x}\Big|_{x=x(p)}
         \frac{{\rm d}x(p)}{{\rm d}p} \\[0.3cm]
         & =& x(p) + p x'(p) - p x'(p) =
            x(p) \, .
\end{eqnarray*}
Bei gegebenem $f(x)$ ist die Funktion $g(p)$ durch
diese Bedingung nur bis auf eine Konstante
festgelegt. Diese Konstante kann durch die
Bedingung
\begin{equation}
    g(p) + f(x) =  p x 
\end{equation}     
festgelegt werden, wobei man in dieser Gleichung sowohl
$x$ als Funktion von $p$ auffassen kann, $x=x(p)$,
oder auch $p$ als Funktion von $x$: $p=p(x)$.

\subsection{Eine geometrische Sichtweise}

Wir k\"onnen die Sichtweise des letzten Abschnitts \ref{sec_Leg_Umkehr} noch
etwas weiter treiben und geometrisch veranschaulichen (dieser Abschnitt verwendet
Ideen aus \cite{Wouter}).

Gegeben seien die beiden Funktionen $x(p)$ und $p(x)$, die Umkehrfunktionen
voneinander sein sollen. Das setzt voraus, dass beide Funktionen in ihrem Definitionsbereich
monoton sind. In einem $px$-Diagramm lassen sich die Graphen durch dieselbe Linie
darstellen: $(p,x(p))$ und $(p(x),x)$, je nachdem, ob man die Abszisse oder die Ordinate als
unabh\"angige Variable auffasst (siehe Abb.\ \ref{fig_Leg_Umkehr}).  

\begin{SCfigure}[50][htb]
\begin{picture}(170,170)(0,0)
\put(20,0){\vector(0,1){160}}
\put(0,20){\vector(1,0){160}}
\put(13,160){\makebox(0,0){$x'$}}
\put(160,13){\makebox(0,0){$p'$}}
\put(137,150){\makebox(0,0){$x(p')$}}
\put(157,138){\makebox(0,0){$p(x')$}}
\put(140,13){\makebox(0,0){$p$}}
\put(13,130){\makebox(0,0){$x$}}
\put(90,50){\makebox(0,0){$g(p)$}}
\put(50,95){\makebox(0,0){$f(x)$}}
\multiput(20,130)(3,0){41}{\makebox(0,0){$\cdot$}}
\multiput(140,20)(0,3){37}{\makebox(0,0){$\cdot$}}
\thicklines
\qbezier(20,20)(55,80)(90,90)
\qbezier(90,90)(130,100)(150,150)
\end{picture}
\caption{\label{fig_Leg_Umkehr}%
Die Integrale \"uber eine Funktion $x(p')$ und ihre Umkehrfunktion $p(x')$, die jeweils
denselben Graphen $(p(x'),x')$ und $(p',x(p'))$ haben, spannen
bei geeignet gew\"ahlten Integrationsgrenzen $x$ und $p$ ein Rechteck der Fl\"ache $px$ auf. 
Die Stammfunktionen $f(x)$ und $g(p)$, die jeweils die Fl\"achen unter den Kurven angeben, 
erf\"ullen daher die Gleichung
$f(x)+g(p)=px$.}
\end{SCfigure}

Die zugeh\"origen Stammfunktionen $f(x)$ und $g(p)$ lassen sich geometrisch
als Fl\"achen unter den Kurven interpretieren und als Integrale \"uber ihre Ableitungen
schreiben:
\begin{equation}
      f(x) = \int_0^x p(x') \,{\rm d}x'  \hspace{1cm} {\rm und} \hspace{1cm}
      g(p) = \int_0^p x(p')\, {\rm d}p' \, . 
\end{equation}
Sind die Punkte $x$ und $p$ so gew\"ahlt, dass $x(p)=x$ und $p(x)=p$, ist die
Gesamtfl\"ache unter beiden Kurven gleich der Fl\"ache des Rechtecks $px$.
Somit gilt:
\begin{equation}
                  f(x) + g(p) = px  \, . 
\end{equation}


\subsection{Die Legendre-Transformation als Beziehung zwischen totalen Differentialen}

Gegeben sei eine Funktion $f(x)$ mit dem totalen Differential
\begin{equation}
            {\rm d} f(x) = \frac{\partial f(x)}{\partial x} {\rm d}x = p(x) \, {\rm d}x \, .
\end{equation}
Gesucht ist eine Funktion $g(p)$ mit der Eigenschaft, dass die Umkehrfunktion
von $p(x)$, d.h.\ die Funktion $x(p)$, als Ableitung im totalen Differential auftritt.
Es soll also gelten: ${\rm d}g = x\,{\rm d}p$. 
(Dies ist die Formulierung der Beziehungen aus Abschnitt \ref{sec_Leg_Umkehr}
durch Differentialformen.)

Der \"ubliche Trick ist, ${\rm d}f(x)$ um $0=x\,{\rm d}p  - x\, {\rm d}p$ zu
erweitern und die Terme umzuordnen:
\begin{equation}
       {\rm d}f = p\, {\rm d} x = p\,{\rm d}x + x\, {\rm d}p - x\,{\rm d}p
                =  {\rm d} (px) - x \,{\rm d}p  \, .
\end{equation}
Daraus ergibt sich
\begin{equation}
      x \,  {\rm d}p 
                =  {\rm d} (px) -  {\rm d}f = {\rm d} \big( px-f \big)  \, .
\end{equation}
Das totale Differential von $g(p)=px-f(x)$, wobei $x$ als Funktion von $p$ aufzufassen ist,
ist somit gleich $x\,{\rm d}p$. 


\subsection{Legendre-Transformation und Einh\"ullende}

Die Legendre-Transformation l\"asst sich auch als eine Transformation von einer
Funktion $f(x)$ zu einer Funktion $g(p)$ auffassen, wobei $g(p)$ die Tangenten
an $f$ charakterisieren. Auf diese Weise kann man die Funktion $f(x)$ als
Einh\"ullende zu ihrer Tangentenschar auffassen. In dieser Interpretation wird das
Konzept der Ber\"uhrungstransformation schon angedeutet.

Sei allgemein eine Kurvenschar $h(x,p)$ gegeben, wobei $h$ als Funktion von $x$
zu einem festen Parameter $p$ eine Kurve beschreibt. $p$ parametrisiert die
verschiedenen Kurven. Wir nehmen nun an, dass diese Kurvenschar eine Einh\"ullende definiert, d.h., 
es gibt eine Funktion, die von jeder Kurve in genau einem Punkt ber\"uhrt wird.
Es sei $x(p)$ dieser Punkt, bei dem f\"ur ein festes $p$ diese Ber\"uhrung
stattfindet. Dann ist in einer gen\"ugend kleinen Umgebung von $p$ (sofern die Einh\"ullende 
dort konvex ist)
\begin{equation}
               h(x(p),p\pm{\rm d}p) \leq h(x(p),p) 
\end{equation}
und damit
\begin{equation}
               \left. \frac{\partial h(x,p)}{\partial p}\right|_{x=x(p)} = 0 \, .
\end{equation}
In Abb.\ \ref{fig_Leg_envelope} ist dieser Fall f\"ur $h(x,p)=xp-g(p)$ dargestellt. 
F\"ur einen festgehaltenen Wert von $p$ beschreibt $h(x,p)$ als Funktion von $x$ eine Gerade in der 
$x,y$-Ebene mit einer Steigung $p$ und einem Schnittpunkt $-g(p)$ mit der $y$-Achse.
Offenbar haben diese Geraden eine Einh\"ullende $f(x)$, d.h.\ jede Gerade ist tangential zu
$f(x)$ in einem bestimmten Punkt $x(p)$ (Abb.\ \ref{fig_Leg_envelope} (links)). 
Der Funktionswert $f(x(p))=h(x(p),p)$ ist (sofern 
$f(x)$ in einer Umgebung von $x(p)$ konvex ist) ein Maximum von $h(x(p),p')$ als Funktion von $p'$
(Abb.\ \ref{fig_Leg_envelope} (rechts)).  
Damit erhalten wir als Bedingung f\"ur den Ber\"uhrungspunkt $x(p)$:
\begin{equation}
             \frac{\partial h(x,p)}{\partial p} = x - \frac{\partial g(p)}{\partial p} = 0
             \hspace{1cm} {\rm oder} \hspace{1cm} x(p) = \frac{\partial g(p)}{\partial p}  \, .
\end{equation} 
Diese Gleichung l\"asst sich zu $p(x)$ aufl\"osen.
Die Einh\"ullende $f(x)$ ergibt sich damit zu
\begin{equation}
                  f(x) = h(x,p(x)) = xp(x) - g(p(x))  \, . 
\end{equation}
Diese Einh\"ullende $f(x)$ ist somit die Legendre-Transformierte von $g(p)$. 

\begin{figure}[htb]
\begin{picture}(210,220)(-10,0)
\put(20,10){\vector(0,1){200}}
\put(10,70){\vector(1,0){180}}

\put(103,75){\line(3,5){80}}
\put(100,91.5){\line(4,5){90}}
\put(15,16.8){\line(1,1){170}}
\put(90,102){\line(4,3){100}}
\put(80,105){\line(2,1){100}}
\put(75,109.3){\line(3,1){110}}
\put(70,113){\line(5,1){120}}
\put(60,119.5){\line(1,0){120}}
\put(40,130.8){\line(5,-1){120}}
\put(30,141){\line(3,-1){120}}
\put(20,155){\line(2,-1){120}}
\put(10,177){\line(4,-3){120}}
\put(10,192){\line(1,-1){120}}
\put(10,205){\line(4,-5){90}}
\put(10,221){\line(3,-5){80}}

\put(10,45){\makebox(0,0){$g(p)$}}
\put(190,63){\makebox(0,0){$x$}}
\put(45,190){\makebox(0,0){$f(x)$}}
\put(26,210){\makebox(0,0){$y$}}
\put(30,15){\makebox(0,0){$1$}}
\put(45,30){\makebox(0,0){$p$}}

\put(20,22){\line(1,0){20}}
\put(40,22){\line(0,1){20}}

\thicklines
\qbezier(30,190)(100,50)(170,190)
\end{picture}
%
\begin{picture}(200,190)(-30,0)
\put(20,10){\vector(0,1){200}}
\put(10,70){\vector(1,0){180}}

\put(112.2,75){\line(3,5){80}}
\put(85,75){\line(1,1){105}}
\put(45,75){\line(3,1){130}}

\put(190,63){\makebox(0,0){$x$}}
\put(45,190){\makebox(0,0){$f(x)$}}
\put(26,210){\makebox(0,0){$y$}}
%\put(30,15){\makebox(0,0){$1$}}
\put(130,64){\makebox(0,0){${\scriptstyle x(p)}$}}
\put(170,64){\makebox(0,0){${\scriptstyle x(p+{\rm d}p)}$}}
\put(65,64){\makebox(0,0){${\scriptstyle x(p-{\rm d}p)}$}}
\put(130,120){\makebox(0,0){${\scriptstyle \bullet}$}}
\put(170,172){\makebox(0,0){${\scriptstyle \bullet}$}}
\put(78,87){\makebox(0,0){${\scriptstyle \bullet}$}}
\put(130,105.5){\makebox(0,0){${\scriptstyle \bullet}$}}
\put(130,103){\makebox(0,0){${\scriptstyle \bullet}$}}
\put(-6,107.6){\makebox(0,0){${\scriptstyle h(x(p),p+{\rm d}p))}$}}
\put(3,120){\makebox(0,0){${\scriptstyle h(x(p),p)}$}}
\put(-6,101.9){\makebox(0,0){${\scriptstyle h(x(p),p-{\rm d}p))}$}}

\put(170,70){\line(0,1){102}}
\put(130,70){\line(0,1){50}}
\put(78,70){\line(0,1){17}}
\put(20,120){\line(1,0){110}}
\put(20,105.5){\line(1,0){92}}
\put(20,103){\line(1,0){88}}

\thicklines
\qbezier(40,80)(120,80)(180,190)

\end{picture}
\caption{\label{fig_Leg_envelope}%
(links) Die Funktion $f(x)$ als Einh\"ullende ihrer Tangenten. Die Funktionenschar $h(x,p)=px-g(p)$, 
aufgefasst als Funktionen von $x$, definieren Geraden mit einer Steigung $p$. 
$g(p)$ ist der Schnittpunkt der Geraden mit der $y$-Achse. (rechts) Wenn $h(x(p))$ der
Funktionswert von $h(x,p)$ an der Stelle ist, wo die Funktion $h(x,p)$ bei festgehaltenem $p$ 
als Funktion von $x$ die Steigung $p$ hat, dann handelt
es sich um einen Extremwert, d.h., die Tangenten zur infinitesimal benachbarten $p$-Werten haben
bei $x(p)$ einen kleineren Funktionswert als $h(x(p))$.}
\end{figure}

\section{Ber\"uhrungstransformationen}

Die Legendre-Transformation ordnet einer Funktion $f(x)$ mit
Argument $x$ eine neue Funktion $g(p)$ mit Argument $p$ zu.
\"Ahnlich wie viele Integraltransformationen (z.B. die Fourier-Transformation)
folgt auch die Legendre-Transformation {\em nicht} aus einer
Punkttransformation (oder Koordinatentransformation) der Ebene
$(x,y)$, in der der Graph der Funktion definiert ist. Allerdings 
handelt es sich um eine einfache Erweiterung einer solchen
Punkttransformation. Bevor wir die so genannten Ber\"uhrungstransformationen
untersuchen, betrachten wir nochmals kurz die Punkttransformationen
und ihre Eigenschaften.

\subsection{Punkttransformationen}

Die folgenden \"Uberlegungen gelten f\"ur Punkt- oder
Koordinatentransformationen in beliebigen Dimensionen, wir
betrachten jedoch immer nur den Fall $d=2$.

Sei $U\subset {\bf R}^2$ ein Teilgebiet der Ebene, parametrisiert
durch $(x,y)$. Eine Punkt- oder Koordinatentransformation besteht
aus einer (2-komponentigen) Abbildung:
\begin{equation}
     (X,Y):  U\subset {\bf R}^2 \rightarrow U'\subset {\bf R}^2
     ~~~{\mit} ~~~ (x,y) \mapsto  (X(x,y),Y(x,y)) \, . 
\end{equation}     
Die Abbildung $(X,Y)$ soll injektiv sein. 

Oft betrachtet man nun das Transformationsverhalten von
Funktionen $v:U\rightarrow {\bf R}$ mit $(x,y) \mapsto v(x,y)$ unter
den Koordinatentransformationen. Wir erhalten eine neue Funktion
$V:U' \rightarrow {\bf R}$ mit $(X,Y) \mapsto V(X,Y)$, die durch die
Bedingung:
\begin{equation}
   v(x,y) = V(X(x,y),Y(x,y))   
\end{equation}     
gegeben ist. 

F\"ur das Folgende sind wir jedoch an dem Transformationsverhalten
von Abbildungen $f:{\bf R} \rightarrow {\bf R}$ mit $x\mapsto f(x)$
bzw.\ deren Graphen $(x,f(x)) \in U$ unter der Transformation
interessiert. Der neue Graph ist durch die Parameterdarstellung
$(X(x,f(x)),Y(x,f(x)))$ gegeben. Durch Aufl\"osung von $X(x)=X(x,f(x))$ nach
$x(X)$ und Einsetzen in den zweiten Term erhalten wir eine
Darstellung der Kurve in der Form $Y=F(X)$, bzw. des Graphen in der
Form $(X,F(X))$. 

Wir behaupten nun, dass unter einer Punkt- bzw. Koordinatentransformation
zwei Kurven $(x,f_1(x))$ und $(x,f_2(x))$, die sich in einem Punkt $(x_0,y_0)$
{\em ber\"uhren}, in zwei Kurven $(X,F_1(X))$ und
$(X,F_2(X))$ abgebildet werden, die sich in dem Punkt $(X_0,Y_0)=
(X(x_0,y_0),Y(x_0,y_0))$ ber\"uhren. (In der passiven Interpretation einer
Koordinatentransformation ist das trivial, aber auch in der aktiven 
Interpretation, bei der eine Kurve mittransformiert wird, ist es
offensichtlich.) 

Hinter dieser Bedingung steckt auch die Definition
f\"ur das Transformationsverhalten von Vektoren auf Mannigfaltigkeiten.
Ein Vektor an einem Punkt $(x_0,y_0)$ ist dort durch \"Aquivalenzklassen 
von Kurven durch  diesen Punkt definiert, wobei zwei Kurven 
\"aquivalent sind, wenn die Richtungsableitungen aller Funktionen
auf der Mannigfaltigkeit entlang der beiden Kurven an diesem Punkt gleich
sind. Das bedeutet, die beiden Kurven haben in diesem Punkt dieselben
Tangentialvektoren. 

Zum Beweis bilden wir die Ableitung von $(X(x,f(x)),Y(x,f(x)))$ am Punkte
$x_0$:
\begin{equation}
 \frac{{\rm d}}{{\rm d}x} (X(x,f(x)),Y(x,f(x))) =
   ( \partial_1 X + \partial_2 X\cdot f'(x)~, \partial_1Y + 
                                 \partial_2Y \cdot f'(x)) \, .
\end{equation}     
Hierbei sind:
\begin{equation}
  \partial_1X = \frac{\partial X}{\partial x}~,~~                                 
 \partial_2X = \frac{\partial X}{\partial y}~,~~
  \partial_1Y = \frac{\partial Y}{\partial x}~,~
   \partial_2Y = \frac{\partial Y}{\partial y} \, .
\end{equation}      
Der Tangentialvektor h\"angt also nur von den partiellen Ableitungen von $X$ und
$Y$ nach ihren Argumenten an dem Punkt $x$ und $y$ ab, die
sind aber an demselben Punkt immer gleich, und von der
Ableitung $f'(x)$ and dem betreffenden Punkt. Wenn aber zwei
Funktionen an einem Punkt dieselbe Ableitung haben, sind diese
Ausdr\"ucke ebenfalls gleich.

Diese Eigenschaft -- Kurven, die sich in einem Punkt ber\"uhren, gehen wieder
in Kurven \"uber, die sich ber\"uhren -- wird bei Ber\"uhrungstransformationen
verallgemeinert.

\subsection{Linienelemente und Ber\"uhrungstransformationen}
 
Ein {\em Linienelement} kann man sich als einen
\glqq infinitesimalen\grqq\ Abschnitt einer Geraden vorstellen.
Es ist spezifiziert durch die Vorgabe eines Punktes $(x,y)$, an dem 
das Linienelement anliegt, und einer Steigung $p=y'$, die das 
Linienelement hat. Ein Linienelement besteht also aus dem
Tripel $(x,y,p)$. Man beachte, dass $p$ hier nicht als die
Ableitung einer Kurve zu verstehen ist, sondern einfach als die
Vorgabe einer Steigung.

Eine {\em Ber\"uhrungstransformation} ist eine Punkttransformation
auf den Linienelementen. Formal handelt es sich einfach um eine
Punkttransformation auf dem ${\bf R}^3$:
\begin{equation}
    (x,y,y') \mapsto  (\, p(x,y,y') ~,~g(x,y,y')~,~g'(x,y,y')\,) \, . 
\end{equation}      

Anmerkungen:
\begin{enumerate}
\item
Offenbar definiert eine Punkttransformation auf einer Ebene
eine spezielle Ber\"uhrungstransformation. In diesem Fall gilt:
\begin{eqnarray*}
    p(x,y,y') &=& X(x,y) ~,~~g(x,y,y')=Y(x,y) ~, \\[0.3cm]
    g'(x,y,y') &=&
  \frac{\partial_1Y(x,y) + \partial_2Y(x,y) \cdot y'}{\partial_1 X(x,y)
      + \partial_2 X(x,y) \cdot y'} \, .
\end{eqnarray*} 
In den ersten beiden Koordinaten handelt es sich um eine
gew\"ohnliche Punkttransformation der Ebene, und die
dritte Koordinate (die Steigung) ist die durch die
Punkttransformation induzierte transformierte Steigung einer
Kurve, die nach dem oben Erw\"ahnten nicht von der
speziellen Wahl der Kurve abh\"angt, sondern nur von ihrer
Steigung in dem betreffenden Punkt.
\item
Eine Legendre-Transformation l\"asst sich in folgender Form
als Ber\"uhrungstransformation auffassen:
\begin{equation}
  p(x,y,y')= y' ~,~~g(x,y,y')= xy'-y ~,~~g'(x,y,y')= x \, . 
\end{equation}     
Man beachte, dass in diesem Fall die Legendre-Transformation
nicht als Transformation einer Kurve aufgefasst wird, sondern
als eine Transformation auf der Menge der Linienelemente in
einer Ebene. Zu der herk\"ommlichen Transformation von
Funktionen wird diese Ber\"uhrungstransformation erst, wenn
wir die Wirkung dieser Transformation auf eine Kurve in der
Ebene zusammen mit ihren Linienelementen an jedem Punkt
(ihrer Ableitung) betrachten. 
\end{enumerate}
Jede Funktion $x\mapsto f(x)$ definiert in der Ebene einen
Graphen (eine Kurve) der Form $(x,f(x))$, und sie definiert im
Raum der Linienelemente ebenfalls eine Kurve:
$(x,f(x),f'(x))$. Wenn sich zwei Kurven in einem Punkt
ber\"uhren, d.h., f\"ur einen speziellen Wert $x_0$ gilt
$(x_0,f(x_0),f'(x_0)) = (x_0,g(x_0), g'(x_0))$, dann sind
auch die Bilder dieser beiden Linienelemente gleich.


\begin{thebibliography}{9}
\bibitem{Lie} Lie, Sophus \& Scheffers, Georg; \textit{Geometrie der Ber\"uhrungstransformationen}
              Verlag B.\,G.\ Teubner, Leipzig, 1986.
\bibitem{Wouter}  Wouter (\url{https://physics.stackexchange.com/users/16660/wouter}), 
           Physical meaning of Legendre transformation, 
           URL (version: 2014-04-01): \url{https://physics.stackexchange.com/q/69374} \\
           Normalerweise w\"urde ich solche Referenzen nicht verwenden, aber dies ist mal
           eine originelle Ausnahme.
\end{thebibliography}



\end{document}



