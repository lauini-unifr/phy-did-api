%\documentclass[german,10pt]{book}      
\usepackage{makeidx}
\usepackage{babel}            % Sprachunterstuetzung
\usepackage{amsmath}          % AMS "Grundpaket"
\usepackage{amssymb,amsfonts,amsthm,amscd} 
\usepackage{mathrsfs}
\usepackage{rotating}
\usepackage{sidecap}
\usepackage{graphicx}
\usepackage{color}
\usepackage{fancybox}
\usepackage{tikz}
\usetikzlibrary{arrows,snakes,backgrounds}
\usepackage{hyperref}
\hypersetup{colorlinks=true,
                    linkcolor=blue,
                    filecolor=magenta,
                    urlcolor=cyan,
                    pdftitle={Overleaf Example},
                    pdfpagemode=FullScreen,}
%\newcommand{\hyperref}[1]{\ref{#1}}
%
\definecolor{Gray}{gray}{0.80}
\DeclareMathSymbol{,}{\mathord}{letters}{"3B}
%
\newcounter{num}
\renewcommand{\thenum}{\arabic{num}}
\newenvironment{anmerkungen}
   {\begin{list}{(\thenum)}{%
   \usecounter{num}%
   \leftmargin0pt
   \itemindent5pt
   \topsep0pt
   \labelwidth0pt}%
   }{\end{list}}
%
\renewcommand{\arraystretch}{1.15}                % in Formeln und Tabellen   
\renewcommand{\baselinestretch}{1.15}                 % 1.15 facher
                                                      % Zeilenabst.
\newcommand{\Anmerkung}[1]{{\begin{footnotesize}#1 \end{footnotesize}}\\[0.2cm]}
\newcommand{\comment}[1]{}
\setlength{\parindent}{0em}           % Nicht einruecken am Anfang der Zeile 

\setlength{\textwidth}{15.4cm}
\setlength{\textheight}{23.0cm}
\setlength{\oddsidemargin}{1.0mm} 
\setlength{\evensidemargin}{-6.5mm}
\setlength{\topmargin}{-10mm} 
\setlength{\headheight}{0mm}
\newcommand{\identity}{{\bf 1}}
%
\newcommand{\vs}{\vspace{0.3cm}}
\newcommand{\noi}{\noindent}
\newcommand{\leer}{}

\newcommand{\engl}[1]{[\textit{#1}]}
\parindent 1.2cm
\sloppy

         \begin{document}  \setcounter{chapter}{1}


\chapter{SI-Einheiten}
% Kap 1
\label{chap_SI}

Auf ihrer 26.\ Versammlung im November 2018 hat die CGPM (Conf\'{e}rence g\'{e}n\'{e}rale des poids et mesure -
General Conference on Weights and Measures) \index{CGPM - General Conference on Weights and Measures} 
ein neues Einheitensystem (SI - Syst\'{e}me international d'unit\'{e}s)\index{SI - Internationales Einheitensystem} 
beschlossen, das am 20.\ Mai 2019 in Kraft trat. Dieses System basiert im Wesentlichen auf der
Festlegung einer Zeiteinheit sowie der Festlegung bestimmter Naturkonstanten, mit denen ausgehend von der
Zeiteinheit andere Grundeinheiten - L\"ange, Masse, Temperatur, Stromst\"arke, 
Mengeneinheit - definiert werden k\"onnen. 


\section{Das neue System der Grundeinheiten}

Das neue System der Grundeinheiten beruht auf den Festlegungen der in Tabelle \ref{tab_SI}
angegebenen Naturkonstanten.\index{Frequenz des Hyperfeinstruktur\"ubergangs in C\"asium-133}%
\index{Lichtgeschwindigkeit}\index{Planck'sche Konstante}\index{Elementarladung}%
\index{Boltzmann-Konstante}\index{Avogadro-Konstante}%

\begin{table}[htb]
\begin{tabular}{l|l|l}
Bezeichnung & Symbol & Wert und Einheit \\ \hline
Frequenz des Hyperfinestruktur\"ubergangs in C\"asium-133 & $\Delta \nu_{\rm Cs}$ &
        9\,192\,631\,770\,{\rm Hz} \\
Lichtgeschwindigkeit im Vakuum & $c$ & 299\,792\,458\,{\rm m/s} \\ 
Planck'sche Konstante & $h$ & $6,626\,070\,15 \cdot 10^{-34}\,{\rm J\,s}$ \\
Elementarladung & $e$ & $1,602\,176\,634 \cdot 10^{-19}\,{\rm C}$ \\
Boltzmann-Konstante & $k_{\rm B}$ & $1,380\,649 \cdot 10^{-23}\,{\rm J/K}$ \\
Avogadro-Konstante & $N_{\rm A}$ & $ 6,022\,140\,76 \cdot 10^{23}\,{\rm mol}^{-1}$ \\
Photometrisches Strahlungs\"aquivalent & $K_{\rm cd}$ & $683\,{\rm lm/W}$ \\[-0.1cm]
(Monochromatische Strahlung von 540\,THz) & &  \\ \hline        
\end{tabular}
\caption{\label{tab_SI}%
Die fundamentalen Konstanten zur Festlegung der SI-Basis.}
\end{table}

Das photometrische Strahlungs\"aquivalent ist eine technische Konstante, die eine physikalische
Gr\"o\ss e (eine Strahlungsleistung, ausgedr\"uckt in Watt) mit einer physiologischen Gr\"o\ss e,
einer wahrgenommenen Helligkeit - ausgedr\"uckt durch die Einheit Lumen - in Verbindung bringt. 
Hierauf werden wir nicht weiter eingehen. 

Die Definition einer fundamentalen Frequenz $\Delta \nu_{\rm Cs}$ \"uber einen bestimmten
\"Ubergang (dem Hyperfeinstruktur\"ubergang im Grundzustand)
 in einem bestimmten Atom (C\"asium-133) gibt gleichzeitig eine physikalische Realisierung
dieser Frequenz an. Atomuhren, bei denen die Frequenz des C\"asium\"ubergangs gemessen wird, 
bezeichnet man als \glqq prim\"are Zeitnormale\grqq.\index{Zeitnormale!prim\"are} 
Es gibt andere Realisierungen (\"Uberg\"ange im Wasserstoff oder in Rubidium etc.),
die aber vorher am C\"asium\"ubergang geeicht werden m\"ussen und mit mindestens derselben
Genauigkeit und Stabilit\"at reproduzierbar sein sollten. Solche Realisierungen bezeichnet man
auch als \glqq sekund\"are Zeitnormale\grqq.\index{Zeitnormale!sekund\"are} 

Die fundamentalen Einheiten - Sekunde, Meter, Kilogramm, Ampere, Kelvin, Mol (und das Candela) -
erh\"alt man nun als Kombination dieser Gr\"o\ss en. Wie diese fundamentalen Einheiten zu bestimmen
sind, ist mit Ausnahme der Sekunde nicht festgelegt. Bei der Boltzmann-Konstanten und der Avogadro-Konstanten
handelt es sich um Proportionalit\"atsfaktoren zwischen historisch festgelegten Gr\"o\ss en, die
man heute nicht mehr unterscheiden m\"usste. Die Planck'sche Konstante und die Lichtgeschwindigkeit
im Vakuum sind universelle Naturkonstanten, die in vielen Naturgesetzen auftreten und somit
auf unterschiedliche Weisen bestimmt werden k\"onnen. Das neue SI-System l\"asst diese
Bestimmung offen.  

\subsection{Die Sekunde}
\label{sec_Sekunde}

Eine Frequenz gibt an,\index{Sekunde}\index{Frequenz} 
wie oft sich ein periodisch wiederkehrendes Ereignis in einer gewissen
Zeiteinheit wiederholt. Die Einheit Hertz (Hz) legt diese Zeiteinheit auf eine Sekunde fest. 
Eine Sekunde ist somit die Zeitdauer, in der sich die Schwingungen, die dem \"Ubergang im
C\"asium entsprechen, 9\,192\,631\,770 mal wiederholen. Oder anders ausgedr\"uckt: Die 
Periodendauer einer Schwingung entspricht\index{Periodendauer} 
\begin{equation}
               T = \frac{1}{\Delta \nu_{\rm Cs}} = \frac{1}{ 9\,192\,631\,770} {\rm s} \, .
\end{equation}  
Damit ist
\begin{equation}
               1\,{\rm s}  =  9\,192\,631\,770 \, T = 9\,192\,631\,770 \, \frac{1}{\Delta \nu_{\rm Cs}} \, .
\end{equation}  
Diese Gleichung kann man auch so interpretieren, dass die Definition
\begin{equation}
               \Delta \nu_{\rm Cs}  =  9\,192\,631\,770 \, {\rm s}^{-1}
\end{equation}  
nach der Einheit Sekunde aufgel\"ost wird. 

\subsection{Das Meter}

Bereits 1975 wurde der Wert der Lichtgeschwindigkeit 
im Vakuum als Naturkonstante,\index{Meter}
wie in Tabelle \ref{tab_SI} angegeben, festgelegt. Damit l\"asst sich die Einheit f\"ur
eine L\"ange - das Meter - auf die Einheit der Zeit zur\"uckf\"uhren:
\begin{equation}
              1\,{\rm m} =                 \frac{c}{299\,792\,458} \cdot 1 \,{\rm s} = 
              \frac{9\,192\,631\,770}{299\,792\,458} \frac{c}{\Delta \nu_{\rm Cs}} 
                 \approx  30,6633189885 \frac{c}{\Delta \nu_{\rm Cs}} \, .
\end{equation}  
Diese Festlegung hat zwei \"aquivalente Interpretationen: Zum einen ist ein Meter der
1/299\,792\,458 Teil der Strecke, die das Licht im Vakuum in einer Sekunde zur\"ucklegt,
bzw.\ das 30,6633... fache der Strecke, die das Licht im Vakuum in der Zeitdauer einer Periode
der Cs-Schwingung zur\"ucklegt. Andererseits entspricht das Verh\"altnis $c/\Delta \nu$
der Wellenl\"ange einer Strahlung, die sich mit der Geschwindigkeit $c$ ausbreitet und eine
Frequenz $\Delta \nu$ hat. Also ist 1 Meter das 30,6633...-fache der 
Wellenl\"ange der Strahlung\index{Wellenl\"ange}
zu dem Cs-\"Ubergang im Vakuum. 

Damit erhalten wir auch eine Vorstellung von der Wellenl\"ange der Strahlung zu dem
Cs-Hyperfeinstruktur\"ubergang: 1/30,6633... Meter oder ungef\"ahr 3,26\,cm. Es handelt sich also
um eine Strahlung im Radiobereich. Per Definition sind\index{Radiowellen} 
Radiowellen alle Formen von Wellen,
deren Frequenz unter 3000\,GHz liegt, was hier offensichtlich der Fall ist. 

\subsection{Das Kilogramm}

Das Produkt aus Planck'schem Wirkungsquantum und Frequenz, also $h\nu$, ist eine Energie.
Es ist die Energie eines einzelnen Photons\index{Photonenergie}\index{Energie} 
mit der Frequenz $\nu$. \"Uber die Einstein'sche
Gleichung $E=mc^2$ k\"onnen wir eine Energie mit einer Masse in Beziehung setzen. Offenbar hat
$h \nu/c^2$ die Einheit einer Masse, und wenn man die Naturkonstanten in SI-Einheiten
ausdr\"uckt, ist die Einheit dieser Masse das Kilogramm. Wenn wir den Ausdruck:
\begin{equation}
              \frac{h \Delta \nu_{\rm Cs}}{c^2} = 
       \frac{6,626\,070\,15 \cdot 10^{-34} \cdot 9\,192\,631\,770}{(299\,792\,458)^2} {\rm kg} 
\end{equation}  
nach der Einheit Kilogramm aufl\"osen, erhalten wir:\index{Kilogramm}
\begin{equation}
           1\,{\rm kg} =
       \frac{(299\,792\,458)^2}{6,626\,070\,15 \cdot 10^{-34} \cdot 9\,192\,631\,770} \, \frac{h \Delta \nu_{\rm Cs}}{c^2}
         \approx  1,475\,5214 \cdot 10^{40} \, \frac{h \Delta \nu_{\rm Cs}}{c^2}  \, .
\end{equation}  
Man kann diese Gleichung folgenderma\ss en interpretieren: Ein Kilogramm entspricht \"uber die Beziehung $E=mc^2$
der Energie von $1,475... \cdot 10^{40}$ Photonen, von denen jedes einzelne zu dem Strahlungs\"ubergang
im Cs-Atom geh\"ort. 

\subsection{Das Ampere}

Das neue SI-System definiert die Elementarladung als fundamentale Naturkonstante. 
Die Einheit - das Coulomb - ist gleich $1 \, {\rm C= 1\, A \cdot s}$. Damit hat das Produkt aus der
Elementarladung und der Cs-Frequenz die Einheit Ampere:\index{Ampere}
\begin{equation}
       e \cdot \Delta \nu_{\rm Cs} = 1,602\,176\,634 \cdot 10^{-19} \cdot 9\,192\,631\,770\, {\rm A} \approx
            1,4728219827 \cdot 10^{-9} \,{\rm A} \, .
\end{equation}
Dieses Produkt ist gleich der Stromst\"arke, die man erh\"alt, wenn eine Elementarladung in der Periodendauer 
einer Cs-Schwingung durch eine vorgegebene Fl\"ache tritt. Damit folgt f\"ur die Einheit Ampere:
\begin{equation}
     1\,{\rm A} = \frac{1}{1,602\,176\,634 \cdot 10^{-19} \cdot 9\,192\,631\,770} \, e \cdot \Delta \nu_{\rm Cs} \approx
           6,789\,6868 \cdot 10^8 \,  e \cdot \Delta \nu_{\rm Cs} \, .
\end{equation}
Die Stromst\"arke von 1 Ampere entspricht also entweder dem Fluss von 
$1/1,602...\cdot 10^{19} \approx 0,624\cdot 10^{19}$ 
Elementarladungen pro Sekunde oder dem Fluss von $6,789... \cdot 10^8$ Elementarladungen
pro Periodendauer einer Cs-Schwingung. 

\subsection{Das Kelvin}

Theoretisch k\"onnte man auf eine eigene Temperaturskala verzichten. In allen relevanten
F\"allen, in denen die Temperatur $T$ mit anderen physikalischen Gr\"o\ss en in Beziehung
gesetzt wird, tritt das Produkt $k_{\rm B}T$ mit der Boltzmann-Konstanten $k_{\rm B}$ 
auf. Dieses Produkt hat die Dimension einer Energie, also
dieselbe Dimension wie $h\nu$. Es sind haupts\"achlich historische Gr\"unde, dass man der
Temperatur nicht die Dimension der Energie gegeben hat. 

Da die Boltzmann-Konstante in den Einheiten J/K angegeben
ist, hat\index{Temperatur}\index{Kelvin}\index{Boltzmann-Konstante}
\begin{equation}
    \frac{h \Delta \nu_{\rm Cs}}{k_{\rm B}} = 
    \frac{6,626\,070\,15 \cdot 10^{-34} \cdot 9\,192\,631\,770}{1,380\,649 \cdot 10^{-23}} \, {\rm K}
\end{equation}
die Dimension einer Temperatur. L\"osen wir diese Gleichung nach der Einheit Kelvin auf, folgt:
\begin{equation}
   1\,{\rm K} =      \frac{1,380\,649 \cdot 10^{-23}}{6,626\,070\,15 \cdot 10^{-34} \cdot 9\,192\,631\,770} \, 
    \frac{h \Delta \nu_{\rm Cs}}{k_{\rm B}} \approx 2,266\,6653 \,  \frac{h \Delta \nu_{\rm Cs}}{k_{\rm B}} \, .
\end{equation}
Die thermische Energie zu einem Kelvin entspricht also ungef\"ahr der Energie des Strahlungs\"ubergangs
bei C\"asium. Etwas anders ausgedr\"uckt: Wenn man jedem thermischen Freiheitsgrad eines Systems
die Energie $h \Delta \nu$ eines Photons aus dem Cs-Strahlungs\"ubergang zuf\"uhrt, erh\"oht sich
seine Temperatur um ungef\"ahr 1 Kelvin. 

Hierbei handelt es sich um \glqq ungef\"ahr\grqq-Werte, also Gr\"o\ss enordnungen. Zum einen h\"angt
die Beziehung zwischen der zugef\"uhrten Energie und der Temperaturerh\"ohung von der spezifischen
W\"arme ab und kann an Phasen\"uberg\"angen sogar \"uberhaupt keine Temperaturerh\"ohung zur
Folge haben (latente W\"arme), andererseits gibt die Beziehung
\begin{equation}
                   \langle \epsilon_{\rm kin} \rangle = \frac{3}{2} k_{\rm B} T 
\end{equation}
eine klare Beziehung zwischen dem Erwartungswert der kinetischen Energie eines Bestandteils einer Substanz
und der Temperatur an, die lediglich die Boltzmann-Konstante als Faktor enth\"alt. 

\subsection{Das Mol}

Die Avogadro-Konstante ist eine historisch gew\"ahlte Konstante 
zwischen der Menge\index{Mol}\index{Avogadro-Konstante}
einer Substanz und der Anzahl ihrer elementaren Bestandteile (Atome, Molek\"ule, Ionen, etc.). Bei
\glqq Substanzen\grqq\ aus makroskopischen Bestandteilen (z.B.\ einer Gruppe von Menschen
oder Billiardkugeln) w\"urde man einfach deren Anzahl angeben. Zu einer Zeit, als die mikroskopische
Natur der Materie noch nicht bekannt war, definierte man die Menge einer Substanz \"uber
bestimmte chemische Reaktionen, bei denen diese Substanz mit einer wohldefinierten Menge
einer anderen Substanz reagierte. Willk\"urlich hat man dann eine bestimmte Menge an Kohlenstoff, 
ausgedr\"uckt in Gramm, als die Einheit mol definiert. Sp\"ater hat man dann durch Messungen die Anzahl
der Kohlenstoffmolek\"ule bestimmt, die dieser Menge entspricht, also die Avogadro-Zahl.

Heute ist ein Mol einer Substanz genau die Menge, die aus $N_{\rm A}= 6,022\,140\,76 \cdot 10^{23}$ 
elementaren Bestandteilen dieser Substand besteht. 

\section{Zur Geschichte der Grundeinheiten}

\subsection{Die Sekunde}

Die Einteilung eines Tages in 24 Stunden finden wir schon im antiken Babylon bzw.\ Mesopotamien. 
Bis ins Mittelalter wurden Tag und Nacht in jeweils 12 Stunden unterteilt,\index{Tag}\index{Stunde} 
was allerdings je nach Jahreszeit zu unterschiedlich
langen Tag- und Nachtstunden f\"uhrte. Diese sogenannten 
\glqq Temporalstunden\grqq\ (horae inequales) konnten\index{Temporalstunde}
sich deutlich unterscheiden und waren nur bei den Tag-und-Nacht-Gleichen - also den \"Aquinoktien um den
20.\ M\"arz und den 23.\ September - gleich. Erst mit dem Aufkommen
von R\"aderuhren zu Beginn des 14.\ Jahrhunderts setzten sich die sogenannten horae aequinoctiales, also
gleichlange Tag- und Nachtstunden, durch.   

Den Begriff der Minute und Sekunde finden wir erst im Mittelalter. 
Dabei ist - fern aller Logik - die Minute\index{Minute}\index{Sekunde} 
eine Abk\"urzung von \glqq pars minuta prima\grqq\ (einmal verminderter Teil) und die Sekunde eine
Abk\"urzung von \glqq pars minuta secunda\grqq\ (zweimal vermindeter Teil). 

Bis 1956 war die Sekunde definiert als der 1/86\,400-ste Teil eines 
mittleren\index{Sonnentag!mittlerer}\index{Sonnentag!wahrer} 
Sonnentages.\footnote{Die sogenannten wahren
Sonnentage - beispielsweise die Zeitdauer zwischen zwei Sonnenh\"ochstst\"anden an aufeinanderfolgenden
Tagen - sind nicht immer gleich lang, siehe das Kapitel zur Zeitgleichung, Kap.\ \ref{chap_Zeitgleichung}.
Der mittlere Sonnentag ist die \"uber ein Jahr gemittelte Dauer eines Tages.} 
Nachdem man aber festgestellt hatte, dass auch die mittleren Sonnentage Schwankungen
unterworfen sind, legte man 1956 die Sekunde als den Bruchteil 1/31\,556\,925,9747
des tropischen Jahres 1900 fest, wobei
man hier ein im Rahmen einer Theorie hochgerechnetes Jahr 1900 aus den Bewegungsdaten der Erde um die
Sonne am 31.\ Januar 1899, 12 Uhr, gew\"ahlt hat. Die so definierte Sekunde bezeichnet man
als Ephemeridensekunde.\index{Ephemeridensekunde} 

Die Definition der Sekunde \"uber den Hyperfeinstruktur\"ubergang in C\"asium-133 l\"oste 1967 die 
Ephemeridensekunde ab. Es wird nicht ausgeschlossen, dass bis 2030
eine Neudefinition der Sekunde \"uber atomare \"Uberg\"ange im optischen Bereich erfolgt, da mit 
diesen eine deutlich h\"ohere Genauigkeit erreicht werden kann.

\subsection{Das Meter}

Bis in die fr\"uhe Neuzeit waren sehr unterschiedliche L\"angenma\ss st\"abe in Gebrauch. 
Elle, Fu\ss, Zoll (Daumenbreite) oder Schritt (heute noch im englischen Yard) konnten sich 
beispielsweise auf den jeweiligen Herrscher beziehen. Eintausend Doppelschritte (Lateinisch
\glqq mille passus\grqq, hiervon leitet sich die Bezeichnung Meile ab)\index{Meile} 
entsprachen im R\"omischen\index{Seemeile}\index{nautische Meile}
Reich einer Meile von etwas \"uber 1,5\,km. Sp\"ater setzte sich die nautische Meile bzw.\ Seemeile
mit 1852 Metern (das entspricht ungef\"ahr einer Bogenminute am \"Aquator) als meist verwendete
Form der Meile durch. 

Ende des 18.\ Jahrhunderts definierte man die Einheit Meter als den 10\,000\,000-sten\index{Meter} 
Teil des Meridianbogens (also des L\"angengrads) durch Paris vom Nordpol zum \"Aquator. Mittlerweile
war durch genaue Vermessungen des Geoids (der Erdform) bekannt, dass die Erde keine Kugelform
hat und somit der Umfang oder auch die L\"ange eines Meridianbogens davon abh\"angen, wo diese
Gr\"o\ss en gemessen werden. Man realisierte das so definierte Meter durch ein 
Urmeter, einen Platinstab,\index{Urmeter}
der in Paris gelagert wurde. Obwohl sich sp\"ater herausstellte, dass die Messungen der L\"ange des
Merdians durch Paris fehlerhaft waren, wurde 1889 international die L\"ange des Meters nach den alten
Prototypen festgelegt und als Urmeter einer Platin-Iridium-Legierung angefertigt. 

Das Meter als die L\"ange des in Paris gelagerten Urmeters war bis 1960 in Gebrauch. Zwischen 1960
und 1975 wurde das Meter \"uber die Wellenl\"ange einer bestimmten Linie einer Krypton-Lampe
definiert. Durch die Festlegung eines Zahlenwerts f\"ur die Lichtgeschwindigkeit im Vakuum wurde ab 1975
das Meter \"uber die Sekunde definiert. 

\subsection{Das Kilogramm}

Urspr\"unglich sollte das Kilogramm\index{Kilogramm} 
der Masse von einem Kubikdezimeter Wasser entsprechen, wobei die Temperatur von manchen
Definitionen auf $0^\circ$C, von anderen auf die Temperatur maximaler Dichte (rund 
$4^\circ$C) festgelegt wurde. Da diese Definitionen nur schwer mit der erforderlichen Genauigkeit
(im Bereich von Milligramm pro Kilogramm) reproduziert werden konnten, stellte man sp\"ater
Prototypen des Kilogramms aus Platin-Iridium-Legierungen her. Die Definition eines Kilogramms
\"uber solch einen Prototyp hatte bis 2018 G\"utligkeit. Allerdings zeigte sich schon seit mehreren
Jahren, dass die Masse des in Paris in einem Safe bei konstanter Temperatur gelagerten Prototypen
und die Massen von urspr\"unglich gleich schweren Kopien signifikante Unterschiede aufwiesen. 
Daher wurde 2018 festgelegt, dass das Kilogramm \"uber die Planck'sche Konstante und die
Lichtgeschwindigkeit auf die Definition der Sekunde zur\"uckgef\"uhrt werden soll. 

\subsection{Das Ampere}

Streng genommen m\"usste man f\"ur die Gesetze des Elektromagnetismus keine neuen
Einheiten einf\"uhren. \"Uber die Coulomb-Kraft k\"onnte man z.B.\ der Ladung eine Einheit
in Bezug auf die drei Grundeinheiten Kilogramm, Meter und Sekunde geben. Dazu definiert man
beispielsweise die Ladung \"uber das Kraftgesetz in folgender Form:
\begin{equation}
                 F = \frac{q_1 q_2}{r^2}  \, .
\end{equation}
Die Ladung erh\"alt dadurch eine Einheit ${\rm kg}^{\frac{1}{2}} {\rm m}^{\frac{3}{2}} {\rm s}^{-1}$. 
Die freie Naturkonstante, die die Ladungen $q_1$ und $q_2$ im Abstand $r$ mit einer Kraft $F$ verbindet,
wird zu 1 definiert. Experimentell \"uberpr\"uft werden kann nur, dass die Kraft proportional zum Produkt
der beiden Ladungen und umgekehrt proportional zum Quadrat des Abstands ist. 
Das sogenannte CGS-System\index{CGS-System}\index{Gau\ss'sche Einheiten}
(bzw.\ die Gau\ss'schen Einheiten) beruht auf dieser Konstruktion. CGS steht f\"ur Zentimeter-Gramm-Sekunde,
d.h.\ es gibt in diesem System zus\"atzlich noch Zehnerpotenzen zwischen den CGS-Einheiten und den 
MKS-Einheiten (Meter-Kilogramm-Sekunde). Eine weitere Freiheit gibt es bei der Proportionalit\"at zwischen 
Ladung und elektrischen Feld sowie bei der Proportionalit\"at zwischen Stromst\"arke und magnetischem
Feld. Neben der Einheit f\"ur die Ladung besteht also eine Freiheit f\"ur die Einheiten des elektrischen und
magnetischen Feldes. 

W\"ahrend im 19.\ Jahrhundert das CGS-System bzw.\ das Gau\ss'sche System verbreitet waren, entschloss
man sich gegen Ende des 19.\ Jahrhunderts, der Stromst\"arke eine eigene Einheit, das Ampere, 
zu geben.\index{Ampere}
Man h\"atte ebenso gut (wie es im Prinzip heute der Fall ist) der Ladung eine eigene Einheit geben k\"onnen.  
Auf diese Weise vermied man nicht ganzzahlige Exponenten f\"ur die Einheiten mancher Gr\"o\ss en (wie
beispielsweise f\"ur die Ladung im CGS-System). Mitte des 20.\ Jahrhunderts definierte man das Ampere
\"uber die Kraft zwischen zwei stromdurchflossenen Leitern. Diese Definition war im Wesentlichen bis
2019 g\"ultig.  

\subsection{Das Kelvin}

Die Definition einer Temperaturskala galt historisch als ein Problem. Rein
subjektiv k\"onnen wir zwischen k\"alteren und w\"armeren Systemen unterscheiden und
es ist eine Erfahrungstatsache, dass es eine Eigenschaft gibt, die mit diesem Empfinden
korreliert, und die f\"ur Systeme, die l\"angere Zeit in Kontakt sind, gleich wird. Diese Eigenschaft
nennen wir Temperatur.\index{Temperatur} 
Verschiedene Systeme reagieren jedoch sehr unterschiedlich auf
Temperatur\"anderungen (Festk\"orper \"andern beipielsweise ihre lineare Ausdehnung, Metalle ihren
elektrischen Widerstand, Gase ihr Volumen oder ihren Druck etc.). Die Festlegung einer Skala erscheint daher
zun\"achst willk\"urlich. Erst durch Experimente im 17.\ und 18.\ Jahrhundert fand man heraus,
dass f\"ur Gase bei hohen Temperaturen und hohen Verd\"unnungen ein von dem jeweiligen Gas
nahezu unabh\"angiges Ausdehnungsverhalten vorliegt. Die Gesetze von 
Boyle-Mariotte (um 1670; f\"ur eine\index{Boyle-Mariotte, Gesetz von}
feste Temperatur ist das Produkt aus Druck und Volumen konstant), 
von Gay-Lussac\index{Gay-Lussac, Gesetz von} 
(um 1800; f\"ur einen festen Druck ist das Volumen proportional zur Temperatur, manchmal bezeichnet
man dieses Gesetz auch als das Gesetz von Charles, der es 1786 entdeckte) 
und von Amontons\index{Amontons, Gesetz von}
(um 1700; f\"ur ein konstantes Volumen ist der Druck proportional zur Temperatur) erlaubten eine
Definition der Temperatur, die in mehrfacher Hinsicht ausgezeichnet war: Sie war unabh\"angig
von dem jeweiligen Gas sowie von anderen Parametern, sofern die Dichte klein genug und
die Temperatur hoch genug war, sodass man verschiedene Systeme miteinander vergleichen 
konnte. Das ideale Gasgesetz in seiner heutigen Form als Zusammenfassung der drei oben genannten
Gesetze wurde allerdings erst 1834 von Beno\^{i}t Clapeyron formuliert.\index{Clapeyron, Beno\^{i}t} 

1824 erschien eine Schrift von\index{Carnot, Nicolas L\'{e}onard Sadi} 
Nicolas L\'{e}onard Sadi Carnot, in der er im Wesentlichen den
Carnot-Prozess beschrieb und eine Schranke f\"ur den Wirkungsgrad von W\"armemaschinen
(also Maschinen, bei denen ein Temperaturunterschied zwischen zwei Systemen und der
damit verbundene W\"armefluss, wenn man diese Systeme in Kontakt bringt, genutzt wird, um 
mechanische Arbeit zu leisten). Eine wichtige Folgerung dieses Prozesses war, dass
man eine Temperaturskala definieren konnte, ohne sich auf eine spezielle Realisierung zu
beziehen. Mathematisch kann man sagen: Die Temperatur ist der integrierende Faktor
zwischen der nicht exakten Einsform W\"arme und der exakten Einsform zur Entropie 
(d.h.\ dem Gradienten der Zustandsgr\"o\ss e Entropie).  

Die heutige\index{Celsius-Skala} 
Celsius-Skala wurde 1742 von Anders Celsius\index{Celsius, Anders} 
formuliert: Er unterteilte die Temperatur
zwischen dem Gefrier- und dem Siedepunkt von Wasser bei Normaldruck in 100 Teile. Allerdings
ordnete Celsius dem Siedepunkt die Temperatur 0 und dem Gefrierpunkt die Temperatur
100 zu. Zwei Jahre sp\"ater (nach dem Tod von Celsius) wurde diese Zuordnung umgedreht.
Schon einige Zeit vor der Celsius-Skala f\"uhrte\index{Fahrenheit-Skala}\index{Fahrenheit, Daniel Gabriel} 
Daniel Gabriel Fahrenheit um 1714 eine Temperaturskala
ein, die auf drei Fixpunkten basierte: $0{\,}^\circ$F entspricht $-17,8{\,}^\circ$C und war damals
die tiefste Temperatur, die man mit einer sogenannten K\"altemischung (aus Eis, Wasser und Salmiak)
erzeugen konnte. Fahrenheit glaubte, auf diese Weise negative Temperaturwerte vermeiden
zu k\"onnen. Der Gefrierpunkt von Wasser wurde zu $32{\,}^\circ$F festgelegt, und die
K\"orpertemperatur eines gesunden Menschen zu $96{\,}^\circ$F, was mit $35,6{\,}^\circ$C etwas
niedrig ist. 

William Thomson (First Baron Kelvin)\index{Thomson, William (First Baron Kelvin)} 
schlug 1848 vor, die Celsius-Skala \glqq zu verschieben\grqq\ und
den Nullpunkt auf den absoluten Nullpunkt festzulegen, den man z.B.\ aus den idealen Gasgesetzen
mit dem universellen Volumenausdehnungskoeffizienten $\gamma=1/T$ zur\"uckrechnen konnte. Die
neue Skala wurde sp\"ater Kelvin genannt.\index{Kelvin-Skala} 
Sie basierte auf dem absoluten Nullpunkt mit $0$\,K (bis
1967 sagte man auch \glqq $0$ Grad Kelvin\grqq) sowie dem Tripelpunkt von Wasser, der zu
$273,16$\,K festgelegt wurde (also bei $0,01{\,}^\circ$C). 

Seit 2019 ist die Einheit Kelvin direkt \"uber die Energieeinheit Joule definiert.
Die Boltzmann-Konstante ist
streng genommen keine Naturkonstante sondern verkn\"upft die historische
Kelvin-Skala mit der Energieskala. Die Temperatur ist direkt proportional zur thermischen
Energie, d.h.\ die thermische Energie ist zu $k_{\rm B}T$ definiert.   

\subsection{Das Mol}

Dass es \"uberhaupt eine Einheit f\"ur die Substanzmenge gibt, geht vermutlich auf die
Chemie zur\"uck. Ohne ihre Gesetze h\"atte man Anfang des 19.\ Jahrhunderts Mengen verschiedener 
Substanzen kaum vergleichen k\"onnen.

John Dalton\index{Dalton, John} 
entdeckte um 1800 das Gesetz der konstanten Proportionen und das
Gesetz der multiplen Proportionen. Ersteres besagt, dass in einer Substanz (also einer aus Molek\"ulen
bestehenden chemischen Zusammensetzung) das Massenverh\"altnis der Bestandteile
(chemischen Elemente) immer gleich ist. Das zweite Gesetz besagt: Wenn zwei chemische Elemente
mehrere Verbindungen eingehen k\"onnen (z.B.\ die Stickoxide ${\rm NO}$, ${\rm NO}_2$, ${\rm N_2O}$,
${\rm N_2O_3}$ etc.), 
geschieht dies immer in Massenverh\"altnissen, die relativ zu einander kleine Zahlen annehmen. Diese 
Gesetze unterst\"utzten die Atomtheorie Daltons, allerdings konnte man zu dieser Zeit keine absoluten
Gr\"o\ss enordnungen angeben. Die Vermutung war aber, dass sich zwei Substanzen $A$ und $B$
im Prinzip in der Form $AB$ oder $AB_2$ oder $A_2B$ etc.\ verbinden k\"onnen. Wenn man somit
willk\"urlich eine bestimmte Menge (im Sinne einer bestimmten Masse) der Substanz $A$ als 
\glqq Mengeneinheit\grqq\ bezeichnet, kann man aus den Masseverh\"altnissen in solchen chemischen 
Verbindungen die Masse der Substanz $B$ bestimmen, die nach Daltons Atomhypothese dieselbe Anzahl 
von atomaren Einheiten haben sollte. 

Kurze Zeit sp\"ater - um 1812 - entdeckte\index{Avogadro'sches Gesetz}\index{Avogadro, Amedeo} 
Amedeo Avogadro das Avogadro'sche Gesetz (oftmals spricht
man auch von der Avogadro'schen Vermutung, da die Atomhypothese damals nicht als erwiesen galt), 
wonach (ideale) Gase bei gleicher Temperatur, gleichem Druck und gleichem Volumen auch dieselbe Anzahl 
von Molek\"ulen enthalten. Seine Vermutung wiederum ging auf eine Beobachtung von Gay-Lussac um 1800
zur\"uck, die darin bestand, dass sich Gase, wenn diese sich zu einer chemischen Substanz verbinden, immer
in entsprechenden Volumenverh\"altnissen miteinander reagierten (z.B.\ zwei Volumeneinheiten Wasserstoff
mit einer Volumeneinheit Sauerstoff zu Wasser, ${\rm H_2O}$). 

Es folgte ein Sprung bis in die Mitte des 19.\ Jahrhunderts:
Josef Loschmidt\index{Loschmidt, Josef}\index{mittlere Wegl\"ange} 
fand 1865 eine interessante Beziehung zwischen der mittleren Wegl\"ange $l$ von
Luftmolek\"ulen bei einer bestimmten Temperatur $T$, dem
Durchmesser $d$ der Luftmolek\"ule und dem Verh\"altnis $V_{\rm l}/V_{\rm g}$ - dem Volumen $V_{\rm l}$
von Luft in seiner fl\"ussigen Form und dem Volumen $V_{\rm g}$ derselben Masse an Luft im gasf\"ormigen
Zustand bei derselben Temperatur $T$ (er nannte dieses Verh\"altnis Kondensationskoeffizient):
\begin{equation}
               d = 8 l \frac{V_{\rm l}}{V_{\rm g}} \, .
\end{equation} 
Loschmidt wusste, dass diese Beziehung wegen mehrerer N\"aherungen, die er gemacht hatte, nicht 
exakt gilt, ihm ging es aber auch nur um eine Bestimmung der
Gr\"o\ss enordnung von $d$. Das Volumen fl\"ussiger Luft war damals zwar noch nicht bekannt,
allerdings verwendete er theoretische \"Uberlegungen, dieses Volumen aus der chemischen Zusammensetzung
und dem Vergleich mit \"ahnlichen Gasen (z.B.\ Wasser, wo dieses Verh\"altnis bekannt war) abzusch\"atzen. 
Maxwell hatte in seiner Arbeit von 1860 unter Bezug auf Messungen von Stokes die mittlere freie Wegl\"ange 
der Luftmolek\"ule bei Zimmertemperatur ziemlich gut auf etwas \"uber 60 Nanometer (heute gibt man meist 68\,nm an) 
abgesch\"atzt. Damit konnte Loschmidt den Durchmesser der Luftmolek\"ule mit rund 1\,nm angeben.
Es war dann wieder Maxwell, der erkannte, 
dass man aus dem Wert von Loschmidt f\"ur die Gr\"o\ss e der Luftmolek\"ule auch deren Anzahl in einem 
bestimmten Volumen bzw.\ einer bestimmten Masse (und damit die heutige Avogadro-Zahl) bestimmen konnte. 
(N\"aheres in Abschnitt \ref{sec_SI_Kur}.)

Auch wenn sich die Verfahren zur Bestimmung der Avogadro-Zahl im Verlauf der Zeit
verbesserten, blieb sie mit vergleichsweise gro\ss en Ungenauigkeiten behaftet. Daher\index{Mol}
behielt man bis 2018 die historische Definition f\"ur ein Mol bei: Das Mol ist die Stoffmenge eines Systems, das aus
ebensoviel Einzelteilchen besteht, wie Atome in 0,012 Kilogramm des Kohlenstoffnuklids ${}^{12}$C enthalten
sind. Durch die Festlegung der Avogadro-Zahl wurde die Einheit Mol eigentlich \"uberfl\"ussig, sie ist jedoch
aus praktischen Gr\"unden immer noch sinnvoll.

\section{Historische Kuriosit\"aten}

Im Zusammenhang mit der Festlegung der fundamentalen Einheiten gibt es einige 
interessante und teilweise am\"usante historische Anmerkungen.

\subsection{Jean Picard und das Meter}

Eine interessante Idee,\index{Picard, Jean}
eine von k\"orperlichen Gliedma\ss en unabh\"angige Definition des Meters zu finden, stammt
von Jean Picard, der 1668 vorschlug, als L\"angeneinheit die L\"ange eines Pendels mit der Halbperiode 
von einer Sekunde zu w\"ahlen. Diesem Vorschlag ging die Beobachtung Galileis um 1630 voraus, dass
ein Pendel f\"ur kleine Auslenkungen eine wohldefinierte Periode (unabh\"angig von dem Grad
der Auslenkung) hat; au\ss erdem hatte\index{Huygens, Christiaan}\index{Pendeluhr} 
Christiaan Huygens um 1656 das Prinzip der Pendeluhr
erfunden. Nach der Formel f\"ur die Periodenl\"ange $T$ eines Pendels der L\"ange $l$ im Schwerefeld
der Erde mit der Erdbeschleunigung $g$ bei kleinen Auslenkungen,
\begin{equation}
       T = 2 \pi \sqrt{\frac{l}{g}}   \, ,
\end{equation}
folgt aus $T/2=1$\,s mit $g=9,81\,{\rm m/s^2}$ die L\"ange $l=0,99396$\,m. Dieser Vorschlag Picards ist insofern
interessant, weil hier ein L\"angenma\ss stab \"uber eine \glqq Naturkonstante\grqq\ ($g$) auf eine
Zeiteinheit zur\"uckgef\"uhrt wurde. Vermutlich trug die schlechte Reproduzierbarkeit der Sekunde
dazu bei, dass sich dieser Vorschlag nicht durchsetzte. Dass $g$ an verschiedenen Orten der Welt
unterschiedliche Werte hat, stellte sich erst sp\"ater heraus, wobei schon Picard als Referenzort
Paris vorgeschlagen hatte. 

\subsection{$\sqrt{2}$ oder $4/3$ - wenn gro\ss e Geister uneins sind}
\label{sec_SI_Kur}

Wenn sich zwei gro\ss e Wissenschaftler - in diesem Fall James Clerk Maxwell und Rudolf Clausius - um 
einen Faktor streiten, kann man davon ausgehen, dass ein interessantes Problem dahintersteckt.  

1858 formulierte\index{Clausius, Rudolf} 
Rudolf Clausius eine Beziehung zwischen der mittleren freien Wegl\"ange $l$ von Atomen 
bzw.\ Molek\"ulen in Gasen und ihrer Gr\"o\ss e \cite{Clausius58}. Die Herleitung dieser Formel
erfolgt heute meist in folgender Form: Angenommen, wir haben einen Beh\"alter vom Volumen $V$ mit
$N$ kugelf\"ormigen Teilchen ($N$ sehr gro\ss, Teilchendichte $\rho=N/V$), die in diesem Volumen zuf\"allig 
verteilt sind und zun\"achst als in Ruhe angenommen werden. F\"ur ein weiteres Teilchen, das sich mit der 
Geschwindigkeit $v$ durch diesen Beh\"alter bewegt, ist $s=vt$ die in der Zeit $t$ zur\"uckgelegte Strecke. 
Die Frage ist: Wie oft st\"o\ss t dieses Teilchen auf 
dieser Strecke mit anderen Teilchen zusammen, wobei der Wirkungsqueschnitt f\"ur einen solchen
Zusammensto\ss\ $\sigma$ (Dimension einer Fl\"ache) sein soll? $\sigma\cdot vt$ ist das Volumen des Zylinders,
den das Teilchen mit seinem Wirkungsquerschnitt f\"ur einen Zusammensto\ss\ mit einem ruhenden Teilchen
in der Zeit $t$ \"uberstrichen hat.
In diesem Zylinder befinden sich $\rho \sigma \cdot v t$ ruhende Teilchen, mit denen das fliegende Teilchen
im Prinzip zusammengesto\ss en ist; dies ist also die Anzahl der Zusammenst\"o\ss e. Die mittlere freie
Wegl\"ange $l$ ist nun die zur\"ckgelegte Stecke $vt$ dividiert durch die mittlere Anzahl der Zusammenst\"o\ss e,
also
\begin{equation}
           l = \frac{vt}{\rho \sigma \cdot vt} = (\rho \sigma)^{-1}  \, .
\end{equation}   
Wenn sich die anderen Teilchen nicht in Ruhe befinden, sondern eine mittlere Geschwindigkeit $u$
haben, z\"ahlt f\"ur das in der Zeit $t$ \"uberstrichene Volumen die mittlere Relativgeschwindigkeit zwischen 
$v$ und $u$; diese ist $v_{\rm rel}=\sqrt{v^2+u^2}$ (das Skalarprodukt $\pmb{u} \cdot \pmb{v}$ verschwindet im Mittel). 
Da $u$ und $v$ im Mittel gleich sind, erhalten wir einen Faktor $\sqrt{2}$:
\begin{equation}
\label{eq_meanfreepath}
           l = \frac{vt}{\rho \sigma \cdot v_{\rm rel}t} = (\sqrt{2}\rho \sigma)^{-1}  \, .
\end{equation}   

Clausius hatte in seiner Arbeit von 1858 \cite{Clausius58} statt des Faktors $\sqrt{2}$ ohne eine weitere Begr\"undung
den Faktor $4/3$ verwendet.\index{Maxwell, James Clerk} 
James Clerk Maxwell hatte in seiner Arbeit von 1860, in der er auch die heute als
Maxwell'sche Geschwindigkeitsverteilung bekannte Verteilungsformel f\"ur die Geschwindigkeiten der
Molek\"ule bzw.\ Atome bei einer bestimmten Temperatur herleitete \cite{Maxwell60}, den Faktor $\sqrt{2}$
nach obiger Argumentation hergeleitet und in einer Notiz vermerkt, dass Clausius statt dessen den Faktor 
$4/3$ verwendet habe. Daraufhin f\"uhlte sich Clausius gen\"otigt, seinen Faktor $4/3$ zu begr\"unden \cite{Clausius60}. 
Bis zum ersten Gleichheitszeichen in
Gl.\ \ref{eq_meanfreepath} stimmen beide Autoren \"uberein. Doch Claudius berechnet die mittlere
relative Geschwindigkeit, indem er \"uber die Richtungen aller Geschwindigkeiten $u$ der anderen Teilchen
mittelt:
\begin{equation}
    v_{\rm rel} = \frac{1}{2} \int_0^\pi \sqrt{ v^2 + u^2 - 2uv \cos \theta} \, \sin \theta \, {\rm d} \theta \, .  
\end{equation}
Das Integrationsma\ss\ $\frac{1}{2} \sin \theta \, {\rm d}\theta$ begr\"undet Clausius mit dem Argument,
dass jede Richtung gleichwahrscheinlich sei und daher proportional zum Fl\"achenelement zu dieser Richtung.
Die relative H\"aufigkeit, in die Richtung $\theta$ abgelenkt zu werden, ist somit $2 \pi \sin \theta \, {\rm d}\theta/4\pi$
(Fl\"achenelement zum Winkel ${\rm d}\theta$ relativ zur Kugeloberfl\"ache). Heute w\"urden wir das Ma\ss\ 
in Winkelkoordinaten f\"ur die Kugeloberfl\"ache schreiben, $\frac{1}{4\pi} \sin \theta \, {\rm d} \theta\, {\rm d}\varphi$, 
und da der Integrand nicht von $\varphi$ abh\"angt, k\"onnen wir das Integral \"uber $\varphi$
ausf\"uhren (es ergibt einen Faktor $2\pi$) und wir erhalten 
dasselbe Ergebnis.

Das Integral l\"asst sich mit $\sin\theta \, {\rm d}\theta = {\rm d}\cos \theta$ leicht berechnen und man
erh\"alt:
\begin{equation}
    v_{\rm rel} = \frac{1}{6uv}  (v^2 + u^2 - 2uv \cos \theta )^{3/2} \Big|_0^\pi  
     =  \frac{1}{6 uv} \big( |v+u|^3 - |v-u|^3 \big)  
     =  \left\{ \begin{array}{ll}  v + \frac{u^2}{3v} &  u\leq v \\  u + \frac{v^2}{3u} & u>v \end{array} \right. \, .
\end{equation}
F\"ur $u=v$ ist somit $v_{\rm rel}= \frac{4}{3}v$. Diese Argumentation klingt sehr vern\"unftig. 

In den \glqq Collected Papers\grqq\ von Maxwell aus dem Jahre 1890 ist auch die 1860er Arbeit 
wiedergegeben \cite{Maxwell90},
allerdings  folgt der Bemerkung, dass Clausius den Faktor 4/3 verwendet, eine Fu\ss note, die der
Herausgeber hinzugef\"ugt hat. Dort hei\ss t es, dass das Ergebnis von Clausius zwar im Prinzip richtig ist, 
Clausius aber immer nur von einer mittleren Geschwindigkeit $u$ bzw.\ $v$ der Teilchen ausgegangen ist und
f\"ur $u=v$ annimmt, dass alle Teilchen dieselbe Geschwindigkeit haben und lediglich ihre Richtungen
gleichverteilt sind. Er ersetzt nun den konstanten Geschwindigkeitsbetrag f\"ur $u$ und $v$ (mit der Mittelung \"uber alle
Richtungen - hier verwendet der Herausgeber die Formeln von Clausius) durch die Maxwell'sche 
Geschwindigkeitsverteilung und behauptet (ohne Beweis), dass dies auf den gew\"unschten Faktor $\sqrt{2}$
statt 4/3 f\"uhre. 


%\begin{equation}
%               d = 8 l \frac{V_{\rm l}}{V_{\rm g}} \hspace{1.5cm} {\rm heute} \hspace{0.6cm}
%               d = \sqrt{2} \cdot  6 \cdot l \frac{V_{\rm l}}{V_{\rm g}} \, .
%\end{equation} 
%Loschmidt wusste, dass es sich bei dieser Beziehung nur um eine N\"aherung handelte,
%da sie beispielsweise den Leerraum zwischen Molek\"ulen in der fl\"ussigen
%Phase vernachl\"assigt, doch er schreibt: \glqq ... [der Wert f\"ur die Gr\"o\ss e der Luftmolek\"ule] 
%ist aber sicher nicht um das zehnfache zu gro\ss\ oder zu klein\grqq, 
%was sich als richtig herausstellte (er lag um einen Faktor 3,5 falsch). 

Loschmidt verwendete die Clausius'sche Beziehung zwischen der mittleren freien
Wegl\"ange und dem Durchmesser (sowie der Dichte) der Molek\"ule. Die \glqq heutige\grqq\ Formel
beruht auf dem von Maxwell korrigierten Faktor. Allerdings sind beide Gleichungen ohnehin nur als
N\"aherungen aufzufassen, da sie beispielsweise den Leerraum zwischen Molek\"ulen in der fl\"ussigen
Phase vernachl\"assigen. 

\begin{thebibliography}{99}
\bibitem{Clausius58} Rudolf Clausius; \textit{\"Uber die mittlere L\"ange der Wege, welche bei der
       Molecularbewegung gasf\"ormiger K\"orper von den einzelnen Molec\"ulen zur\"uckgelegt werden;
       nebst einigen anderen Bemerkungen \"uber die mechanische W\"armetheorie}; Annalen der Physik 181 (1958),
       p.\ 239--258. \\
       \url{https://era-prod11.ethz.ch/zut/ch19/content/zoom/15344023}
\bibitem{Clausius60} Rudolf Clausius; \textit{On the dynamical theory of gases}; Philosophical Magazine 19,
         4.\ Series (1860); p.\ 434--436. 
\bibitem{Maxwell60} James Clerk Maxwell; \textit{Illustrations of the dynamical theory of gases -- Part I.\ on the
           motions and collisions of perfectly elastic spheres}; Philosophical Magazine 19; 4.\ Series (1860); 
           p.\ 19--32.               
\bibitem{Maxwell90} James Clerk Maxwell; \textit{The Scientific Papers of James Clerk Maxwell}; Cambridge
              University Press, 1890.            
\bibitem{Loschmidt} Johann Josef Loschmidt; \textit{Zur Gr\"o\ss e der Luftmolec\"ule}; Sitzungsbericht der kais.\ Akad.\
           der Wissenschaften, Band 52 (1866) Abt.\ II; p.\ 395--413.  
           \url{https://mpoweruk.com/timekeepers.htm}
\end{thebibliography}


%\end{document}

