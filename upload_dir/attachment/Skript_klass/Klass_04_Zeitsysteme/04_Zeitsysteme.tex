%\documentclass[german,10pt]{book}      
\usepackage{makeidx}
\usepackage{babel}            % Sprachunterstuetzung
\usepackage{amsmath}          % AMS "Grundpaket"
\usepackage{amssymb,amsfonts,amsthm,amscd} 
\usepackage{mathrsfs}
\usepackage{rotating}
\usepackage{sidecap}
\usepackage{graphicx}
\usepackage{color}
\usepackage{fancybox}
\usepackage{tikz}
\usetikzlibrary{arrows,snakes,backgrounds}
\usepackage{hyperref}
\hypersetup{colorlinks=true,
                    linkcolor=blue,
                    filecolor=magenta,
                    urlcolor=cyan,
                    pdftitle={Overleaf Example},
                    pdfpagemode=FullScreen,}
%\newcommand{\hyperref}[1]{\ref{#1}}
%
\definecolor{Gray}{gray}{0.80}
\DeclareMathSymbol{,}{\mathord}{letters}{"3B}
%
\newcounter{num}
\renewcommand{\thenum}{\arabic{num}}
\newenvironment{anmerkungen}
   {\begin{list}{(\thenum)}{%
   \usecounter{num}%
   \leftmargin0pt
   \itemindent5pt
   \topsep0pt
   \labelwidth0pt}%
   }{\end{list}}
%
\renewcommand{\arraystretch}{1.15}                % in Formeln und Tabellen   
\renewcommand{\baselinestretch}{1.15}                 % 1.15 facher
                                                      % Zeilenabst.
\newcommand{\Anmerkung}[1]{{\begin{footnotesize}#1 \end{footnotesize}}\\[0.2cm]}
\newcommand{\comment}[1]{}
\setlength{\parindent}{0em}           % Nicht einruecken am Anfang der Zeile 

\setlength{\textwidth}{15.4cm}
\setlength{\textheight}{23.0cm}
\setlength{\oddsidemargin}{1.0mm} 
\setlength{\evensidemargin}{-6.5mm}
\setlength{\topmargin}{-10mm} 
\setlength{\headheight}{0mm}
\newcommand{\identity}{{\bf 1}}
%
\newcommand{\vs}{\vspace{0.3cm}}
\newcommand{\noi}{\noindent}
\newcommand{\leer}{}

\newcommand{\engl}[1]{[\textit{#1}]}
\parindent 1.2cm
\sloppy

         \begin{document}  \setcounter{chapter}{3}


\chapter{Zeitsysteme}
% Kap 4
\label{chap_Zeitsysteme}

\info{Thomas Filk}{28.03.2024}%
Seit 1967 ist die Sekunde\index{Sekunde} 
durch die Frequenz $\Delta \nu_{\rm Cs}$ des Hyperfeinstruktur\"ubergangs\index{Hyperfeinstruktur\"ubergang}
in 133-C\"asium im Grundzustand definiert. Es gilt:
\begin{equation}
               1\,{\rm s}  =  9\,192\,631\,770 \, T = 9\,192\,631\,770 \, \frac{1}{\Delta \nu_{\rm Cs}} \, .
\end{equation}  
$T$ ist die Periodenl\"ange einer Schwingung zu diesem \"Ubergang.
Die Definition der Sekunde \"uber den Hyperfeinstruktur\"ubergang in 133-C\"asium l\"oste 1967 die 
Ephemeridensekunde ab, die \"uber die Position der Erde relativ zur Sonne definiert 
war.\index{Zeitsysteme|(} 

Man k\"onnte meinen, dass mit der Definition der SI-Sekunde und der Konstruktion von
Uhren, die diese Sekunde genau genug messen k\"onnen, das Problem der Zeitmessung
gel\"ost sei. Im Detail ergeben sich jedoch Probleme, die zu verschiedenen
Zeitsystemen gef\"uhrt haben. Zum einen muss man zwischen theoretischen Zeitsystemen,
die auf idealen Uhren basieren, und den Realisierungen solcher Zeitsysteme durch
tats\"achlich existierende und mit Fehlern behaftete Uhren unterscheiden. In diesem Zusammenhang
ist zu definieren, wie man aus den fehlerbehafteten Werten realer Messinstrumente eine Zeitskala
definiert kann. Au\ss erdem muss
man einen Referenzpunkt w\"ahlen, da nach der speziellen und allgemeinen Relativit\"atstheorie
Uhren an verschiedenen Orten (im Gravitationsfeld) oder in verschiedenen Bewegungszust\"anden
unterschiedliche Zeitsysteme definieren. Und schlie\ss lich m\"ochte man nat\"urlich, dass die
Zeitsysteme trotz ihrer derzeitigen Definition \"uber atomare Eigenschaften noch etwas mit dem 
Tag-und-Nacht-Rhythmus der Erde zu tun haben.

Auch heute noch sind verschiedene Zeitsysteme in Gebrauch. Die folgende Liste bzw.\ Behandlung
dieser Zeitsysteme kann daher nur einen groben \"Uberblick geben. 

\section{Das Julianische Datum}

Die Julianische Tagesz\"ahlung (englisch \textit{Julian day}) z\"ahlt, unabh\"angig von einem 
speziellen Kalender oder der Definition eines Jahres,\index{Julianische Tagesz\"ahlung}
die Tage seit einem Anfangsdatum, das als der 1.\ Januar des Jahres 4713 v.\,Chr.\ nach dem
zur\"uckextrapolierten Julianischen Kalender gew\"ahlt wird. Genauer beginnt diese Z\"ahlung mittags
um 12 Uhr (Sonnenh\"ochstand am heutigen Nullmeridian durch Greenwich). W\"ahlt man nicht den
zur\"uckgerechneten Julianischen Kalender sondern den zur\"uckgerechneten Gregorianischen
Kalender, so beginnt diese Zeitrechnung mit dem 24.\ November des Jahres 4714 v.\,Chr. Und da es
weder im Julianischen noch im Gregorianischen Kalender das Jahr 0 gibt, entspricht das Jahr 4713 dem
Jahr $-$4712 im astronomischen Kalender, der das Jahr 0 hinzunimmt. 

Die Wahl dieses Datums hat historische Gr\"unde. Zun\"achst werden hier drei Zyklen zu einer
Epoche zusammengefasst: Der Menton-Zyklus\index{Menton-Zyklus} 
oder Mondzyklus von 19 Jahren, der Sonnenzyklus\index{Sonnenzyklus}
von 28 Jahren\footnote{Im Julianischen Kalender haben 4 Jahre - die Periode dieser Kalenderz\"ahlung - 
eine Dauer von 1461 Tagen, dabei verschieben
sich die Wochentage um 5 Tage - der Rest bei einer Teilung von 1461 durch 7. 
Alle $4\times 7=28$ Jahre fallen also die Wochentage wieder auf dasselbe Datum. Im
Gregorianischen Kalender entspricht eine Periode 400 Jahren mit 146\,097 Tagen. Dies ist glatt durch 7 teilbar,
d.h.\ alle 400 Jahre fallen im Gregorianischen Kalender dieselben Daten wieder auf dieselben Wochentage.}
und der sogenannte Indiktionszyklus\index{Indiktionszyklus} 
von 15 Jahren (ein Zyklus, der im Altertum und Mittelalter
oft verwendet wurde und urspr\"unglich mit Neuberechnungen der Steuern einherging). 
Das Produkt dieser drei Zyklenl\"angen sind
7980 Jahre; dies bezeichnet man als eine Epoche. Das Jahr 4713 v.Chr.\ wurde im Mittelalter errechnet
als ein Jahr, in dem alle drei genannten Zyklen nach damaliger Rechnung begannen. Es hat den Vorteil, dass
es vor jeder historischen Datierung liegt, d.h.\ es gibt keine datierten historischen Ereignisse vor diesem
Jahr (au\ss er astronomische Ereignisse, die man heute zur\"uckrechnen kann). 

Dem Julianische Tag, der am 1.\ Januar 2000 mittags 12 Uhr (Universal time im Gregorianischen Kalender) begann, 
entspricht nach der Julianischen Tagesz\"ahlung  der Tag 2\,451\,545. Soviel Tage sind seit dem 1.\ Januar 4713 v.\,Chr.\
nach dem zur\"uckgerechneten Julianischen Kalender vergangen. Diese Z\"ahlung ist so einfach, dass
sie nicht nur in der Astronomie gerne verwendet wird, sondern auch in abgewandelter
Form in vielen Computersystemen. 

Will man neben dem Tag auch die Uhrzeit ber\"ucksichtigen, werden Nachkommastellen angegeben.
Der Tag hat 86\,400 Sekunden, sodass der 1.\ Januar 2000, 22 Uhr 25 Minuten und 10 Sekunden dem
Julianischen Datum 2\,451\,545,43414 entspricht. Sowohl die Julianische Tagesz\"ahlung (im Englischen
\textit{Julian Day}) als auch das\index{Julianisches Datum}  
Julianische Datum (\textit{Julian Date}) werden mit JD abgek\"urzt. 
Man beachte, dass diese Tagesz\"ahlung abgesehen von der Angabe des Anfangsdatums
nicht davon abh\"angt, ob ansonsten der Julianische oder Gregorianische Kalender (oder welcher andere 
Kalender auch immer) verwendet wird. 

Eine h\"aufig verwendete Abwandlung ist das sogenannte\index{Julianisches Datum!Modfiziertes} 
Modifizierte Julianische Datum, abgek\"urzt mit
MJD. Dies wird beispielsweise auch von dem BIPM (Bureau International des Poids et Mesures) in seinen
\textit{Circular T},\index{Circular T}\index{BIPM, Bureau International des Poids et Mesures} 
den monatlichen Berichten zur Bestimmung der Internationalen Atomzeit TAI,
verwendet. Das MJD unterscheidet sich vom JD in zwei Punkten:
\begin{enumerate}
\item
Der Tagesanfang wird auf 0 Uhr Mitternacht (statt auf 12 Uhr mittags) gelegt.
\item  
Der Beginn ist 0 Uhr, 17.\ November 1858. Das bedeutet, vom Julianischen Tag werden 2\,400\,000,5 Tage
abgezogen: ${\rm MJD=JD}-2\,400\,000,5$.       
\end{enumerate}
Damit hat der 1.\ Januar 2000, 12 Uhr mittags, den Wert: ${\rm MJD}= 51\,544,5$.  

\section{Die Ephemeridenzeit - ET}

Die\index{Ephemeridenzeit} 
Sekunde als der 1/86\,400-ste Teil eines Tages ist erst seit dem Mittelalter in Gebrauch
und bis 1956 war die Sekunde als der 1/86\,400-ste Teil eines mittleren Sonnentages definiert. 
Man wusste zu diesem Zeitpunkt allerdings schon, dass auch der mittlere Sonnentag nicht wirklich
vollkommen gleichm\"a\ss ig verl\"auft bzw.\ dass der mittlere Sonnentag davon abh\"angt, \"uber
welchen Zeitraum genau gemittelt wird. Daher verwendete man schon seit l\"angerem die Stellung
der Erde relativ zur Sonne zur Definition einer Sekunde. So war in der Zeit zwischen 1956 und 1968
die Sekunde definiert als der 1/31\,556\,925,9747-ste Teil des tropischen Jahres 1900 
(genauer: des theoretisch hochgerechneten tropischen
Jahres 1900, bestimmt aus den Bahndaten der Erde zum Zeitpunkt 31.12.1899, mittags 12 Uhr).
Diese Definition der Sekunde bezeichnet man auch als die 
Ephemeridensekunde.\index{Ephemeridensekunde}

Schon in babylonischer Zeit (um 1000 v.\,Chr.) gab es Tabellen zu den Positionen von Sonne, Mond und
Planeten, und der sogenannte\index{Almagest (Ptolem\"aus)} 
Almagest von Ptolem\"aus mit seinen genauen Beschreibungen der
Planetenbahnen gilt als H\"ohepunkt antiker Astronomie. Im Mittelalter wurden weitere 
Tabellen\index{Ephemeridentabellen} 
erstellt (bekannt sind die Toledaner Tafeln aus dem 12.\ Jahrhundert, die Alfonsinischen Tafeln 
aus dem 13.\ Jahrhundert, die Ephemeridentafeln von Regiomontanus von 1474, oder 
auch die Rudolfinischen Tafeln von 1627 von Johannes Kepler).  
  
Seit dem Mittelalter gab es Tabellen, in denen die Position der Sonne, des Mondes und der Planeten
relativ zu einem Koordinatensystem (meist durch Angabe der Rektazension und Deklination) 
und/oder relativ zueinander mit Daten und Tageszeiten korreliert wurden. 

\section{TT, TCG und TCB}
 
TT (\textit{Terrestial Time} oder \textit{Temps terrestre}), TCG (\textit{Geocentric Coordinated Time} oder
\textit{Temps-coordonn\'{e}e geocentrique}) und TCB (\textit{Barycentric Coordinate Time} oder
\textit{Temps-coordonn\'{e}e barycentrique}) sind idealisierte Zeitsysteme, manchmal spricht man auch von
Platonischen Zeitsystemen,\index{Zeitsystem!Platonisches} 
die auf idealen Uhren unabh\"angig von konkreten Realisierungen beruhen.
Sie dienen meist der theoretischen Beschreibung von Objekten in unserem Sonnensystem. 

TCB (baryzentrische koordinierte Zeit)\index{TCB, baryzentrisch koordinierte Zeit} 
ist die Zeit einer idealen, die SI-Sekunde anzeigenden Uhr, 
die sich in einem idealisierten hypothetischen 
System befindet, das sich parallel zum Schwerpunkt (dem Baryzentrum) des Sonnensystems bewegt,
also dieselben Bewegungen wie dieser Schwerpunkt ausf\"uhrt, aber keinem Gravitationspotenzial
unterliegt. Es wird typischerweise f\"ur die Beschreibung von Bahnkurven von Planeten oder Kometen
verwendet sowie von Raketen oder Satelliten, die den gravitativen Einfluss der Erde verlassen. 
Entsprechend ist\index{TCG, geozentrisch koordinierte Zeit} 
TCG (geozentrische koordinierte Zeit) die Zeit einer idealen, die SI-Sekunde anzeigenden 
Uhr, die sich in einem idealisierten System befindet, das sich parallel zum Schwerpunkt des Erdmittelpunkts 
bewegt, aber keinem Gravitationspotenzial unterliegt. Sie dient zur Beschreibung von Bahnkurven im 
gravitativen Einfluss der Erde, beispielsweise den Bahnkurven von Satelliten, dem Mond oder Raketen. 

TT (terrestrische Zeit)\index{TT, terrestrische Zeit} 
ist die Zeit einer idealen Uhr, die sich auf der Oberfl\"ache der Erde befindet. Hierbei ist die Oberfl\"ache
definiert durch die gravitative \"Aquipotentialfl\"ache (also konstantes Gravitationspotenzial) zur mittleren 
Meeresh\"ohe, wobei der Effekt der Erddrehung (Potenzial zur Fliehkraft) 
einbezogen wird, nicht jedoch der Einfluss von Str\"omungen oder Gezeiten. Diese Fl\"ache bezeichnet man
auch als das Geoid der Erde. Da die mittlere Meeresh\"ohe\index{Meeresh\"ohe, mittlere}
(z.B.\ im Zusammenhang mit dem Klimawandel) keine Konstante ist, hat man f\"ur das Potenzial
den Wert $U/c^2=6,969\,290\,134\cdot 10^{-10}$ definiert (der Quotient aus einem Potenzial
und $c^2$ ist eine dimensionslose Konstante), was bei einer homogenen Kugel von der Masse der Erde
einem Radius von etwas \"uber 6\,360\,km entspricht. 

TT und TCG unterscheiden sich also um einen multiplikativen Faktor $( 1 - U/c^2)$, um den TT langsamer
ist als TCG. Au\ss erdem wurde definiert, dass die beiden Zeiten am 1.\ Januar 1977, um 0 Uhr 32,184 Sekunden,
gleich waren. Die seltsame Sekundenzahl ergibt sich daraus, dass TT m\"oglichst exakt an die vorher
gebr\"auchliche Ephemeridenzeit angepasst werden sollte. Au\ss erdem entspricht diese Zeit exakt 0 Uhr nach 
der Internationalen Atomzeit TAI (siehe n\"achsten Abschnitt). Fr\"uher verwendete man f\"ur die terrestrische
Zeit TT die Bezeichnung TDT\index{TDT, terrestrial dynamical time} 
(terrestrial dynamical time); diese Bezeichnung findet man gelegentlich immer noch.  

\section{Die Internationale Atomzeit - TAI}

Die Internationale Atomzeit TAI\index{TAI, internationale Atomzeit} 
(\textit{Temps atomique international}) ist eine Realisierung von TT. Hierbei
handelt es sich um eine Zeit, die an existierenden Uhren abgelesen wird. Da eine einzelne Uhr immer
ungenau ist und auch mal St\"orungen haben kann, handelt es sich bei TAI um einen gewichteten
Mittelwert von derzeit \"uber 400 hoch pr\"azisen Uhren an \"uber 80 Orten. In der Physikalisch Technischen
Bundesanstalt (PTB)\index{PTB, Physikalisch Technische Bundesanstalt} 
in Braunschweig stehen zwei solche Atomuhren - genannt CSF1 und CSF2, hierbei handelt
es sich um C\"asium Fountain Clocks -, die zur
TAI beitragen. In regelm\"assigen Abst\"anden \"ubermitteln diese Uhren ihre Zeiten an das
BIPM. Dort werden die Daten um verschiedene Faktoren korrigiert (dazu z\"ahlen Laufzeiten der \"Ubertragung
sowie relativistische Korrekturen aufgrund der H\"ohenunterschiede der verschiedenen Uhren), die
Uhren mit den gr\"o\ss ten Abweichungen werden nicht ber\"ucksichtigt und von rund 300 Uhren wird
ein gewichtetes Mittel gebildet. Die Gewichtung der einzelnen Uhren richtet sich nach ihrem gesch\"atzten 
Fehler (der sich aus einer Theorie ihrer Funktionsweise bestimmt) sowie Ungenauigkeiten oder
Schwankungen in der Vergangenheit. Einmal im Monat wird im
\textit{BIPM Circular T} ver\"offentlicht, um wie viel die einzelnen Uhren von dem berechneten Mittelwert
zu bestimmten Zeitpunkten (die dann rund einen Monat zur\"uckliegen) abwichen. 

TAI versucht also eine m\"oglichst genaue Realisation einer idealen, die SI-Sekunde anzeigenden Uhr
zu sein, die sich auf der Oberfl\"ache des Geoids befindet. 
Insofern ist sie eine Realisation von TT, bis auf die oben erw\"ahnten 32,184 Sekunden, um die
sich die beiden Zeiten am 1.\ Januar 1977, 0 Uhr (TAI-Zeit), unterschieden. TT sollte dabei an die Ephemeridenzeit
angepasst werden, an die TAI schon 1958 angepasst worden war. Die 32,184 Sekunden Unterschied
erkl\"aren sich also daher, dass zwischen diesen beiden Zeitpunkten (1958 und 1977) schon ein
solcher Unterschied zwischen der Atomzeit und der Ephemeridenzeit bestand. 

\section{Universal Times - UT0, UT1, UT2}

Die\index{UT, Universal Times} 
Universal Times beziehen sich direkt auf die Lage der Erde und sind somit eine Fortf\"uhrung
der Ephemeridenzeit. Die Universal Time ist definiert \"uber den Winkel, den der Nullmeridian auf der
Erde relativ zu einem ausgezeichneten Punkt auf dem Himmels\"aquator (der Projektion des Erd\"aquators
vom Mittelpunkt der Erde aus auf die Himmelskugel) hat. Dieser Punkt auf dem Himmels\"aquator
ist heute definiert \"uber ein Referenzsystem am Himmel, das \"uber die Lage von festen Punkten am
Himmel - meist Quasare, die auch im Radiowellenbereich nachweisbar sind, sodass eine VLBI (Very Long
Baseline Interferometry) eine sehr genaue Richtungsbestimmung auch am Tag erm\"oglicht - definiert ist. Dieses
Referenzsystem bezeichnet\index{ICRF, International Celestial Reference Frame} 
man auch als ICRF (\textit{International Celestial Reference Frame}).   

Die Ephemeridenzeit bezog sich auf einen gemittelten Tag, wie er durch das (vom 31.12.1899, 12 Uhr, hochgerechnete) Jahr 1900 definiert war.\index{GMT, Greenwich Mean Time} 
Diese Zeit nannte man auch GMT (\textit{Greenwich Mean Time}). Sie entspricht genau
dem, was man sp\"ater UT0 nannte: Eine Zeitbestimmung \"uber einen gemittelten Sonnentag, also die Zeit
eines realen Sonnentags, die durch die Zeitgleichung - also die Einfl\"usse der elliptische Bahn der Erde sowie ihrer Neigung
gegen\"uber der Ekliptik - korrigiert wird und somit zur Zeit eines mittleren Sonnentags geh\"ort. 
Der Winkel ERA (\textit{Earth Rotation Angle})\index{ERA, Earth Rotation Angle} 
des Nullmeridians relativ zu dem Referenzsystem am Himmel  
ist bis auf eine Konstante direkt proportional zur UT. Die Proportionalit\"atskonstante ber\"ucksichtigt, dass es sich
hierbei um eine Messung der Erdrotation relativ zu einem Himmelssystem (siderischen Referenzsystem)
handelt, und diese Zeit korrigiert werden muss, um einen Sonnentag zu erhalten. Dementsprechend entspricht
dieser Proportionalit\"atsfaktor ziemlich genau dem Faktor $(1+1/365,25)$, um den der siderische Tag im 
Vergleich zum Sonnentag korrigiert werden muss. Genauer lautet die Beziehung:
\begin{equation}
      {\rm ERA} = 2\pi (0,7790572732640 + 1,00273781191135448 \cdot T_{\rm u}) \, {\rm rad}  \, ,
\end{equation}
wobei $T_{\rm u} = ({\rm JD(UT1)} - 2\,451\,545,0)$ ist, und ${\rm JD(UT1)}$ dem Julianischen Datum
nach der UT1-Uhrzeit entspricht. ERA ist die gemessene Variable (ein Winkel) und JD(UT1) - ein Datum
mit einem Zeitpunkt - wird daraus berechnet. ERA, also der gemessene Winkel, wird dabei um zwei
Faktoren korrigiert: (1) die Zeitgleichung, die den Unterschied zwischen mittlerem und wahrem 
Sonnentag angibt, und (2) Schwankungen in der Drehachse der Erde (sogenannte Polarbewegungen), durch die
der L\"angengrad, beispielsweise eines Observatoriums, nicht eindeutig bestimmt ist. Weitere jahreszeitliche
Schwankungen, die beispielsweise in dem Zeitsystem UT2 ber\"ucksichtigt werden, werden f\"ur UT1 nicht
ber\"ucksichtigt.       

\section{Universal Coordinated Time - UTC}

Wir haben nun zwei realisierte Zeitsysteme (d.h., Zeitsysteme, die durch Messungen an realen physikalischen Systemen bestimmt werden): 
Die TAI, die \"uber Atomuhren bestimmt wird, die m\"oglichst nahe an der SI-Sekunde
arbeiten, und UT1, die \"uber die Lage der Erde relativ zur Sonne bestimmt wird. UT1 hat den Vorteil, dass 12 Uhr
mittags mit unserer Vorstellung von \glqq Sonnenh\"ochststand\grqq\ zusammenf\"allt, wohingegen TAI auf einer
m\"oglichst genauen und gleichm\"a\ss igen Realisierung einer Sekunde beruht. Damit entsteht das Problem:
An welches Zeitsystem sollen wir uns halten, wenn diese beiden Zeitsysteme auseinanderlaufen. UT1 hat den Vorteil,
unseren Vorstellung von Tag und Nacht zu entsprechen, TAI hat den Vorteil, dass die Sekunden immer gleich
lang sind. 

Um ein Auseinanderlaufen der beiden Zeitsysteme auszugleichen, hat man sich (nach anf\"anglichen Problemen bei
der Namensgebung wie auch bei der genauen Definition) 1972 auf das Zeitsystem 
der UTC\index{UTC, Universal Coordinated Time}
(\textit{Universal Coordinated Time}) geeinigt. UCT verl\"auft parallel zu TAI, d.h., UTC richtet sich bez\"uglich
der genauen Zeitangabe nach der besten Realisierung der SI-Sekunde auf der Erdoberfl\"ache, und das ist
die TAI. Bevor allerdings UT1 und TAI um 0,9 Sekunden auseinanderlaufen (weil die Erddrehung gewissen
Schwankungen unterworfen ist), wird f\"ur UTC eine\index{Schaltsekunde} 
Schaltsekunde entweder eingef\"ugt oder weggelassen.
In der Vergangenheit wurden nur Schaltsekunden eingef\"ugt, da die Erdrotation etwas langsamer ist als
die TAI-Zeit. Als UTC im Jahre 1972 eingef\"uhrt wurde bestand schon ein Unterschied von 10 Sekunden
zwischen UT1 und TAI, sodass damals definiert wurde UTC = TAI - 10\,s. Seitdem wurden insgesamt 27
weitere Schaltsekunden eingef\"ugt, die letzte am 31.\ Dezember 2016. S\"amtliche Schaltsekunden wurden
in der Vergangenheit entweder am 31.\ Juni oder am 31.\ Dezember eingef\"ugt. In den letzten Jahren hat
die Geschwindigkeit der Eigendrehung der Erde wieder etwas zugenommen, sodass nicht ausgeschlossen
wird, dass in der nahen Zukunft zum ersten Mal in der Geschichte der UTC eine Sekunde \glqq herausgenommen\grqq\ wird. 

Das Einf\"ugen von Schaltsekunden f\"uhrte dazu, dass im Zeitsystem der UTC gelegentlich eine Minute
61 Sekunden hat. Statt nach der 59.-sten Sekunde die 0.te Sekunde folgen zu lassen, z\"ahlt man 
eine 60.ste Sekunde hinzu und beginnt dann den neuen Monat mit der Sekunde 0. Da in der Vergangenheit
noch nie eine Schaltsekunde entfernt wurde, gibt es gewisse Bedenken, ob alle Computersysteme mit einem
solchen Schritt zurecht kommen. Da auch beim Einf\"ugen von Schaltsekunden in der Vergangenheit immer
wieder Probleme mit verschiedenen digitalen Systemen auftraten, wurde in j\"ungerer Zeit \"uberlegt, die
Definition von UTC nochmals zu \"andern. Es wird angestrebt, bis 2035 eine neue Definition zu finden, bei
der nur alle paar Jahrhunderte eine Korrektur notwendig wird \cite{BIPM2022}. Beispielsweise k\"onnte man die Differenz
zwischen UT1 und TAI auf mehrere Minuten anwachsen lassen, bevor Korrekturen vorgenommen werden.
Rein subjektiv werden wir als Menschen die langsame Verschiebung der Mittagsstunde ohnehin erst
bemerken, wenn sich die Differenz auf die Gr\"o\ss enordnung einer Stunde summiert hat. 

Die UTC ist die Zeit, die wir beispielsweise \"uber Funkuhren, das Radio oder Fernsehen empfangen.
Diese Zeit ist bis auf 0,9 Sekunden an die Orientierung der Erde relativ zum Referenzsystem des Himmels
angepasst, l\"auft parallel zur Atomzeit TAI, d.h.\ ist sehr regelm\"a\ss ig, hat aber den Nachteil, 
dass eine Minute gelegentlich eine Sekunde mehr oder weniger hat. Wegen der Unregelm\"a\ss igkeit
der Erddrehung k\"onnen Schaltsekunden nicht langfristig vorhergesagt werden, sondern erst in einem
Zeitraum von einem halben Jahr. 

\section{Lokalzeit - Local Time}

Die Lokalzeit\index{Lokalzeit} 
ist die Zeit, die von Radio- oder Fernsehstationen, Funkuhren bzw.\ \"offentlichen Uhren
an einem Ort angezeigt wird. Sie richtet sich nach der UTC, unterliegt also unter anderem der 
Einf\"ugung oder L\"oschung von Schaltsekunden, unterscheidet sich von der UTC aber in zweierlei
Hinsicht:
\begin{enumerate}
\item
Sie\index{Zeitzone} 
ber\"ucksichtigt die lokale Zeitzone: UTC ist die Fortsetzung der GMT (Greenwich Mean Time) und
richtet sich nach dem Nullmeridian durch Greenwich, d.h.\ nach dem L\"angengrad, wo im Jahresmittel
mittags um 12 Uhr die Sonne ihren H\"ochststand hat. Damit man an nahezu allen Orten um die Mittagszeit
den Sonnenh\"ochststand hat, wurde die Erde in Zeitzonen eingeteilt, die einseits von L\"angengraden,
andererseits aber auch von L\"andergrenzen berandet sind. Die meisten Zeitzonen unterscheiden sich
von der Zeitzone von Greenwich um ein Vielfaches einer vollen Stunde. Alle 15 L\"angengrade beginnt also rein rechnerisch
eine neue Zeitzone, wobei die meisten L\"ander (Ausnahmen sind die USA, Kanada, Russland und
Australien) nur eine Zeitzone haben, was z.B.\ f\"ur China bedeutet, dass es sich rechnerisch \"uber 
fast vier Zeitzonen erstreckt aber im gesamten Land dieselbe Zeitzone gilt. 
Es gibt aber auch einige L\"ander mit halbst\"undiger Zeitverschiebung (Iran $+3\frac{1}{2}$, Afghanistan $+4\frac{1}{2}$,
Indien $+5\frac{1}{2}$, Burma $+6\frac{1}{2}$, Zentralaustralien $+9\frac{1}{2}$) sowie   
viertelst\"undiger Zeitverschiebung (Nepal $+5\frac{3}{4}$, kleine Teile von Australien $+8\frac{3}{4}$
sowie die Chatham Islands $+10\frac{3}{4}$). 

Zentraleuropa (mit Ausnahme von Portugal und\index{MEZ, mitteleurop\"aische Zeit}
den Inselgruppen Kanaren, Madeira, Island und Azoren) verwenden die Mitteleurop\"aische Zeit (MEZ), die
sich von der UTC um +1 Stunde unterscheidet, d.h., wenn es in Greenwich 12 Uhr mittags ist, ist es in
Mitteleuropa bereits 1 Uhr mittags. 
\item
In den Sommermonaten\index{Sommerzeit} 
wechseln viele L\"ander auf die Sommerzeit. Dazu verschiebt man die Zeitzone um eine
weitere Stunde nach Osten, d.h., relativ zum Sonnenstand ist es eine Stunde sp\"ater. Damit verbunden ist etwas
l\"angere Dunkelheit am Morgen und etwas l\"angere Helligkeit am Abend. In diesem Fall spricht man in
Mitteleuropa von der\index{MESZ, Mitteleurop\"aische Sommerzeit} 
MESZ - Mitteleurop\"asische Sommerzeit, die sich von der UTC um +2 Stunden unterscheidet.      
\end{enumerate}
Ungef\"ahr entlang des 180-sten L\"angengrads (im Pazifik, durch die Fiji-Inseln) verl\"auft die Internationale Datumsgrenze.\index{Datumsgrenze}
\"Uberquert man diese Grenze von West nach Ost, muss man das Datum um einen Tag zur\"uckstellen, \"uberquert
man sie von Ost nach West stellt man das Datum um einen Tag vor. 

\section{GPS Time}

Abschlie\ss end soll noch kurz auf die GPS-Zeit\index{GPS-Zeit}\index{GPS, Global Positioning System} 
des Global Positioning Systems (GPS) eingegangen werden. 
Das GPS besteht unter anderem aus 24 aktiven Satelliten (es befinden sich derzeit - Dezember 2022 - 32
Satelliten im Orbit, davon sind 31 einsatzbereit, sieben der Satelliten dienen als Backup), die st\"andig ein
Signal aussenden, das die genaue Zeit sowie den genauen Ort dieser Satelliten angibt. Mindestens vier dieser
Satelliten befinden sich jederzeit in \glqq Sichtlinie\grqq\ von jedem beliebigen Punkt der Erde aus. Aus den
Signalen kann ein geeigneter Empf\"anger seine genaue Position sowie die genaue Zeit bestimmen. 

Die vom GPS-System verwendete GPS-Zeit richtet sich im Wesentlichen nach der TAI, d.h.\ es wird die
auf das Geoid der Erde bezogene Eigenzeit verwendet, ohne Einschub oder Wegnahme von Schaltsekunden. 
Das bedeutet unter anderem, dass die Uhren in den GPS-Satelliten \glqq falsch gehen\grqq: Es handelt sich
nicht um SI-Uhren, die die Eigenzeit in dem Satelliten messen, sondern diese Eigenzeiten werden um die
relativistischen Effekte aufgrund der Bewegung des Satelliten sowie des Gravitationsfelds der Erde
korrigiert, sodass diese Uhren die Zeit angeben, die an einem ruhenden Ort auf dem Geoid der Erde
gilt. Damit stimmen diese Uhren in ihrem Zeittakt mit der Atomzeit TAI \"uberein. 

Als Startpunkt der GPS-Zeit wurde 0 Uhr am 6.\ Januar 1980 festgelegt. Zu diesem Zeitpunkt unterschieden
sich UTC und TAI bereits um 19 Schaltsekunden. Da die GPS-Zeit zu diesem Zeitpunkt an die UTC angepasst
wurde, unterscheiden sich GPS-Zeit und TAI also dauerhaft um 19 Sekunden: GPS = TAI - 19\,s. Derzeit (Dezember
2022) unterscheiden sich UTC und GPS-Zeit um 18 Sekunden; diese 18 Sekunden wurden seit dem 6.\ Januar
1980 bei der UTC als Schaltsekunden eingef\"ugt. Damit geht UTC relativ zur GPS-Zeit um 18 Sekunden nach,
d.h.\ es gilt: GPS = UTC + 18\,s. Diese Zeitdifferenz \"andert sich aber, falls bei der UTC Schaltsekunden eingef\"ugt
oder weggelassen werden.   

Die Z\"ahlung bei der GPS-Zeit verwendet mehrere Einheiten: Epochen, Wochen, Tage und Sekunden.
\begin{enumerate}
\item
Eine Epoche\index{Epoche} 
besteht aus 1024 Wochen. Die Wochenzahl wird als 10-Bit Zeichenfolge \"ubertragen. 
Der Woche 1023 folgt die Woche 0. Dies bezeichnet man auch als Rollover.\index{Rollover} 
Eine Epoche dauert somit
7168 Tage oder etwas \"uber 19,6 Jahre. Da diese Rollover (bisher fanden
zwei solche Rollover statt - in der Nacht vom 21.\ auf den 22.\ August 1999 und in der Nacht vom 
6.\ auf den 7.\ April 2019) zu Problemen bei manchen Anwendern gef\"uhrt haben, will
man in naher Zukunft die Zeichenfolge f\"ur die Wochen auf 13 Bit erweitern, sodass nur ungef\"ahr alle 157 Jahre
ein solcher Rollover stattfindet. Derzeit (am 11.\ Dezember 2022) befinden wir uns in der GPS-Woche 2240 (dies ist
die Anzahl der Wochen, die seit dem 6.\ Januar 1980 vergangen sind), also in der 192.\ Woche der Epoche 2.
\item
Die Wochen\index{Woche} 
werden mit Beginn vom 6.\ Januar 1980 gez\"ahlt. Da es jedoch derzeit noch wegen der 10-Bit-Folge
f\"ur die Angabe der Wochen zu Rollover kommt, gibt die GPS-Zeit nur die Wochenzahl der laufenden Epoche
wieder (am 11.\ Dezember 2022 war das die Woche 192). 
\item
F\"ur jede Woche wird der Tag angegeben, also ein Wert zwischen 0 und 6. Die neue Woche beginnt
mit dem Sonntag, dem Tag 0. 
\item
F\"ur jeden Tag werden die Sekunden angegeben. Der Tag beginnt um Mitternacht 0 Uhr.  
\end{enumerate}
Damit besteht eine volle Angabe der GPS-Zeit aus: Wochenzahl (gesamt), Epoche + Woche innerhalb der
Epoche (ein Wert zwischen 0 und 1023), Tag (ein Wert zwischen 0 und 6) und Sekunden an diesem Tag
(ein Wert zwischen 0 und 86399). Die Gesamtzahl der Wochen bzw.\ die Epoche wird allerdings nicht im GPS-Signal kodiert. 
Der Empf\"anger bzw.\ Anwender muss also wissen, in welcher Epoche man sich befindet. 

\section{Kuriosit\"aten}

\subsection{Die Datumsgrenze}

Es wurde oben erw\"ahnt,\index{Datumsgrenze} 
dass man bei der \"Uberquerung der Datumsgrenze von West nach Ost das Datum
um einen Tag zur\"uckstellen muss (also 24 Stunden subtrahieren muss), bei der \"Uberquerung von Ost nach West
muss das Datum entsprechend um einen Tag vorgestellt werden. Als Scherzfrage f\"ur Kinder in den unteren Klassen
bietet sich nun folgendes Gedankenexperiment an: Angenommen, man k\"onnte mit einem sehr schnellen Flugzeug
immer von West nach Ost um die Erde reisen und dabei die Datumsgrenze in kurzer Zeit mehrfach von West nach
Ost \"uberqueren, dann w\"urde man jedesmal das Datum um einen Tag zur\"ucksetzen. Kann man auf diese Weise
in die Vergangenheit reisen, also ist man beispielsweise nach f\"unfmaliger \"Uberquerung der Datumsgrenze um
f\"unf Tage zur\"uckgereist? Eine \"ahnliche Frage kann man nat\"urlich auch bez\"uglich der Reisen in die
Zukunft stellen, da man bei Reisen um die Erde von Ost nach West jedesmals beim \"Uberqueren der Datumsgrenze
das Datum um einen Tag vorstellen muss. 

Nat\"urlich geht das nicht: Wenn man von West nach Ost reist muss man ja jedesmal, wenn man in eine neue
Zeitzone gelangt, die Uhr um eine Stunde vorstellen. Hat man die Erde dann einmal umrundet, wurde die Uhr um
insgesamt 24 Stunden vorgestellt. Beim \"Uberqueren der Datumsgrenze gleicht man dies wieder aus, indem die
Uhr um 24 Stunden zur\"uckgesetzt bzw.\ das Datum um 1 Tag zur\"uckgesetzt wird. 

\subsection{Verschiebung der Datumsgrenze zum Millenium}

Die Kiribati Inseln\index{Kiribati Inseln} 
(offiziell die Republik Kiribati) bilden eine Inselgruppe in der Mitte des Pazifiks, die sich
entlang des \"Aquators vom ungef\"ahr 170.\ L\"angengrad Ost (die Insel Banaba, westlich der Gilbert Islands) 
bis zum 150.\ L\"angengrad West (die Insel Caroline, heute Millenium Island, in der Inselgruppe der Line Islands) 
erstreckt. Die Datumsgrenze verlief bis Mitte der 90er Jahre des letzten Jahrhunderts durch diese Inselgruppe, sodass
auf verschiedenen Inseln zum selben Zeitpunkt ein unterschiedliches Datum herrschte. Zum 1.\ Januar 1995 wurde
die Datumsgrenze von der Republik Kiribati so verlegt, dass sie nun \"ostlich um die Line Inseln heruml\"auft und damit
zwei Zeitzonen in das alte Datum hineinragt (das bei Inseln in der N\"ahe, z.B.\ auch auf Hawai, das
auf demselben L\"angengrad liegt, g\"ultig ist). Als offizieller Grund wurde
angegeben, dass ein unterschiedliches Datum innerhalb eines Landes zu Problemen in landes\"ubergreifenden
Angelegenheiten f\"uhre. Inoffiziell wurde allerdings auch nie bestritten, dass ein wirtschaftlicher Grund dahinter
steckte: Auf diese Weise wurden die Line Islands die ersten Gebiete, die zum Millenium ins neue Jahrtausend
wechselten. Man erhoffte sich dadurch als touristische Attraktion das erste Land der Welt zu sein, in dem der
Sonnenaufgang im neuen Jahrtausend beobachtet werden konnte.  
\index{Zeitsysteme|)} 

\begin{thebibliography}{99}
\bibitem{BIPM2022} Resolution 4 of the 27th CGPM (General Conference on Weights and Measures) 2022
\url{https://www.bipm.org/en/cgpm-2022/resolution-4}
\end{thebibliography}

%\end{document}

