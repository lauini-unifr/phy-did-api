\documentclass[german,10pt]{book}      
\usepackage{makeidx}
\usepackage{babel}            % Sprachunterstuetzung
\usepackage{amsmath}          % AMS "Grundpaket"
\usepackage{amssymb,amsfonts,amsthm,amscd} 
\usepackage{mathrsfs}
\usepackage{rotating}
\usepackage{sidecap}
\usepackage{graphicx}
\usepackage{color}
\usepackage{fancybox}
\usepackage{tikz}
\usetikzlibrary{arrows,snakes,backgrounds}
\usepackage{hyperref}
\hypersetup{colorlinks=true,
                    linkcolor=blue,
                    filecolor=magenta,
                    urlcolor=cyan,
                    pdftitle={Overleaf Example},
                    pdfpagemode=FullScreen,}
%\newcommand{\hyperref}[1]{\ref{#1}}
%
\definecolor{Gray}{gray}{0.80}
\DeclareMathSymbol{,}{\mathord}{letters}{"3B}
%
\newcounter{num}
\renewcommand{\thenum}{\arabic{num}}
\newenvironment{anmerkungen}
   {\begin{list}{(\thenum)}{%
   \usecounter{num}%
   \leftmargin0pt
   \itemindent5pt
   \topsep0pt
   \labelwidth0pt}%
   }{\end{list}}
%
\renewcommand{\arraystretch}{1.15}                % in Formeln und Tabellen   
\renewcommand{\baselinestretch}{1.15}                 % 1.15 facher
                                                      % Zeilenabst.
\newcommand{\Anmerkung}[1]{{\begin{footnotesize}#1 \end{footnotesize}}\\[0.2cm]}
\newcommand{\comment}[1]{}
\setlength{\parindent}{0em}           % Nicht einruecken am Anfang der Zeile 

\setlength{\textwidth}{15.4cm}
\setlength{\textheight}{23.0cm}
\setlength{\oddsidemargin}{1.0mm} 
\setlength{\evensidemargin}{-6.5mm}
\setlength{\topmargin}{-10mm} 
\setlength{\headheight}{0mm}
\newcommand{\identity}{{\bf 1}}
%
\newcommand{\vs}{\vspace{0.3cm}}
\newcommand{\noi}{\noindent}
\newcommand{\leer}{}

\newcommand{\engl}[1]{[\textit{#1}]}
\parindent 1.2cm
\sloppy

         \begin{document}  \setcounter{chapter}{6}


\chapter{\glqq Koan\grqq-Aufgaben}
% Kap x
\label{chap_Koan}

In verschiedenen buddhistischen Schulen gelten Koans als scheinbar sinnlose oder paradoxe
Aussagen bzw.\ Probleme, bei denen der Sch\"uler durch die Meditation \"uber dieses Koan zu tieferen
Einsichten im Sinne des Buddhismus gelangt. Das eigentliche Ziel ist weniger eine Antwort als der Weg
zu dieser Antwort. 

In \"ahnlicher Weise gibt es Fragen, Aufgaben oder Probleme in der Physik, die zun\"achst verbl\"uffen. 
Wenn man diesen
Fragen aber nachgeht und die Antworten findet, hat man auf dem Erkenntnisweg meist sehr viel
gelernt. Daher ist es auch wichtig, dass man die Fragen selbstst\"andig l\"ost. Ein Nachschlagen
oder \glqq googeln\grqq\ der Antworten verfehlt diesen Zweck. Es ist nicht wichtig, schnell eine
L\"osung zu finden. Man kann auch mehrere Tage oder gar Wochen \"uber diese Probleme nachdenken, 
aber man sollte sich bem\"uhen, die Antworten selbst zu finden. Falls das dauerhaft
nicht gelingt, fehlen wesentliche
physikalische Grundlagen und dann sollte man diese nachholen bzw.\ auffrischen. 

\section{Die Relativit\"at physikalischer Eigenschaften}

Die Geschwindigkeit und der Impuls sind relative Konzepte, d.h., f\"ur Beobachter in
verschiedenen Inertialsystemen (mit verschiedenen relativen Geschwindigkeiten) ist
die Geschwindigkeit bzw.\ der Impuls eines beobachteten Objekts unterschiedlich. Damit sind
Geschwindigkeit und Impuls keine Eigenschaften eines Objekts sondern Eigenschaften, die ein Objekt
relativ zu einem Beobachter hat. Da dies f\"ur die Geschwindigkeit und den Impuls gilt, gilt es auch f\"ur
die kinetische Energie: Impuls und kinetische Energie sind keine
Eigenschaften eines Objekts, sondern nur relativ zu einem Beobachter definiert. 

Einstein war ein Meister, die \glqq Gleichartigkeit der Physik in verschiedenen Bezugssystemen\grqq\
auszunutzen. Es sollte keine Rolle spielen, wer ein physikalisches Ph\"anomen beobachtet bzw.\
aus welchem Bezugssystem dieses Ph\"anomen beobachtet wird - die Physik sollte immer dieselbe
sein. Doch was bedeutet es dann, wenn unterschiedliche Beobachter demselben Objekte 
unterschiedliche Impulse oder kinetische Energien zuschreiben?

\subsection{Die kinetisch Energie}

Jeder Beobachter schreibt einem Objekt eine andere Geschwindigkeit zu. Damit ist
aber auch die kinetische Energie eines Objekts beobachterabh\"angig. Bedeutet
dies, dass die kinetische Energie keine physikalische Eigenschaft eines Objekts ist?

\subsection{Die deBroglie-Wellenl\"ange} 

Die deBroglie-Wellenl\"ange eines Teilchens h\"angt von seinem Impuls ab. Da der
Impuls aber wiederum vom Beobachter abh\"angt, ist auch die deBroglie-Wellenl\"ange
beobachterabh\"angig. Damit ist aber auch die Wellenfunktion, die man einem Teilchen
zuordnet, abh\"angig vom Bewegungszustand des Beobachters. Was bedeutet dies
f\"ur die Realit\"at der Wellenfunktion? Und weshalb beobachten wir ein scharfes
Interferenzmuster, wenn wir Elektronen oder andere Teilchen an einem Doppelspalt oder
Gitter streuen? Wenn die deBroglie-Wellenl\"ange beobachterabh\"angig ist, ist dann
auch das beobachtete Interferenzmuster von Teilchen am Doppelspalt beobachterabh\"angig?

\section{Aufgaben zur Optik}

\subsection{Weshalb sehen wir \"uberhaupt Licht?}

Die meisten (z.B.\ thermischen) Lichtquellen (Gl\"ubirne) haben eine riesige Anzahl an
Emissionszentren f\"ur die abgegebene Strahlung (die Gr\"o\ss enordnung liegt
ungef\"ahr im Bereich von $10^{18}$ bis $10^{20}$ solcher Zentren). Jedes dieser Zentren
sendet f\"ur kurze Augenblicke (eine grobe Gr\"o\ss enordnung ist $10^{-8}$ Sekunden)
eine Strahlung aus. Im Mittel finden wir in dieser 
Strahlung f\"ur jede Richtung alle m\"oglichen Phasen, sodass man meinen k\"onnte, im
Mittel heben sich alle elektromagnetischen Anregungen weg. Weshalb ist das nicht der Fall?
Weshalb sehen wir Licht? (Die Energieerhaltung ist nat\"urlich ein einfaches Argument, aber
man kann es auch auf dem Niveau der elektromagnetischen Felder erkl\"aren.) 

\section{Aufgaben zur Relativit\"atstheorie}

\section{Aufgaben zur Quantentheorie}

\section{Aufgaben zur Thermodynamik und Statistischen Mechanik}

\subsection{Die ideale Gas-Gleichung I}

Die Temperatur ist ein direktes Ma\ss\ f\"ur die mittlere kinetische Energie der Teilchen in
einem Gas. Der Druck ist ein Ma\ss\ f\"ur die Kraft, die bei einem Sto\ss\ der Teilchen gegen
eine der Gef\"a\ss w\"ande auf diese Wand \"ubertragen wird. Die ideale Gas-Gleichung
besagt, dass bei konstantem Volumen und konstanter Teilchenzahl der Druck proportional
zur Temperatur ist. Wenn sich die Temperatur verdoppelt, verdoppelt sich somit der Druck.
Doch eine doppelte Temperatur bedeutet, dass die Geschwindigkeit um einen Faktor
$\sqrt{2}$ zugenommen hat, und damit auch der Impuls\"ubertrag auf eine Wand. 
Wieso hat sich der Duck verdoppelt?

\subsection{Die ideale Gas-Gleichung II}

In einem Beh\"alter befinden sich zwei Gase, deren Molek\"ule eine Masse $m_1$ und $m_2$
haben. Die Gase befinden sich im thermischen Gleichgewicht. Die Anzahl der Gasatome von
beiden Substanzen seien gleich. Wie verhalten sich die Partialdr\"ucke der beiden Gase? 

\section{Astrophysik}

Weshalb ist ein tropisches Jahr um 20 Minuten und 24 Sekunden k\"urzer als ein
siderisches Jahr, aber ein tropischer Monat nur um 6,8 Sekunden k\"urzer als ein
siderischer Monat?


\end{document}

