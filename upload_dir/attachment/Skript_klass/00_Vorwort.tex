%\documentclass[german,12pt]{book}   
\usepackage{makeidx}
\usepackage{babel}            % Sprachunterstuetzung
\usepackage{amsmath}          % AMS "Grundpaket"
\usepackage{amssymb,amsfonts,amsthm,amscd} 
\usepackage{mathrsfs}
\usepackage{rotating}
\usepackage{sidecap}
\usepackage{graphicx}
\usepackage{color}
\usepackage{fancybox}
\usepackage{tikz}
\usetikzlibrary{arrows,snakes,backgrounds}
\usepackage{hyperref}
\hypersetup{colorlinks=true,
                    linkcolor=blue,
                    filecolor=magenta,
                    urlcolor=cyan,
                    pdftitle={Overleaf Example},
                    pdfpagemode=FullScreen,}
%\newcommand{\hyperref}[1]{\ref{#1}}
%
\definecolor{Gray}{gray}{0.80}
\DeclareMathSymbol{,}{\mathord}{letters}{"3B}
%
\newcounter{num}
\renewcommand{\thenum}{\arabic{num}}
\newenvironment{anmerkungen}
   {\begin{list}{(\thenum)}{%
   \usecounter{num}%
   \leftmargin0pt
   \itemindent5pt
   \topsep0pt
   \labelwidth0pt}%
   }{\end{list}}
%
\renewcommand{\arraystretch}{1.15}                % in Formeln und Tabellen   
\renewcommand{\baselinestretch}{1.15}                 % 1.15 facher
                                                      % Zeilenabst.
\newcommand{\Anmerkung}[1]{{\begin{footnotesize}#1 \end{footnotesize}}\\[0.2cm]}
\newcommand{\comment}[1]{}
\setlength{\parindent}{0em}           % Nicht einruecken am Anfang der Zeile 

\setlength{\textwidth}{15.4cm}
\setlength{\textheight}{23.0cm}
\setlength{\oddsidemargin}{1.0mm} 
\setlength{\evensidemargin}{-6.5mm}
\setlength{\topmargin}{-10mm} 
\setlength{\headheight}{0mm}
\newcommand{\identity}{{\bf 1}}
%
\newcommand{\vs}{\vspace{0.3cm}}
\newcommand{\noi}{\noindent}
\newcommand{\leer}{}

\newcommand{\engl}[1]{[\textit{#1}]}
\parindent 1.2cm
\sloppy

     \begin{document}

\chapter*{Vorwort}
\addcontentsline{toc}{chapter}{Vorwort}
\thispagestyle{empty}

Der vorliegende Text entstand im Zusammenhang 
mit dem Projekt, f\"ur das mir im Juni 2022 von der Wilhelm und Else Her\"aus-Stiftung
eine Seniorprofessur an der Universit\"at Freiburg verliehen wurde. Dieses Projekt
besteht in der Ausarbeitung einer Vorlesung mit dem (vorl\"aufigen) Titel
\glqq Ausgew\"ahlte Kapitel der Modernen Physik\grqq. Die Vorlesung richtet sich
speziell an Studierende  f\"ur das Lehramt Physik und soll einige Themen abdecken,
die im regul\"aren Studium oft nur am Rande angesprochen werden, die aber f\"ur
die Unterrichtspraxis relevant sein k\"onnen, auch wenn sie teilweise deutlich \"uber
die Vorgaben der Lehrpl\"ane hinausgehen.  

Die einzelnen Kapitel in diesem Text sind weitgehend unabh\"angig voneinander und
k\"onnen jeweils f\"ur sich gelesen werden. Sie bestehen aus Kurztexten zu bestimmten
Themen. Daher kommt es gelegentlich vor, dass mehrere Kapitel in einem ihrer Unterabschnitte
denselben physikalischen Sachverhalt in unterschiedlicher Ausf\"uhrlichkeit beschreiben.
Wenn dieser Sachverhalt f\"ur ein bestimmtes Thema relevant ist, wird er in diesem
Zusammenhang auch kurz beschrieben, wobei dann aber f\"ur eine ausf\"uhrlichere
Behandlung auf einen anderen Kurztext verwiesen werden kann. Aus diesem Grund hat
auch jedes Kapitel seine eigenen Literaturangaben. 

Gelegentlich wird man sich wundern, dass in einer Vorlesung zur \glqq modernen Physik\grqq\
auch die Bestimmung des Erdumfangs nach Eratosthenes oder die Physik der
Gezeiten thematisiert wird. Aber die Bestimmung des Erdumfangs ist der erste Schritt in
der kosmischen Entfernungsleiter, und dies ist ein aktuelles Forschungsgebiet, das
beispielsweise mit der Suche nach Dunkler Energie verkn\"upft ist. Die Physik der
Gezeiten h\"angt eng mit den l\"anger werdenden Tagen und dem zunehmenden Abstand
zwischen Erde und Mond zusammen, die erst im letzten Jahrhundert entdeckt wurden. 
Insofern bezieht sich \glqq moderne Physik\grqq\ nicht nur auf die Quantentheorie und die
Relativit\"atstheorie. 

Die Sammlung wird st\"andig erweitert. Mittlerweile haben auch andere Dozentinnen und
Dozenten angeboten, in dem vorgegebenen Sinne solche Kurztexte zu erstellen. Auch die bereits
vorhandenen Texte werden weiter \"uberarbeitet und an die parallel stattfindende
Vorlesung angepasst. 

%{\small
\vspace{0.3cm}

\noindent
Freiburg, Fr\"uhjahr 2023\\
Thomas~Filk
%}
%\end{document}
