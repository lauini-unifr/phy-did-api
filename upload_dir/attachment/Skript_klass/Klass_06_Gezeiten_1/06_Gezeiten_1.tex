\documentclass[german,10pt]{book}      
\usepackage{makeidx}
\usepackage{babel}            % Sprachunterstuetzung
\usepackage{amsmath}          % AMS "Grundpaket"
\usepackage{amssymb,amsfonts,amsthm,amscd} 
\usepackage{mathrsfs}
\usepackage{rotating}
\usepackage{sidecap}
\usepackage{graphicx}
\usepackage{color}
\usepackage{fancybox}
\usepackage{tikz}
\usetikzlibrary{arrows,snakes,backgrounds}
\usepackage{hyperref}
\hypersetup{colorlinks=true,
                    linkcolor=blue,
                    filecolor=magenta,
                    urlcolor=cyan,
                    pdftitle={Overleaf Example},
                    pdfpagemode=FullScreen,}
%\newcommand{\hyperref}[1]{\ref{#1}}
%
\definecolor{Gray}{gray}{0.80}
\DeclareMathSymbol{,}{\mathord}{letters}{"3B}
%
\newcounter{num}
\renewcommand{\thenum}{\arabic{num}}
\newenvironment{anmerkungen}
   {\begin{list}{(\thenum)}{%
   \usecounter{num}%
   \leftmargin0pt
   \itemindent5pt
   \topsep0pt
   \labelwidth0pt}%
   }{\end{list}}
%
\renewcommand{\arraystretch}{1.15}                % in Formeln und Tabellen   
\renewcommand{\baselinestretch}{1.15}                 % 1.15 facher
                                                      % Zeilenabst.
\newcommand{\Anmerkung}[1]{{\begin{footnotesize}#1 \end{footnotesize}}\\[0.2cm]}
\newcommand{\comment}[1]{}
\setlength{\parindent}{0em}           % Nicht einruecken am Anfang der Zeile 

\setlength{\textwidth}{15.4cm}
\setlength{\textheight}{23.0cm}
\setlength{\oddsidemargin}{1.0mm} 
\setlength{\evensidemargin}{-6.5mm}
\setlength{\topmargin}{-10mm} 
\setlength{\headheight}{0mm}
\newcommand{\identity}{{\bf 1}}
%
\newcommand{\vs}{\vspace{0.3cm}}
\newcommand{\noi}{\noindent}
\newcommand{\leer}{}

\newcommand{\engl}[1]{[\textit{#1}]}
\parindent 1.2cm
\sloppy

    \begin{document}  \setcounter{chapter}{5}


\chapter{Die Gezeiten - Ebbe und Flut}
\index{Gezeiten}% Kap 6
\label{chap_Gezeiten}

Die Gezeiten - Ebbe und Flut - sind jedem bekannt, der mal an einer Ozeank\"uste
war. Die Erscheinungen k\"onnen jedoch sehr unterschiedlich sein: Meist erlebt man
zweimal an einem Tag Flut und zweimal Ebbe, es gibt jedoch auch K\"usten, an denen
je nach Jahreszeit nur einmal am Tag Ebbe und Flut auftreten. Die H\"ohenunterschiede
- der Tidenhub -\index{Tidenhub} 
k\"onnen zwischen \glqq kaum sp\"urbar\grqq\ bis hin zu deutlich \"uber 10 Metern schwanken.  
Der vermutlich h\"ochste Tidenhub ist in der Bay of Fundy in Kanada.\index{Bay of Fundy} 
Dort wurden schon \"uber 20 Meter gemessen. 

Schon im Altertum war den Seefahrern bekannt, dass Ebbe und Flut irgendwie mit
dem Stand von Mond und Sonne zu tun haben. Sowohl das deutsche Wort \glqq Gezeiten\grqq\
als auch der Ausdruck \glqq Tiden\grqq\ (niederdeutsch f\"ur \glqq Zeiten\grqq), 
der besonders in Norddeutschland \"ublich ist,
deuten den engen Zusammenhang zur \glqq Zeit\grqq\ an, der immer schon mit dem Stand
der Gestirne in Verbindung gebracht wurde. Sonne und Mond sind f\"ur die Gezeiten 
verantwortlich, wobei - wie wir noch sehen werden - der Einfluss des Monds ungef\"ahr 
doppelt so gro\ss\ ist wie der Einfluss der Sonne.

Versucht man die Einzelheiten zu verstehen, erkennt man bald, dass die Gezeiten ein
sehr komplexes Ph\"anomen darstellen. Hier kann nur ein elementarer Einblick gegeben werden.
Ausf\"uhrlichere Informationen findet man z.B.\ in den Referenzen \cite{Hicks,Kowalik,Parker}.

In Tabelle \ref{tab_Tide} sind die wichtigsten Gr\"o\ss en zusammengefasst, die in diesem
Kapitel ben\"otigt werden.

\begin{table}[htb]
\begin{tabular}{r|l}
Gravitationskonstante & $G= 6,67 \cdot 10^{-11}\,{\rm \frac{m^3}{kg\cdot s^2}}$  \\  
Masse der Erde & $M_{\rm Erde} = 5,97\cdot 10^{24}$\,kg  \\ 
Masse des Monds &  $ M_{\rm Mond} =7,35 \cdot 10^{22}$\,kg  \\
Masse der Sonne &  $ M_\odot = 2\cdot 10^{30}$\,kg  \\
Abstand Erde-Mond &  $R_{EM} = 380\,000$\,km  \\[-0.2cm] 
 & (zwischen $363\,000$ und $405\,500$\,km) \\
Abstand Erde-Sonne &  $R_{ES} = 150\,000\,000$\,km   \\
Erdradius &   $R_{\rm Erde} =  6\,375$\, km \\
Neigung der Erdachse zur Ekliptik&   $\alpha = 23,44^\circ $  \\
\end{tabular}
\caption{\label{tab_Tide}%
Die wichtigsten physikalischen Gr\"o\ss en, die im Zusammenhang mit den Gezeiten
auftreten. Es handelt sich um ungef\"ahre bzw.\ gemittelte Angaben, die f\"ur eine grobe Absch\"atzung
der Gezeitenkr\"afte ausreichen.}
\end{table}
\index{Erdmasse}\index{Mondmasse}\index{Abstand!Erde-Mond}\index{Abstand!Erde-Sonne}\index{Erdradius}%
\index{Neigung der Erdachse}\index{Gravitationskonstante}

Anmerkung: Ich werde in diesem Kapitel oft von Fliehkr\"aften sprechen,\index{Fliehkraft} 
obwohl es sich dabei f\"ur
viele nicht um wirkliche Kr\"afte handelt. Andererseits ist das Konzept der Kraft ohnehin ein
Hilfskonstrukt, dessen \glqq Wirklichkeit\grqq, insbesondere im Zusammenhang mit der Gravitation,
durchaus in Frage gestellt werden kann. Wer den Begriff Fliehkraft vermeiden m\"ochte, kann dies
immer durch \glqq Richtungs\"anderung des Impulses\grqq\ ersetzen. 


\section{Gezeitenkr\"afte}

Ausgangspunkt der Erkl\"arungen sind immer die\index{Gezeitenkraft} 
Gezeitenkr\"afte (engl.\ \textit{tidal forces}) 
des Monds bzw.\ der Sonne. Gezeitenkr\"afte sind\index{Differenzielle Kraft} 
sogenannte \glqq differenzielle Kr\"afte\grqq,
d.h., sie geben die Differenz eines Kraftfelds bzw.\ die Differenz der Kr\"afte zwischen zwei Punkten an. 
Betrachten wir zun\"achst die gew\"ohnliche Schwerkraft.\index{Schwerkraft}\index{Gravitationskraft}

Die Schwerkraft $F$ eines Objekts der Masse $M$
auf einen Gegenstand der Masse $m$ im Abstand $R$ ist
\begin{equation}
                         F = G \frac{M m}{R^2} \, .
\end{equation} 
Bildet man $F/m$ erh\"alt man eine Beschleunigung. Dieser Wert ist unabh\"angig von der Masse $m$
des \glqq Probek\"orpers\grqq:
\begin{equation}
                         a = G \frac{M}{R^2} \, .
\end{equation} 
Setzt man die Werte f\"ur die Gravitationskonstante $G$, die Masse des Monds $M_{\rm Mond}$
und den mittleren Abstand $R_{EM}$ zwischen Erde und Mond ein, erh\"alt man:
\begin{equation}
                         a_M = 6,67\cdot 10^{-11} \frac{{\rm m}^3}{\rm kg \cdot s^2} 
                         \cdot \frac{7,35 \cdot 10^{22} \, {\rm kg}}{ 3,8^2 \cdot 10^{16}\, {\rm m}^2} 
                         \approx  3,4 \cdot 10^{-5} \, \frac{\rm m}{\rm s^2} \, ,
\end{equation}
wobei dieser Wert aufgrund der elliptischen Form der Mondbahn und dem damit verbundenen variierenden
Abstand zwischen Erde und Mond zwischen $2,98 \cdot 10^{-5}\, \frac{\rm m}{\rm s^2}$ und 
$3,71 \cdot 10^{-5}\, \frac{\rm m}{\rm s^2}$ schwanken kann.

Entsprechend erhalten wir f\"ur den Einfluss Sonne: 
\begin{equation}
                         a_\odot = 6,67\cdot 10^{-11} \frac{{\rm m}^3}{\rm kg \cdot s^2} 
                         \cdot \frac{2 \cdot 10^{30} \, {\rm kg}}{ 1,5^2 \cdot 10^{22}\, {\rm m}^2} 
                         \approx 5,9 \cdot 10^{-3}  \, \frac{\rm m}{\rm s^2}  \, .
\end{equation} 
Der gravitative Einfluss der Sonne auf die Erde bzw.\ auf Gegenst\"ande auf der Erde ist also 
\"uber 170-mal gr\"o\ss er als der Einfluss des Monds. Der f\"uhrende, konstante Teil 
dieser Kraft wirkt jedoch auf alle
Gegenst\"ande auf der Erde gleicherma\ss en, d.h., wir sp\"uren ihn nicht, da alle Gegenst\"ande
derselben Beschleunigung unterliegen und somit keine relativen Verschiebungen auftreten. 
Wir w\"urden ihn sp\"uren, wenn die Erde (durch was auch immer f\"ur einen \"uberirdischen 
Mechanismus) in ihrem Zentrum an einem Punkt im Raum
\glqq festgehalten\grqq\ w\"urde. Alle Gegenst\"ande (insbesondere auch alle Wassermassen)
w\"urden in diesem Fall mit der obigen Beschleunigung zur Sonne hingezogen. 

F\"ur die Gezeiten sind jedoch die Gezeitenkr\"afte verantwortlich, d.h.\ die Unterschiede in
der Schwerkraft des Monds (bzw.\ der Sonne) auf Gegenst\"ande, die sich an
verschiedenen Orten auf der Erde befinden. Der Unterschied zwischen der Gravitationsbeschleunigung des
Monds auf den Schwerpunkt der Erde und einen Punkt an der Erdoberfl\"ache, der dem Mond
zugewandt ist, betr\"agt:\hyperref[secA]{(Herleitung)}
\begin{equation}
\label{eq_Delta_a}
                         \Delta a = G \frac{M_{\rm Mond}}{(R_{EM}-R_{\rm Erde})^2}  - 
                          G \frac{M_{\rm Mond}}{R_{EM}^2}  \approx 
                          G \frac{M_{\rm Mond}}{R_{EM}^3} 2 R_{\rm Erde}    \, .
\end{equation} 
Im letzten Schritt wurde nur der f\"uhrende Term in $R_{\rm Erde}/R_{EM} \approx 1/60$ genommen,
entsprechend kleiner sind die Korrekturen. Setzt man Zahlen f\"ur das Erde-Mond-System ein,
erh\"alt man f\"ur diese differenzielle\index{Gezeitenkraft!Mond} 
Beschleunigung:
\begin{equation}
        \Delta a_M \approx 1,14 \cdot 10^{-6}\, \frac{\rm m}{\rm s^2} \, ,
\end{equation}     
wobei auch hier der Wert wieder zwischen $0,94 \cdot 10^{-6}\, \frac{\rm m}{\rm s^2}$ und
$1,3\cdot 10^{-6}\, \frac{\rm m}{\rm s^2}$ schwanken kann. 

Wegen des wesentlich gr\"o\ss eren Abstands zwischen Erde und Sonne und weil dieser
Abstand kubisch eingeht, ist diese Beschleunigung nun f\"ur die Sonne kleiner:\index{Gezeitenkraft!Sonne} 
\begin{equation}
        \Delta a_\odot \approx 5\cdot 10^{-7}\, \frac{\rm m}{\rm s^2} \, .
\end{equation}  
W\"ahrend also die absolute Schwerkraft der Sonne auf die Erde rund 170 mal gr\"o\ss er ist als die des Monds,
ist die Gezeitenkraft an der Oberfl\"ache der Erde rund 2,3 (schwankend zwischen 1,9 und 2,6) mal
schw\"acher als die des Monds. Wie schon erw\"ahnt, sp\"uren wir die absolute Schwerkraft der
Sonne und des Monds nicht, da sich die Erde auf ihrer Bahn \glqq im freien Fall\grqq\ befindet. 
Die differenzielle Schwerkraft, also die Gezeitenkraft, ist jedoch wahrnehmbar,
da der Schwerpunkt der Erde, und damit der Punkt im freien Fall, einer anderen Beschleunigung
unterliegt als die Punkte an der Erdoberfl\"ache. Die Punkte auf der dem Mond abgewandten    
Seite der Erde sp\"uren eine entsprechend geringere Schwerebeschleunigung im Vergleich zum
Mittelwert. 

\section{Der dem Mond abgewandte Gezeitenberg}

Der dem Mond zugewandte Wasserberg der Gezeiten wird durch die h\"ohere Gravitationskraft 
des Monds auf diese Wassermassen im Vergleich zum Erdschwerpunkt erkl\"art.
F\"ur den Wasserberg auf der dem Mond abgewandten Seite findet man zwei zun\"achst scheinbar 
verschiedene Erkl\"arungen, nach denen einmal die h\"ohere Fliehkraft an dieser Seite der Erde f\"ur den
Wasserberg verantwortlich ist und einmal die schw\"achere Gravitationskraft. Beide Erkl\"arungen sind
richtig, allerdings muss man hier vorsichtig sein, keine Fehlvorstellungen zu generieren.
Wir berechnen zun\"achst die Fliehkr\"afte an beliebigen Punkten der Erde und 
beschr\"anken uns dabei auf das Erde-Mond-System, das f\"ur die Gezeiten den gr\"o\ss ten
Einfluss hat. Die Effekte der Sonne lassen sich ebenso erkl\"aren und \"uberlagern sich den
Einfl\"ussen des Monds.

\subsection{Der Einfluss der Fliehkraft}

Wir werden sehen, dass sich die Fliehkraft an jedem Punkt der Erde in zwei Anteile aufspalten
l\"asst: Ein Anteil ist von einer Achse durch das Zentrum der Erde radial nach au\ss en gerichtet, der
zweite Anteil ist \"uberall auf der Erde (und auch in ihrem Inneren) derselbe und bezieht sich
auf die Bewegung des Erdzentrums um den Schwerpunkt des Erde-Mond-Systems. Bildet man die
vektorielle Summe dieses zweiten Anteils der Fliehkraft und der Gravitationskr\"afte des Monds, bleiben
gerade die Gezeitenkr\"afte mit einer Wirkung nach au\ss en \"ubrig. Diese erzeugen die Gezeiten.

\subsubsection{Der Schwerpunkt des Erde-Mond-Systems}

Erde und Mond drehen sich um eine Achse, die durch den gemeinsamen Schwerpunkt $D$
(Abb.\ \ref{fig_Schwerpunkt})
verl\"auft und senkrecht auf der Erde-Mond-Umlaufbahn steht. Der Schwerpunkt berechnet sich aus
der Bedingung\index{Schwerpunkt!Erde-Mond-System}
\begin{equation}
\label{eq_Schwerpunkt}
             r_1 M_1 = r_2 M_2  \hspace{1cm} {\rm oder} \hspace{1cm}  
                 r_1  = \frac{M_2}{M_1+M_2} R  \hspace{1cm} {\rm mit} ~~ R=r_1+r_2 \, .
\end{equation}
Setzen wir f\"ur $M_2$ die Masse des Monds, f\"ur $M_1$ die Masse der Erde und f\"ur
$R=r_1+r_2$ den Abstand Erde-Mond ein, erhalten wir f\"ur $r_1$ - den Abstand vom Erdmittelpunkt $Z$
zum Schwerpunkt $D$ des Erde-Mond-Systems, $r_1 = R_{ZD} \approx 4\,620$\,km. Das ist etwas weniger als
3/4-tel des Erdradius. Der Schwerpunkt des Erde-Mond-Systems liegt also innerhalb der Erde.   

\begin{figure}[htb]
\begin{tikzpicture}
\draw[thick] (2,1.5) circle (1.2);
\draw[thick] (14,1.5) circle (0.6);
\draw (2,1.5) -- (14,1.5);
\filldraw[black] (6,1.5) circle (0.1);
\filldraw[black] (2,1.5) circle (0.05);
\filldraw[black] (14,1.5) circle (0.05);
\draw (6,1.2) node {$D$};
\draw (2,1.8) node {$Z$};
\draw (2,1.0) node {$M_1$};
\draw (14,1.2) node {$M_2$};
\draw (4,1.2) node {$r_1$};
\draw (10,1.2) node {$r_2$};
\end{tikzpicture}
%
\caption{\label{fig_Schwerpunkt}%
Zwei Massen $M_1$ und $M_2$ drehen sich um einen gemeinsamen Schwerpunkt $D$. Die
Abst\"ande $r_1$ und $r_2$ ergeben sich aus dem Hebelgesetz. $Z$ ist der Mittelpunkt der
einen Masse (Erde).}
\end{figure}

F\"ur das System Erde-Sonne liegt dieser Schwerpunkt\index{Schwerpunkt!Erde-Sonne-System} 
rund 450\,km vom
Zentrum der Sonne entfernt, also tief im Inneren der Sonne. Auch wenn sich die Situation f\"ur
das Erde-Mond-System in dieser Hinsicht vollkommen vom Erde-Sonne-System unterscheidet, bleibt
die Argumentation f\"ur die Gezeiten im Wesentlichen die Gleiche. Diese Argumentation h\"angt nicht
von der genauen Lage des gemeinsamen Schwerpunkts ab.  

\subsubsection{Radiale Fliehkr\"afte}

Wir stellen uns nun das Erde-Mond-System als einen starren K\"orper vor, bei dem sich Erde
und Mond um eine feste Achse durch den gemeinsamen Schwerpunkt $D$ drehen und sich dabei
immer dieselbe Seite zuwenden. F\"ur den Mond ist das richtig und wir werden in
Abschnitt \ref{sec_EMWW} auch eine Begr\"undung daf\"ur finden, f\"ur die Erde gilt dies jedoch
nicht: Sie dreht sich
zus\"atzlich noch um eine Achse durch ihren Mittelpunkt $Z$. Drehungen der Erde um eine Achse
durch ihren Mittelpunkt haben aber (unter den hier angenommenen idealisierten Bedingungen einer
kugelf\"ormigen Erde) keinen Einfluss auf die Gezeiten, da ihr Effekt - die Fliehkraft zu dieser Drehung - 
in radialer Richtung von der Drehachse durch $Z$ nach au\ss en zeigt und im selben Abstand von der
Drehachse auch denselben Wert hat. Diese Kr\"afte f\"uhren zu einer Abplattung der Erde, die dadurch am 
\"Aquator etwas dicker ist als entlang von Gro\ss kreisen durch die Pole.\index{Erde!Form} 

\subsubsection{Die Fliehkr\"afte auf der Erde}

Der Schwerpunkt des Erde-Mond-Systems sei also ein fester Punkt der Erde, den wir mit
$D$ bezeichnen; er markiert eine Drehachse durch diesen Punkt (siehe Abbildung \ref{fig_Balance1}).   
Allgemein ist die Fliehkraft auf einen Gegenstand der Masse $m$ an einem Punkt $C$ durch
\begin{equation}
                 \vec{F} = m \omega^2 \vec{R}_{DC} 
\end{equation}
gegeben, wobei $\vec{R}_{DC}$ der Verbindungsvektor von der Drehachse $D$ zum Punkt $C$ ist und
$\omega$ die Winkelfrequenz der Drehung bezeichnet (sie entspricht einer Umlaufzeit des
Erde-Mond-Systems um den gemeinsamen Schwerpunkt).
Auch hier bietet es sich an, die Beschleunigung $\vec{a}=\omega^2\vec{R}_{DC}$
aufgrund dieser Kraft zu betrachten. Da die Winkelfrequenz f\"ur alle 
punktf\"ormigen Objekte auf der Erde dieselbe
ist, spielt nur der Abstandsvektor $\vec{R}_{DC}$ vom Drehzentrum $D$ zum Punkt $C$ eine Rolle. 
Auf der dem Mond zugewandten Seite der Erde (Punkt $B$) ist dieser Abstand sehr klein, 
$R_{DB} \approx 1\,755$\,km, im Vergleich zur abgewandten Seite (Punkt $A$), 
$R_{DA}\approx 11\,000$\,km. Im Zentrum der Erde heben sich die
Fliehkraft und die Anziehungskraft des Monds gerade auf. Oft hei\ss t es nun, dass sich auf der dem 
Mond zugewandten Seite die gr\"o\ss ere Gravitationskraft des Monds und die kleinere Fliehkraft addieren, 
w\"ahrend auf der abgewandten Seite die Fliehkraft gr\"o\ss er sei, sodass eine Nettokraft \"ubrig bliebe, 
selbst wenn man die kleinere Gravitationskraft abzieht. Diese Erkl\"arung f\"ur die beiden
Wasserberge ist so nicht ganz richtig, da die f\"ur die Gezeiten relevanten Fliehkr\"afte an allen Punkten
der Erde gleich sind, wie die folgende \"Uberlegung zeigt. 

\begin{figure}[htb]
\setlength{\unitlength}{0.8pt}
\begin{picture}(400,205)(-50,100)
\put(120,200){\makebox(0,0){$\bullet$}}    %  Z
\put(195,200){\makebox(0,0){$\bullet$}}    %  D
\put(165,290){\makebox(0,0){$\bullet$}}    %  C
\put(20,200){\makebox(0,0){$\bullet$}}      %  A
\put(220,200){\makebox(0,0){$\bullet$}}    %  B
%
\put(120,192){\makebox(0,0){{\footnotesize $Z$}}}
\put(195,192){\makebox(0,0){{\footnotesize $D$}}}
\put(163,298){\makebox(0,0){{\footnotesize $C$}}}
\put(12,200){\makebox(0,0){{\footnotesize $A$}}}
\put(228,200){\makebox(0,0){{\footnotesize $B$}}}
\put(170,192){\makebox(0,0){{\footnotesize $\vec{R}_{ZD}$}}}
\put(123,240){\makebox(0,0){{\footnotesize $\vec{R}_{ZC}$}}}
\put(198,240){\makebox(0,0){{\footnotesize $\vec{R}_{DC}$}}}
\put(135,208){\makebox(0,0){{\footnotesize $\vec{e}_1$}}}
\qbezier(20,200)(20.7,241.1)(49.3,270.7)
\qbezier(49.3,270.7)(78.8,299.5)(120,300)
\qbezier(120,300)(161.2,299.5)(190.7,270.7)
\qbezier(190.7,270.7)(219.3,241.1)(220,200)
\qbezier(220,200)(219.3,158.9)(190.7,129.3)
\qbezier(190.7,129.3)(161.2,100.5)(120,100)
\qbezier(120,100)(78.8,100.5)(49.3,129.3)
\qbezier(49.3,129.3)(20.7,158.9)(20,200)
%
\put(300,170){\vector(1,0){100}}
\put(300,200){\vector(1,0){100}}
\put(300,230){\vector(1,0){100}}
\put(350,210){\makebox(0,0){Mond}}
\thicklines
\put(120,200){\vector(1,0){30}}
\put(120,200){\vector(1,0){75}}
\put(120,200){\vector(1,2){44}}
\put(195,200){\vector(-1,3){29.5}}
\put(120,200){\line(1,0){75}}
\end{picture}
\caption{\label{fig_Balance1}%
Die Fliehkraft bzw.\ -beschleunigung auf einen allgemeinen Punkt $C$. 
Der Schwerpunkt des Erde-Mond-Systems und damit der Mittelpunkt der Erde-Mond-Umlaufbahn ist 
der Drehpunkt $D$. $Z$ bezeichnet den Mittelpunkt der Erde. 
Die Ansicht ist \glqq von oben\grqq,
d.h., bei $Z$ und $D$ handelt es sich eigentlich um Drehachsen.}
\end{figure}

Dazu berechnen wir die Fliehbeschleunigung auf einen beliebigen Punkt $C$ (er muss nicht an der
Erdoberfl\"ache liegen). Diese Fliehbeschleunigung ist durch
\begin{equation}
                   \vec{a} = \omega^2 \vec{R}_{DC} = \omega^2 (\vec{R}_{ZC} - \vec{R}_{ZD})  
\end{equation}
gegeben (siehe Abb.\ \ref{fig_Balance1}).  

Wie man sieht, kann man diese\index{Fliehkraft} 
Fliehbeschleunigung in zwei Anteile aufteilen: Ein Anteil
zeigt vom Erdmittelpunkt $Z$ radial nach au\ss en (Richtung $\vec{R}_{ZC}$) - dieser Anteil addiert 
sich zu der t\"aglichen Drehung der Erde um ihre Achse und tr\"agt nicht zur Gezeitenwirkung 
bei.%
\footnote{Hier muss man eigentlich etwas vorsichtiger sein: Die t\"agliche Drehung der Erde erfolgt
um ihre Rotationsachse, die relativ zur Ekliptik um 23,4 Grad geneigt ist. Die Drehung der Erde, von
der hier die Rede ist, erfolgt einmal im Monat um eine Achse durch das Erdzentrum, die parallel zur
Drehachse des Erde-Mond-Systems durch ihren gemeinsamen Schwerpunkt $D$ ist. Diese beiden Achsen sind 
nicht identisch, auch wenn sie beide durch den Erdschwerpunkt verlaufen. Beide Drehungen haben
keinen Einfluss auf die Gezeiten.} %
Der zweite Anteil ist unabh\"angig vom Punkt $C$, also f\"ur alle Punkte der Erde
derselbe. Er ist immer parallel zur Verbindungslinie vom gemeinsamen Schwerpunkt $D$ in
die dem Mond abgewandte Richtung, d.h.\ in die Richtung der Achse durch den Erdmittelpunkt $Z$. 
Dieser zweite Anteil ist gleich der Gravitationskraft auf den Mittelpunkt $Z$ der Erde. F\"ur die
Gravitationsbeschleunigung bedeutet das:
\begin{equation}
\label{eq_aZ}
              \vec{a}_Z = G \frac{M_{\rm Mond}}{R_{EM}^2} \vec{e}_1 - \omega^2  \vec{R}_{ZD} = 0  \hspace{1cm} {\rm oder}
                  \hspace{1cm}    G \frac{M_{\rm Mond}}{R_{EM}^2} \vec{e}_1   = \omega^2  \vec{R}_{ZD}  \, .
\end{equation}
Hierbei ist $\vec{e}_1$ ein Einheitsvektor von der zentralen Drehachse durch den Erdmittelpunkt $Z$
in Richtung des Monds. 

\begin{figure}[htb]
\setlength{\unitlength}{0.6pt}
\begin{picture}(420,350)(-60,50)
\thicklines
\put(120,200){\makebox(0,0){$\bullet$}}
\put(128,190){\makebox(0,0){{\footnotesize $Z$}}}
\put(205,200){\makebox(0,0){$\bullet$}}
\put(213,190){\makebox(0,0){{\footnotesize $D$}}}
\qbezier(20,200)(20.7,241.1)(49.3,270.7)
\qbezier(49.3,270.7)(78.8,299.5)(120,300)
\qbezier(120,300)(161.2,299.5)(190.7,270.7)
\qbezier(190.7,270.7)(219.3,241.1)(220,200)
\qbezier(220,200)(219.3,158.9)(190.7,129.3)
\qbezier(190.7,129.3)(161.2,100.5)(120,100)
\qbezier(120,100)(78.8,100.5)(49.3,129.3)
\qbezier(49.3,129.3)(20.7,158.9)(20,200)
%
\thinlines
\put(20,200){\vector(-1,0){60}}
\put(50,270){\vector(-1,1){44}}
\put(220,200){\vector(1,0){60}}
\put(190,270){\vector(1,1){44}}
\put(120,300){\vector(0,1){60}}
\put(50,130){\vector(-1,-1){44}}
\put(120,100){\vector(0,-1){60}}
\put(190,130){\vector(1,-1){44}}
%
\put(20,200){\vector(-1,0){45}}
\put(50,270){\vector(-1,0){45}}
\put(220,200){\vector(-1,0){45}}
\put(190,270){\vector(-1,0){45}}
\put(120,300){\vector(-1,0){45}}
\put(50,130){\vector(-1,0){45}}
\put(120,100){\vector(-1,0){45}}
\put(190,130){\vector(-1,0){45}}
\put(120,200){\vector(-1,0){45}}
\end{picture}
%
\begin{picture}(200,350)(0,50)
\thicklines
\put(120,200){\makebox(0,0){$\bullet$}}
\put(128,190){\makebox(0,0){{\footnotesize $Z$}}}
\put(205,200){\makebox(0,0){$\bullet$}}
\put(213,190){\makebox(0,0){{\footnotesize $D$}}}
\qbezier(20,200)(20.7,241.1)(49.3,270.7)
\qbezier(49.3,270.7)(78.8,299.5)(120,300)
\qbezier(120,300)(161.2,299.5)(190.7,270.7)
\qbezier(190.7,270.7)(219.3,241.1)(220,200)
\qbezier(220,200)(219.3,158.9)(190.7,129.3)
\qbezier(190.7,129.3)(161.2,100.5)(120,100)
\qbezier(120,100)(78.8,100.5)(49.3,129.3)
\qbezier(49.3,129.3)(20.7,158.9)(20,200)
%
\thinlines
%\put(20,200){\vector(-1,0){40}}
%\put(50,270){\vector(-1,1){29}}
%\put(220,200){\vector(1,0){40}}
%\put(190,270){\vector(1,1){29}}
%\put(120,300){\vector(0,1){40}}
%\put(50,130){\vector(-1,-1){29}}
%\put(120,100){\vector(0,-1){40}}
%\put(190,130){\vector(1,-1){29}}
%
\put(20,200){\vector(-1,0){20}}
\put(50,270){\vector(-1,0){15}}
\put(220,200){\vector(1,0){20}}
\put(190,270){\vector(1,0){15}}
%\put(120,300){\vector(-1,0){45}}
\put(50,130){\vector(-1,0){15}}
%\put(120,100){\vector(-1,0){45}}
\put(190,130){\vector(1,0){15}}
%\put(120,200){\vector(-1,0){45}}
\end{picture}
\caption{\label{fig_Balance2}%
(links) Die Fliehkr\"afte bzw.\ -beschleunigungen lassen sich an jedem Punkt in zwei
Anteile aufspalten: ein Anteil, der radial nach au\ss en zeigt und proportional zum Abstand vom
Erdmittelpunkt ist - dieser Anteil tr\"agt nicht zu den Gezeiten bei. Ein zweiter Anteil, der an jedem
Punkt der Erde derselbe ist und gleich der Fliehkraft auf das Zentrum $Z$ der Erde.
(rechts) L\"asst man den radialen Anteil der Fliehkr\"afte weg - er tr\"agt nicht zu den Gezeiten bei - 
und addiert man zu dem konstanten vom Mond weggerichteten Teil der Fliehkaft die
Gravitationskraft des Monds, heben sich die Fliehkraft und der zentrale Teil der Gravitationskraft
weg. Es bleiben nur die Gezeitenanteile der Gravitation. Diese sind f\"ur Ebbe und Flut auf der
Erde verantwortlich.}
\end{figure}

Bilden wir nun die Summe der beiden Beschleunigungen und nutzen dabei Gl.~\ref{eq_aZ},
erhalten wir f\"ur den Punkt $C$:
\begin{equation}
         \vec{a}_C =   2 G \frac{M_{\rm Mond}}{R_{EM}^3}  (\vec{e}_1 \cdot \vec{R}_{ZC}) \vec{e}_1 \, .
\end{equation} 
Anmerkungen:
\begin{enumerate}
\item
Wir haben es bei den obigen Betrachtungen mit drei verschiedenen Drehachsen zu tun:
(1) die Drehachse der Erde durch ihren Mittelpunkt $Z$ - sie ist um etwas \"uber 23 Grad zur
Eklipik geneigt; (2) die Drehachse des Erde-Mond-Systems durch den Schwerpunkt $D$, wegen
der Neigung der Mondumlaufbahn\index{Neigung der Mondumlaufbahn} 
relativ zur Ekliptik von rund 5 Grad schwankt diese Neigung
relativ zur Drehachse der Erde zwischen 18 und 28 Grad; (3) eine Achse parallel
zur Drehachse Erde-Mond durch das Zentrum $Z$ der Erde, um diese Achse dreht sich die
Erde einmal monatlich bei einem Umlauf des Erde-Mond-Systems relativ zum Fixsternhimmel.
\item  
Wir hatten schon mehrfach erw\"ahnt, dass die Fliehkr\"afte zu Drehungen um Achsen durch
den Erdmittelpunkt nicht zur Gezeitenwirkung beitragen, da diese Fliehkr\"afte radial von der
Drehachse weg nach au\ss en wirken und ihr Betrag nur vom Abstand von der Drehachse
abh\"angt. Diese Kr\"afte sind also symmetrisch zur Drehachse.
Durch die t\"agliche Drehung der Erde um ihre Achse wirkt am
\"Aquator eine Beschleunigung von $a=R_{\rm Erde} \omega^2$, wobei $\omega$ einer Umdrehung
am Tag entspricht, also
$\omega=(2\pi)/(24\cdot 60\cdot 60)\,{\rm s}^{-1}$. Diese Beschleunigung betr\"agt rund
$a=0,0337\,\frac{\rm m}{{\rm s}^2}$, ist also um ein Vielfaches gr\"o\ss er als die Gezeitenkr\"afte. 
Diese Beschleunigung tr\"agt zu einer Abplattung der Erde bei: Der Umfang der Erde am 
\"Aquator ist gr\"o\ss er als entlang der Pole.   
\end{enumerate}

\subsection{Die Gezeitenkr\"afte}

Eine zweite Erkl\"arung der Gezeiten betont einen anderen Gesichtspunkt, ist aber letztendlich
\"aquivalent zu der Erkl\"arung im letzten Abschnitt. 

Wir stellen uns statt der Erde drei Objekte im Abstand von einem punktf\"ormig angenommenen 
Massezentrum (z.B.\ dem Mond) vor. Der Einfachheit wegen sei dieses anziehende Massezentrum weit 
von diesen drei Objekten entfernt. Das mittlere der drei Objekte entspreche der Erde; zwei weitere 
Objekte - eines dem Mond zugewandt,
das andere dem Mond abgewandt - entspreche Wassermassen auf der dem Mond zugewandten
bzw.\ abgewandten Seite der Erde (siehe Abb.\ \ref{fig_Erkl2}). 

\begin{figure}[htb]
\begin{picture}(430,80)(0,0)
\put(50,30){\circle{40}}
\put(77,30){\circle*{10}}
\put(23,30){\circle*{10}}
\put(330,20){\vector(1,0){100}}
\put(330,40){\vector(1,0){100}}
\put(380,30){\makebox(0,0){Mond}}
\thicklines
\put(23,55){\vector(1,0){10}}
\put(50,55){\vector(1,0){15}}
\put(77,55){\vector(1,0){20}}
\end{picture}\\
\begin{picture}(300,60)(0,0)
\put(160,30){\circle{40}}
\put(197,30){\circle*{10}}
\put(123,30){\circle*{10}}
\put(330,20){\vector(1,0){100}}
\put(330,40){\vector(1,0){100}}
\put(380,30){\makebox(0,0){Mond}}
\thicklines
\put(123,55){\vector(1,0){25}}
\put(160,55){\vector(1,0){35}}
\put(197,55){\vector(1,0){45}}
\end{picture}\\
%
\begin{picture}(300,60)(0,0)
\put(270,30){\circle{40}}
\put(320,30){\circle*{10}}
\put(220,30){\circle*{10}}
\put(330,20){\vector(1,0){100}}
\put(330,40){\vector(1,0){100}}
\put(380,30){\makebox(0,0){Mond}}
\end{picture}

\caption{\label{fig_Erkl2}%
Die Gezeitenkr\"afte ziehen einen Gegenstand auseinander, sofern er nicht
durch andere Kr\"afte zusammengehalten wird.}
\end{figure}


Auf alle drei Objekte wirkt in erster N\"aherung die Schwerkraft des Massezentrums. Bei dem vorderen
Objekt (in Richtung des Massezentrums) kommt die Gezeitenkraft hinzu, da es n\"aher am Massezentrum
liegt, bei dem hinteren Objekt ist die Gesamtkraft um die Gezeitenkraft geringer, da es weiter vom
Massezentrum entfernt ist. W\"urden diese drei Objekte im freien Fall auf das Massezentrum zufallen,
w\"urde sich der Abstand zwischen ihnen vergr\"o\ss ern: Das vordere Objekt f\"allt schneller, das hintere
langsamer als das mittlere Objekt (die Erde als Ganzes). 

\begin{figure}[htb]
\begin{picture}(400,100)(-40,0)
\put(20,100){\line(1,0){360}}
\put(20,10){\line(4,1){360}}
\put(20,100){\vector(0,-1){90}}
\put(20,10){\vector(4,1){15}}
\put(185,100){\vector(1,0){15}}
\put(187,52){\vector(4,1){15}}
\qbezier(20,100)(20,52)(33,12.5)
\put(-8,50){\makebox(0,0){$\Delta \vec{s}_0=\vec{v}t$}}
\put(45,3){\makebox(0,0){$\Delta \vec{s}=\frac{1}{2}\vec{g}t^2$}}
\put(180,93){\makebox(0,0){$R$}}
\put(183,58){\makebox(0,0){$R$}}
\end{picture}
\caption{\label{fig_Fliehkraft}%
Eine Zentripetalkraft zieht einen Gegenstand von einer Geraden gerade so ab, dass die
Beschleunigung diesen Gegenstand auf einer Kreisbahn h\"alt.}
\end{figure}

Zu diesem freien Fall kommt nun eine Kreisbewegung hinzu, die gerade so ist, dass die Beschleunigung
des mittleren Objekts dieses auf der Kreisbahn h\"alt. Die Bedingung daf\"ur ist, dass die in der (infinitesimalen)
Zeitdauer $t$ zur\"uckgelegte tangentiale Strecke $\Delta \vec{s}_0=\vec{v} t$ abz\"uglich der Strecke 
aufgrund der Beschleunigung ($\Delta \vec{s} = \frac{1}{2} \vec{g} t^2$)
zum Zentrum der Zentripetalkraft gerade wieder dem Abstand $R$ von diesem Zentrum entspricht (siehe Abb.\
\ref{fig_Fliehkraft}). Nach dem Satz von Pythagoras gilt somit:
\begin{equation}
            R^2 + (vt)^2  = \left( R + \frac{1}{2}gt^2 \right)^2 = R^2 + R g t^2 + \frac{1}{4}g^2 t^4  \, .
\end{equation}
Vernachl\"assigen wir den Term $t^4$, da $t$ infinitesimal klein sein soll, folgt die Bedingung:
\begin{equation}
               (vt)^2  =  R g t^2   \hspace{1cm} {\rm oder} \hspace{1cm}  g = \frac{v^2}{R}  \, .
\end{equation}
F\"ur einen vollen Umlauf ben\"otige der K\"orper die Zeit $T$, in der die Strecke $U=2\pi R$ (der
Umfang der Kreisbahn) zur\"uckgelegt wird. Es ist also $vT=2\pi R$ oder $v=\omega R$ mit der
Umlauf(winkel)frequenz $\omega= 2\pi/T$. Damit ergibt sich schlie\ss lich die Bedingung, die schon mehrfach
verwendet wurde:
\begin{equation}
               g =  \omega^2 R  \, .
\end{equation}
Das dem Massezentrum zugewandte Objekt
bewegt sich auf einer kleineren Kreisbahn, d.h., bei ihm ist die Anziehung durch die Masse etwas
gr\"o\ss er als die Kreisbeschleunigung. Bei dem abgewandten Objekt ist es umgekehrt: Seine Kreisbahn
hat einen gr\"o\ss eren Radius, bei ihm ist somit die Anziehung durch das Massezentrum etwas
geringer als es seiner Kreisbeschleunigung entspricht. Es w\"urde also nach Au\ss en getrieben, wenn
es nicht durch andere Kr\"afte an die mittlere Masse gebunden w\"are. 

\section{Spring- und Nipptide}

Die Einfl\"usse\index{Springtide} 
von Sonne und Mond \"uberlagern sich, sodass die Gezeitenkr\"afte besonders
intensiv sind, wenn Sonne, Erde und Mond auf einer Linie liegen. Dabei ist es zun\"achst nicht
wichtig, ob sich Sonne und Mond von der Erde aus gegen\"uberliegen (also Vollmond ist), oder
ob Sonne und Mond von der Erde aus auf einer Seite sind (also bei Neumond). In beiden F\"allen
erh\"alt man eine sogenannte Springflut oder Springtide. Diese tritt rund zweimal in einem Monat
auf. 

Andererseits ist der Einfluss auf die Gezeiten besonders schwach, wenn Sonne und Mond von der
Erde aus betrachtet unter einem Winkel von 90 Grad erscheinen, d.h.\ bei zunehmendem oder
abnehmendem Halbmond. Man spricht in diesem Fall von einer Nipptide.\index{Nipptide} 

Die Intensit\"at einer Springflut kann durch verschiedene Faktoren noch verst\"arkt werden. Zum einen
handelt es sich um rein geometrische Faktoren im Sonnen- und Mondstand: Zum Beispiel, wenn
der Mond sich gerade in seiner Periapsis, also dem erdn\"achsten Punkt seiner elliptischen 
Umlaufbahn um die Erde (bzw.\ den gemeinsamen Schwerpunkt des Erde-Mond-Systems) befindet. 
Weitere wichtige Faktor sind allerdings Wetterbedingungen, z.B.\ wenn zeitgleich zur Springflut
auch ein landeinw\"artiger Sturm weht. 

\section{Die Neigung der Erdachse}

Da der Wasserstand gerade bei Hafeneinfahrten f\"ur die Schifffahrt von Bedeutung ist, wurde
dieser teilweise schon seit Jahrhunderten gemessen. Diese Aufzeichnungen sind nicht nur f\"ur
den Klimawandel (Anhebung des Meeresspiegels) von Bedeutung, sondern verdeutlichen
auch die Komplexit\"at der Gezeiten. 
In einer harmonischen Analyse (also einer Frequenzanalyse oder Fourier-Zerlegung) der Wasserst\"ande
kann man viele hundert Anteile erkennen. Neben dem idealisierten Einfluss von Sonne und Mond, die
nicht exakt dieselbe Frequenz haben - ein Sonnentag dauert 24 Stunden, ein Mondtag ist jedoch
um rund 50 Minuten l\"anger -, spielt auch die Elliptizit\"at der Mond- und Sonnenbahnen
eine wichtige Rolle. Au\ss erdem kommen ortsabh\"angige Str\"omungsverh\"altnisse hinzu. 

Ein wichtiger Faktor ist aber auch die Neigung der Erdachse
um 23,5 Grad relativ zur Ekliptik sowie die Neigung der Mondumlaufbahn relativ zur Ekliptik von
etwas \"uber 5 Grad. Dieser Einfluss kann an manchen Orten der Erde und zu manchen
Jahreszeiten dazu f\"uhren, dass nur eine Ebbe und eine Flut am Tag auftreten.   

\begin{figure}[htb]
\setlength{\unitlength}{0.8pt}
\begin{picture}(400,220)(-20,90)
\put(20,200){\makebox(0,0){$\bullet$}}
\put(180,120){\makebox(0,0){$\bullet$}}
\put(61,280){\makebox(0,0){$\bullet$}}
\put(220,200){\makebox(0,0){$\bullet$}}
%
\put(52,285){\makebox(0,0){$A$}}
\put(209,195){\makebox(0,0){$A'$}}
\put(30,206){\makebox(0,0){$B$}}
\put(188,115){\makebox(0,0){$B'$}}
%
\qbezier(20,200)(20.7,241.1)(49.3,270.7)
\qbezier(49.3,270.7)(78.8,299.5)(120,300)
\qbezier(120,300)(161.2,299.5)(190.7,270.7)
\qbezier(190.7,270.7)(219.3,241.1)(220,200)
\qbezier(220,200)(219.3,158.9)(190.7,129.3)
\qbezier(190.7,129.3)(161.2,100.5)(120,100)
\qbezier(120,100)(78.8,100.5)(49.3,129.3)
\qbezier(49.3,129.3)(20.7,158.9)(20,200)
%
\qbezier(40,140)(-40,200)(40,260)
\qbezier(200,140)(280,200)(200,260)
%
\put(70,100){\line(1,2){100}}
\put(31,244.5){\line(2,-1){178}}
\multiput(61,277)(4,-2){40}{$\cdot$}
\multiput(21,197)(4,-2){40}{$\cdot$}
\put(300,170){\vector(1,0){100}}
\put(300,200){\vector(1,0){100}}
\put(300,230){\vector(1,0){100}}
\put(350,210){\makebox(0,0){Mond}}
\end{picture}
\caption{\label{fig_Neigung}%
Durch die Neigung der Erdachse zur Ekliptik ist einer der Flutwulste oberhalb
des \"Aquators, der gegen\"uberliegende unterhalb. An bestimmten Orten, z.B. $A$ oder $B$, 
kann es vorkommen, dass nur einer der beiden Flutberge auftritt. Zwischen den jeweiligen
Positionen, $A$ bzw.\ $B$ und $A'$ bzw.\ $B'$ liegen etwas \"uber 12 Stunden.}
\end{figure}

Wie man in Abb.\ \ref{fig_Neigung} erkennt, 
gibt es Orte auf der Erdoberfl\"ache, die\index{diurnale Gezeiten}
zu bestimmten Zeiten, wenn die Erdachse zum Mond gerichtet ist (und nat\"urlich auch,
wenn sie von ihm weggerichtet ist), mitten im Flutberg befinden, wohingegen sie etwas
\"uber 12 Stunden sp\"ater (ein Mond-Tag hat 24\,h plus 50\,min) vergleichsweise weit
entfernt von dem gegen\"uberliegenden Flutberg sind. Das kann den Effekt haben, dass
man an solchen Orten nur einmal am Tag eine Flut und nur einmal am Tag eine Ebbe
wahrnimmt. 

Allerdings ist dies nur ein geometrischer Effekt auf die Gezeiten. In Wirklichkeit spielen
sehr viele Faktoren eine weitaus dominantere Rolle, wann, wo und wie intensiv Gezeiten an
einem Ort auftreten. Insbesondere spielen die Tiefenverh\"altnisse des Meeres und
auch die geographischen Verh\"altnisse der K\"uste eine sehr wichtige Rolle oder auch
ob es sich um eine Ost- oder Westk\"uste handelt. Auch die Dynamik der Str\"omungen spielt
hier eine wichtige Rolle. Tritt an einem Ort nur einmal innerhalb von 24 Stunden eine Flut oder Ebbe
auf, spricht man von diurnalen Gezeiten, ansonsten von semi-diurnalen Gezeiten. 


\section{Herleitung einiger Gleichungen}
\subsection{Gezeitenbeschleunigung}
\label{secA}

Berechnet werden soll in f\"uhrender Ordnung (Gl.\ \ref{eq_Delta_a})
\begin{equation}
                         \Delta a = G \frac{M_{\rm Mond}}{(R_{EM}-R_{\rm Erde})^2}  - 
                          G \frac{M_{\rm Mond}}{R_{EM}^2}   \, .
\end{equation} 
Dazu wird aus dem ersten Term auf der rechten Seite der Abstand Erde-Mond ausgeklammert,
\begin{equation}
           G \frac{M_{\rm Mond}}{(R_{EM}-R_{\rm Erde})^2} 
                  = G \frac{M_{\rm Mond}}{R_{EM}^2} \frac{1}{(1-\frac{R_{\rm Erde}}{R_{EM}})^2} \, , 
\end{equation} 
und der hintere Term in Potenzen von $\frac{R_{\rm Erde}}{R_{EM}}$ entwickelt. Die Entwicklung
lautet allgemein:
\begin{equation}
           \frac{1}{(1-x)^2} = 1 + 2x + O(x^2)  \, .
\end{equation}
Damit folgt:
\begin{equation}
           G \frac{M_{\rm Mond}}{(R_{EM}-R_{\rm Erde})^2} 
                  = G \frac{M_{\rm Mond}}{R_{EM}^2} \left( 1 + 2 \frac{R_{\rm Erde}}{R_{EM}} + ...\right)  
\end{equation} 
und wir erhalten f\"ur die Gezeitenbeschleunigung (die f\"uhrenden Terme heben sich weg): 
\begin{equation}
          \Delta a = G \frac{M_{\rm Mond}}{(R_{EM}-R_{\rm Erde})^2}  - 
       G \frac{M_{\rm Mond}}{R_{EM}^2}  =  2 G \frac{M_{\rm Mond}}{R_{EM}^2} \left( \frac{R_{\rm Erde}}{R_{EM}} 
          + O\left( \left(\frac{R_{\rm Erde}}{R_{EM}} \right)^2 \right) \right) \, .
\end{equation} 

\begin{thebibliography}{99}

\bibitem{Hicks} Hicks, S.D.; \textit{Understanding Tides}, NOAA Report; 2006; 
                  \url{https://repository.oceanbestpractices.org/handle/11329/594} (aufgerufen am 13.5.2023).
\bibitem{Kowalik} Kowalik, Z., Luick, J.L.; \textit{Modern Theory and Practice of Tide Analysis and
                 Tidal Power}, Austides Consulting, Eden Hills, 2019; \url{https://austides.com/downloads/} 
                 (aufgerufen am 13.5.2023).
\bibitem{Parker} Parker, B.B.; \textit{Tidal Analysis and Prediction}; NOAA Special Publication NOS CO-OPS 3;
                 2007; \url{https://repository.oceanbestpractices.org/handle/11329/632} (aufgerufen am 13.5.2023).                 

\end{thebibliography}

\end{document}
