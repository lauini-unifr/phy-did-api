\documentclass[german,10pt]{book}      
\usepackage{makeidx}
\usepackage{babel}            % Sprachunterstuetzung
\usepackage{amsmath}          % AMS "Grundpaket"
\usepackage{amssymb,amsfonts,amsthm,amscd} 
\usepackage{mathrsfs}
\usepackage{rotating}
\usepackage{sidecap}
\usepackage{graphicx}
\usepackage{color}
\usepackage{fancybox}
\usepackage{tikz}
\usetikzlibrary{arrows,snakes,backgrounds}
\usepackage{hyperref}
\hypersetup{colorlinks=true,
                    linkcolor=blue,
                    filecolor=magenta,
                    urlcolor=cyan,
                    pdftitle={Overleaf Example},
                    pdfpagemode=FullScreen,}
%\newcommand{\hyperref}[1]{\ref{#1}}
%
\definecolor{Gray}{gray}{0.80}
\DeclareMathSymbol{,}{\mathord}{letters}{"3B}
%
\newcounter{num}
\renewcommand{\thenum}{\arabic{num}}
\newenvironment{anmerkungen}
   {\begin{list}{(\thenum)}{%
   \usecounter{num}%
   \leftmargin0pt
   \itemindent5pt
   \topsep0pt
   \labelwidth0pt}%
   }{\end{list}}
%
\renewcommand{\arraystretch}{1.15}                % in Formeln und Tabellen   
\renewcommand{\baselinestretch}{1.15}                 % 1.15 facher
                                                      % Zeilenabst.
\newcommand{\Anmerkung}[1]{{\begin{footnotesize}#1 \end{footnotesize}}\\[0.2cm]}
\newcommand{\comment}[1]{}
\setlength{\parindent}{0em}           % Nicht einruecken am Anfang der Zeile 

\setlength{\textwidth}{15.4cm}
\setlength{\textheight}{23.0cm}
\setlength{\oddsidemargin}{1.0mm} 
\setlength{\evensidemargin}{-6.5mm}
\setlength{\topmargin}{-10mm} 
\setlength{\headheight}{0mm}
\newcommand{\identity}{{\bf 1}}
%
\newcommand{\vs}{\vspace{0.3cm}}
\newcommand{\noi}{\noindent}
\newcommand{\leer}{}

\newcommand{\engl}[1]{[\textit{#1}]}
\parindent 1.2cm
\sloppy

         \begin{document}  \setcounter{chapter}{6}
\newcommand{\solution}[1]{#1}
%\newcommand{\solution}[1]{}

\chapter{Quantenkryptographie - Das Eckert-Protokoll}
% Kap x
\label{chap_Eckert}

Das Eckert-Protokoll ist neben dem BB84-Protokoll eines der bekanntesten
Quantenkryptographieprotokolle. In diesem Kapitel wird vorausgesetzt, dass
eine zuf\"allige Folge von klassischen Bits als Schl\"ussel ein sicheres
Verschl\"usselungsverfahren ist (siehe Kap.\ \ref{chap_BB84}), sofern nur
der Sender und Empf\"anger diesen Schl\"ussel kennen. Ebenso wie BB84
handelt es sich bei dem Eckert-Protokoll um einen quantenbasierten
Schl\"usselaustausch. Im Gegensatz zum BB84-Protokoll werden hierbei
verschr\"ankte Photonen verwendet. Das hat den Vorteil, dass die Folge
der Zufallszahlen wirklich zufallsverteilt ist und dass die Bits, die wegen
der ungleichen Wahl der Basen von Alice und Bob verworfen werden zum
Nachweis verwendet werden k\"onnen, dass die Folge nicht abgelauscht
wurde.  



\end{enumerate}


\begin{thebibliography}{99}
\bibitem{Bennett} Bennett, C.H., Brassard, G., \textit{Quantum cryptography: Public
        key distribution and coin tossing}; in \textit{Proceedings of IEEE International
        Conference on Computers, Systems and Signal Processing}, Vol.\ 175 (1984).       
\end{thebibliography}


\end{document}

