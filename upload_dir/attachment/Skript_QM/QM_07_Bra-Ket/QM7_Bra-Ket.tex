%\documentclass[german,10pt]{book}      
\usepackage{makeidx}
\usepackage{babel}            % Sprachunterstuetzung
\usepackage{amsmath}          % AMS "Grundpaket"
\usepackage{amssymb,amsfonts,amsthm,amscd} 
\usepackage{mathrsfs}
\usepackage{rotating}
\usepackage{sidecap}
\usepackage{graphicx}
\usepackage{color}
\usepackage{fancybox}
\usepackage{tikz}
\usetikzlibrary{arrows,snakes,backgrounds}
\usepackage{hyperref}
\hypersetup{colorlinks=true,
                    linkcolor=blue,
                    filecolor=magenta,
                    urlcolor=cyan,
                    pdftitle={Overleaf Example},
                    pdfpagemode=FullScreen,}
%\newcommand{\hyperref}[1]{\ref{#1}}
%
\definecolor{Gray}{gray}{0.80}
\DeclareMathSymbol{,}{\mathord}{letters}{"3B}
%
\newcounter{num}
\renewcommand{\thenum}{\arabic{num}}
\newenvironment{anmerkungen}
   {\begin{list}{(\thenum)}{%
   \usecounter{num}%
   \leftmargin0pt
   \itemindent5pt
   \topsep0pt
   \labelwidth0pt}%
   }{\end{list}}
%
\renewcommand{\arraystretch}{1.15}                % in Formeln und Tabellen   
\renewcommand{\baselinestretch}{1.15}                 % 1.15 facher
                                                      % Zeilenabst.
\newcommand{\Anmerkung}[1]{{\begin{footnotesize}#1 \end{footnotesize}}\\[0.2cm]}
\newcommand{\comment}[1]{}
\setlength{\parindent}{0em}           % Nicht einruecken am Anfang der Zeile 

\setlength{\textwidth}{15.4cm}
\setlength{\textheight}{23.0cm}
\setlength{\oddsidemargin}{1.0mm} 
\setlength{\evensidemargin}{-6.5mm}
\setlength{\topmargin}{-10mm} 
\setlength{\headheight}{0mm}
\newcommand{\identity}{{\bf 1}}
%
\newcommand{\vs}{\vspace{0.3cm}}
\newcommand{\noi}{\noindent}
\newcommand{\leer}{}

\newcommand{\engl}[1]{[\textit{#1}]}
\parindent 1.2cm
\sloppy

         \begin{document}  \setcounter{chapter}{6}

\chapter{Die Bra-Ket-Notation}
% Kap x
\label{chap_BraKet}

\info{Thomas Filk}{27.03.2024}%
Die Bra-Ket-Notation von Paul Adrien Maurice Dirac (1902--1984) \cite{Dirac} hat\index{Bra-Ket-Notation|(} 
sich mittlerweile in der Quantentheorie\index{Dirac, Paul}
als au\ss erordentlich hilfreich etabliert. Ihre N\"utzlichkeit erweist sich besonders dann,
wenn Zust\"ande oder Operatoren basisunabh\"angig oder aber in wechselnden Basissystemen ausgedr\"uckt
werden sollen. Gerade in vielen modernen Zweigen der Quantentheorie, beispielsweise in der
Quanteninformatik, sind solche Basiswechsel wichtig. 

Physikalisch repr\"asentiert ein Ket-Vektor $| \psi \rangle$ einen Zustand
(wobei der Buchstabe $\psi$ zun\"achst rein symbolisch den Zustand beschreibt oder charakterisiert),
mathematisch repr\"ansentiert $|\psi \rangle$ einen normierten
Vektor in einem Hilbert-Raum (d.h.\ einem komplexen Vektorraum mit einem Skalarprodukt). 
Der zugeh\"orige Bra-Vektor $\langle \psi |$ ist ein Element des Dualraums, dem
wegen des Skalarprodukts einem Element des Vektorraums zugeordnet werden kann:
Sei $|\psi \rangle$ ein Vektor im Hilbert-Raum und $g(\cdot, \cdot)$ das Skalarprodukt auf
diesem Hilbert-Raum (wobei $\cdot$ jeweils durch einen Vektor zu ersetzen ist), dann ist
$\langle \psi| = g(| \psi \rangle, \cdot)$ der zugeh\"orige Vektor im Dualraum.

Diese rein mathematischen Beziehungen sind vielen Lehrkr\"aften nicht sehr vertraut, zum
einen, weil sie in der Schule keine Verwendung finden und daher nie ben\"otigt werden,
zum anderen aber auch, weil sie in Physikvorlesungen, besonders in fr\"uheren Zeiten,
nicht sehr pr\"azise verwendet oder eingef\"uhrt wurden. Konzepte wie der Dualraum eines
Vektorraums sind ohnehin vielen Studierenden der Physik eher \glqq suspekt\grqq. 

Die Bra-Ket-Notation kann aber auch symbolisch f\"ur verschiedene physikalische Operationen
verstanden werden -- unabh\"angig von ihrer mathematischen Bedeutung in der linearen
Algebra. Dabei sind meist nur sehr wenige und einfache Regeln zu beachten, denen man au\ss erdem
eine anschauliche Deutung geben kann. Um diese symbolische Bedeutung soll es in diesem 
Kurztext gehen. 

Zur Notation: Wenn man von einem \glqq Zustand\grqq\ spricht, ist oftmals nicht klar, 
was\index{Zustand}\index{Quantenzustand}
gemeint ist. Insbesondere sollte man zwischen der mathematischen Repr\"asentation und der
physikalischen Realisierung unterscheiden. Bei der physikalischen Realisierung ist wiederum
zwischen der Realisierung durch ein einzelnes System und der Realisierung durch ein
Ensemble von Systemen zu unterscheiden (siehe auch den Kurztext 
\glqq \hyperref[chap_Axiom]{Axiomatischer Zugang zur Physik})\grqq. 
Da im Folgenden viel von Wahrscheinlichkeiten
die Rede ist, repr\"asentiert $|\psi\rangle$ (sofern nicht explizit anders erw\"ahnt) in diesem Kurztext
immer ein Einzelsystem, das in dem Zustand $\psi$ pr\"apariert wurde.

\"Ahnliches gilt f\"ur den Begriff der Observablen.\index{Observable} 
Hier unterscheide ich zwischen der physikalischen
Realisierung einer Observablen in Form eines Messprotokolls, ausgedr\"uckt durch das Symbol
${\cal A}$, und der mathematischen Repr\"asentation der Observablen durch einen selbst-adjungierten
Operator, $A$. F\"ur den physikalischen Zustand, der dem Zustand zu einem Eigenwert $\lambda_n$ 
einer Observablen  entspricht, schreibe ich oft $\psi_n$; $|\lambda_n\rangle$ bezeichnet den 
zugeh\"origen normierten Vektor.

\section{Ket- und Bra- als \glqq Pr\"aparation\grqq\ und \glqq Messung\grqq}

Rein symbolisch kann ein normierter Ket-Vektor $|\psi\rangle$ folgenderma\ss en interpretiert werden:
Ein physikalisches System wird in dem Zustand $\psi$ pr\"apariert und wird nun durch diesen
Zustand beschrieben. Der Ket-Vektor $|\psi\rangle$ repr\"asentiert somit die Pr\"aparation eines Systems in einem
bestimmten Zustand $\psi$.\index{Ket-Vektor} 

Entsprechend repr\"asentiert ein normierter Bra-Vektor $\langle \phi|$ die\index{Bra-Vektor}
Messung an einem System, die das System auf den Zustand $\phi$ testet. $\langle \phi|$
repr\"asentiert somit den Prozess, bei dem ein Detektor anzeigt, dass der Zustand $\phi$ registriert wurde. 
Man kann einen solchen Detektor dadurch realisieren, dass man zun\"achst einen Filter
f\"ur den Zustand $\phi$ aufstellt, d.h., wenn dieser Filter das System durchgelassen hat, befindet
es sich anschlie\ss end im Zustand $|\phi\rangle$.
Dahinter steht ein Detektor, dessen \glqq Klick\grqq\ anzeigt, dass das System
den Filter passiert hat. 

Wenn man einen Ket-Vektor $|\psi\rangle$  im $\mathbb{C}^n$ bez\"uglich einer Orthonormalbasis
durch einen Spaltenvektor mit Komponenten $x_1, ..., x_n$ darstellt, 
dann entspricht dem Bra-Vektor der Zeilenvektor mit den jeweils komplex konjugierten Komponenten:
\begin{equation}
\label{eq_BK_Komponenten}
          |\psi\rangle \simeq \left( \begin{array}{c} x_1 \\ x_2 \\ \vdots \\ x_n \end{array} \right)
          \hspace{1cm} \langle \psi | \simeq (x_1^* , x_2^*, \cdots , x_n^* ) \, .
\end{equation}
Aus der linearen Algebra ist vermutlich die Notation $\pmb{x}$ f\"ur den Spaltenvektor und
$\pmb{\bar{x}}^T$ f\"ur den komplex konjugierten Zeilenvektor vertrauter. Mathematisch haben 
Ket- und Bra-Vektoren die gleiche Bedeutung. 

Die Hintereinanderausf\"uhrung dieser beiden Operationen -- Pr\"aparation und Messung -- 
liefert eine komplexe Zahl, die man als Wahrscheinlichkeitsamplitude zu
diesem Prozess -- System wurde im Zustand $\psi$ pr\"apariert und anschlie\ss end im
Zustand $\phi$ gemessen -- interpretiert. Das Absolutquadrat dieser Wahrscheinlichkeitsamplitude
ist die Wahrscheinlichkeit $w(\psi \rightarrow \phi)$, dass dieser Prozess auftritt. Dies ist
die Born'sche Regel:\index{Born'sche Regel}
\begin{equation}
       | \psi \rangle  \begin{picture}(14,0)(0,0) \put(3,2.4){\line(1,0){20}} \end{picture} > | \phi \rangle
        ~~ \simeq  ~~  \langle \phi | \psi \rangle  \hspace{1cm}
         w(\psi \rightarrow \phi) = | \langle \phi | \psi \rangle |^2  \, .
\end{equation}                    
Wichtig ist, dass man das Skalarprodukt $\langle \phi | \psi \rangle$ als eine Amplitude auffasst,
die dem Prozess $\psi \rightarrow \phi$ zugeordnet werden kann. Die mathematische Schreibweise hat
f\"ur diese symbolische Interpretation
den Nachteil, dass mathematische Ausdr\"ucke von rechts nach
links zu lesen sind. Die \glqq Zeit\grqq-Achse, in der die Bra-Ket-Ausdr\"ucke als Prozesse 
interpretiert werden k\"onnen, verl\"auft also waagerecht von rechts nach links.

Aus dieser Interpretation ergeben sich die wichtigsten Regeln f\"ur den Umgang mit der
Bra-Ket-Notation. Gew\"ohnlich leitet man diese Regeln aus den Eigenschaften des Skalarprodukts
ab. Wenn ein System im Zustand $|\psi\rangle$ pr\"apariert wurde und die Wahrscheinlichkeit,
an diesem System den Zustand $|\phi\rangle$ zu messen, verschwindet, ist $\langle \phi|\psi\rangle =0$.
Man bezeichnet diese beiden Zust\"ande dann als orthogonal. Diese Bezeichnung darf man
beispielsweise bei linearen Polarisationszust\"anden durchaus w\"ortlich nehmen: Eine horizontale
Polarisation ist orthogonal zu einer vertikalen Polarisation und es gilt $\langle v|h\rangle=0$,
wobei $|h\rangle$ und $|v\rangle$ jeweils eine horizontale bzw.\ vertikale Polarisation eines Photons
bezeichnen. Das Gleiche gilt f\"ur die Polarisationen $+45^\circ$ und $-45^\circ$, d.h., $\langle -|+\rangle=0$.
Sind andererseits die beiden Zust\"ande $|\psi\rangle $und $|\phi\rangle$ identisch, sodass die
Wahrscheinlichkeit, an einem im Zustand $|\psi\rangle$ pr\"aparierten System den Zustand $|\phi\rangle$ zu
messen, gleich eins ist, gilt $\langle \phi|\psi\rangle=1$ ($\langle \psi|\psi\rangle = 1$ ist die 
Normierungsbedingung f\"ur Vektoren $|\psi\rangle$, die Zust\"ande repr\"asentieren). 
Allgemein gilt das verallgemeinerte Gesetz\index{Malus, Gesetz von}
von Malus: $|\langle \phi|\psi\rangle|^2 = \cos^2 \alpha$, wobei $\alpha$ in reellen Vektorr\"aumen
der Winkel zwischen den beiden Strahlen zu $|\phi\rangle$ und $|\psi\rangle$ ist. 
 
Als Beispiel betrachten wir die Wellenfunktion $\psi(x)$, die in der Bra-Ket-Notation folgende
Darstellung hat:\index{Wellenfunktion}
\begin{equation} 
            \psi(x) = \langle x | \psi \rangle  \, . 
\end{equation}
Im Sinne unserer symbolischen Interpretation hat die Wellenfunktion also folgende 
Bedeutung: Ein System wird im Zustand $\psi$ pr\"apariert und anschlie\ss end wird
getestet, ob eine Messung das System am Ort $x$ registriert. Das Absolutquadrat
dieser Gr\"o\ss e, $|\langle x|\psi\rangle|^2 = |\psi(x)|^2$, ist die Wahrscheinlichkeit(-sdichte),
bei einer Ortsmessung das System am Ort $x$ vorzufinden.

Eine ganz \"ahnliche Interpretation hat die Fourier-Transformierte der Ortswellenfunktion,
die Funktion
\begin{equation} 
            \tilde{\psi}(p) = \langle p | \psi \rangle  \, . 
\end{equation}
(Dass es sich hierbei um die Fourier-Transformierte von $\psi(x)$ handelt, wird in Abschnitt
\ref{sec_BK_Vollstaendig} gezeigt.) 
Sie beschreibt die Amplitude f\"ur den Prozess, bei dem ein System im Zustand $\psi$
pr\"apariert wurde und anschlie\ss end bei einer Impulsmessung der Impuls $p$
gefunden wird. Das Absolutquadrat $|\langle p|\psi\rangle|^2 = |\tilde{\psi}(p)|^2$
ist wieder die Wahrscheinlichkeit(-sdichte), an einem System, das im Zustand $\psi$
pr\"apariert wurde, bei einer Impulsmessung den Impuls $p$ vorzufinden. 

Hier wird auch ein kleiner Vorteil der Bra-Ket-Notation deutlich: Die beiden Funktionen 
$\psi(x)=\langle x|\psi\rangle$ und $\tilde{\psi}(p)=\langle p|\psi\rangle$ beziehen sich auf
denselben Zustand $\psi$, an dem lediglich eine andere Messung (Ort $x$ bzw.\ Impuls $p$)
vorgenommen wird. Die funktionale Abh\"angigkeit von $x$ bzw.\ $p$ ist nat\"urlich eine
andere, deshalb handelt es sich um verschiedene Funktionen, was durch die $\tilde{~}$
ausgedr\"uckt wird, doch dadurch entsteht auch leicht der Eindruck, als ob es sich um 
verschiedene Zust\"ande handelte. W\"urde man jedoch einfach $\psi(x)$ und $\psi(p)$ schreiben,
w\"urde dasselbe Funktionssymbol f\"ur verschiedene mathematische Funktionen verwendet. 

\section{Projektionsoperatoren als \glqq Filter\grqq}

Projektionsoperatoren\index{Projektionsoperator} 
sind lineare Abbildungen -- in endlich dimensionalen Vektorr\"aumen
mit vorgegebener Basis also darstellbar als Matrizen --, die einen beliebigen Vektor auf einen
linearen Unterraum projizieren. Eindimensionale Projektionsoperatoren kann man mit einem
eindimensionalen Unterraum identifizieren, sie beschreiben somit einen reinen Zustand. Andererseits
kann man (auch mehrdimensionale) Projektionsoperatoren als Filter interpretieren, die nur
bestimmte Zust\"ande durchlassen. 

\subsection{Die Projektion als lineare Abbildung}

Hat eine Projektion eines Vektors auf einen linearen Unterrraum einmal stattgefunden, \"andert
eine erneute Projektion nichts, da der projizierte Vektor schon in dem betreffenden Unterraum
liegt. Die definierende Gleichung eines Projektionsoperators ist somit $P^2=P$ (und
damit gilt nat\"urlich auch $P^n=P$ f\"ur $n\geq 1, n\in \mathbb{N}$, d.h., man kann eine Projektion
beliebig oft anwenden, es z\"ahlt nur das erste Mal). Aus dieser Bedingung folgt, dass die
Eigenwerte von $P$ selbst nur $0$ oder $1$ sein k\"onnen, da die Eigenwerte einer linearen
Abbildung dieselbe Gleichung erf\"ullen m\"ussen, wie diese Abbildung selbst.

Da auf Hilbert-R\"aumen immer ein Skalarprodukt definiert ist, k\"onnen wir auch entscheiden, was eine
orthogonale Projektion ist. In diesem Fall sind die Vektoren, die auf die Null (den Nullvektor)
projiziert werden, orthogonal zu den Vektoren, die bereits in dem durch den Projektionsoperator
ausgezeichneten Unterraum liegen und daher
durch die Projektion nicht mehr ver\"andert werden. F\"ur diese Bedingung muss $P$ selbst-adjungiert
sein, es muss also $P^\dagger = P$ gelten ($P^\dagger$ bedeutet bez\"uglich einer Orthonormalbasis die
komplex konjugierte transponierte Matrix: $(P^\dagger)_{ij} = (P_{ji})^*$). 
Diese beiden Bedingungen -- $P^\dagger =P$ und $P^2=P$ --
werden im Folgenden f\"ur Projektionsoperatoren immer vorausgesetzt. Die Spur eines Projektionsoperators
ist gleich der Anzahl der Eigenwerte, die den Wert 1 haben und somit gleich der Anzahl der 
Dimensionen des Unterraums, auf den der Projektionsoperator projiziert. 

Wenn es nicht explizit erw\"ahnt wird, betrachten wir in diesem Kurztext eindimensionale Projektionsoperatoren,
also Projektionsoperatoren, die auf eindimensionale Unterr\"aume (Strahlen) projizieren. Projektionsoperatoren
auf h\"oher-dimensionale Unterr\"aume lassen sich immer als Summe von paarweise orthogonalen
eindimensionalen Projektionsoperatoren darstellen.

\subsection{Projektionsoperatoren als Darstellung von Zust\"anden}

In der Quantentheorie sind die Projektionsoperatoren auf einen eindimensionalen Unterraum\index{Zustand}
von besonderer Bedeutung. Sie charakterisieren diesen eindimensionalen Unterraum und somit
einen quantenmechanischen Zustand. Im Gegensatz zur Repr\"asentation eines reinen Zustands durch
einen normierten Vektor haben sie sogar den Vorteil, dass die Freiheit in der Wahl der Phase nicht
auftritt. Allerdings l\"asst sich mit Projektionsoperatoren das vertraute Superpositionsprinzip nicht
mehr so einfach ausdr\"ucken wie durch Vektoren, d.h., die\index{Quantenzustand}
Summe bzw.\ allgemeiner die Linearkombination von zwei Projektionsoperatoren ist im Allgemeinen
kein Projektionsoperator mehr und somit kein Repr\"asentant f\"ur einen reinen Zustand. 

Wenn ein normierter 
Ket-Vektor $|\psi\rangle$ bez\"uglich einer Orthonormalbasis im $\mathbb{C}^n$ durch einen
Spaltenvektor und der zugeh\"orige Bra-Vektor durch den entsprechenden (komplex konjugierten)
Zeilenvektor dargestellt werden (vgl.\ Gl.\ \ref{eq_BK_Komponenten}), dann hat der eindimensionale
Projektionsoperator auf den von $|\psi\rangle$ aufgespannten linearen Unterraum die Form:
\begin{equation}
     P_\psi = | \psi \rangle \langle \psi | \simeq \left( \begin{array}{cccc}
      x_1 x_1^*  & x_1 x_2^* & \cdots & x_1 x_n^* \\
      x_2 x_1^* & x_2 x_2^* & \cdots & x_2 x_n^*  \\
      \vdots & \vdots & \ddots & \vdots  \\
      x_n x_1^* & x_n x_2^* & \cdots & x_n x_n^*  \end{array} \right)  \, .
\end{equation}
In der linearen Algebra verwendet man meist die Darstellung $P_\psi = \pmb{x}\cdot \pmb{\bar{x}}^T$, wobei
$\pmb{\bar{x}}^T$ wieder den komplex konjugierten Zeilenvektor bezeichnet und $\cdot$ das
Matrixprodukt (\glqq Summe von Komponenten in einer Zeile mal Komponenten in einer Spalte\grqq).

Die definierende Eigenschaft des Projektionsoperators, $P^2=P$, ist (zumindest f\"ur eindimensionale 
Projektionsoperatoren) in der Bra-Ket-Schreibweise offensichtlich:
\begin{equation}
                  P^2_\psi = | \psi \rangle \langle \psi | \psi \rangle \langle \psi | = | \psi \rangle \langle \psi |  = P_\psi \, ,
\end{equation}
wohingegen ihr Beweis in der Komponentenschreibweise aufwendiger ist.
Angewandt auf einen beliebigen Zustand $|\phi\rangle$ folgt:
\begin{equation}
                 P_\psi |\phi \rangle =| \psi \rangle \langle \psi | \phi \rangle \, . 
\end{equation}
Auch hier ist offensichtlich das Ergebnis ein Vektor, der in Richtung von $|\psi\rangle$ zeigt,
also in dem von $|\psi \rangle$ aufgespannten eindimensionalen linearen Unterraum liegt. 

\subsection{Projektionsoperatoren als Repr\"asentanten von Filtern}

Das physikalische System, das durch einen Projektionsoperator repr\"asentiert wird, ist ein
Filter.\index{Filter} 
Ein (idealer) Filter hat die Eigenschaft, physikalischen Systeme in bestimmten Zust\"anden ungehindert 
durchzulassen und Systeme in anderen, orthogonalen Zust\"anden vollst\"andig zu absorbieren. Das
einfachste Beispiel f\"ur einen Filter ist ein Polarisationsfilter f\"ur Licht, der Licht einer bestimmten
Polarisation ungehindert durchl\"asst und Licht der orthogonalen Polarisation vollst\"andig absorbiert. 

Die definierende Eigenschaft f\"ur Projektionsoperatoren, $P^2=P$, hat f\"ur Filter die Interpretation,
dass ein zweiter Filter hinter einem ersten Filter, wobei beide dieselben Polarisationsrichtung haben
sollen, keine Wirkung hat: Alles Strahlung, die den ersten Filter passiert hat, passiert (im Idealfall --
reale Filter haben immer einen kleinen Absorptionskoeffizienten) auch den zweiten Filter. 

F\"ur zwei eindimensionale Projektionsoperatoren zu orthogonalen Zust\"anden $|\psi\rangle$ und $|\phi\rangle$
(also $\langle \psi|\phi \rangle =0$) gilt:
\begin{equation}
            P_\psi P_\phi = | \psi \rangle \langle \psi | \phi \rangle \langle \phi | = 0 \, .             
\end{equation}
Die anschauliche Bedeutung dieser Gleichung ist, dass zwei zueinander orthogonale Filter nichts durchlassen. 
In diesem Fall spielt auch die Reihenfolge der Filter keine Rolle, was f\"ur die zugeh\"origen
Projektionsoperatoren bedeutet, dass diese kommutieren. 

\subsection{Die Summe von Projektionsoperatoren}
\label{sec_BK_SP}

In endlich dimensionalen
Vektorr\"aumen l\"asst sich auch jeder Projektionsoperator auf einen mehrdimensionalen Unterraum
durch eine Summe orthogonaler Projektionsoperatoren auf eindimensionale Unterr\"aume darstellen. 
Sind $|\psi\rangle$ und $|\phi\rangle$ orthogonal, gilt also $\langle \psi|\phi \rangle=0$, dann handelt
es sich bei
\begin{equation}
                  P_2 = P_\psi + P_\phi =  |\psi \rangle \langle \psi | + |\phi \rangle \langle \phi |
\end{equation}
wieder um einen Projektionsoperator, da
\begin{equation}
                  P_2^2 = P_\psi^2 +  P_\psi P_\phi + P_\phi P_\psi + P_\phi^2 
                  =  P_\psi + P_\phi \, .
\end{equation}
Die gemischten Produkte verschwinden, $P_\psi P_\phi=0=P_\phi P_\psi$, au\ss erdem 
wurde die definierende Eigenschaft der Projektoren verwendet. Man kann sich leicht \"uberlegen,
dass diese Eigenschaft f\"ur eine beliebige Summe von paarweise orthogonalen Projektionsoperatoren
gilt. Die Interpretation von $P_2$ ist die eines Filters, der sowohl die Zust\"ande $|\psi\rangle$ als
auch die Zust\"ande $|\phi \rangle$ sowie jede Linearkombination dieser beiden Zust\"ande
ungehindert durchl\"asst und alle dazu orthogonalen Zust\"ande absorbiert. Zur ersten Eigenschaft
zeigt man
\begin{eqnarray}
                  P_2 (\alpha |\psi\rangle + \beta |\phi \rangle ) &=& 
               \big( |\psi \rangle \langle \psi | + |\phi \rangle \langle \phi | \big)~ (\alpha |\psi\rangle + \beta |\phi \rangle )  \\
               &=& 
               \alpha |\psi \rangle \langle \psi |\psi\rangle + 
               \beta |\psi \rangle \langle \psi |\phi\rangle + 
              \alpha |\phi \rangle \langle \phi | \psi\rangle + \beta |\phi \rangle \langle \phi |\phi \rangle   \\
              &=&  \alpha |\psi\rangle + \beta |\phi \rangle 
\end{eqnarray}
Die mittleren Terme verschwinden wegen der Orthogonalit\"at der Zust\"ande und bei den \"au\ss eren
beiden Termen wurde ausgenutzt, dass die Vektoren normiert sein sollen. Das \glqq + \grqq-Zeichen zwischen
zwei Projektionsoperatoren hat
also hier die Bedeutung von \glqq sowohl Eigenschaft 1 als auch Eigenschaft 2 sowie jede
Linearkombination dieser beiden Eigenschaften wird durchgelassen\grqq. 
Die Multiplikation von zueinander orthogonalen Projektionsoperatoren
hat die Bedeutung der Hintereinanderschaltung der Filter und liest sich 
als \glqq nur wenn beide Eigenschaften vorliegen, wird das System durchgelassen\grqq. 

Als Beispiel f\"ur die letzte Aussage betrachten wir zwei Projektionsoperatoren in einem 
3-dimensionalen Vektorraum, von denen einer auf die von den orthogonalen Zust\"anden
$|1\rangle$ und $|2\rangle$ aufgespannten Ebene projiziert und der zweite auf die
von $|2\rangle$ und $|3\rangle$ aufgespannten Ebene projiziert. Die drei Vektoren
$|1\rangle$, $|2\rangle$ und $|3\rangle$ seien paarweise orthogonal, ansonsten aber beliebig.
Dann gilt:
\begin{eqnarray}
       P_{23} P_{12} &=& \Big( |2\rangle \langle 2| + |3\rangle \langle 3| \Big) \Big( |1\rangle \langle 1| + |2\rangle \langle 2| \Big) \\
           &=& |2\rangle \langle 2|1\rangle \langle 1| + |2\rangle \langle 2|2\rangle \langle 2| +
           |3\rangle \langle 3|1\rangle \langle 1| + |3\rangle \langle 3|2\rangle \langle 2| 
           = |2\rangle \langle 2|
\end{eqnarray}
Alle anderen Terme verschwinden wegen der Orthogonalit\"at. Nur die \glqq Eigenschaft\grqq\ 2, die beiden
Projektionsoperatoren gemeinsam ist, bleibt bestehen. 

\subsection{Der Doppelspalt}

Das Doppelspaltexperiment\index{Doppelspaltexperiment} 
l\"asst sich mit den bisher beschriebenen Mitteln schon recht gut
beschreiben. Wir stellen uns eine Quelle vor, die Teilchen (oder Licht bzw.\ Photonen) mit einem
Impuls $p$ (bzw.\ einer Wellenl\"ange $\lambda$) in $z$-Richtung erzeugt (vgl.\ Abbildung \ref{fig_BK_Doppel})
und auf einen Schirm mit einem Doppelspalt lenkt. Die beiden Spalte befinden sich bei den Punkten 1
und 2. Diese beiden Spalte wirken wie zwei Filter, welche die Teilchen bez\"uglich der $x$-Richtung
auf die Punkte 1 bzw.\ 2 projizieren (d.h., nur Teilchen an diesen Punkten werden von den Spalten
durchgelassen). Hinter den beiden Spalten breiten sich, ausgehend von Punkt 1 bzw.\ Punkt 2, 
Kugelwellen (bzw.\ Zylinderwellen) aus. Diese treffen schlie\ss lich auf eine Detektorplatte. An der
Stelle $x$ auf dieser Detektorplatte befindet sich ein Detektor, der registriert, wenn ein Teilchen bei $x$
auf die Platte auftrifft. 

\begin{figure}[htb]
\begin{picture}(420,130)(0,0)
\multiput(110,50)(10,0){7}{\line(0,1){40}}
\put(170,10){\makebox(0,0){$|\psi\rangle$}}
\put(100,10){\makebox(0,0){$|1\rangle\langle 1| + |2\rangle\langle 2|$}}
\put(15,120){\makebox(0,0){$x$}}
\put(5,47){\makebox(0,0){$x$}}
\put(160,40){\makebox(0,0){$z$}}
\put(10,10){\makebox(0,0){$\langle x|$}}
\put(96,57){\makebox(0,0){${\scriptstyle 1}$}}
\put(96,80){\makebox(0,0){${\scriptstyle 2}$}}
\put(195,70){\makebox(0,0){\footnotesize Quelle}}
\put(150,40){\vector(-1,0){20}}
\put(10,47){\line(6,1){88}}
\put(10,47){\line(3,1){88}}
\put(180,50){\line(1,0){30}}
\put(180,50){\line(0,1){40}}
\put(180,90){\line(1,0){30}}
\thicklines
\put(10,20){\vector(0,1){100}}
\put(100,20){\line(0,1){41}}
\put(100,64){\line(0,1){11}}
\put(100,78){\line(0,1){42}}
\put(298,115){\makebox(0,0){\mbox{$|\psi\rangle$ beschreibt eine ebene Welle}}}
\put(315,105){\makebox(0,0){\mbox{zu festem Impuls in $z$-Richtung}}}
\put(320,90){\makebox(0,0){\mbox{$|1\rangle \langle 1|$ beschreibt einen Filter am Punkt 1}}}
\put(320,75){\makebox(0,0){\mbox{$|2\rangle \langle 2|$ beschreibt einen Filter am Punkt 2}}}
\put(320,60){\makebox(0,0){\mbox{$\langle x|$ beschreibt eine Messung am Punkt $x$}}}
\end{picture}
\caption{\label{fig_BK_Doppel}%
Das Doppelspaltexperiment. Qualitativ lassen sich die wesentlichen Elemente in der Bra-Ket-Notation
einfach darstellen. $|\psi\rangle$ beschreibt die Pr\"aparation eines Strahls von Teilchen (oder Licht)
zu festem Impuls bzw.\ fester Wellenl\"ange. Der Doppelspalt entspricht zwei Filtern, die Teilchen an
den beiden Orten der Spalte, 1 und 2, durchlassen. Anschlie\ss end breiten sich von diesen beiden Orten
Kugelwellen aus. Die beiden Anteile treffen auf einen Detektorschirm. Am Ort $x$ werden Teilchen
registriert.} 
\end{figure}

Eine rein qualitative Beschreibung ber\"ucksichtigt nur den pr\"aparierten Zustand einer ebenen
Welle, die auf den Schirm mit zwei Spalten zul\"auft, die beiden Spalte, die jeweils einen r\"aumlichen
Filter darstellen -- sie lassen nur Teilchen am Ort 1 bzw.\ 2 durch -- und die abschlie\ss ende Messung
des Orts der Teilchen auf dem Detektorschirm. Eine quantitative Beschreibung muss noch die Ausbreitung
der Wellen ber\"ucksichtigen, insbesondere die Ausbreitung der Kugelwellen von den Orten 1 bzw.\ 2 bis
zum Detektorschirm. 

Insgesamt ergibt sich f\"ur diesen Prozess der folgende Ausdruck:
\begin{equation}
    |\psi \rangle \longrightarrow  \big( |1\rangle \langle 1| + |2\rangle \langle 2|\big)~ |\psi\rangle \longrightarrow
       \langle x |~\big( |1\rangle \langle 1| + |2\rangle \langle 2|\big) ~ |\psi\rangle =
       \langle x |1\rangle \langle 1| \psi \rangle + \langle x |2\rangle \langle 2|\psi\rangle 
\end{equation}
Dies entspricht dem Prozess
\begin{equation}
  \mbox{pr\"aparierte Welle} \longrightarrow \mbox{trifft auf die beiden Filter} \longrightarrow 
      \mbox{wird am Ort $x$ gemessen} \, .
\end{equation}
Da es sich bei $|\psi\rangle$ um den Zustand zu einer ebenen Welle handelt, sind die
Amplituden $\langle 1|\psi\rangle$ und $\langle 2|\psi\rangle$ f\"ur den Prozess \glqq ebene Welle
propagiert zu den Spalten 1 bzw.\ 2\grqq\ gleich. Normiert auf die Teilchenanzahl, die einen der
beiden Spalte passieren, erhalten wir jeweils die halbe Wahrscheinlichkeit oder
$\langle 1|\psi\rangle=\langle 2|\psi\rangle = 1/\sqrt{2}$. F\"ur die Prozesse $\langle x| 1 \rangle$ bzw.\
$\langle x|2\rangle$ m\"ussten wir eigentlich Kugelwellen von Spalt 1 zum Ort $x$ bzw.\ vom Spalt 2 zum
Ort $x$ ansetzen. Ist der Abstand zwischen den beiden Spalten klein und der Abstand zur
Detektorebene im Vergleich dazu sehr gro\ss, spielt
nur der relative Gangunterschied $\delta$ der Wellen eine Rolle, d.h., 
$\langle x|2\rangle ={\rm e}^{{\rm i}\delta} \langle x|1\rangle$. Wir erhalten so f\"ur die Amplitude der Gesamtwelle
am Ort $x$:
\begin{equation}
             \Psi_{\rm ges}(x) = \frac{N}{\sqrt{2}} \Big( 1 + {\rm e}^{{\rm i}\delta(x)} \Big) \, , 
\end{equation} 
wobei $N$ eine geeignete Normierungskonstante ist. F\"ur das Absolutquadrat, d.h.\ die
Wahrscheinlichkeit(sdichte), am Ort $x$ ein Teilchen nachzuweisen, erhalten wir:
\begin{equation}
             w(x) = \frac{N^2}{2} \Big( 2 + {\rm e}^{{\rm i}\delta(x)} + {\rm e}^{-{\rm i}\delta(x)} \Big)
                  = N^2 \big( 1 + \cos \delta(x) \big)   \, . 
\end{equation} 
Dies entspricht dem Interferenzmuster auf dem Schirm. Der Gangunterschied $\delta(x)$ l\"asst sich durch 
die Parameter des Aufbaus -- Abstand $d$ der beiden Spalte, Wellenl\"ange $\lambda$ und
der Winkel $\alpha(x)$ des Punkts $x$ zur Normalenrichtung des Strahls -- ausdr\"ucken: 
\begin{equation}
             \delta(x) = 2\pi \frac{d}{\lambda} \sin \alpha(x) \, . 
\end{equation} 
Die hier verwendete Notation $\langle x|1\rangle$ und $\langle x|2\rangle$ f\"ur die beiden Teilstrahlen
nach dem Spalt 1 bzw.\ 2 bezeichnet nat\"urlich dasselbe, was man gew\"ohnlich als Wellen
$\psi_1(x)$ und $\psi_2(x)$ bezeichnet. Korrekterweise bezieht sich diese Notation allerdings auf 
eine Messung der Teilchen am Ort $x$ auf dem Schirm. 

\subsection{Die Vollst\"andigkeitsrelation}
\label{sec_BK_Vollstaendig}

Sei $\{ |e_i\rangle \}_{i=1,...,n}$ eine Orthonormalbasis eines Hilbert-Raums und\index{Vollst\"andigkeitsrelation}
seien $P_i = | e_i \rangle \langle e_i|$ die zugeh\"origen Projektionsoperatoren, dann gilt
\begin{equation}
            \sum_{i=1}^n P_i = \sum_{i=1}^n | e_i \rangle \langle e_i| = {\bf 1} \, .
\end{equation}
Dies folgt letztendlich aus der Tatsache, dass der gesamte Vektorraum von der Basis
aufgespannt wird und ist eine direkte Verallgemeinerung dessen, was in Abschnitt
\ref{sec_BK_SP} zur Summe von orthogonalen Projektionsoperatoren gesagt wurde. 

Da die Eigenr\"aume von selbst-adjungierten Operatoren orthogonal sind, erlauben diese
die Definition einer vollst\"andigen Orthonormalbasis. Sei $A$ ein beliebiger selbst-adjungierter
Operator und seien $\{\lambda_i\}$ die Eigenwerte und $|\lambda_i\rangle$ die zugeh\"origen
normierten Eigenvektoren (der Einfachheit wird hier angenommen, dass die Eigenwerte nicht
entartet sind, die Verallgemeinerung ist aber offensichtlich), dann gilt 
\begin{equation}
             \sum_{i=1}^n | \lambda_i \rangle \langle \lambda_i| = {\bf 1} \, .
\end{equation}
Diese Identit\"at kann immer zwischen Bra- und Ket-Vektoren eingef\"ugt werden. Beispielsweise
gilt f\"ur einen beliebigen normierten Vektor $|\psi\rangle$:
\begin{equation}
      \langle \psi| \psi \rangle = \sum_{i=1}^n \langle \psi | \lambda_i \rangle \langle \lambda_i \psi \rangle 
              = \sum_{i=1}^n | \langle \lambda_i | \psi \rangle |^2   = 1 \,  .
\end{equation}
Diese Relation bedeutet, dass f\"ur einen beliebigen Zustand $|\psi\rangle$ und eine beliebige
Observable ${\cal A}$, repr\"asentiert durch den selbst-adjungierten Operator $A$, die
Wahrscheinlichkeit, einen der m\"oglichen Eigenwerte $\lambda_i$ als Messwert zu finden und
damit das System in den Zustand $|\lambda_i\rangle$ zu versetzen, gleich 1 ist. Man erh\"alt
in der Quantentheorie bei einer Messung immer ein definitives Ergebnis, das einem der
Eigenwerte der Operators zu der Messung entspricht. Dies ist einer der Wesensz\"uge der
Quantentheorie.\index{Wesensz\"uge der Quantentheorie}

Eine entsprechende Relation gilt auch f\"ur ein kontinuierliches Spektrum eines
selbst-adjungierten Operators, beispielsweise das Spektrum des Ortsoperators:
\begin{equation}
             \int_{-\infty}^\infty | x \rangle \langle x |  \, {\rm d}x= {\bf 1} \, .
\end{equation}
Damit folgt beispielsweise
\begin{equation}
      \langle \psi| \psi \rangle = \int_{-\infty}^\infty \langle \psi |x \rangle \langle x | \psi \rangle \, {\rm d}x
              = \int_{-\infty}^\infty  | \langle x | \psi \rangle |^2 \,{\rm d}x  
              = \int_{-\infty}^\infty  | \psi(x)|^2 \,{\rm d}x = 1 \,  .
\end{equation}
Dies ist die \"ubliche Normierungsbedingung f\"ur Wellenfunktionen.

Ebenso folgt daraus, dass $\tilde{\psi}(p)$ die Fourier-Transformierte von $\psi(x)$ ist:
\begin{equation}
      \tilde{\psi}(p) = \langle p | \psi \rangle = \int_{-\infty}^\infty \langle p |x \rangle \langle x | \psi \rangle \, {\rm d}x
              = \int_{-\infty}^\infty   \langle p | x \rangle  \psi(x) \,{\rm d}x = \frac{1}{\sqrt{2\pi \hbar}}
              \int_{-\infty}^\infty  {\rm e}^{-{\rm i}px/\hbar} \, \psi(x) \,{\rm d}x  \,  ,
\end{equation}
wobei ausgenutzt wird, dass
\begin{equation}
           \langle p|x\rangle = \langle x|p\rangle^* \hspace{1cm} {\rm und} \hspace{0.6cm}
           \langle x | p \rangle = \frac{1}{\sqrt{2\pi \hbar}} {\rm e}^{{\rm i} p x/\hbar}  \, .
\end{equation}
Die Wahrscheinlichkeitsamplitude, dass ein mit dem Impuls $p=\hbar/2\pi \lambda$ pr\"apariertes
Teilchen am Ort $x$ gemessen wird, ist (bis auf eine Normierung) durch eine reine Welle gegeben.

\section{Unit\"are Prozesse}

Physikalische Prozesse, bei denen ein System bestehen bleibt (also beispielsweise nicht
absorbiert werden kann), allerdings der Zustand des Systems ver\"andert werden kann,
werden durch unit\"are Operatoren bzw.\ unit\"are Matrizen beschrieben. Diese erhalten
das Skalarprodukt, d.h.\index{unit\"arer Prozess}\index{Prozess!unit\"arer}
\begin{equation}
          \langle \psi | U^\dagger U| \phi \rangle = \langle \psi | \phi \rangle 
\end{equation} 
und damit insbesondere auch die Norm. (Umgekehrt ist jede Isometrie, d.h.\ normerhaltende
lineare Abbildung in endlich dimensionalen Hilbert-R\"aumen auch eine unit\"are Abbildung.)
Da diese Bedingung f\"ur beliebige Vektoren $|\psi\rangle$ und $|\phi\rangle$ gelten soll, 
folgt $U^\dagger U = {\bf 1}$ oder $U^\dagger = U^{-1}$. 

Jede lineare Abbildung kann als Linearkombination von Ausdr\"ucken der Form 
$|\psi\rangle \langle \phi|$ dargestellt werden. Ein solcher Ausdruck  l\"asst sich folgenderma\ss en interpretieren:
Bei dem Prozess wird festgestellt, dass sich das System im Zustand $|\phi\rangle$ befindet und es wird
der Zustand $|\psi\rangle$ erzeugt bzw.\ pr\"apariert. Der Faktor in einer solchen Linearkombination 
ist eine (Wahrscheinlichkeits-)Amplitude, die diesem speziellen Prozess zugeordnet wird.

Im Folgenden betrachten wir ganz speziell den unit\"aren Prozess, der durch einen Strahlteiler
bei Photonen gegeben ist. 

\subsection{Der Strahlteiler}

Ein Strahlteiler\index{Strahlteiler} 
teilt einen Lichtstrahl in zwei gleiche Anteile auf. Es handelt sich dabei oft
um einen Glasw\"urfel, der aus zwei Prismen zusammengesetzt ist, die an ihrer Grenzfl\"ache, wo
sie aufeinanderliegen, eine besondere Beschichtung haben. Wir betrachten im Folgenden
polarisationsunabh\"angige Strahlteiler. Typischerweise besitzen diese Strahlteiler zwei
Eingangsstrahlen sowie zwei Ausgangsstrahlen. Wir bezeichnen diese beiden Strahlrichtungen
mit 1 bzw.\ 2 und nummerieren die Stahlen so, dass gegen\"uberliegende Seiten dieselbe
Bezeichnung haben. Das bedeutet, die geradeaus durchgelassenen Strahlen haben jeweils dieselbe 
Nummer (siehe Abb.\ \ref{fig_BK_ST}). 

\begin{SCfigure}[70][htb]
\begin{picture}(100,100)(0,0)
\put(30,30){\line(1,0){40}}
\put(30,30){\line(0,1){40}}
\put(70,30){\line(0,1){40}}
\put(30,70){\line(1,0){40}}
\put(30,30){\line(1,1){40}}
\put(45,15){\makebox(0,0){$1$}}
\put(45,85){\makebox(0,0){$1$}}
\put(15,44){\makebox(0,0){$2$}}
\put(85,44){\makebox(0,0){$2$}}
\thicklines
\put(50,5){\vector(0,1){20}}
\put(5,50){\vector(1,0){20}}
\put(75,50){\vector(1,0){20}}
\put(50,75){\vector(0,1){20}}
\end{picture}
\caption{\label{fig_BK_ST}%
Der Strahlteiler teilt einfallende Strahlen in zwei ausfallende Strahlen auf. Die einfallenden
Strahlen k\"onnen dabei Strahl 1 oder 2 sein, die ausfallenden sind so nummeriert, dass 
geradeaus durchgehende Strahlen die gleiche Bezeichnung behalten. Die Wirkung eines solchen Strahlteilers
kann durch folgende Bra-Ket-Darstellung beschrieben werden: \newline
$  U_{\rm ST} = \frac{1}{\sqrt{2}} \Big(  | 1 \rangle \langle 1 | + | 2 \rangle \langle 2 | +
          {\rm i}  | 1 \rangle \langle 2 | + {\rm i} | 2 \rangle \langle 1 |  \Big) \, . $ }
\end{SCfigure}

Wir behandeln ein solches System f\"ur den Zustand eines Photons als
Zwei-Zustandssystem: Das Photon befindet sich bei einer Messung entweder in Strahl 1 oder in Strahl 2. 
Die zugeh\"origen Ket-Vektoren sind $|1\rangle$ und $|2\rangle$. (Falls das System mehr
als zwei Strahleng\"ange hat, m\"ussen entsprechend mehr Zust\"ande eingef\"uhrt
werden; dieser Fall kann beispielsweise auftreten, wenn die ausfallenden Strahlen 1 und/oder 2 durch weitere
Strahlteiler nochmals aufgespalten werden.) 

Da alle Photonen, die in den Strahlteiler treten, diesen auch wieder verlassen sollen, muss
ein solcher Strahlteiler durch eine unit\"are Matrix dargestellt werden. Die allgemeinste
$2\times 2$ unit\"are Matrix, welche die Intensit\"at der beiden einfallenden Strahlen zu jeweils
der H\"alfte aufteilt, hat die Form\footnote{Hierbei
handelt es sich um eine speziell unit\"are Matrix mit der Determinante 1. Eine
gemeinsame Phase l\"asst sich immer durch eine gleiche Phasenverschiebung beider
einfallenden oder ausfallenden Teilstrahlen kompensieren.}
\begin{equation}
            U_{\rm ST} = \frac{1}{\sqrt{2}} 
            \left( \begin{array}{cc} {\rm e}^{{\rm i} \delta} & {\rm e}^{{\rm i} \alpha} \\
            -{\rm e}^{-{\rm i} \alpha} & {\rm e}^{- {\rm i} \delta}  \end{array} \right) \, .
\end{equation}
Wir k\"onnen den Strahlteiler immer so bauen, dass die beiden\index{Strahlteiler!Darstellung durch unit\"are Matrix}
geradeaus durchgelassenen Strahlen in ihrer Phase unver\"andert bleiben ($\delta=0$) und
die beiden abgelenkten Strahlen jeweils die gleiche Phase erhalten ($\alpha=\pi/2$). Dies l\"asst uns nur die
Matrix
\begin{equation}
            U_{\rm ST} = \frac{1}{\sqrt{2}} 
            \left( \begin{array}{cc} 1 & {\rm i} \\
           {\rm i}  & 1  \end{array} \right) \, .
\end{equation}
Das entspricht auch der oft verwendeten Regel, dass die Ablenkung eines Strahls um 
$90^\circ$ eine Phasenverschiebung von $90^\circ$ bewirkt (und damit der Multiplikation mit
einem Faktor i). 

Statt aber in der Matrix- und Vektorschreibweise zu arbeiten, stellen wir die unit\"are Matrix
durch ihre Elemente in der Bra-Ket-Schreibweise dar:
\begin{equation}
\label{eq_BK_UTerme}
            U_{\rm ST} = \frac{1}{\sqrt{2}} \Big(  | 1 \rangle \langle 1 | + | 2 \rangle \langle 2 | +
          {\rm i}  | 1 \rangle \langle 2 | + {\rm i} | 2 \rangle \langle 1 |  \Big) \, .
\end{equation}
Der Bra-Teil dieser Darstellung (der rechte Anteil) stellt jeweils fest, durch welche Seite das Photon in den
Strahlteiler eintritt, er \glqq misst den Eingangszustand\grqq, und der Ket-Teil (linker Anteil) gibt an,
durch welche Seite das Photon aus dem Strahlteiler austritt, d.h., in welchem Strahl das
Photon pr\"apariert wurde. Die \glqq + \grqq-Zeichen bedeuten wieder Alternativen.

\begin{figure}[htb]
%     Bild 1
\begin{picture}(100,90)(0,0)
\put(30,30){\line(1,0){40}}
\put(30,30){\line(0,1){40}}
\put(70,30){\line(0,1){40}}
\put(30,70){\line(1,0){40}}
\put(30,30){\line(1,1){40}}
\put(45,20){\makebox(0,0){$1$}}
\put(45,80){\makebox(0,0){$1$}}
\put(50,0){\makebox(0,0){$\frac{1}{\sqrt{2}}\,|1\rangle \langle 1|$}}
\thicklines
\put(50,15){\line(0,1){35}}
\put(50,50){\vector(0,1){35}}
\end{picture}
%         Bild 2
\begin{picture}(100,90)(0,0)
\put(30,30){\line(1,0){40}}
\put(30,30){\line(0,1){40}}
\put(70,30){\line(0,1){40}}
\put(30,70){\line(1,0){40}}
\put(30,30){\line(1,1){40}}
\put(45,20){\makebox(0,0){$1$}}
\put(80,43){\makebox(0,0){$2$}}
\put(50,0){\makebox(0,0){$\frac{\rm i}{\sqrt{2}}\,|2\rangle \langle 1|$}}
\thicklines
\put(50,15){\line(0,1){35}}
\put(50,50){\vector(1,0){35}}
\end{picture}
%         Bild 3
\begin{picture}(100,90)(0,0)
\put(30,30){\line(1,0){40}}
\put(30,30){\line(0,1){40}}
\put(70,30){\line(0,1){40}}
\put(30,70){\line(1,0){40}}
\put(30,30){\line(1,1){40}}
\put(20,45){\makebox(0,0){$2$}}
\put(80,43){\makebox(0,0){$2$}}
\put(50,0){\makebox(0,0){$\frac{1}{\sqrt{2}}\,|2\rangle \langle 2|$}}
\thicklines
\put(15,50){\line(1,0){35}}
\put(50,50){\vector(1,0){35}}
\end{picture}
%         Bild 4
\begin{picture}(100,90)(0,0)
\put(30,30){\line(1,0){40}}
\put(30,30){\line(0,1){40}}
\put(70,30){\line(0,1){40}}
\put(30,70){\line(1,0){40}}
\put(30,30){\line(1,1){40}}
\put(20,45){\makebox(0,0){$2$}}
\put(45,80){\makebox(0,0){$1$}}
\put(50,0){\makebox(0,0){$\frac{\rm i}{\sqrt{2}}\,|2\rangle \langle 1|$}}
\thicklines
\put(15,50){\line(1,0){35}}
\put(50,50){\vector(0,1){35}}
\end{picture}
\caption{\label{fig_BK_Uvier}%
Jeder Weg eines Photons durch den Strahlteiler entspricht einem Ket-Bra-Term. Bei einer Reflektion
an der diagonalen Trennlinie wird die Phase der Photonwelle um $90^\circ$ verschoben, was einem
Faktor i entspricht. Au\ss erdem wurde eine Strahlaufteilung zu gleichen Intensit\"aten angenommen,
was f\"ur die Amplitude jeweils einen Faktor $1/\sqrt{2}$ bedeutet.} 
\end{figure}

Die vier Terme in Gl.\ \ref{eq_BK_UTerme} entsprechen den vier m\"oglichen Wegen, die
ein Photon durch den Strahlteiler nehmen kann (siehe Abb.\ \ref{fig_BK_Uvier}): 
zwei Eing\"ange und zwei Ausg\"ange.
Sie entsprechen damit ebenfalls den vier Termen in der unit\"aren $2\times 2$-Matrix, die
den Strahlteiler beschreibt.

Tritt ein Photon nur durch den ersten Strahlengang in den Strahlteiler, befindet es sich also
zun\"achst in dem Zustand $|1\rangle$, so befindet es sich nach dem Durchtritt durch den
Strahlteiler im Zustand
\begin{equation}
             | 1 \rangle ~~ \longrightarrow ~~  \frac{1}{\sqrt{2}} \left( \begin{array}{cc}
                 1 & {\rm i} \\ {\rm i} & 1 \end{array} \right)  \left( \begin{array}{c}
         1 \\ 0  \end{array} \right)  = \frac{1}{\sqrt{2}} \left( \begin{array}{c}  1 \\ {\rm i} \end{array} \right) 
\end{equation}
was in der Bra-Ket-Notation folgenderma\ss en geschrieben wird:
\begin{eqnarray}
             | 1 \rangle  & \longrightarrow &  \frac{1}{\sqrt{2}} \Big(  |1\rangle \langle 1| +  
            {\rm i} |2\rangle \langle 1| + |2\rangle \langle 2| + {\rm i} |1\rangle \langle 2| \Big) ~ |1\rangle \\
             &=&    \frac{1}{\sqrt{2}}  \Big(  |1\rangle \langle 1| 1\rangle  +
            {\rm i} |2\rangle \langle 1|1\rangle + |2\rangle \langle 2|1\rangle + {\rm i} |1\rangle \langle 2|1\rangle \Big)
            =  \frac{1}{\sqrt{2}} \Big( |1\rangle + {\rm i} |2\rangle \Big) 
\end{eqnarray}

\subsection{Das Mach-Zehnder-Interferometer}

Mit dem beschriebenen Formalismus k\"onnen wir nun das Mach-Zehnder-Interferometer
behandeln.\index{Mach-Zehnder-Interferometer} 
Es besteht aus zwei Strahlteilern: Der erste teilt einen Lichtstrahl in zwei Anteile,
der zweite f\"uhrt die beiden Strahlg\"ange wieder zusammen
und bringt sie zur Interferenz (siehe Abb.\ \ref{fig_BK_MZ}). Es entspricht im Prinzip einem Doppelspalt, 
allerdings k\"onnen die beiden \glqq Spalte\grqq\ nun nahezu beliebig weit auseinander gebracht 
werden. Wichtig ist jedoch, dass die optischen Wegl\"angen der beiden Strahlen innerhalb der
Koh\"arenzl\"ange des Lichts gleich sind. 

\begin{SCfigure}[50][htb] 
\begin{picture}(285,170)(-5,0)
\thicklines
\put(40,10){\line(1,1){20}}
\put(200,10){\line(1,1){20}}
\put(201,9){\line(1,1){20}}
\put(40,120){\line(1,1){20}}
\put(39,121){\line(1,1){20}}
\put(200,120){\line(1,1){20}}
\thinlines
\put(40,10){\line(1,0){20}}
\put(40,10){\line(0,1){20}}
\put(60,10){\line(0,1){20}}
\put(40,30){\line(1,0){20}}

\put(200,120){\line(1,0){20}}
\put(200,120){\line(0,1){20}}
\put(220,120){\line(0,1){20}}
\put(200,140){\line(1,0){20}}

\put(0,20){\line(1,0){210}}
\put(50,20){\line(0,1){110}}
\put(210,20){\vector(0,1){150}}
\put(50,130){\line(1,0){40}}
\put(100,130){\vector(1,0){150}}
\put(90,120){\line(1,0){10}}
\put(90,120){\line(0,1){20}}
\put(100,120){\line(0,1){20}}
\put(90,140){\line(1,0){10}}
%
\put(20,20){\vector(1,0){0}}
\put(150,20){\vector(1,0){0}}
\put(150,130){\vector(1,0){0}}
\put(50,70){\vector(0,1){0}}
\put(210,70){\vector(0,1){0}}
%
\put(25,30){\makebox(0,0){$| 1 \rangle$}}
\put(140,30){\makebox(0,0){$\frac{1}{\sqrt{2}}\,| 1 \rangle$}}
\put(64,70){\makebox(0,0){$\frac{\rm i}{\sqrt{2}}\,| 2 \rangle$}}
\put(140,140){\makebox(0,0){$\frac{\rm i}{\sqrt{2}}{\rm e}^{{\rm i}\delta}\,| 2 \rangle$}}
\put(224,70){\makebox(0,0){$\frac{1}{\sqrt{2}}\,| 1 \rangle$}}
\put(240,160){\makebox(0,0){$\frac{1}{2}(1 -{\rm e}^{{\rm i}\delta})| 1 \rangle$}}
\put(250,122){\makebox(0,0){$\frac{\rm i}{2}(1+{\rm e}^{{\rm i}\delta})| 2 \rangle$}}
\put(95,130){\makebox(0,0){${\scriptstyle {\rm e}^{{\rm i}\delta}}$}}
%
%\put(210,170){\makebox(0,0){\footnotesize D dunkel}}
%\put(270,130){\makebox(0,0){\footnotesize C hell}}
\put(50,4){\makebox(0,0){\footnotesize Strahlteiler}}
\put(185,114){\makebox(0,0){\footnotesize Strahlteiler}}
\put(70,146){\makebox(0,0){\footnotesize Spiegel}}
\put(230,37){\makebox(0,0){\footnotesize Spiegel}}
%
\end{picture} 
\caption{\label{fig_BK_MZ}%
Das Mach-Zehnder-Inter\-fero\-meter als 2-Zustands\-system. Ein einzelnes Photon kann 
entweder im Strahl 1 oder im Strahl 2 nachgewiesen werden. Quantenmechanisch kann ein Superpositionszustand
aus den beiden Zust\"anden vorliegen. In Stahl 2 befindet sich noch ein \glqq Phasenschieber\grqq, mit dem die
optische Wegl\"ange des Stahls 2 relativ zu Strahl 1 um $\delta$ ver\"andert werden kann.} 
\end{SCfigure}

Wir beschreiben dieses System f\"ur Einzelphotonen als Zwei-Zustandssystem: Bei einer Messung kann sich
das Photon entweder im Strahl 1 oder im Strahl 2 befinden. Dementsprechend bezeichnen wir den
Zustand des Photons mit $|1\rangle$ bzw.\ $|2\rangle$. Diese beiden Zust\"ande sind orthogonal, sodass
$\langle 1|2\rangle = \langle 2|1\rangle=0$. 
Wir k\"onnen nun den Prozess, bei dem
ein Photon das Mach-Zehnder-Interferometer durchquert, schrittweise beschreiben: Wir bezeichnen den
Strahl, durch den das Photon in den ersten Strahlteiler dringt, als
Strahl 1. Das Photon befindet sich somit im Zustand $|1\rangle$. Die
Wirkung des ersten Strahlteilers ist:
\begin{equation}
  |1\rangle \longrightarrow \frac{1}{\sqrt{2}} \Big(  | 1 \rangle \langle 1 | + | 2 \rangle \langle 2 | +
          {\rm i}  | 1 \rangle \langle 2 | + {\rm i} | 2 \rangle \langle 1 |  \Big)~ |1\rangle =
       \frac{1}{\sqrt{2}} \Big(  | 1 \rangle  + {\rm i} | 2 \rangle \Big) \, . 
\end{equation}
Ohne eine Messung erhalten wir einen Superpositionszustand, bei dem das Photon eine Komponente
in Strahl 1 und eine Komponente in Strahl 2 hat.
Die beiden Spiegel, an denen die Strahlen abgelenkt werden, tragen einen gemeinsamen Faktor
i bei, den wir unber\"ucksichtigt lassen, da er zum Ergebnis nicht beitr\"agt. 
Allerdings sollten wir eine relative Phasenverschiebung zwischen den beiden Strahlg\"angen 
ber\"ucksichtigen. Dies geschieht durch den Operator
$\big( |1\rangle \langle 1 | + {\rm e}^{{\rm i}\delta} |2\rangle \langle 2 | \big)$. Damit wird aus dem Zustand:
\begin{equation}
    \frac{1}{\sqrt{2}} \Big(  | 1 \rangle  + {\rm i} | 2 \rangle \Big)  \longrightarrow
     \Big( |1\rangle \langle 1 | + {\rm e}^{{\rm i}\delta} |2\rangle \langle 2 | \Big) 
      \frac{1}{\sqrt{2}} \Big(  | 1 \rangle  + {\rm i} | 2 \rangle \Big)  =   
      \frac{1}{\sqrt{2}} \Big(  | 1 \rangle  + {\rm i} {\rm e}^{{\rm i}\delta} | 2 \rangle \Big) 
\end{equation}
Die Superposition der beiden Zust\"ande
trifft nun auf den zweiten Strahlteiler:
\begin{eqnarray}
       \frac{1}{\sqrt{2}} \Big(  | 1 \rangle  + {\rm i} {\rm e}^{{\rm i}\delta} | 2 \rangle \Big) & \longrightarrow &
         \frac{1}{\sqrt{2}} \Big(  | 1 \rangle \langle 1 | + | 2 \rangle \langle 2 | +
          {\rm i}  | 1 \rangle \langle 2 | + {\rm i} | 2 \rangle \langle 1 |  \Big)~  \frac{1}{\sqrt{2}} 
          \Big(  | 1 \rangle  + {\rm i} {\rm e}^{{\rm i}\delta} | 2 \rangle \Big)  \\
         & = &  \frac{1}{2} \Big(  | 1 \rangle  + {\rm i} {\rm e}^{{\rm i}\delta} | 2 \rangle  
                        + {\rm i}^2 {\rm e}^{{\rm i}\delta} |1\rangle +{\rm i} |2\rangle \Big) \\
         &=&   \frac{1}{2} \Big(  | 1 \rangle  + {\rm i}{\rm e}^{{\rm i}\delta} | 2 \rangle  - {\rm e}^{{\rm i}\delta} |1\rangle +{\rm i} |2\rangle \Big) \\
         & = & \frac{1}{2} \Big( (1-{\rm e}^{{\rm i}\delta}) |  1 \rangle + {\rm i} (1 + {\rm e}^{{\rm i}\delta}) |2\rangle \Big)  \, . 
\end{eqnarray}
Je nach dem Wert von $\delta$ verschwindet der eine oder der andere Zustand: F\"ur $\delta=0~({\rm mod}\, 2\pi)$ 
gibt es nur den Zustand $|2\rangle$, f\"ur $\delta = \pi~ ({\rm mod}\,2\pi)$ gibt es nur den Zustand $|1\rangle$. 
F\"ur die Prozesse $|1\rangle \rightarrow \fbox{MZ} \rightarrow |j\rangle$, wobei $\fbox{MZ}$ f\"ur das
Mach-Zehnder-Interferometer steht, erhalten wir die Wahrscheinlichkeitsamplituden:
\begin{equation}
   \langle 1 | ~ \fbox{MZ} ~ |  1\rangle  = \frac{1}{2} (1-{\rm e}^{{\rm i}\delta}) \hspace{0.6cm}
   \langle 2 | ~ \fbox{MZ} ~ |  1\rangle = \frac{1}{2} {\rm i} (1+{\rm e}^{{\rm i}\delta})   
\end{equation}
und wenn das Photon durch den Strahl 2 in das Mach-Zehnder-Interferometer tritt entsprechend:
\begin{equation}
   \langle 2 | ~ \fbox{MZ} ~ |  2\rangle  = \frac{1}{2} (1-{\rm e}^{{\rm i}\delta}) \hspace{0.6cm}
   \langle 1 | ~ \fbox{MZ} ~ |  2\rangle = \frac{1}{2} {\rm i} (1+{\rm e}^{{\rm i}\delta})   \, .
\end{equation}
F\"ur die Wahrscheinlichkeiten dieser Prozesse folgt: 
\begin{equation}
    w(1\rightarrow 1)  = w(2\rightarrow 2) = \frac{1}{2} (1-\cos \delta) \hspace{0.6cm}
    w(1 \rightarrow 2) = w(2\rightarrow 1) = \frac{1}{2} (1+\cos\delta)   
\end{equation}


F\"ur das \glqq Knallerexperiment\grqq\ m\"ussen\index{Knallerexperiment} 
wir den Operator $\big(|1\rangle \langle 1| + 0\cdot |2\rangle \langle 2| \big)$
dazwischenschieben. Dies bedeutet, dass ein Photon, welches sich in Strahl 2 befindet, absorbiert wird
(weil es auf die Bombe getroffen ist). Dies ist kein unit\"arer Operator mehr sondern ein Projektionsoperator auf
den Zustand $|1\rangle$. Die Bombe hat also einen \"ahnlichen Effekt, wie ein Filter: Befindet sich das Photon
in Zustand $|2\rangle$, wird es absorbiert. Als Ergebnis erhalten wir nun $\frac{1}{2} \big( |1\rangle + {\rm i} |2\rangle \big)$. 
Mit der jeweiligen Wahrscheinlichkeit $1/4$ landet das Photon also schlie\ss lich in Detektor 1 oder in Detektor 2. Mit
der Wahrscheinlichkeit $1/2$ wurde es von der Bombe absorbiert.  

\section{Erwartungswerte}

Letztendlich ist das Ziel in der Quantentheorie (und allgemeiner in der Physik) immer
die Vorhersage f\"ur das Eintreffen bestimmter Ereignisse. 
In der Quantentheorie besteht\index{Erwartungswerte}
diese Vorhersage meist in einer Wahrscheinlichkeitsaussage. Bisher haben wir die Bra-Ket-Notation
verwendet, um Prozesse zu beschreiben und wir haben gesehen, wie man f\"ur einen Prozess eine
Wahrscheinlichkeitsamplitude und damit schlie\ss lich eine Wahrscheinlichkeit berechnen kann.

In diesem Abschnitt betrachten wir Erwartungswerte von Observablen.
Der Grund f\"ur diese Unterscheidung ist im Wesentlichen folgender: Die selbstadjungierten
Operatoren $A$, die einer Observablen ${\cal A}$, definiert durch die Messvorschrift,
entsprechen, beschreiben nicht den zeitlichen Ablauf der Messung und die Dynamik
des Messvorgangs, sondern sie enthalten lediglich die Informationen, die bei einer
Messung an einem System gewonnen werden k\"onnen: die m\"oglichen Messwerte 
$\{ \lambda_n\}$ und die Zust\"ande $\{ |\lambda_n\rangle \}$, 
durch welche die Systeme nach einer Messung und der Feststellung
eines bestimmten Messresultats beschrieben werden. Sie beschreiben also nicht
den Mess\-prozess. Andere Operatoren, beispielsweise die unit\"aren Operatoren,
beschreiben Prozesse, bei denen sich die Zust\"ande eines Systems ver\"andern.
Eine Zwischenrolle nehmen Projektionsoperatoren ein, die einerseits bestimmte
absorbierende Prozesse beschreiben, bei denen das physikalische Systeme durch
einen Filter tritt, andererseits aber auch als Observable aufgefasst werden k\"onnen.

Eine Observable ${\cal A}$, definiert durch ein Messprotokoll, wird in der Quantentheorie 
durch einen selbstadjungierten Operator $A$\index{Observable}
dargestellt. Aus der Mathematik ist bekannt, dass ein selbstadjungierter Operator immer
reelle Eigenwerte hat und dass die Eigenr\"aume zu verschiedenen Eigenwerten
orthogonal sind. Au\ss erdem sind die Eigenr\"aume vollst\"andig, d.h., sie spannen den gesamten
Vektorraum auf. Auf die Besonderheiten bzw.\ Verallgemeinerungen dieser Aussagen f\"ur
Operatoren mit einem kontinuierlichen Spektrum in unendlich dimensionalen Hilbert-R\"aumen   
soll hier nicht eingegangen werden.

Bekannterma\ss en sind die Eigenwerte von Observablen die m\"oglichen Messwerte und
die zugeh\"origen Eigenr\"aume die entsprechenden Zust\"ande. Sofern ein Eigenwert entartet
ist, d.h., der zugeh\"orige Eigenraum hat mehr als eine Dimension, repr\"asentiert jeder eindimensionale
Teilraum dieses Eigenraums einen Zustand zu diesem Eigenwert. In allen F\"allen l\"asst sich
eine Orthonormalbasis f\"ur den Vektorraum aus normierten Eigenvektoren eines selbst-adjungierten
Operators bilden.

Die Eigenwertgleichung eines Operators $A$ zu einem Eigenwert $\lambda_n$ lautet:
\begin{equation}
                                A | \lambda_n \rangle = \lambda_n | \lambda_n \rangle  \, , 
\end{equation}
wobei $|\lambda_n\rangle$ ein Eigenvektor (normiert auf den Betrag eins) von $A$ zum Eigenwert
$\lambda_n$ ist. Die Tatsache, dass bei selbstadjungierten Operatoren die Eigenvektoren
zu verschiedenen Eigenwerten orthogonal sind, bedeutet
\begin{equation}
                 \langle \lambda_m | \lambda_n \rangle = \delta_{mn} =
                 \left\{ \begin{array}{ll}  1 & m=n \\ 0 & \mbox{sonst}  \end{array} \right.  \, . 
\end{equation}
Die symbolische Interpretation dieser Bedingung ist, dass ein System, das in einem Zustand
$|\lambda_n\rangle$ pr\"apariert wurde, unmittelbar nach dieser Pr\"aparation mit Sicherheit 
nicht im Zustand $|\lambda_m\rangle$
gemessen wird, sofern $m\neq n$, und mit Sicherheit in diesem Zustand gemessen wird, sofern
$m=n$. 

Der Operator $A$ l\"asst sich durch die Projektionsoperatoren zu seinen Eigenwerten
darstellen:
\begin{equation}
\label{eq_Spektralzerlegung}
          A = \sum_n \lambda_n  | \lambda_n \rangle \langle \lambda_n |  = \sum_n \lambda_n P_n \, .
\end{equation}
Zum Beweis \"uberzeuge man sich, dass dieser Operator die Eigenwertgleichungen erf\"ullt, und
da die Eigenvektoren den Vektorraum aufspannen, ist damit die Gleichheit der linearen Operatoren
gezeigt. Diese Darstelllung bezeichnet man auch als die Spektralzerlegung eines selbst-adjungierten
Operators.\index{Spektralzerlegung}

Der Ausdruck $\langle \psi |A| \psi \rangle$ ist der Erwartungswert von ${\cal A}$
im Zustand $\psi$, d.h., wenn die Observable
${\cal A}$ an sehr vielen Systemen, die alle im Zustand $\psi$ pr\"apariert wurden, gemessen wird,
erh\"alt man als Erwartungswert f\"ur die Messwerte diese Gr\"o\ss e:
\begin{equation}
     \langle \psi | A | \psi \rangle  
         = \sum_n \lambda_n  \langle \psi| \lambda_n \rangle \langle \lambda_n | \psi \rangle
         = \sum_n \lambda_n  w(\psi \rightarrow n)  = \langle {\cal A} \rangle \, .
\end{equation}
Die Summe \"uber alle Messwerte, wobei jeder Messwert mit der Wahrscheinlichkeit seines 
Nachweises gewichtet wird, ist gleich dem Erwartungswert dieser Messwerte. 

Wir k\"onnen diese \"Uberlegungen auch umkehren: Wenn ein Operator $A$ die Eigenschaft
haben soll, dass f\"ur jeden beliebigen Zustand $\psi$ der Ausdruck $\langle \psi |A| \psi \rangle$
der Erwartungswert f\"ur die zugeh\"orige Observable ${\cal A}$ ist, dann muss $A$ die Form in 
Gl.\ \ref{eq_Spektralzerlegung} haben.  
\index{Bra-Ket-Notation|)}



\begin{thebibliography}{99}
\bibitem{Dirac} Dirac, P.A.M.; {\em A new notation for quantum mechanics}; Mathematical
        Proceedings of the Cambridge Philosophical Society, {\bf 35} (1939) 416-418.

\end{thebibliography}

\section{Anhang}
\subsection{Unit\"are Matrizen}
\label{A_BK_Unitary}
Eine allgemeine $2\times 2$ unit\"are Matrix mit der Determinante ${\rm e}^{{\rm i}\varphi}$
l\"asst sich folgenderma\ss en parametrisieren und zerlegen:
\begin{eqnarray}
     U &=& {\rm e}^{{\rm i}\varphi/2} 
     \left( \begin{array}{cc}  {\rm e}^{{\rm i}(\alpha+\beta)} \cos \theta & {\rm e}^{{\rm i}(\alpha - \beta)} \sin \theta \\
     - {\rm e}^{-{\rm i}(\alpha - \beta)} \sin \theta & {\rm e}^{-{\rm i}(\alpha+\beta)} \cos \theta \end{array} \right)  \\
     &=&
     \left( \begin{array}{cc}  {\rm e}^{{\rm i}\varphi/2} & 0    \\
     0  & {\rm e}^{{\rm i}\varphi/2} \end{array} \right) 
          \left( \begin{array}{cc}  {\rm e}^{{\rm i}\alpha} & 0    \\
     0  & {\rm e}^{-{\rm i}\alpha} \end{array} \right) 
          \left( \begin{array}{cc}  \cos \theta & \sin \theta    \\
     - \sin \theta  & \cos \theta \end{array} \right) 
               \left( \begin{array}{cc}  {\rm e}^{{\rm i}\beta} & 0    \\
     0  & {\rm e}^{-{\rm i}\beta} \end{array} \right) \, .
\end{eqnarray}
F\"ur einen Strahlteiler, bei dem die Intensit\"at eines einfallenden Strahls zur H\"alfte
auf die beiden ausfallenden Strahlen verteilt werden soll, ist $\cos \theta = \sin \theta = 1/\sqrt{2}$
bzw.\ $\theta = \pi/2$. Ansonsten kann man \"uber den Winkel $\theta$ die relativen Intensit\"aten 
(diese sind $I_\perp=\sin^2\theta$ f\"ur den abgelenkten Strahl und $I_\parallel = \cos^2 \theta$ f\"ur den
geradeaus durchgelassenen Strahl) parametrisieren. Eine unit\"are Matrix der Form
\begin{equation}
           U = \left( \begin{array}{cc}  {\rm e}^{{\rm i}\psi} & 0  \\  0  & {\rm e}^{{\rm i}\phi} \end{array} \right) 
\end{equation}
bedeutet lediglich eine Phase f\"ur jeden Teilstrahl (vor bzw.\ hinter dem Strahlteiler) und l\"asst
sich physikalisch durch eine einfache Wegl\"angenver\"anderung (oder durch d\"unne Glaspl\"attchen)
des entsprechenden Strahls erreichen. Wir k\"onnen einen Strahlteiler somit durch die Matrix
($\varphi=\alpha=\beta=0$)
\begin{equation}
          U_{\rm ST~1} =  \frac{1}{\sqrt{2}}  \left( \begin{array}{cc}  1 & 1  \\ -1  & 1 \end{array} \right) 
\end{equation}
darstellen. W\"ahlen wir jedoch in obiger Darstellung $\varphi=0$, $\alpha = \pi/4$ und $\beta=-\pi/4$, so
erhalten wir die Form
\begin{equation}
          U_{\rm ST~2} =  \frac{1}{\sqrt{2}}  \left( \begin{array}{cc}  1 & {\rm i}  \\ {\rm i}  & 1 \end{array} \right) 
\end{equation}
Letztendlich k\"onnen wir also durch geeignete Phasenverschiebungen vor und hinter einem
Strahlteiler und durch Variation der Intensit\"atsanteile (durchgelassen/abgelenkt) jede beliebige 
$2\times 2$ unit\"are 
Matrix durch einen Strahlteiler realisieren.


%\end{document}

