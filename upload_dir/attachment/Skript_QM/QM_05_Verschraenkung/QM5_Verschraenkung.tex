\documentclass[german,10pt]{book}      
\usepackage{makeidx}
\usepackage{babel}            % Sprachunterstuetzung
\usepackage{amsmath}          % AMS "Grundpaket"
\usepackage{amssymb,amsfonts,amsthm,amscd} 
\usepackage{mathrsfs}
\usepackage{rotating}
\usepackage{sidecap}
\usepackage{graphicx}
\usepackage{color}
\usepackage{fancybox}
\usepackage{tikz}
\usetikzlibrary{arrows,snakes,backgrounds}
\usepackage{hyperref}
\hypersetup{colorlinks=true,
                    linkcolor=blue,
                    filecolor=magenta,
                    urlcolor=cyan,
                    pdftitle={Overleaf Example},
                    pdfpagemode=FullScreen,}
%\newcommand{\hyperref}[1]{\ref{#1}}
%
\definecolor{Gray}{gray}{0.80}
\DeclareMathSymbol{,}{\mathord}{letters}{"3B}
%
\newcounter{num}
\renewcommand{\thenum}{\arabic{num}}
\newenvironment{anmerkungen}
   {\begin{list}{(\thenum)}{%
   \usecounter{num}%
   \leftmargin0pt
   \itemindent5pt
   \topsep0pt
   \labelwidth0pt}%
   }{\end{list}}
%
\renewcommand{\arraystretch}{1.15}                % in Formeln und Tabellen   
\renewcommand{\baselinestretch}{1.15}                 % 1.15 facher
                                                      % Zeilenabst.
\newcommand{\Anmerkung}[1]{{\begin{footnotesize}#1 \end{footnotesize}}\\[0.2cm]}
\newcommand{\comment}[1]{}
\setlength{\parindent}{0em}           % Nicht einruecken am Anfang der Zeile 

\setlength{\textwidth}{15.4cm}
\setlength{\textheight}{23.0cm}
\setlength{\oddsidemargin}{1.0mm} 
\setlength{\evensidemargin}{-6.5mm}
\setlength{\topmargin}{-10mm} 
\setlength{\headheight}{0mm}
\newcommand{\identity}{{\bf 1}}
%
\newcommand{\vs}{\vspace{0.3cm}}
\newcommand{\noi}{\noindent}
\newcommand{\leer}{}

\newcommand{\engl}[1]{[\textit{#1}]}
\parindent 1.2cm
\sloppy

         \begin{document}  \setcounter{chapter}{6}
\newcommand{\solution}[1]{#1}
%\newcommand{\solution}[1]{}

\chapter{Verschr\"ankung}
% Kap x
\label{chap_Entanglement}

M\"ochte man in der Quantentheorie Systeme beschreiben, die aus
mehreren Teilsystemen, beispielsweise mehreren Teilchen, bestehen, so ergeben
sich zwei Besonderheiten: (1) Der Zusammenhang zwischen dem Spin
von Teilchen und ihrer Statistik (das Spin-Statistik-Theorem) f\"ur
Systeme aus identischen Teilchen, was beispielsweise f\"ur Elektronen das Pauli'sche
Ausschlie\ss ungsprinzip impliziert, und
(2) die sogenannten Quantenkorrelationen von verschr\"ankten
Systemen. Gerade die M\"oglichkeit solcher neuartiger
Korrelationen, die sich nicht auf eine direkte kausale 
Beziehung und auch nicht im klassischen Sinn auf eine gemeinsame Vergangenheit
mit einer durchg\"angigen Kette von Ursache-Wirkungs-Relationen
zur\"uckf\"uhren lassen, ist immer noch Gegenstand grundlegender
Diskussionen. Auf das Spin-Statistik-Theorem wird an anderer Stelle
eingegangen. In diesem Kurztext geht es in erster Linie um das Konzept
der Verschr\"ankung. 

\section{Das Ph\"anomen der Verschr\"ankung}

Eine Voraussetzung, um \"uberhaupt von Verschr\"ankung sprechen zu k\"onnen,
sind zwei Systeme unabh\"angiger Freiheitsgrade. Nur dann ist es sinnvoll von Zust\"anden
zu sprechen, f\"ur die diese beiden Freiheitsgrade miteinander verschr\"ankt sind. 
\glqq Unabh\"angig\grqq\ bedeutet hier,
dass s\"amtliche Observablen des einen Sys\-tems von Freiheitsgraden mit allen Observablen 
des anderen Systems kommutieren, d.h., je zwei Observablen, von denen sich 
eine auf das eine System und die andere auf das andere System beziehen, lassen sich gleichzeitig
messen.

Das typische Beispiel sind die Freiheitsgrade zu zwei verschiedenen Teilchen: 
S\"amtliche Observablen, die sich an dem einen Teilchen messen lassen (Ort, Impuls,
Spin, etc.) kommutieren mit allen Observablen, die sich an dem anderen Teilchen
messen lassen. Aber auch die r\"aumlichen Observablen (Ort, Impuls und Funktionen
dieser beiden Observablen) eines Teilchens kommutieren mit seinen inneren
Freiheitsgraden (z.B.\ den Spinkomponenten). Auch in solchen F\"allen kann man
von Verschr\"ankung (z.B.\ zwischen dem Ort eines Teilchens und seinem Spin) sprechen.
Zun\"achst beschr\"anken wir uns jedoch auf den Fall von zwei Teilchen.

Da an jedem der beiden Teilchen gleichzeitig jeweils eine Observable gemessen werden
kann, k\"onnen wir auch nach Korrelationen zwischen den Ergebnissen solcher
Messungen fragen. Betrachten wir als Beispiel zwei Elektronen in einem gro\ss en
Abstand (sodass diese beiden Elektronen aufgrund ihrer verschiedenen Orte 
unterscheidbar sind), dann k\"onnen wir an beiden Elektronen gleichzeitig eine
beliebige Spin-Komponente messen (z.B.\ mit einem Stern-Gerlach-Experiment). 
Wir bezeichnen mit $w(a,\pmb{n}_1;b,\pmb{n}_2)$ die Wahrscheinlichkeit, an Teilchen
1 den Wert $a=\pm1$ (in Einheiten von $\hbar/2$) 
bei einer Messung des Spins in $\pmb{n}_1$-Richtung zu erhalten
und gleichzeitig an Teilchen 2 den Wert $b=\pm1$ bei einer Messung des Spins in 
$\pmb{n}_2$-Richtung zu erhalten. Sind diese Wahrscheinlichkeiten 
unabh\"angig von einander, d.h.\ sind diese Wahrscheinlichkeiten das
Produkt der Einzelwahrscheinlichkeiten, 
\begin{equation}
          w(a,\pmb{n}_1;b,\pmb{n}_2) = w_1(a,\pmb{n}_1) w_2(b,\pmb{n}_2) \, ,
\end{equation}
dann liegt keine Verschr\"ankung vor. 
Sind die Wahrscheinlichkeiten nicht unabh\"angig, d.h., sind die konditionierten
Wahrscheinlichkeiten f\"ur ein Ereignis an einem Teilchen abh\"angig von dem
Ereignis, das an dem anderen Teilchen gemessen wurde, \emph{und} handelt es sich
um einen \emph{reinen} Zustand der beiden Teilchen, dann liegt eine Verschr\"ankung vor. 

Das Ph\"anomen konditionierter Wahrscheinlichkeiten gibt es auch in der klassischen
Physik. Man denke an folgende Situation: Jeden Tag werden an zwei Personen zwei Emails
verschickt, die einen identischen Inhalt haben -- der Einfachheit halber nur
entweder die Zahl $+1$ oder die Zahl $-1$. Sobald eine Person 
ihre Email liest und somit die Zahl in ihrer Email kennt, wei\ss\ sie, was die andere
Person liest oder lesen wird. Das Lesen einer Email
erlaubt eine neue Vorhersage f\"ur den Inhalt der anderen Email. 

Allerdings handelt es sich hier um ein Gemisch von Emailpaaren: ein Gemisch aus 
Emailpaaren, bei denen beide Emails den Inhalt \glqq $+1$\grqq\ haben, und Emailpaaren, bei denen
beide Emails den Inhalt \glqq$-1$\grqq\ haben. 
H\"atten alle Emails immer
denselben Inhalt, w\"usste man schon im Voraus, was die andere Person lesen wird.
Durch das Lesen der eigenen Email w\"urde in diesem Fall diese Vorhersage nicht ge\"andert. Nur weil
es sich um Emails mit potenziell verschiedenen Inhalten handelt (entweder beide $+1$ oder beide
$-1$), kommt es zu einer wirklichen Vorhersage, nachdem die eine Email gelesen wurde.

Betrachten wir ganz konkret einen sogenannten $\Phi^+$ Bell-Zustand, der ein verschr\"ankter
reiner Zustand ist:
\begin{equation}
            | \Phi^+ \rangle = \frac{1}{\sqrt{2}} 
            \big(  | \uparrow_z \rangle_1 |\uparrow_z \rangle_2 +  | \downarrow_z \rangle_1 |\downarrow_z \rangle_2 \big) \, .
\end{equation}
Hierbei bezeichnet $|\uparrow_z\rangle_1$ einen Eigenzustand von Teilchen 1 zum Spin in
$z$-Richtung mit Eigenwert $+\hbar/2$. Entsprechend bezeichnet $\downarrow_z$ den Eigenwert $-\hbar/2$
bez\"uglich der $z$-Richtung. Das Gleiche gilt f\"ur Teilchen 2, gekennzeichnet durch den Index an der Klammer. 

Nimmt man an einem der beiden Teilchen, die von diesem Zustand beschrieben werden, eine Spinmessung
entlang der $z$-Richtung
vor, kann man vorhersagen, dass eine entsprechende Spinmessung am anderen Teilchen
denselben Wert ergibt. Die konditionierten Wahrscheinlichkeiten sind
also gleich $+1$ f\"ur gleiche Spinrichtungen und $0$ f\"ur entgegengesetzte. Die
urspr\"ungliche Wahrscheinlichkeit, $0,5$ f\"ur jede der beiden Spinrichtungen, wird also
durch die Kenntnis des Ergebnisses der ersten Messung revidiert, indem man nun den
Spin f\"ur das andere Teilchen mit Sicherheit vorhersagen kann.

Dasselbe Ergebnis erh\"alt man allerdings auch, wenn ein gleiches Gemisch
aus Zust\"anden $| \uparrow_z \rangle_1 |\uparrow_z \rangle_2$ und 
$| \downarrow_z \rangle_1 |\downarrow_z \rangle_2$ vorliegt. In diesem Fall w\"urde man
den Zustand allerdings durch eine Dichtematrix ausdr\"ucken. Dieser Fall ist vergleichbar
mit einer klassischen Korrelation, die eine gemeinsame Ursache hat. Er entspricht den
beiden Arten von Emails (beide mit Inhalt $+1$ oder beide mit Inhalt $-1$). 

Wie kann man diese beiden F\"alle experimentell
unterscheiden? Eine Unterscheidung ist nur m\"oglich, indem man auch andere Observablen an
dem System aus zwei Teilchen misst. Beispielsweise ist der $\Phi^+$-Zustand ein Zustand, der
sich unter reellen Koordinatentransformationen (Drehungen des Koordinatensystems um die
Ausbreitungsachse der Teilchen) nicht
\"andert. Die Korrelation gilt bez\"uglich jeder beliebigen Richtung in der $xz$-Ebene, in 
welcher der Spin gemessen wird \hyperref[sec_Entanglement_A]{(Herleitung)}. Diese
Aussage gilt nicht f\"ur das Gemisch. Misst man dort die Spin-Komponenten beispielsweise
in $x$-Richtung, erh\"alt man auch Ergebnisse, bei denen die Werte auf beiden Seiten verschieden
sind. Dies zeigt auch eine einfache Rechnung, wenn man die $z$-Eigenzust\"ande nach 
$x$-Eigenzust\"anden entwickelt \hyperref[sec_Entanglement_B]{(Herleitung 2)}. 
Eine Korrelation bez\"uglich aller (reellen) Richtungen l\"asst sich
nicht durch ein echtes Gemisch realisieren.

\section{Das Tensorprodukt von Vektorr\"aumen}
\label{sec_Tensorprodukt}

In der klassischen Mechanik ist der Phasenraum von $N$
Teilchen das kartesische Produkt der Einteilchenphasenr\"aume,
d.h., ein reiner $N$-Teilchenzustand entspricht einem Punkt
in einem $6N$-dimensionalen Phasenraum: jeweils drei Orts-
und Impulskomponenten f\"ur jedes Teilchen.  
In der Quantentheorie wird der Zustandsraum \"uber
einen Hilbert-Raum definiert. Bei zwei Vektorr\"aumen 
ist der Produktraum das sogenannte\index{Tensorprodukt} 
{\em Tensorprodukt} der
beiden einzelnen Vektorr\"aume: Dazu bildet man zun\"achst das 
kartesische Produkt von zwei Basen der Teilr\"aume; dies
definiert eine Basis des Produktraums. Der Produkt-Vektorraum
besteht aus s\"amtlichen Linearkombinationen, die von
dieser Basis aufgespannt werden.

In diesem Abschnitt wird das Tensorprodukt auf vergleichsweise abstraktem
Niveau beschrieben. Wer zun\"achst ein konkretes Beispiel m\"ochte, kann
Abschnitt \ref{sec_Entanglement_Produkt} vorziehen.

%\begin{definition}
Seien ${\cal H}_1$ und ${\cal H}_2$ zwei Vektorr\"aume \"uber demselben
K\"orper mit
jeweiligen Basisvektoren $\{|e_i\rangle \}_{i\in I}$ und $\{|f_j\rangle\}_{j\in J}$
(die Indexmengen $I$ f\"ur $i$ und $J$ f\"ur $j$ m\"ussen nicht gleich sein). 
Das {\em Tensorprodukt} ${\cal H}={\cal H}_1\otimes{\cal H}_2$
ist definiert als der Vektorraum, der durch die Paare von Basisvektoren
$\{ |e_i,f_j\rangle \}_{i\in I, j\in J}$ aufgespannt wird. Ein beliebiger Vektor 
$|x\rangle \in {\cal H}$ l\"asst sich somit immer in der Form
\begin{equation}
     |x \rangle = \sum_{i,j} x_{ij} |e_i,f_j\rangle 
\end{equation}
darstellen.
%\end{definition}

Statt $|e_i,f_j\rangle$ schreibt man 
manchmal auch $|e_i\rangle |f_j\rangle$
oder $|e_i\rangle \otimes |f_j\rangle$. Seien
$|a\rangle=\sum_i a_i |e_i\rangle \in {\cal H}_1$ 
und $|b\rangle = \sum_j b_j |f_j\rangle \in {\cal H}_2$ 
Vektoren aus den einzelnen Vektorr\"aumen, 
dann ist
\begin{equation}
     |a\rangle \otimes |b\rangle = \Big( \sum_i a_i |e_i\rangle \Big)
       \otimes \Big( \sum_j b_j | f_j\rangle \Big) =
       \sum_{i,j} a_i b_j \, |e_i, f_j\rangle
\end{equation}
das Tensorprodukt dieser beiden Vektoren.
Haben ${\cal H}_1$ und ${\cal H}_2$ jeweils die Dimensionen
$d_1$ bzw.\ $d_2$, so hat ${\cal H}_1\otimes {\cal H}_2$
die Dimension $d_1 \cdot d_2$. 

Man unterscheide zwischen dem
Tensorprodukt von zwei Vektorr\"aumen und dem kartesischen
Produkt der Vektorr\"aume (das man auch als direkte Summe
${\cal H}_1 \oplus{\cal H}_2$ schreibt). Das kartesische Produkt
besteht aus allen Paaren $(x,y)$ von Vektoren $x\in {\cal H}_1$ 
und $y\in {\cal H}_2$ mit komponentenweiser
Addition und hat die Dimension $d_1+d_2$. F\"ur eine Basis $\{|e_i\rangle\}$ von
${\cal H}_1$ und eine Basis $\{|f_j\rangle\}$ von ${\cal H}_2$ ist die 
mengentheoretische Vereinigung $\{|e_i \rangle\}\cup \{|f_j\rangle\}$ eine Basis
in ${\cal H}_1\oplus {\cal H}_2$ und jeder Vektor $z$ in ${\cal H}_1\oplus {\cal H}_2$
l\"asst sich in der Form
\begin{equation}
       z = \sum_{i=1}^{d_1} x_i |e_i\rangle + \sum_{j=1}^{d_2} y_j |f_j\rangle
\end{equation}
schreiben. Diese beiden Konstruktionen eines Vektorraums aus zwei
gegebenen Vektorr\"aumen sollten nicht verwechselt werden.

Ist auf den beiden Vektorr\"aumen jeweils ein Skalarprodukt definiert, so kann man
auch auf dem Produktraum ein Skalarprodukt definieren:
F\"ur zwei Vektoren des Tensorproduktraums
\begin{equation}
|x\rangle, |y\rangle \in {\cal H} ;~~~
|x\rangle = \sum_{i,j} x_{ij}|e_i,f_j\rangle ~;~~~~  
|y\rangle = \sum_{k,l} y_{kl}|e_k,f_l\rangle  
\end{equation}
gilt:
\begin{equation}
  \langle y | x \rangle = \sum_{i,j,k,l} 
         y^*_{lk} x_{ij} \,  \langle e_i|e_k\rangle \langle f_j|f_l\rangle
\end{equation}
und, sofern es sich bei $\{ | e_k\rangle \}$ und $|f_l\rangle \}$ um Orthonormalbasen handelt
\begin{equation}
  \langle y | x \rangle = \sum_{i,j} 
         y^*_{ji} x_{ij}   \, . 
\end{equation}
Auf diese Weise wird auch der Produktraum von zwei Hilbert-R\"aumen wieder zu
einem Hilbert-Raum. 

Diese Definition des Tensorprodukts von Vektorr\"aumen
(bzw.\ von Hilbert-R\"aumen) ist nicht sehr elegant, da sie explizit auf Basissysteme
in den jeweiligen Vektorr\"aumen Bezug nimmt. Es gibt in der
Mathematik zwar elegantere M\"oglichkeiten, das Tensorprodukt von
zwei Vektorr\"aumen ohne Bezug auf eine Basis zu definieren, diese
Definitionen sind aber f\"ur konkrete Berechnungen sehr unhandlich.
Der Nachteil der obigen Definition ist, dass man von
relevanten Konzepten (beispielsweise dem Konzept der
Verschr\"ankung, s.u.) eigentlich beweisen muss, dass sie  nicht von der
Wahl der Basisvektoren bei der Konstruktion des Tensorraums abh\"angen. 
F\"ur das Folgende sollte man diese Aussage einfach glauben oder 
ein entsprechendes Mathematikbuch konsultieren.

Seien $A_1$ ein linearer Operator auf ${\cal H}_1$ und $A_2$ ein
linearer Operator auf ${\cal H}_2$, dann ist $A_1\otimes A_2$ ein 
linearer Operator auf ${\cal H}$, der folgenderma\ss en auf die Basisvektoren
wirkt:
\begin{equation}
    (A_1 \otimes A_2) ( |e_i\rangle \otimes | f_j\rangle)  = 
                A_1|e_i\rangle \otimes \,  A_2|f_j\rangle 
\end{equation}
Wegen der Linearit\"at der Abbildung ist durch 
diese Definition festgelegt, wie ein solcher Operator auf einen
beliebigen Vektor im Tensorproduktraum wirkt. Insbesondere kann man
sich leicht \"uberzeugen, dass zwei Operatoren, die auf verschiedene
Vektorr\"aume wirken, im Tensorprodukt immer kommutieren:
\begin{equation}
     (A_1 \otimes {\bf 1}_2 ) ({\bf 1}_1 \otimes A_2) =
    ({\bf 1}_1 \otimes A_2)  (A_1 \otimes {\bf 1}_2 ) 
\end{equation}
Hier bezeichnet ${\bf 1}_i$ die Identit\"atsabbildung in Vektorraum ${\cal H}_i$.
Oft findet man daf\"ur vereinfacht die Schreibweise
\begin{equation}
       [A_1,A_2] = 0   \, ,
\end{equation}
wobei aber betont werden muss, dass sich die Indizes auf verschiedene
Vektorr\"aume beziehen und mit $A_1$ eigentlich $A_1\otimes {\bf 1}_2$ 
gemeint ist und entsprechend $A_2$ f\"ur ${\bf 1}_1\otimes A_2$ steht. 

Allgemein l\"asst sich ein Operator $B$ auf ${\cal H}$ immer als eine
Linearkombination solcher \glqq Produktoperatoren\grqq\ schreiben,
d.h.
\begin{equation}
       B = \sum_{ij} b_{ij} (A_i \otimes A_j)  \, .
\end{equation}

Nun wird auch die Bra-Ket-Notation f\"ur Operatoren einsichtiger:
Lineare Abbildungen von einem Vektorraum
$V$ in einen Vektorraum $W$ kann man abstrakt
als Elemente von $W\otimes V^*$ auffassen,
wobei $V^*$ der Dualraum von $V$ ist. Ein solches
Element aus $W\otimes V^*$ hat die
Eigenschaft, dass man es auf ein Element auf $V$
anwenden muss (das ist der $V^*$-Anteil, er liefert
zun\"achst eine Zahl) und das
Ergebnis ist ein Vektor in $W$ (der $W$-Anteil, der
mit der vorher erhaltenen Zahl multipliziert wird). 
Diese sehr abstrakte Sichtweise steckt implizit hinter
der Bra-Ket-Notation f\"ur Operatoren.

\section{Separable Zust\"ande und verschr\"ankte Zust\"ande}

Ist ein Vektorraum ${\cal H}$ das Tensorprodukt von zwei
Vektorr\"aumen ${\cal H}_1$ und ${\cal H}_2$, kann man
f\"ur Vektoren in dem Tensorproduktraum folgende Eigenschaften definieren:

%\begin{definition}
Ein Vektor $|\Psi\rangle\in {\cal H}_1 \otimes {\cal H}_2$ 
hei\ss t {\em separabel},\index{separabel}\index{Quantenzustand!separabler}
wenn es Vektoren $|\phi\rangle \in {\cal H}_1$ und 
$|\psi\rangle \in {\cal H}_2$ gibt, sodass
\begin{equation}
       | \Psi \rangle = |\phi \rangle \otimes |\psi \rangle  \, ,
\end{equation}
andernfalls hei\ss t der Vektor $|\Psi\rangle$ (bez\"uglich des
vorgegebenen Tensorprodukts) 
{\em verschr\"ankt}.\index{verschr\"ankt}\index{Quantenzustand!versch\"ankter}
%\end{definition}

Hierzu ein paar Anmerkungen:
\begin{enumerate}
\item
Die Begriffe {\em separabel} bzw.\ {\em verschr\"ankt} sind
nur in Bezug auf eine Tensorproduktdarstellung eines Vektorraums
definiert. Es macht keinen Sinn, von einem verschr\"ankten Vektor bzw.\
Zustand {\em per se} zu sprechen. Ein Vektor kann immer nur
{\em verschr\"ankt in Bezug auf eine Partitionierung} (d.h.\ Auf\-teilung
des Gesamtsys\-tems in Teilsysteme) sein. 
\item
Die Eigenschaften \glqq separabel\grqq\ und \glqq verschr\"ankt\grqq\ wurden zwar f\"ur Vektoren
definiert, aber man kann sich leicht davon \"uberzeugen, dass sie
auch f\"ur die Darstellung reiner physikalischer Zust\"ande (also Strahlen bzw.\ 1-dimensionale
lineare Unterr\"aume im Hilbert-Raum) sinnvoll sind: Ist ein Vektor
separabel (bzw.\ verschr\"ankt), dann ist auch ein beliebiges komplexes
Vielfaches dieses Vektors separabel (bzw.\ verschr\"ankt). Die Produktzerlegung
von Strahlen ist sogar eindeutig, wohingegen die Zerlegung eines
separablen Vektors nicht eindeutig ist, da z.B.\ $\vec{x}\otimes \vec{y}$ und
$(\frac{1}{\alpha}\vec{x})\otimes(\alpha \vec{x})$ denselben Produktvektor
definieren.

Ein reiner Zustand eines aus zwei Teilsystemen zusammengesetzten Gesamtsystems
hei\ss t verschr\"ankt, wenn er durch einen verschr\"ankten Zustandsvektor
in dem zugeh\"origen Produkt-Hilbert-Raum beschrieben wird. Entsprechend hei\ss t
ein solcher Zustand separabel, wenn der zugeh\"orige Zustandsvektor separabel ist. 
\item
Es ist wichtig zu betonen, dass die Konzepte der Separabilit\"at
und Verschr\"ankt\-heit nicht von den gew\"ahlten Basissystemen 
in den Hilbert-R\"aumen
abh\"angen. Auch die Definition des Tensorprodukts h\"angt nicht 
von der Wahl der Basen ab.
\item
Auch wenn die Definitionen f\"ur einen separablen bzw.\ verschr\"ankten
Vektor sehr einfach sind,
kann es in einem konkreten Fall durchaus schwierig sein zu
entscheiden, ob ein gegebener Vektor in Bezug auf eine 
Tensorproduktzerlegung separabel oder verschr\"ankt
ist. In Abschnitt \ref{sec_Teilreduktion} wird ein Kriterium
eingef\"uhrt, mit dem sich diese Entscheidung 
zumindest im Prinzip treffen l\"asst.  
\item
Ein Problem im Zusammenhang mit verschr\"ankten Vektoren ist, dass
weder die separablen noch die verschr\"ankten Vektoren einen Unterraum
(also Vektorraum) von ${\cal H}$ bilden. Die Linearkombination zweier
separabler Vektoren ist meist verschr\"ankt und die Linearkombination
verschr\"ankter Vektoren kann separabel sein. 
\end{enumerate}

\section{Beispiel: Der Produktraum $\mathbb{C}^2 \otimes \mathbb{C}^2$}
\label{sec_Entanglement_Produkt}

Sehr viele Beispiele beziehen sich auf Zwei-Zustandssysteme, die durch 
den komplexen zweidimensionalen Hilbert-Raum $\mathbb{C}^2$ beschrieben
werden. Das Tensorprodukt von zwei solchen Vektorr\"aumen beschreibt
zwei Zwei-Zustandssysteme. Dieses besteht aus den Vektoren
mit Komponenten $x_{ij}$ mit $i,j=1,2$. Das Tensorprodukt von zwei
Vektoren $\pmb{x}$ mit Komponenten $x_i$ und $\pmb{y}$ mit Komponenten
$y_i$ ist
\begin{equation}
       \pmb{x} \otimes \pmb{y} = \left( \begin{array}{c} x_1 \\ x_2 \end{array} \right)
       \otimes  \left( \begin{array}{c} y_1 \\ y_2 \end{array} \right) =
        \left( \begin{array}{c} x_1 y_1 \\ x_1y_2 \\ x_2 y_1 \\ x_2 y_2 \end{array} \right) \, .
\end{equation}
Davon zu unterscheiden ist die direkte Summe bzw.\ das kartesische Produkt
der beiden Vektoren:
\begin{equation}
       \pmb{x} \oplus \pmb{y} = \left( \begin{array}{c} x_1 \\ x_2 \end{array} \right)
       \oplus  \left( \begin{array}{c} y_1 \\ y_2 \end{array} \right) =
        \left( \begin{array}{c} x_1 \\ x_2 \\  y_1 \\  y_2 \end{array} \right) \, .
\end{equation}
F\"ur den Spezialfall von zwei zweidimensionalen Vektorr\"aumen ist die 
Dimension der Summe und des Produkts der Vektorr\"aume gleich
($2+2=2\times 2=4$), sodass in diesem Fall eine Verwechslung nochmals leichter
m\"oglich ist. 

Nicht jeder Vektor im $\mathbb{C}^2 \otimes \mathbb{C}^2$ l\"asst sich als
Produkt von zwei Vektoren im $\mathbb{C}^2$ schreiben. F\"ur solche Vektoren muss
offensichtlich gelten $x_{11} x_{22} = x_{12} x_{21}$, d.h., das Produkt der beiden \"au\ss eren
Komponenten ist gleich dem Produkt der beiden mittleren Komponenten. Es zeigt sich, dass beim
$\mathbb{C}^2 \otimes \mathbb{C}^2$ diese Bedingung nicht nur notwendig sondern
sogar hinreichend ist, damit ein Vektor separabel ist, sich also als Tensorprodukt von
zwei Vektoren darstellen l\"asst. Ist bei einem Vektor im $\mathbb{C}^2 \otimes \mathbb{C}^2$
diese Bedingung nicht erf\"ullt, ist der Vektor verschr\"ankt. 

Der eingangs erw\"ahnte $\Phi^+$ Bell-Zustand hat in dieser Schreibweise die Form
\begin{equation}
       |\Phi^+\rangle = \frac{1}{\sqrt{2}} \big( \pmb{e}_1 \otimes \pmb{e}_1 + \pmb{e}_2 \otimes \pmb{e}_2 \big)
         = \frac{1}{\sqrt{2}} \left[ \left( \begin{array}{c} 1 \\ 0 \end{array} \right)
       \otimes   \left( \begin{array}{c} 1 \\ 0 \end{array} \right) +  \left( \begin{array}{c} 0 \\ 1 \end{array} \right)
       \otimes   \left( \begin{array}{c} 0 \\ 1 \end{array} \right) \right]
       =   \frac{1}{\sqrt{2}} 
        \left( \begin{array}{c} 1 \\ 0 \\ 0 \\ 1 \end{array} \right) \, ,
\end{equation}
und f\"ur den sogenannten EPR-Zustand (benannt nach Einstein, Podolsky und Rosen, wobei man auch
Bohm hinzunehmen sollte), den man auch als $\Psi^-$ Bell-Zustand bezeichnet, folgt:
\begin{equation}
       |\Psi^-\rangle = \frac{1}{\sqrt{2}} \big( \pmb{e}_1 \otimes \pmb{e}_2 - \pmb{e}_2 \otimes \pmb{e}_1 \big)
         = \frac{1}{\sqrt{2}} \left[ \left( \begin{array}{c} 1 \\ 0 \end{array} \right)
       \otimes   \left( \begin{array}{c} 0 \\ 1 \end{array} \right) -  \left( \begin{array}{c} 0 \\ 1 \end{array} \right)
       \otimes   \left( \begin{array}{c} 1 \\ 0 \end{array} \right) \right]
       =   \frac{1}{\sqrt{2}} 
        \left( \begin{array}{c} 0 \\ 1 \\ -1 \\ 0 \end{array} \right) \, .
\end{equation}
In beiden F\"allen ist offensichtlich, dass das Produkt der beiden mittleren Komponenten nicht
gleich dem Produkt der beiden \"au\ss eren Komponenten ist. 

\section{Beispiele f\"ur Tensorprodukte und Verschr\"ankungen}

\subsection{Zwei Teilchen}

Wie schon erw\"ahnt ist das Standardbeispiel f\"ur ein System, bei dem eine Verschr\"ankung
auftreten kann, ein Zwei-Teilchensystem. S\"amtliche Observablen an dem einen Teilchen
vertauschen mit s\"amtlichen Observablen an dem anderen Teilchen. Das gilt f\"ur die
r\"aumlichen Freiheitsgrade (meist beschrieben durch Wellenfunktionen) ebenso wie f\"ur diskrete
Freiheitsgrade wie den Spin.

Wird der r\"aumliche Anteil eines Ein-Teilchensystems durch den Hilbert-Raum
${\cal L}_2(\mathbb{R}^3,{\rm d^3 x)}$ beschrieben, so ist der Hilbert-Raum f\"ur ein
Zwei-Teilchensystem das Tensorprodukt dieses Hilbert-Raums mit sich selbst, 
und das ist der Hilbert-Raum ${\cal L}_2(\mathbb{R}^6,{\rm d}^6 x)$, also der
Hilbert-Raum der Funktionen $\psi(\pmb{x},\pmb{y})$ mit der Eigenschaft
\begin{equation}
       \int_{\mathbb{R}^3} \int_{\mathbb{R}^3} | \psi(\pmb{x},\pmb{y}) |^2 \, {\rm d}^3x \,{\rm d}^3y = 1 \, .
\end{equation}
Es ist also
\begin{equation}
          {\cal L}_2(\mathbb{R}^3,{\rm d^3 x)} \otimes {\cal L}_2(\mathbb{R}^3,{\rm d^3 x)} =
          {\cal L}_2(\mathbb{R}^3\times \mathbb{R}^3,{\rm d^6 x)} \, .
\end{equation}
Etwas vereinfacht kann man sagen, dass das Tensorprodukt eines Funktionenraums in $n$
Variablen mit dem Funktionenraum in $m$ Variablen der Funktionenraum in $n+m$ Variablen
ist. 

Eine Funktion $\psi(\pmb{x},\pmb{y})$ hei\ss t verschr\"ankt, wenn es keine Funktionen 
$\phi(\pmb{x})$ und $\varphi(\pmb{y})$ gibt, sodass $\psi(\pmb{x},\pmb{y})=\phi(\pmb{x}) \varphi(\pmb{y})$.

Dies l\"asst sich auf beliebige Systeme verallgemeinern, die in nat\"urlicher Weise in Subsysteme
unterteilt werden k\"onnen. 

\subsection{Ort und Spin}

Auch die r\"aumlichen Observablen Ort und Impuls kommutieren mit den Observablen zum 
Spin, beispielsweise bei einem Elektron oder bei einem Photon (wo der Spin
der Polarisation entspricht). Damit kann es auch zu einer Verschr\"ankung zwischen
Orts- und Spin-Freiheitsgraden kommen. 

Ein typisches Beispiel ist das Stern-Gerlach-Experiment. Tritt ein Spin-1/2-Teilchen durch einen
Stern-Gerlach-Magneten, wird es nach oben oder unten (relativ zur Richtung des inhomogenen
Magnetfelds) abgelenkt. Elektronen, die nach oben abgelenkt wurden, haben Spin $+\frac{1}{2}\hbar$,
Elektronen, die nach unten abgelenkt wurden, haben Spin $-\frac{1}{2}\hbar$. Der Zustand
nach dem Durchtritt durch den Magneten aber vor dem Auftreffen auf dem Schirm ist somit
\begin{equation}
        | e \rangle = 
         | \psi_+ \rangle | \uparrow \rangle + | \psi_- \rangle |\downarrow\rangle
        \simeq  \left( \begin{array}{c}  \psi_+(\pmb{x}) \\  \psi_-(\pmb{x})  \end{array} \right) 
        \hspace{0.7cm} {\rm mit} \hspace{0.3cm}
        \int_{\mathbb{R}^3} \big( | \psi_+(\pmb{x})|^2 + |\psi_-(\pmb{x})|^2 \big)\, {\rm d}^3x =1  \, ,  
\end{equation} 
wobei $\psi_+(\pmb{x})$ seinen Tr\"ager entlang der Ablenkungsrichtung 
nach oben und $\psi_-(\pmb{x})$ seinen 
Tr\"ager entlang der Ablenkungsrichtung nach unten hat. Formal handelt es sich hierbei um
ein Tensorprodukt aus einem Funktionenraum (beispielsweise dem Hilbert-Raum der
quadratintegrierbaren Funktionen ${\cal L}_2(\mathbb{R}^3,{\rm d^3 x)}$) und 
dem $\mathbb{C}^2$. Man gelangt so zu zweikomponentigen Spinoren. 

Ein \"ahnlicher Fall tritt auf, wenn ein Lichtstrahl auf einen Doppelspalt trifft, der \glqq markiert\grqq\ ist,
z.B.\ indem hinter dem linken Spalt ein Polarisationsfilter in $h$-Richtung (horizontal) und hinter
dem rechten Spalt ein Polarisationsfilter in $v$-Richtung (vertikal) platziert ist, sodass
der Lichtstrahl hinter dem Doppelspalt eine Superposition aus einem horizontal polarisierten
Strahl, der durch den linken Spalt getreten ist, und einem vertikal polarisierten Strahl, der durch
den rechten Spalt getreten ist, darstellt. Der Zustand ist somit:
\begin{equation}
                  | \gamma \rangle = \frac{1}{\sqrt{2}}\big( |L \rangle | h \rangle + | R \rangle | v\rangle \big) \, ,  
\end{equation} 
Hierbei stehen $|L\rangle$ und $|R\rangle$ symbolisch f\"ur den r\"aumlichen Zustand eines
Photons, das durch den linken bzw.\ rechten Spaltgetreten ist, und
$|h\rangle$ bzw.\ $|v\rangle$ f\"ur eine Polarisation in horizontaler bzw.\ vertikaler Richtung. 

\subsection{Verschiedene Koordinatenrichtungen}

Eine Wellenfunktion $\Psi(\pmb{x})$ im $\mathbb{R}^3$ ist ein Element des dreifachen Tensorprodukts
von Wellenfunktionen \"uber $\mathbb{R}$, also entlang der $x$-, $y$- bzw.\ $z$-Richtung. 
Wenn eine solche Wellenfunktion nicht in ihre Anteile entlang verschiedener Koordinatenachsen
faktorisiert, kann man hier theoretisch von einer Verschr\"ankung zwischen den Anteilen zu verschiedenen 
Koordinatenrichtungen sprechen. 
Damit sind alle Wellenfunktionen im Wasserstoffatom, die Eigenfunktionen zu
den Quantenzahlen $n$, $l$ und $m$ sind, verschr\"ankt. Dies zeigt eine gewisse Willk\"ur
in der Wahl der Partition eines Systemes. Offenbar kann man auch Partitionen vornehmen,
die physikalisch im Allgemeinen nicht sinnvoll sind. 

\subsection{Der Messprozess}

Eine besondere Form der Verschr\"ankung, die ebenfalls immer noch Anlass zu Diskussionen
gibt, ist der Messprozess. Gegeben sei ein Quantensystem $S$, an dem mit einem Messinstrument $M$
eine Messung vorgenommen werden soll. Man kann dies als ein Gesamtsystem ($Q+M$) auffassen,
das in nat\"urlicher Weise in zwei Teilsysteme $Q$ und $M$ partitioniert werden kann. Vor der
Messung sei die Messapparatur in einem neutralen Zustand $|\varphi_0\rangle$ und das
Quantensystem $S$ befinde sich in einem Superpositionszustand bez\"uglich der Eigenzust\"ande
der Messapparatur $| s \rangle = \sum_i a_i | s_i \rangle$. 

Durch den Messprozess geht das Gesamtsystem in einen verschr\"ankten Zustand \"uber, bei
dem die Zust\"ande $|s_i\rangle$ des Quantensystems korreliert sind mit den Zeigerstellungen
$|\varphi_i\rangle$ des Messinstruments:
\begin{equation}
                    | s \rangle | \varphi_0\rangle = \sum_i a_i |s_i\rangle |\varphi_0\rangle 
                       ~~ \overset{\textrm{Messprozess}}{\longrightarrow} ~~
                        \sum_i a_i |s_i\rangle |\varphi_i\rangle \, . 
\end{equation}
Wir nutzen dann diesen verschr\"ankten Zustand, um an dem Messapparat eine Messung vorzunehmen:
Wir lesen die Zeigerstellung $\varphi_i$ am Messapparat ab und schlie\ss en aus dem Ergebnis auf den
Zustand $|s_i\rangle$ des Quantensystems. Dieser erste Schritt des Messprozesses -- die 
Verschr\"ankung zwischen Quantensystem und Messapparatur -- l\"asst sich mit einer
Schr\"odinger-Gleichung beschreiben und ist meist recht gut verstanden. Der anschlie\ss ende
Schritt - die Reduktion der Superposition zu einem reinen Zustand, bei dem das Messinstrument
eine wohldefinierte Zeigerstellung hat -- l\"asst sich nicht durch eine Schr\"odinger-Gleichung
beschreiben und bildet das eigentliche Problem des Messprozesses. 

\section{Die Teilreduktion von Zust\"anden}
\label{sec_Teilreduktion}

In der Quantentheorie l\"asst sich jeder
Zustand durch eine Dichtematrix darstellen, wobei reine
Zust\"ande den Projektionsoperatoren entsprechen. S\"amtliche
Eigenschaften allgemeiner Dichtematrizen gelten nat\"urlich
auch in Tensorproduktr\"aumen. Eine besondere
Konstruktion ist jedoch,\index{Teilreduktion} 
dass man \glqq Teilreduktionen\grqq\ vornehmen
kann. Man bildet dabei die Spur \"uber nur einen
der beiden Hilbert-R\"aume. Physikalisch kann man das so
deuten, dass man alle Information \"uber den ausreduzierten Teilraum (bzw.\ alle 
Korrelationen mit diesem Teilraum) \glqq vergisst\grqq\ und sich nur auf die Informationen
beschr\"ankt, die man durch lokale Messungen an dem verbliebenen
Teilsystem gewinnen kann.

Sei $A$ eine Matrix in einem Produkt-Hilbert-Raum, dann l\"asst
sich (in Bezug auf beliebige orthonormale 
Basen $\{|e_i\rangle\}$ f\"ur ${\cal H}_1$
und $\{|f_j\rangle\}$ f\"ur ${\cal H}_2$) die Matrix folgenderma\ss en
schreiben:
\begin{equation}
     A = \sum_{i,j,k,l}  a_{ij\,kl} ~ | e_i\rangle | f_j\rangle \,
           \langle f_k|  \langle e_l|  
\end{equation} 
Wir definieren nun die\index{Teilspur} 
\textit{Spur} (oder auch \textit{Teilspur}) \"uber den
Hilbert-Raum ${\cal H}_1$ durch
\begin{eqnarray}
        A_2 &=&  {\rm Sp}_1\, A = \sum_{n} \langle e_n |A | e_n \rangle \\
         &=&
          \sum_{ijkl} a_{ijkl} ~ \sum_n \langle e_n| e_i \rangle 
          |f_j\rangle \langle f_k| \langle e_l|e_n\rangle   \\
          &=& \sum_{jk} \left( \sum_n  a_{njkn} \right) |f_j\rangle \langle f_k| \, .
\end{eqnarray}
$A_2$ ist ein Operator auf ${\cal H}_2$, der auf ${\cal H}_1$ 
wirkende Anteil von $A$ wurde \glqq ausgespurt\grqq. 

Ganz entsprechend kann die Spur \"uber
${\cal H}_2$ gebildet werden und man erh\"alt einen Operator auf ${\cal H}_1$:
\begin{eqnarray}
        A_1 &=&  {\rm Sp}_2\, A = \sum_{m} \langle f_m |A | f_m \rangle \\
         &=&
          \sum_{ijkl} a_{ijkl} ~ \sum_m \langle f_m| f_j \rangle 
          |e_i\rangle \langle e_l | \langle f_k|f_m\rangle   \\
          &=& \sum_{il} \left( \sum_m  a_{imml} \right) |e_i\rangle \langle e_l| 
\end{eqnarray}

Diese Operationen des teilweisen Ausspurens kann man
auch f\"ur Dichtematrizen vornehmen. Handelt es sich um die
Dichtematrix zu einem separablen reinen Zustand, also um
das Tensorprodukt von zwei Projektionsoperatoren,
\begin{equation}
    \rho =  | \psi \rangle |\phi \rangle \,   
       \langle \phi| \langle \psi|  =
        P_\psi \otimes P_\phi   \, ,
\end{equation}
dann erh\"alt man durch die Teilspuren wieder Projektionsoperatoren
zu reinen Zu\-st\"anden:
\begin{equation}
   \rho_1 = {\rm Sp}_2\, \rho = |\psi \rangle \langle \psi| = P_\psi
   ~~~  {\rm und} ~~~ 
   \rho_2 = {\rm Sp}_1\, \rho = |\phi \rangle \langle \phi| = P_\phi 
\end{equation}
Handelt es sich jedoch bei $\rho$ um die Dichtematrix zu einem
verschr\"ankten Zustand (d.h., $\rho$ ist zwar ein Projektionsoperator
in Gesamt-Hilbert-Raum, aber es
gibt keine Projektionsoperatoren $P_\phi$ und $P_\psi$, sodass
$\rho = P_\phi \otimes P_\psi$), dann beschreiben $\rho_1$ 
und $\rho_2$ gemischte Zust\"ande. Auf diese Weise erh\"alt
man ein praktisches Kriterium zur \"Uberpr\"ufung, ob ein Zustand
separabel oder verschr\"ankt ist: Handelt es sich bei den Dichtematrizen 
der ausreduzierten Teilr\"aume um Projektionsoperatoren, ist der Zustand 
separabel; sind es Dichtematrizen zu gemischten Zust\"anden, ist
der Zustand verschr\"ankt. 
Die {\em von Neumann-Entropie}\index{von Neumann-Entropie}
der reduzierten Dichtematrizen
\begin{equation}
    S = \sum_i p_{\alpha,\,i} \ln p_{\alpha,\,i} =
     {\rm Sp}\, \rho_\alpha \ln \rho_\alpha   
    ~~~~~~~  (\alpha=1,2)
\end{equation}
wird oft als ein Verschr\"ankungsma\ss\ angesehen (f\"ur reine
Zust\"ande ist die von Neumann-Entropie null). Man kann
beweisen, dass die von null verschiedenen Eigenwerte 
$\{p_{\alpha, \,i} \}$ der
beiden reduzierten Dichtematrizen gleich sind, sodass auch
die beiden von Neumann-Entropien gleich sind.

Physikalisch l\"asst sich die teilreduzierte Dichtematrix 
folgenderma\ss en verstehen: Durch lokale Messungen an einem Teilsystem
(z.B.\ Teilsystem 1) kann man
den verschr\"ankten Zustand $|\Psi\rangle$ nicht von einem gemischten
Zustand zu der ausreduzierten Dichtematrix 
unterscheiden. Anders ausgedr\"uckt: F\"ur alle Observablen
der Form $A\otimes \identity$ sind die Erwartungswerte in
einem verschr\"ankten Zustand dieselben wie f\"ur die
Observable $A$ bez\"uglich der reduzierten Dichtematrix 
$\rho_1=\mathrm{Sp}_2\, P_\Psi A$, d.h., es gilt:
\begin{equation}
     \langle \Psi | A \otimes \identity |  \Psi   \rangle = 
     \mathrm{Sp}\, (\rho_1 A )
\end{equation} 
Man kann $\rho_1$ sogar \"uber diese Forderung definieren; das oben
angegebene Verfahren ist aber konstruktiver.
Eine entsprechende Beziehung gilt f\"ur die Dichtematrix, die man
f\"ur das Teilsystem 2 durch Ausreduktion von Teilsystem 1 erh\"alt.

Man kann auch Verschr\"ankungsma\ss e f\"ur gemischte
Zust\"ande formulieren, allerdings handelt es sich hier um ein
sehr komplexes Thema, das immer noch Teil der
aktuellen Forschung ist. Ebenfalls ein schwieriges und noch
nicht abgeschlossenes Forschungsgebiet ist die Untersuchung
von Verschr\"ankungsma\ss en von 
(reinen und gemischten) Zust\"anden in
Hilbert-R\"aumen, die das Tensorprodukt von mehr als zwei
Zustandsr\"aumen sind, d.h.\ bez\"uglich einer Partition des
Gesamtsystems in mehr als zwei Teilsysteme. Ein guter
\"Ubersichtsartikel zu Verschr\"ankungsma\ss en ist
\cite{Horodecki}. 

\section*{Anmerkungen}
\subsection*{Die Richtungsunabh\"angigkeit des $\Phi^+$-Zustands (Herleitung)}
\label{sec_Entanglement_A}

Eine allgemeine unit\"are Transformation in einem 2-dimensionalen reellen
Vektorraum hat die Form einer Drehung
\begin{equation}
             R = \left( \begin{array}{cc} a & b \\ - b & a \end{array} \right)
             \hspace{1cm} {\rm mit} \hspace{0.6cm}  a^2 + b^2 = 1 \, . 
\end{equation}
(Typischerweise parametrisiert man eine solche Transformation durch $a=\cos \alpha$ und
$b=\sin \alpha$.)
Auf eine beliebige Basis $\{ |\pmb{e}_1 \rangle, |\pmb{e}_2\rangle\}$ wirkt diese Transformation
in der Form:
\begin{equation}
    | \pmb{f}_1 = a | \pmb{e}_1 \rangle + b |\pmb{e}_2 \rangle \hspace{0.5cm} , \hspace{0.5cm}
     | \pmb{f}_2 = -b | \pmb{e}_1\rangle + a |\pmb{e}_2 \rangle \, . 
\end{equation}
F\"ur den korrelierten verschr\"ankten Zustand $\Phi^+$ (ausgedr\"uckt in der $\pmb{f}$-Basis) folgt:
\begin{eqnarray}
    \frac{1}{\sqrt{2}} \big(  | \pmb{f}_1 \rangle \otimes | \pmb{f}_1 \rangle + 
        | \pmb{f}_2 \rangle \otimes | \pmb{f}_2 \rangle \big)  & = & \\ 
        & &  \hspace*{-5cm} =
    \frac{1}{\sqrt{2}} \Big(  \big( a | \pmb{e}_1 \rangle + b |\pmb{e}_2 \rangle \big) 
        \otimes  \big(  a | \pmb{e}_1\rangle + b |\pmb{e}_2 \rangle \big)
         +  \big( -b | \pmb{e}_1\rangle + a |\pmb{e}_2 \rangle \big) 
            \otimes \big(  -b | \pmb{e}_1 \rangle + a |\pmb{e}_2 \rangle \big)  \Big)       \\   
        & &  \hspace*{-5cm} =
  \frac{1}{\sqrt{2}} ( a^2 + b^2) \big( | \pmb{e}_1 \rangle \otimes | \pmb{e}_1 \rangle +
        | \pmb{e}_2 \rangle \otimes | \pmb{e}_2 \rangle \big)  \\
                & &  \hspace*{-5cm} =
  \frac{1}{\sqrt{2}} \big( | \pmb{e}_1 \rangle \otimes | \pmb{e}_1 \rangle + 
        | \pmb{e}_2 \rangle \otimes | \pmb{e}_2 \rangle \big)  
\end{eqnarray}

\subsection*{Die Richtungsabh\"angigkeit des gemischten Zustands (Herleitung 2)}
\label{sec_Entanglement_B}

Auf die beiden Zust\"ande des Gemischs wirkt die Drehung in der Form
\begin{eqnarray}
    | \pmb{f}_1 \otimes |\pmb{f}_1\rangle &=&
    = \big( a | \pmb{e}_1 \rangle + b |\pmb{e}_2 \rangle \big) \otimes
                     ( a | \pmb{e}_1 \rangle + b |\pmb{e}_2 \rangle \big)  \\
    &=& \big( a^2 | \pmb{e}_1 \rangle \otimes | \pmb{e}_1 \rangle + ab | \pmb{e}_1 \rangle \otimes | \pmb{e}_2 \rangle +
                   ba | \pmb{e}_2 \rangle \otimes | \pmb{e}_1 \rangle + b^2 | \pmb{e}_2 \rangle \otimes | \pmb{e}_2 \rangle  \\
    | \pmb{f}_2 \otimes |\pmb{f}_2\rangle &=&
    = \big( -b | \pmb{e}_1 \rangle + a |\pmb{e}_2 \rangle \big) \otimes
                     ( -b | \pmb{e}_1 \rangle + a |\pmb{e}_2 \rangle \big)  \\
    &=& \big( b^2 |\pmb{e}_1 \rangle \otimes | \pmb{e}_1 \rangle - ba |\pmb{e}_1 \rangle \otimes |\pmb{e}_2 \rangle -
                   -ab | \pmb{e}_2 \rangle \otimes | \pmb{e}_1 \rangle + a^2 |\pmb{e}_2 \rangle \otimes |\pmb{e}_2 \rangle  
\end{eqnarray}
Man erkennt, dass in beiden F\"allen auch antikorrelierte Ergebnisse m\"oglich sind.

\begin{thebibliography}{99}
\bibitem{Horodecki} Horodecki, R., Horodecki, P., Horodecki, M.,
         Horodecki, K.; {\em Quantum entanglement}; Rev.\ Mod.\ Phys.\ {\bf 81}
         (2009) 865--942.       
\end{thebibliography}


\end{document}

