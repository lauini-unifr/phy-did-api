\documentclass[german,10pt]{book}      
\usepackage{makeidx}
\usepackage{babel}            % Sprachunterstuetzung
\usepackage{amsmath}          % AMS "Grundpaket"
\usepackage{amssymb,amsfonts,amsthm,amscd} 
\usepackage{mathrsfs}
\usepackage{rotating}
\usepackage{sidecap}
\usepackage{graphicx}
\usepackage{color}
\usepackage{fancybox}
\usepackage{tikz}
\usetikzlibrary{arrows,snakes,backgrounds}
\usepackage{hyperref}
\hypersetup{colorlinks=true,
                    linkcolor=blue,
                    filecolor=magenta,
                    urlcolor=cyan,
                    pdftitle={Overleaf Example},
                    pdfpagemode=FullScreen,}
%\newcommand{\hyperref}[1]{\ref{#1}}
%
\definecolor{Gray}{gray}{0.80}
\DeclareMathSymbol{,}{\mathord}{letters}{"3B}
%
\newcounter{num}
\renewcommand{\thenum}{\arabic{num}}
\newenvironment{anmerkungen}
   {\begin{list}{(\thenum)}{%
   \usecounter{num}%
   \leftmargin0pt
   \itemindent5pt
   \topsep0pt
   \labelwidth0pt}%
   }{\end{list}}
%
\renewcommand{\arraystretch}{1.15}                % in Formeln und Tabellen   
\renewcommand{\baselinestretch}{1.15}                 % 1.15 facher
                                                      % Zeilenabst.
\newcommand{\Anmerkung}[1]{{\begin{footnotesize}#1 \end{footnotesize}}\\[0.2cm]}
\newcommand{\comment}[1]{}
\setlength{\parindent}{0em}           % Nicht einruecken am Anfang der Zeile 

\setlength{\textwidth}{15.4cm}
\setlength{\textheight}{23.0cm}
\setlength{\oddsidemargin}{1.0mm} 
\setlength{\evensidemargin}{-6.5mm}
\setlength{\topmargin}{-10mm} 
\setlength{\headheight}{0mm}
\newcommand{\identity}{{\bf 1}}
%
\newcommand{\vs}{\vspace{0.3cm}}
\newcommand{\noi}{\noindent}
\newcommand{\leer}{}

\newcommand{\engl}[1]{[\textit{#1}]}
\parindent 1.2cm
\sloppy

         \begin{document}  \setcounter{chapter}{9}


\chapter{Interferometer}
% Kap x
\label{chap_Interferometer}

Interferometer spielen \"uberall in der Optik eine wichtige Rolle. Sie sind eines
der zentralen Instrumente der Experimentalphysik f\"ur optische und quantenoptische Versuche. Verwendet man
Laserlicht, zeigen Interferometer unter anderem die Wellennatur des Lichts. Mit Einzelphotonen
lassen sich jedoch vollkommen neuartige und interessante Effekte erzielen. 

Zwei der bekanntesten Interferometer sind das Mach-Zehnder-Interferometer und
das Michelson-Interferometer. Beide beruhen auf einem \"ahnlichen Prinzip: Ein Strahl
wird mit einem Strahlteiler geteilt, \"uber ein Spiegelsystem auf einen zweiten Strahlteiler gelenkt
und dort wieder zusammengebracht, wobei dann die Interferenzmuster entstehen. Beim Michelson-Interferometer
gibt es nur einen Strahlteiler, der sowohl die Aufspaltung als auch die sp\"atere
Zusammenf\"uhrung \"ubernimmt, d.h.\ das Spiegelsystem reflektiert die aufgeteilten 
Strahlen wieder zur\"uck auf den ersten Strahlteiler, hinter dem es dann zur Interferenz kommen kann. Beim 
Mach-Zehnder-Interferometer lenken die Spiegel die beiden Strahlen auf einen
zweiten Strahlteiler, hinter dem dann eventuell Interferenzen beobachtet werden k\"onnen.  


Beide Interferometer lassen sich sowohl mit gew\"ohnlichen Strahlteilern als auch
mit polarisationsabh\"angigen Strahlteilern betreiben. Allerdings ist dies beim Michelson-Interferometer
etwas problematischer, da dort die vorgegebene Orientierung die Strahlen in die Lichtquelle
lenken w\"urde, wohingegen man einen Ausgang m\"ochte, der senkrecht dazu steht. 
Das Michelson-Interferometer wird auch beim Nachweis von Gravitationswellen verwendet. 

Viele Experimente lassen sich am Mach-Zehnder-Interferometer leichter beschreiben,
aber am Michelson-Interferometer leichter durchf\"uhren. Daher sollen hier beide Interferometer
besprochen werden. In gewissen Hinsicht sind diese Interferometer Erweiterungen des
Doppelspaltexperiments f\"ur koh\"arentes Licht: Der Strahl bzw.\ der Zustand eines einzelnen Photons
wird aufgeteilt und die beiden Anteile breiten sich entlang unterschiedlicher Wege aus bevor
sie wieder zusammengef\"uhrt und zur Interferenz gebracht werden. Allerdings kann man an
diesen Interferometern die beiden Teile eines Strahls bzw.\ eines Photonzustands im Prinzip beliebig weit
auseinander bringen, was den Effekt erh\"oht. Au\ss erdem kann man optische
Ger\"ate oder Hinternisse in den Weg einer der Teilstrahlen bringen. In der Schule werden viele
der (Gedanken-)Experimente mit Mach-Zender- bzw.\ Michelson-Interferometern unter dem
Gesichtspunkt \glqq Ist Welcher-Weg-Information bekannt?\grqq\ behandelt, d.h.\ die Aufgabe
der Sch\"uler*innen besteht darin zu entscheiden, ob man je nach experimentellem Aufbau
im Prinzip die Information dar\"uber, welchen Weg ein Photon genommen hat, gewinnen kann
oder nicht. In Abh\"angigkeit davon kann man dann entscheiden, ob ein Interferenzmuster beobachtet werden
kann oder nicht.  

Beide Interferometer werden f\"ur den Fall normaler (d.h., nicht polarisationsabh\"angiger) 
Strahlteilern beschrieben. Ein anderes Kapitel widmet sich dem Mach-Zehnder-Interferometer 
mit polarisationsabh\"angigen Strahlteilern. Wir beginnen allerdings mit einer
kurzen Beschreibung der mathematischen Darstellung eines Strahlteilers.

\section{Strahlteiler}

Strahlteiler sind die wesentlichen Elemente eines Mach-Zehnder- bzw.\ Michelson-Interferometers.
Es gibt polarisationsabh\"angige Strahlteiler, bei denen ein einfallender Lichtstrahl in zwei
Teilstrahlen mit zueinander orthogonalen Polarisationszust\"anden aufgespalten wird. H\"aufiger
verwendet man jedoch nicht polarisationsabh\"angige Strahlteiler, d.h., der einfallende Strahl wird koh\"arent
in zwei gleichartig polarisierte Teilstrahlen aufgeteilt. 

Physikalisch besteht ein Strahlteiler meist aus einer beschichteten Grenzfl\"ache. Die Beschichtung ist
gerade so gew\"ahlt, dass die H\"alfte der Lichtintensit\"at durchgelassen und die andere H\"alfte
abgelenkt wird. Es gibt auch Strahlteiler mit einer asymmetrischen Aufspaltung, wir beschr\"anken uns
hier aber auf die Halbe-Halbe-Strahlteiler.

Wir k\"onnen die Wirkung eines Strahlteilers 
durch eine unit\"are Matrix beschreiben (unit\"ar, weil zumindest in der Theorie keine Intensit\"at 
verloren geht, d.h.,\ alle Photonen, die in den Strahlteiler eindringen, kommen auch wieder heraus). 
Diese Matrix ergibt sich auch direkt aus der \"Uberlegung, wie zwei einfallende Strahlen am
Strahlteiler aufgeteilt werden. Diese beiden Vorschriften sind:
\begin{equation}
\label{eq_ST1}
       |  1 \rangle_{\rm i}  \longrightarrow  \frac{1}{\sqrt{2}} \big( |1 \rangle_{\rm f} + {\rm i} | 2 \rangle_{\rm f} \big)
       \hspace{2cm}
    |  2 \rangle_{\rm i}  \longrightarrow  \frac{1}{\sqrt{2}} \big( | 2 \rangle_{\rm f} + {\rm i} | 1 \rangle_{\rm f} \big)       
\end{equation}
Hierbei wurde die in der Streutheorie \"ubliche Konvention verwendet, die einlaufenden (`initial')
Zust\"ande durch i und die auslaufenden (`final') Zust\"ande durch f zu kennzeichnen. Die
(Streu-)Matrix zum Strahlteiler ist dann durch $S_{ij}={}_{\rm f}\langle j | i \rangle_{\rm i}$ gegeben, also
durch die Wahrscheinlichkeitsamplituden zu dem \"Ubergang von einem Anfangzustand $i$  
zu einem Ausgangszustand $j$ (siehe auch Abb.\ \ref{fig_ST}). Die Faktoren i in Gl.\ \ref{eq_ST1}
beschreiben eine relative Phasenverschiebung um $90^\circ$. Diese Phasenverschiebung tritt immer
auf, wenn eine Reflektion an einer einer Grenzschicht stattfindet, bei der der Winkel zwischen
einfallender und ausfallender Welle gerade $90^\circ$ betr\"agt. Der Faktor $1/\sqrt{2}$ ergibt sich 
daraus, dass das Quadrat dieses Faktors nach der Born'schen
Regel die Wahrscheinlichkeit angibt, das Photon in dem jeweiligen Sektor zu messen (hier
jeweils $1/2$). F\"ur
Laserlicht bedeutet er einfach, dass jeder Teilstrahl die H\"alfte der Eingangsintensit\"at besitzt. 

\begin{SCfigure}[30][htb]
\begin{picture}(100,100)(-5,0)
\put(10,50){\line(1,0){80}}
\put(50,50){\line(0,1){40}}
\put(30,50){\vector(1,0){0}}
\put(80,50){\vector(1,0){0}}
\put(50,80){\vector(0,1){0}}
\put(90,58){\makebox(0,0){$|1\rangle_{\rm f}$}}
\put(42,90){\makebox(0,0){$|2\rangle_{\rm f}$}}
\put(10,58){\makebox(0,0){$|1\rangle_{\rm i}$}}
\thicklines
\put(29,29){\line(1,1){40}}
\put(31,31){\line(1,1){40}}
\end{picture}
%
\begin{picture}(110,100)(0,0)
\put(50,50){\line(1,0){40}}
\put(50,10){\line(0,1){80}}
\put(50,30){\vector(0,1){0}}
\put(80,50){\vector(1,0){0}}
\put(50,80){\vector(0,1){0}}
\put(90,58){\makebox(0,0){$|1\rangle_{\rm f}$}}
\put(42,90){\makebox(0,0){$|2\rangle_{\rm f}$}}
\put(58,15){\makebox(0,0){$|2\rangle_{\rm i}$}}
\thicklines
\put(29,29){\line(1,1){40}}
\put(31,31){\line(1,1){40}}
\end{picture}
\caption{\label{fig_ST}%
Die beiden Strahleng\"ange in einem Strahlteiler in Abh\"angigkeit vom
Anfangszustand. Ordnet man jeder Reflektion
einen Faktor i zu, erh\"alt man die Matrix in Gl.\ \ref{eq_STrep1}.}
\end{SCfigure}


Damit erhalten wir die folgende unit\"are Matrix, die das Verhalten von Photonen bzw.\
Licht an einem Halbe-Halbe-Strahlteiler beschreibt:
\begin{equation}
\label{eq_STrep1}
                 S = \frac{1}{\sqrt{2}}  \left( \begin{array}{cc}  1 & {\rm i} \\ {\rm i} & 1 \end{array} \right)  \, .
\end{equation}

Man kann noch relative optische Wegl\"angen\"anderungen (Phasen) vor- bzw.\ nachschalten.
Diese werden durch eine Diagonalmatrix mit reinen Phasen beschrieben. Auf diese Weise erhalten
wir eine allgemeine Darstellung f\"ur $S$ in der Form:
\begin{equation}
        S(\alpha,\beta) = \frac{1}{\sqrt{2}} 
        \left( \begin{array}{cc}  {\rm e}^{{\rm i}\alpha} & 0 \\ 0 & {\rm e}^{-{\rm i}\alpha} \end{array} \right)
          \left( \begin{array}{cc}  1 & {\rm i} \\ {\rm i} & 1 \end{array} \right)
          \left( \begin{array}{cc}  {\rm e}^{{\rm i}\beta} & 0 \\ 0 & {\rm e}^{-{\rm i}\beta} \end{array} \right)
  =  \frac{1}{\sqrt{2}}   
    \left( \begin{array}{cc}  {\rm e}^{{\rm i}(\alpha+\beta)} & {\rm i} {\rm e}^{{\rm i}(\beta-\alpha)} \\ 
                   {\rm i} {\rm e}^{{\rm i}(\alpha-\beta)} & {\rm e}^{-{\rm i}(\alpha+\beta)} \end{array} \right)
\end{equation}
Durch die spezielle Wahl $\alpha = - \beta = \pi/4$ erhalten wir eine andere Darstellung f\"ur
einen Strahlteiler, die ebenfalls h\"aufig verwendet wird:
\begin{equation}
\label{eq_STrep2}
                 S' = \frac{1}{\sqrt{2}}  \left( \begin{array}{cc}  1 & 1 \\ -1 & 1 \end{array} \right)  \, .
\end{equation}
Das Minuszeichen in der Matrix ist 
wesentlich, da es sich andernfalls nicht um eine unit\"are Matrix handeln w\"urde. Dieses Minuszeichen
beschreibt letztendlich die relative Phasenverschiebung von $180^\circ$ zwischen den beiden
reflektierten Strahlen. Anders ausgedr\"uckt: Die Summe der relativen Phasen der reflektierten Strahlen 
ist $180^\circ$, die Summe der relativen Phasen der durchgelassenen Strahlen ist $0^\circ$. 
Dieses Minuszeichen ist aber gerade Sch\"uler*innen kaum einfach zu vermitteln,
wohingegen die obige Erkl\"arung einsichtiger ist und auf dasselbe Ergebnis f\"uhrt. 

\section{Das Mach-Zehnder-Interferometer}

Beim Mach-Zehnder-Interferometer trifft ein Lichtstrahl zun\"achst auf einen Strahlteiler.
Ein Teil des Lichtstrahls wird abgelenkt, der andere Teil durchgelassen. \"Uber zwei Spiegel,
an denen die Lichtstrahlen um jeweils $90^\circ$ reflektiert werden, treffen beide
Strahlen von verschiedenen Seiten auf einen zweiten Strahlteiler, wo sie wieder
zusammengef\"uhrt werden. Hinter dem zweiten Strahlteiler kann man die Intensit\"at
des Lichts an zwei m\"oglichen Ausg\"angen messen. 

\begin{SCfigure}[30][htb] 
%\unitlength 2pt
\begin{picture}(300,170)(-5,0)
\thicklines
\put(40,10){\line(1,1){20}}
\put(250,10){\line(1,1){20}}
\put(251,9){\line(1,1){20}}
\put(40,120){\line(1,1){20}}
\put(39,121){\line(1,1){20}}
\put(250,120){\line(1,1){20}}
\thinlines
\put(0,20){\line(1,0){260}}
%\put(60,20){\circle*{10}}
\put(50,20){\line(0,1){110}}
\put(260,20){\line(0,1){110}}
\put(50,130){\line(1,0){210}}
\put(20,20){\vector(1,0){0}}
\put(150,20){\vector(1,0){0}}
\put(150,130){\vector(1,0){0}}
\put(50,70){\vector(0,1){0}}
\put(260,70){\vector(0,1){0}}
\put(260,130){\vector(1,0){20}}
\put(260,130){\vector(0,1){20}}
%
\put(25,30){\makebox(0,0){$|\gamma \rangle$}}
\put(150,30){\makebox(0,0){$| {\rm t} \rangle$}}
\put(60,70){\makebox(0,0){$|{\rm r} \rangle$}}
\put(250,150){\makebox(0,0){$| {\rm D} \rangle$}}
\put(275,120){\makebox(0,0){$|{\rm C} \rangle$}}
%
%\put(260,170){\makebox(0,0){\footnotesize D dunkel}}
%\put(300,130){\makebox(0,0){\footnotesize C hell}}
\put(70,10){\makebox(0,0){\footnotesize Strahlteiler}}
\put(220,120){\makebox(0,0){\footnotesize Strahlteiler}}
\put(70,146){\makebox(0,0){\footnotesize Spiegel}}
\put(280,37){\makebox(0,0){\footnotesize Spiegel}}
%
\end{picture} 
\caption{\label{fig_MachZehnder}%
Mach-Zehnder-Interferometer. Der Strahl trifft auf einen ersten Strahlteiler. Die beiden
Teil\-strahlen r und t werden von Spiegeln auf einen zweiten Strahlteiler gelenkt, hinter dem
es zu konstruktiver bzw.\ destruktiver Interferenz kommen kann.} 
\end{SCfigure}

Die Interferenz am Ausgang hinter dem zweiten Strahlteiler ergibt sich aus unterschiedlichen
optischen Wegl\"angen der zwei Strahlen, die dort zusammentreffen. Meist findet man
folgende, sehr vereinfachte Erkl\"arung: Am Ausgang $D$ treffen zwei Strahlen aufeinander,
von denen einer einmal um $90^\circ$ reflektiert wurde und der andere dreimal. Da jede
Reflektion um $90^\circ$ eine Phasenverschiebung von $90^\circ$ zur Folge hat 
ergibt sich zwischen den beiden Strahlen eine relative Phasenverschiebung
von $180^\circ$ und somit destruktive Interferenz. Bei Ausgang $C$ treffen zwei Strahlen
aufeinander, die jeweils einmal reflektiert wurden und somit in Phase sind; es kommt also zu
konstruktiver Interferenz. 

Im mathematischen Formalismus f\"ur Einzelphotonen k\"onnen wir dies folgenderma\ss en
beschreiben. Ein einfallendes Photon wird durch den Vektor $|\gamma\rangle$ beschrieben.
Am ersten Strahlteiler (ST) wird dieser Zustand in eine Superposition von zwei Zust\"anden 
zerlegt: einer dieser Zust\"ande ist der Zustand $|{\rm t}\rangle$ (f\"ur transmittiert), der andere ist der
Zustand des reflektierten Anteils $|{\rm r}\rangle$. Der reflektierte Anteil erh\"alt noch eine Phase von
$90^\circ$, was durch einen Faktor i ausgedr\"uckt werden kann. Hinter dem ersten Strahlteiler
erhalten wir also folgenden Zustand:
\begin{equation}
           |\gamma \rangle  \longrightarrow  \frac{1}{\sqrt{2}} \big( |{\rm t}\rangle + {\rm i} |{\rm r}\rangle \big) \, .
\end{equation}
Der Faktor $1/\sqrt{2}$ ergibt sich daraus, dass das Quadrat dieses Faktors nach der Born'schen
Regel die Wahrscheinlichkeit angibt, das Photon in dem jeweiligen Sektor zu messen. F\"ur
Laserlicht bedeutet er einfach, dass jeder Teilstrahl die H\"alfte der Eingangsintensit\"at besitzt. 

Beide Anteile des Strahls werden nun an einem Spiegel reflektiert, erhalten somit einen 
Faktor i, der f\"ur beide Teile gleich ist. Die beiden
Anteile treffen anschlie\ss end auf den zweiten Strahlteiler, wobei jeder der beiden Anteile
entweder in die Richtung von D oder von C gelenkt werden kann. Dies dr\"ucken wir durch die
Zust\"ande $|{\rm D}\rangle$ bzw.\ $|{\rm C}\rangle$ aus, und je nachdem, ob eine Reflektion stattgefunden
hat oder nicht, erhalten wir einen weiteren Faktor i.
\begin{eqnarray}
  |\gamma \rangle & \stackrel{\rm 1.~ST}{\longrightarrow}  & 
    \frac{1}{\sqrt{2}} \big( |{\rm t}\rangle + {\rm i}   |{\rm r}\rangle \big) 
    ~ \stackrel{\rm Spiegel}{\longrightarrow} ~ 
    \frac{\rm i}{\sqrt{2}} \big( |{\rm t}\rangle + {\rm i}    |{\rm r}\rangle \big) \\
\label{eq_Feynman1}    
    & \stackrel{\rm 2.~ST}{\longrightarrow} &
    \frac{\rm i}{\sqrt{2}} \left(   \frac{1}{\sqrt{2}} \big(  |{\rm D}\rangle + {\rm i}  |{\rm C}\rangle \big)
    + \frac{\rm i}{\sqrt{2}} \big({\rm i}     |{\rm D}\rangle  + |{\rm C}\rangle \big) \right)  \\
\label{eq_DCDC}    
    & =  &
    \frac{\rm i}{2}  |{\rm D}\rangle  +  \frac{{\rm i}^2}{2} |{\rm C}\rangle 
    + \frac{{\rm i}^3}{2}  |{\rm D}\rangle  + \frac{{\rm i}^2}{2} |{\rm C}\rangle \\
    & = &  - |{\rm C} \rangle 
\end{eqnarray}
Das \"ubliche Interferenzmuster im Sinne von \glqq alternierenden Helligkeitsstreifen\grqq\ erh\"alt man
hier, wenn man die optische Wegl\"ange eines der beiden Teilstrahlen beispielsweise durch eine
Verschiebung des reflektierenden Spiegels \"andert. Man erhalt so alternierend mal den Ausgang C und
mal den Ausgang D als den Ausgang maximaler Helligkeit, wobei der jeweils andere Ausgang dunkel ist.  
  
In der Praxis, beispielsweise bei der Verwendung von sichtbarem Licht, l\"asst sich die optische Wegl\"ange 
kaum so ausmessen, dass man die relativen Phasenunterschiede einzig auf die Anzahl der Reflektionen
zur\"uckf\"uhren kann. Man wird die optische Wegl\"ange eines der Teilstrahlen durch Verschieben der
optischen Elemente so anpassen, dass man beispielsweise hinter einem der Ausg\"ange kein Licht 
(destruktive Interferenz) und entsprechend hinter dem anderen Ausgange volle Helligkeit (konstruktive
Interferenz) findet. 

\section{Feynman's Summation \"uber alle Wege}

Richard Feynman hat eine sehr eing\"angige Darstellung f\"ur die Bestimmung einer
Amplitude $\langle b | a \rangle$ gefunden. Formal kann man schreiben:
\begin{equation}
       \langle b | a \rangle = \sum_{\mbox{\tiny Wege $a \rightarrow b$}}  A(\mbox{Weg}) \, ,
\end{equation}
wobei $A(\mbox{Weg})$ eine komplexe Amplitude ist, die sich als Produkt einzelner Beitr\"age entlang
eines Weges schreiben l\"asst. Summiert wird \"uber alle \glqq Wege\grqq\ (M\"oglichkeiten), unter
Einhaltung der physikalischen Gesetze von dem Zustand $a$ zu dem Zustand $b$ zu gelangen.
Diese Darstellung wollen wir uns am Mach-Zehnder-Interferometer anschauen. 

Der Ausgangszustand ist der Zustand $|\gamma \rangle$ von einem Photon, das am
ersten Strahlteiler in das Interferometer trifft. Die m\"oglichen Endzust\"ande sind $|D\rangle$ und $|C\rangle$,
bei denen ein Photon in Detektor D bzw.\ C nachgewiesen wird. F\"ur die beiden
Amplituden $\langle D | \gamma \rangle$ und $\langle C | \gamma \rangle$
gibt es jeweils zwei Wege:
\begin{eqnarray}
\nonumber
     \langle D | \gamma \rangle &=&  |\gamma \rangle \longrightarrow \mbox{ST}_1 \longrightarrow
       | {\rm r} \rangle \longrightarrow \mbox{Spiegel} \longrightarrow |{\rm r}\rangle \longrightarrow 
       \mbox{ST}_2 \longrightarrow |{\rm D}\rangle  \\
     & &  + ~ |\gamma \rangle \longrightarrow \mbox{ST}_1 \longrightarrow
       | {\rm t} \rangle \longrightarrow \mbox{Spiegel} \longrightarrow |{\rm t}\rangle \longrightarrow 
       \mbox{ST}_2 \longrightarrow |{\rm D}\rangle  \\
 \nonumber      
     \langle C | \gamma \rangle &=&  |\gamma \rangle \longrightarrow \mbox{ST}_1 \longrightarrow
       | {\rm r} \rangle \longrightarrow \mbox{Spiegel} \longrightarrow |{\rm r}\rangle \longrightarrow 
       \mbox{ST}_2 \longrightarrow |{\rm C}\rangle  \\
     & &  + ~  |\gamma \rangle \longrightarrow \mbox{ST}_1 \longrightarrow
       | {\rm t} \rangle \longrightarrow \mbox{Spiegel} \longrightarrow |{\rm t}\rangle \longrightarrow 
       \mbox{ST}_2 \longrightarrow |{\rm C}\rangle  \, .  
\end{eqnarray}
Zur Berechnung der Phase verwenden wir folgende Regeln:
\begin{itemize}
\item
Propagation entlang einer geraden Strecke: $A = 1$.\\
Dies ist nicht ganz korrekt, eigentlich m\"usste man hier als reine Phase die optische Wegl\"ange
nehmen. Da wir nur an relativen Phasenunterschieden interessiert sind k\"onnen wir diesen Beitrag unber\"ucksichtigt lassen.
\item
F\"ur jede Reflektion um einen Winkel $\alpha$ erhalten wir eine Phase $A={\rm e}^{{\rm i}\alpha}$. 
F\"ur Reflektionen um $90^\circ$ ist dies ein Faktor i.
\item
F\"ur jede Verzweigung (z.B.\ in einem Strahlteiler) erhalten wir einen Faktor $A=1/\sqrt{2}$ (das Quadrat
dieses Faktors ist gleich der Wahrscheinlichkeit, in eine der beiden Richtungen abgelenkt zu werden).
\end{itemize}
Damit k\"onnen wir die Phasen f\"ur die obigen Wege ausrechnen:
\begin{eqnarray}
     \langle D | \gamma \rangle &=&  \Big( \frac{\rm i}{\sqrt{2}}\Big) \Big( {\rm i} \Big) \Big( \frac{\rm i}{\sqrt{2}}\Big) 
       +  \Big( \frac{1}{\sqrt{2}}\Big) \Big( {\rm i} \Big) \Big( \frac{1}{\sqrt{2}}\Big) 
          = - \frac{\rm i}{2} + \frac{\rm i}{2} = 0     \\
     \langle C | \gamma \rangle &=&  \Big( \frac{\rm i}{\sqrt{2}}\Big)\Big( {\rm i}\Big)\Big( \frac{1}{\sqrt{2}} \Big)
     + \Big( \frac{1}{\sqrt{2}}\Big)\Big( {\rm i} \Big)\Big(\frac{\rm i}{\sqrt{2}} \Big)
     = - \frac{1}{2} - \frac{1}{2} = -1  \, .  
\end{eqnarray}
Wir finden diese Faktoren auch in den Gleichungen \ref{eq_Feynman1} und \ref{eq_DCDC} wieder.

Feynmans Darstellung einer Amplitude als Summe \"uber Beitr\"age von Wegen ist oft
ganz hilfreich, um sich die m\"oglichen Wege vor Augen zu halten. In einer Kontinuumstheorie wird
aus der Summation \"uber Wege ein sogenanntes Funktionalintegral. Die Idee hinter dieser
Darstellung beruht auf folgender \"Uberlegung: Ein physikalischer Prozess wird in der Quantentheorie
durch den sogenannten Zeitentwicklungsoperator $\exp \big( \frac{\rm i}{\hbar} H t\big) $ (oder, wie
beispielsweise beim Mach-Zehnder-Interferometer, durch einen \"aquivalenten unit\"aren Operator)
beschrieben. Diesen unit\"aren Operator kann man als Produkt von Operatoren zu einzelnen
Abschnitten der Wege (oder sehr kurzen Zeitabschnitten) schreiben. Die Matrixmultiplikation entspricht
aber einer \glqq Summation \"uber alle Wege im Indexraum\grqq. Um das zu verdeutlichen, betrachten
wir der Einfachheit halber das Produkt von drei Matrizen $A$, $B$ und $C$, wobei wir die Bra-Ket-Notation
von Dirac verwenden:
\begin{equation}
     \langle m | (CBA)| n \rangle 
            = \sum_{kl}  \langle m | C | l \rangle \langle l |B | k \rangle \langle  k| A| n \rangle  
\end{equation} 
F\"ur einen festen Wert von $n$ und $m$ (Anfangs- und Endzustand) erhalten wir somit
eine Darstellung des gesamten Matrixelements als Summe \"uber alle Wege 
(summiert wird \"uber alle $k$ und $l$) 
$ n \rightarrow k \rightarrow l \rightarrow m$ und zu jedem Teilprozess eines dieser Wege erhalten
wir als \glqq Amplitude\grqq\ das entsprechende Element der zugeh\"origen unit\"aren Matrix.
Die Wahrscheinlichkeit f\"ur einen \"Ubergang $n\rightarrow m$ bei dem Prozess $CBA$ 
ist dann das Quadrat dieser Amplitude. 
 
Beim Mach-Zehnder-Interferometer multiplizieren wir drei Matrizen: Die erste Matrix beschreibt
den ersten Strahlteiler, die zweite Matrix die Reflektion an einem Spiegel (dies ergibt im
Wesentlichen nur einen Faktor i) und die dritte Matrix
den zweiten Strahlteiler. 

\section{Das Michelson-Interferometer}

Beim Michelson-Interferometer trifft ein einfallender Strahl bzw.\ der Zustand eines einfallenden
Photons auf einen Strahlteiler und wird dort in zwei Anteile aufgespalten, die sich entlang
unterschiedlicher Wege ausbreiten. Beide Teilstrahlen werden an Spiegel totalreflektiert und
laufen wieder zur\"uck. Sie treffen schlie\ss lich wieder
auf den Strahlteiler, wo sie zusammengef\"uhrt werden und interferieren k\"onnen (siehe
Abb.\ \ref{fig_Michelson}).

\begin{SCfigure}[30][htb] 
%\unitlength 2pt
\begin{picture}(260,200)(0,0)
\thicklines
\put(70,60){\line(1,1){20}}
\put(190,60){\line(0,1){20}}
\put(192,60){\line(0,1){20}}
\put(70,180){\line(1,0){20}}
\put(70,182){\line(1,0){20}}
%\put(250,120){\line(1,1){20}}
\thinlines
\put(70,0){\line(1,0){20}}
\put(70,0){\line(0,1){20}}
\put(90,0){\line(0,1){20}}
\put(70,20){\line(1,0){20}}
%
\put(10,70){\line(1,0){180}}
\put(80,70){\line(0,1){110}}
\put(82,70){\line(0,1){110}}
\put(80,72){\line(1,0){110}}
\put(81,20){\line(0,1){50}}
\put(83,20){\line(0,1){50}}
\put(50,70){\vector(1,0){0}}
\put(150,70){\vector(1,0){0}}
\put(150,72){\vector(-1,0){0}}
\put(80,120){\vector(0,1){0}}
\put(82,120){\vector(0,-1){0}}
\put(81,40){\vector(0,-1){0}}
\put(83,40){\vector(0,-1){0}}
%
\put(55,80){\makebox(0,0){$|\gamma \rangle$}}
\put(150,80){\makebox(0,0){$| {\rm t} \rangle$}}
\put(70,120){\makebox(0,0){$|{\rm r} \rangle$}}
%\put(250,150){\makebox(0,0){$| {\rm D} \rangle$}}
%\put(275,120){\makebox(0,0){$|{\rm C} \rangle$}}
%
%\put(260,170){\makebox(0,0){\footnotesize D dunkel}}
%\put(300,130){\makebox(0,0){\footnotesize C hell}}
\put(45,57){\makebox(0,0){\footnotesize Strahlteiler}}
\put(80,190){\makebox(0,0){\footnotesize Spiegel}}
\put(210,70){\makebox(0,0){\footnotesize Spiegel}}
\put(108,10){\makebox(0,0){\footnotesize Detektor}}
%
\end{picture} 
\caption{\label{fig_Michelson}%
Michelson-Inter\-fero\-meter. Nachdem der Strahl von einem Strahlteiler
geteilt wurde, treffen die beiden Strahlteile auf Spiegel, die sie auf den
Strahlteiler zur\"ucklenken. Hinter dem Strahlteiler kann man mit einem
Detektor die Lichtintensit\"at messen. Durch winzige Variationen in den
optischen Wegl\"angen kann man Interferenzen beobachten.} 
\end{SCfigure}

Beim Michelson-Interferometer erh\"alt man zun\"achst nur einen Ausgang, hinter
dem man die Interferenz nachweisen kann. Das ist insofern kein Informationsverlust,
da der zweite Ausgang ohnehin nur die am ersten Ausgang fehlende Intensit\"at 
(zur einlaufenden Gesamtintensit\"at) aufzeigt. Der zweite Ausgang zeigt wieder zur\"uck
zum einfallenden Licht und sollte daher nicht mit Detektoren zugebaut werden. Es gibt
aber unterschiedliche Varianten dieses Aufbaus, bei denen auch hinter dem zweiten Ausgang
Detektoren stehen k\"onnen.

\section{Das \glqq Knallerexperiment\grqq}

Das sogenannte \glqq Knallerexperiment\grqq\ geht auf eine Arbeit von Avshalom Elitzur und
Lev Vaidman aus dem Jahr 1993 mit dem Titel ``Quantum mechanical interaction-free measurements'' 
zur\"uck \cite{Elitzur}.
Sie beschreiben darin die M\"oglichkeit, eine Superbombe auf ihre Funktionsf\"ahigkeit zu testen, ohne
diese Bombe zur Explosion bringen zu m\"ussen. F\"ur schulische Zwecke wurde aus der Superbombe
ein \glqq Knaller\grqq. 

Hier ist allerdings zu erw\"ahnen, dass die Ideen von Elitzur und Vaidman nicht neu waren, sie
haben sie lediglich in einen \glqq werbewirksamen\grqq\ Rahmen gepackt. Schon 1960 hat 
Mauritius Renninger in einem Artikel \glqq Messung ohne St\"orung des Messobjekts\grqq\ auf diese
M\"oglichkeit in der Quantentheorie aufmerksam gemacht \cite{Renninger}. Und schon 1934 schreibt 
Schr\"odinger in einem Artikel: \textit{\"Ubrigens ist es auch an sich nichts weniger als einleuchtend,
da\ss\ das Me\ss instrument ohne Wechselwirkung keine Aussagge \"uber das Objekt machen kann.
Eine Zielscheibe, die nicht getroffen wird, schlie\ss t zumindest gewisse Flugbahnen des Projektils
aus. Und wenn die Scheibe den Sch\"utzen als Hohlkugel umgibt, die blo\ss\ ein kleines Loch hat,
so gibt sie, ungetroffen, sehr genauen Aufschlu\ss\ \"uber die Flugbahn} \cite{Schroedinger}. Damit
ist die wesentliche Idee des Artikels von Elitzur und Vaidman schon vorweggenommen. 

Elitzur und Vaidman beschreiben folgende Situation: Gegeben ist ein gro\ss es Lager von
Superbomben, die einen sehr empfindlichen Ausl\"oser haben. Sobald ein Photon auf diesen
Ausl\"oser trifft und seine Energie \"ubertr\"agt, 
explodiert die Bombe. Andererseits hat man festgestellt, dass bei manchen
Bomben die Ausl\"oser fehlen und bei diesen Bomben ein Photon, das normalerweise auf einen 
Ausl\"oser treffen w\"urde, einfach hindurchgeht und an einer Seite wieder austritt. Ein normaler
Test, bei dem nachgeschaut wird, ob ein Ausl\"oser vorhanden ist, w\"urde die intakten Bomben 
explodieren lassen. Man m\"ochte jedoch m\"oglichst viele intake Bomben behalten k\"onnen.

\begin{SCfigure}[30][htb] 
%\unitlength1cm
\begin{picture}(130,125)(-5,0)
\thicklines
\put(15,15){\line(1,1){10}}
\put(95,15){\line(1,1){10}}
\put(96,14){\line(1,1){10}}
\put(15,95){\line(1,1){10}}
\put(14,96){\line(1,1){10}}
\put(95,95){\line(1,1){10}}
\thinlines
\put(0,20){\line(1,0){100}}
%\put(60,20){\circle*{10}}
\put(20,20){\line(0,1){80}}
\put(100,20){\line(0,1){80}}
\put(20,100){\line(1,0){80}}
\put(10,20){\vector(1,0){0}}
\put(60,20){\vector(1,0){0}}
\put(60,100){\vector(1,0){0}}
\put(20,60){\vector(0,1){0}}
\put(100,60){\vector(0,1){0}}
\put(100,100){\vector(1,0){10}}
\put(100,100){\vector(0,1){10}}
%
\put(110,116){\makebox(0,0){\footnotesize D dunkel}}
\put(125,97){\makebox(0,0){\footnotesize C hell}}
%
\end{picture} \hspace{1cm}
\begin{picture}(130,125)(0,0)
\thicklines
\put(15,15){\line(1,1){10}}
\put(95,15){\line(1,1){10}}
\put(96,14){\line(1,1){10}}
\put(15,95){\line(1,1){10}}
\put(14,96){\line(1,1){10}}
\put(95,95){\line(1,1){10}}
\thinlines
\put(0,20){\line(1,0){53}}
\put(60,20){\circle*{10}}
\put(20,20){\line(0,1){80}}
%\put(100,20){\line(0,1){80}}
\put(20,100){\line(1,0){80}}
\put(10,20){\vector(1,0){0}}
\put(40,20){\vector(1,0){0}}
\put(60,100){\vector(1,0){0}}
\put(20,60){\vector(0,1){0}}
%\put(100,60){\vector(0,1){0}}
\put(100,100){\vector(1,0){10}}
\put(100,100){\vector(0,1){10}}
%
\put(110,116){\makebox(0,0){\footnotesize D}}
\put(125,97){\makebox(0,0){\footnotesize C}}
%
\end{picture}
\caption{\label{fig_Bombe}%
Mach-Zehnder Interferometer ohne und mit
Hindernis in einem Strahlgang. Ohne Hindernis sollte der Detektor bei
D nie ansprechen, mit Hindernis in rund einem Viertel der F\"alle.} 
\label{fig_MZInterferometer}
\end{SCfigure}

Elitzur und Vaidman schlagen nun vor, ein Mach-Zehnder-Interferometer zum Test
der Bomben zu verwenden. Die Bomben werden so in den Strahlengang gelegt, dass
der Ausl\"oser, sofern vorhanden, ein Hindernis darstellt und, wenn er von einem Photon
getroffen wird, die Bombe ausgel\"ost wird. Bei Bomben, die defekt sind und bei denen 
kein Ausl\"oser vorhanden ist, ist der Strahlengang frei und ein Photon k\"onnte ungehindert
durch den fehlenden Ausl\"oser hindurchtreten. 

Falls die Bomben defekt sind, handelt es sich bei dem Aufbau um ein normales Mach-Zehnder-Interferometer.
Wenn die optischen Wegl\"angen entsprechend eingestellt sind, sollten Photonen letztendlich nur
in Detektor C nachgewiesen werden. Man k\"onnte entweder eine ausreichende Anzahl von Photonen
durch die Anordnung treten lassen und feststellen, dass immer nur Detektor C anspricht. Man kann
aber auch das Photon hinter dem Ausgang C durch entsprechende Spiegel wieder in das 
Mach-Zehnder-Interferometer leiten und w\"urde feststellen, dass ein Photon \glqq immer im Kreis l\"auft\grqq.
Dies w\"are das Zeichen f\"ur eine defekte Bombe.

Bei einer intakten Bombe befindet sich ein Hinternis in einem Strahlengang. In rund der H\"alfte der F\"alle
wird ein Photon auf die Bombe treffen und diese ausl\"osen. In der anderen H\"alfte der F\"alle 
kann man sich vorstellen, dass das Photon den anderen Strahlengang nimmt und dort hinter
dem zweiten Strahlteiler entweder in Detektor C oder Detektor D abgelenkt wird. Landet das Photon
in Detektor D wei\ss\ man, dass ein Hindernis vorhanden ist, und damit ist auch bekannt, dass die
Bombe intakt ist. Da das Photon aber im Detektor nachgewiesen wurde, ist es nicht auf das Hindernis
getroffen, es hat keine Wechselwirkung mit dem Hindernis stattgefunden und die Bombe bleibt intakt. 
Dies ist bei einer intakten Bombe in rund einem Viertel der F\"alle der Fall. Schlie\ss lich kann das 
Photon auch in Detektor C landen (ebenfalls ein Viertel der F\"alle); in diesem Fall erh\"alt man keine 
Information \"uber die Bombe. Ein solches Photon kann nochmals in das Mach-Zehnder-Interferometer
gelenkt werden, wieder wird in der H\"alfte der F\"alle die Bombe ausgel\"ost und in jeweils einem Viertel der
F\"alle landet das Photon in Detektor D (damit ist bekannt, dass die Bombe intakt ist) oder in Detektor
C (keine Information). Auf diese Weise kann man letztendlich rund ein Drittel aller intakten Bomben als
intakt erkennen und retten, wohingegen rund zwei Drittel aller intakten Bomben explodieren. 

Wir k\"onnen auch in diesem Fall zur Bestimmung der Wahrscheinlichkeiten die Feynman'sche
Summe \"uber alle M\"oglichkeiten nutzen. Bei einer intakten Bombe gibt es neben den 
M\"oglichkeiten, dass ein Photon in Detektor D oder C landet noch die M\"oglichhkeit, auf die Bombe
zu treffen und diese auszul\"osen. Dies geschieht mit Wahrscheinlichkeit 1 f\"ur einen Weg, der das
Photon auf die Bombe leitet. Zu den Amplituden f\"ur die Detektoren tr\"agt nun jeweils nur ein Weg bei:
\begin{eqnarray}
\nonumber
     \langle D | \gamma \rangle &=&  |\gamma \rangle \longrightarrow \mbox{ST} \longrightarrow
       | {\rm r} \rangle \longrightarrow \mbox{Spiegel} \longrightarrow |{\rm r}\rangle \longrightarrow 
       \mbox{ST} \longrightarrow |{\rm D}\rangle  \\
     & =  &   \Big( \frac{\rm i}{\sqrt{2}}\Big) \Big( {\rm i} \Big) \Big( \frac{\rm i}{\sqrt{2}}\Big) 
       =   - \frac{\rm i}{2} \\
 \nonumber      
     \langle C | \gamma \rangle &=&  |\gamma \rangle \longrightarrow \mbox{ST} \longrightarrow
       | {\rm r} \rangle \longrightarrow \mbox{Spiegel} \longrightarrow |{\rm r}\rangle \longrightarrow 
       \mbox{ST} \longrightarrow |{\rm C}\rangle  \\
     & = &   \Big( \frac{\rm i}{\sqrt{2}}\Big)\Big( {\rm i}\Big)\Big( \frac{1}{\sqrt{2}} \Big)
      =   - \frac{1}{2}  \\
 \langle \mbox{Bombe explodiert}|\gamma \rangle &=& 
\nonumber 
         |\gamma \rangle \longrightarrow \mbox{ST} \longrightarrow
       | {\rm t} \rangle \longrightarrow  \mbox{Bombe}  \\
        &=&   \Big( \frac{1}{\sqrt{2}}\Big)  \, . 
\end{eqnarray}
Das Absolutquadrat dieser Amplituden ergibt die Wahrscheinlichkeiten. Damit erhalten wir f\"ur
die beiden Wahrscheinlichkeiten, dass ein Photon in einem der Detektoren landet, jeweils
$1/4$, wohingegen die Wahrscheinlichkeit auf die Bombe zu treffen und diese
auszul\"osen gleich $1/2$ ist. 

F\"ur die experimentelle Realisation verwendet man meist ein Michelson-Interferometer (Abb.\ \ref{fig:MZ3}).
Die Bombe kann durch einen Spiegel ersetzt werden, der in den Strahlengang geschoben wird
und ein Photon, das eigentlich eine intakte Bombe ausl\"osen w\"urde, auf einen Detektor lenkt. 

\begin{figure}[htb]
\begin{picture}(280,160)(-10,0)
\thicklines
\put(0,100){\line(1,0){30}}
\put(18,100){\vector(1,0){1}}
\put(45,100){\line(2,1){30}}
\put(45,100){\line(1,0){225}}
\put(180,100){\line(0,-1){90}}
\put(61,108){\vector(2,1){1}}
\put(110,100){\vector(1,0){1}}
\put(120,100){\vector(-1,0){1}}
\put(180,70){\vector(0,-1){1}}
\put(180,60){\vector(0,1){1}}
\put(220,100){\vector(1,0){1}}
\put(230,100){\vector(-1,0){1}}
\thinlines
\put(165,50){\vector(1,0){10}}
\put(30,95){\line(1,0){15}}
\put(30,95){\line(0,1){10}}
\put(30,105){\line(1,0){15}}
\put(45,95){\line(0,1){10}}
\put(80,117){\circle{10}}
\put(170,110){\line(1,-1){20}}
\put(170,10){\line(1,0){20}}
\put(170,8){\line(1,0){20}}
\put(270,90){\line(0,1){20}}
\put(272,90){\line(0,1){20}}
\put(170,40){\line(-1,1){14}}
\put(171.5,41.5){\line(-1,1){14}}
\put(210,48){\circle{10}}
\put(180,130){\circle{10}}
%
\put(80,130){\makebox(0,0){\footnotesize Detektor}}
\put(180,143){\makebox(0,0){\footnotesize Detektor}}
%\put(180,143){\makebox(0,0){\footnotesize (dunkel)}}
\put(225,60){\makebox(0,0){\footnotesize Detektor}}
%\put(230,52){\makebox(0,0){\footnotesize (Bombe)}}
\put(215,87){\makebox(0,0){\footnotesize Strahlteiler}}
\put(180,2){\makebox(0,0){\footnotesize Spiegel}}
\put(280,117){\makebox(0,0){\footnotesize Spiegel}}
\put(148,40){\makebox(0,0){\footnotesize Spiegel}}
\put(148,30){\makebox(0,0){\footnotesize (Bombe)}}
%
\put(15,108){\makebox(0,0){\footnotesize UV}}
\put(37,88){\makebox(0,0){\footnotesize BBO}}
\end{picture}
\caption{Experimentelle Realisierung des \glqq Knallerexperiments\grqq\ in einem
Michelson-Interferometer.
In einem BBO-Kristall werden zwei Photonen erzeugt, von denen eines
nachgewiesen wird. Dadurch ist bekannt, dass sich ein zweites
Photon in der experimentellen Anordnung befindet.
Der bewegliche Spiegel (\glqq Bombe\grqq) kann nach rechts in den 
Strahlengang geschoben werden und lenkt das Photon auf den Detektor.
Der obere Detektor (dunkel) sollte nur Ereignisse anzeigen,
bei denen ein Hindernis im zweiten Strahlengang vorhanden ist. 
Der untere Detektor zeigt an, ob das Hindernis getroffen wurde.}
\label{fig:MZ3}
\end{figure}




\begin{thebibliography}{99}
\bibitem{Elitzur} Elitzur, A.C., Vaidman, L.; {\em Quantum mechanical
        interaction-free measurements}, Found.\ of Phys.\ {\bf 23}
        (1993) 987.
\bibitem{Renninger} Renninger, M.; {\em Messung ohne St\"orung des
        Me\ss objekts}; Z.\ Physik 158 (1960) 417.      
\bibitem{Schroedinger} Schr\"odinger, E.; {\em \"Uber die
        Unanwendbarkeit der Geometrie im Kleinen}; Die Naturwissenschaften
        31 (1934) 518--520.                 
\end{thebibliography}

\end{document}

