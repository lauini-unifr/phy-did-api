\documentclass[german,10pt]{article}      
\usepackage{makeidx}
\usepackage{babel}            % Sprachunterstuetzung
\usepackage{amsmath}          % AMS "Grundpaket"
\usepackage{amssymb,amsfonts,amsthm,amscd} 
\usepackage{mathrsfs}
\usepackage{rotating}
\usepackage{sidecap}
\usepackage{graphicx}
\usepackage{color}
\usepackage{fancybox}
\usepackage{tikz}
\usetikzlibrary{arrows,snakes,backgrounds}
\usepackage{hyperref}
\hypersetup{colorlinks=true,
                    linkcolor=blue,
                    filecolor=magenta,
                    urlcolor=cyan,
                    pdftitle={Overleaf Example},
                    pdfpagemode=FullScreen,}
%\newcommand{\hyperref}[1]{\ref{#1}}
%
\definecolor{Gray}{gray}{0.80}
\DeclareMathSymbol{,}{\mathord}{letters}{"3B}
%
\newcounter{num}
\renewcommand{\thenum}{\arabic{num}}
\newenvironment{anmerkungen}
   {\begin{list}{(\thenum)}{%
   \usecounter{num}%
   \leftmargin0pt
   \itemindent5pt
   \topsep0pt
   \labelwidth0pt}%
   }{\end{list}}
%
\renewcommand{\arraystretch}{1.15}                % in Formeln und Tabellen   
\renewcommand{\baselinestretch}{1.15}                 % 1.15 facher
                                                      % Zeilenabst.
\newcommand{\Anmerkung}[1]{{\begin{footnotesize}#1 \end{footnotesize}}\\[0.2cm]}
\newcommand{\comment}[1]{}
\setlength{\parindent}{0em}           % Nicht einruecken am Anfang der Zeile 

\setlength{\textwidth}{15.4cm}
\setlength{\textheight}{23.0cm}
\setlength{\oddsidemargin}{1.0mm} 
\setlength{\evensidemargin}{-6.5mm}
\setlength{\topmargin}{-10mm} 
\setlength{\headheight}{0mm}
\newcommand{\identity}{{\bf 1}}
%
\newcommand{\vs}{\vspace{0.3cm}}
\newcommand{\noi}{\noindent}
\newcommand{\leer}{}

\newcommand{\engl}[1]{[\textit{#1}]}
\parindent 1.2cm
\sloppy

         \begin{document} % \setcounter{chapter}{6}

\begin{itemize}
\item[6]
Pkt.\ 4: Weshalb $\psi i$ und nicht $\psi_i$?

Pkt.\ 5: Ich w\"urde das \glqq sofern m\"oglich\grqq\ hinter \glqq Eine direkt
wiederholte Messung\grqq.

Pkt.\ 6: Dies ist trickreich. Eigentlich vergleicht man ja zwei Zust\"ande
$\psi_1$, dies ist der Zustand, der pr\"apariert wurde, d.h., den wir vor der
Messung dem System zuschreiben, und $\psi_2$, dies ist der Zustand, den
wir aufgrund der Messergebnisse dem System nach der Messung zuschreiben.
Dann betrachten wir $|\langle \psi_2|\psi_1\rangle|^2$ als die Detektionswahrscheinlichkeit
f\"ur Zustand $\psi_2$ bei einer Messung an einem System, das sich in dem Zustand
$\psi_1$ befand. Wenn wir von der Wellenfunktion sprechen, hat diese ja ein Argument
(z.B.\ $x$ im Ortsraum oder $p$ im Impulsraum) und wir betrachten $|\psi(x)|^2$
(wobei $\psi(x) = \langle x | \psi \rangle$), d.h.\ die Detektionswahrscheinlichkeit(sdichte),
bei einer Messung eines Quantenobjekts im Zustand $\psi$ dieses am Ort $x$ zu
finden.   
\item[7]
Hier sollte man vielleicht erw\"ahnen, dass $\psi$ nur bis auf die Multiplikation mit
einer (komplexen) Zahl durch den Zustand bestimmt ist. 

Pkt.\ 1: Auch in der klassischen Physik sollte man zwischen Zustand (einem Punkt im
Phasenraum) und Observablen (Funktionen \"uber dem Phasenraum) und Messungen
(Protokoll zur Durchf\"uhrung eines Experiments) unterscheiden. Eine Observable
(sowohl in der klassischen Physik als auch der QPh) repr\"asentiert ja nicht den
Messvorgang, sondern die Informationen, die bei einer Messung gewonnen werden k\"onnen. 
\item[9]
Pkt.\ 5: siehe Seite 6 zu Pkt.\ 5.

Pkt.\ 6: Hier sollte man vielleicht erw\"ahnen, dass ein Zustand streng genommen
durch ein Ensemble gleichartig pr\"aparierter Systeme realisiert wird. Der Ausdruck
\glqq ein und derselbe Zustand\grqq\ bezieht sich auf zwei verschiedene Systeme aus diesem Ensemble.
\item[22]
Bemerkung zu \glqq Konsequenz\grqq: Hier w\"are ich vorsichtig. Wegen der Ununterscheidbarkeit
der Photonen ist bei koh\"arentem Licht streng genommen nicht mehr definiert, ob ein Photon 
mit sich selbst oder mit anderen interferiert.
\item[33]
Hier sieht man sch\"on, wie unser Denken unsere Sprache beeinflusst: F\"ur einen strengen
\glqq Kopenhagener\grqq\ oder \glqq Q-Bayesianer\grqq\ ist der Zustand vor der Messung 
der Zustand, durch den wir vor der Messung ein System beschrieben haben. Von irgendeinem
Zustand \glqq an sich\grqq\ zu sprechen ist sinnlos. Daher ist es auch sinnlos, im Nachhinein
einen \glqq Zustand vor der Messung\grqq\ bestimmen zu wollen. Ein Zustand beschreibt
unsere Erwartungen bez\"uglich zuk\"unftiger Messung (es handelt sich um ein Erwartungswertfunktional).
\item[34]
Auch hier sieht man, wie sehr unsere Vorstellungen \"uber das hinausgehen, was die QPh wirklich
sagt (bzw.\ was sich nachweisen l\"asst). Auch im oberen Experiment k\"onnen wir nicht mit
Bestimmtheit sagen, dass ein Teilchen, das in Detektor X landet, durch den unteren Spalt
getreten ist. In der Bohm'schen Mechanik ist das auch nicht der Fall: Es gibt ein sogenanntes
no-crossing Theorem, das besagt, dass sich Bahnkurven in der Bohm'schen Mechanik nicht
schneiden k\"onnen. Ein Teilchen in Detektor X ist daher durch den oberen Spalt getreten.
Wir k\"onnen nicht nachweisen, dass es anders ist (au\ss er, indem wir einen Spalt zuhalten, aber
dann \"andern wir den experimentellen Aufbau). 

G\"abe es in der QPh eine M\"oglichkeit, experimentell zu beweisen, dass ein Teilchen in
Detektor X durch den unteren Spalt getreten ist, k\"onnten wir zeigen, dass die Bohm'sche
Mechanik falsch ist.
\item[42]
Die Sache mit den Kombinationen scheint mir sehr verwirrend. Au\ss erdem ist die Anzahl der
Terme in einer Entwicklung von der Basis abh\"angig. Ich denke, gemeint ist letztendlich, dass sich
bei einer Messung an einem Teilsystem (ich ziehe eigentlich \glqq Freiheitsgrad\grqq\ vor, da es
auch Verschr\"ankung an einem Einzelsystem - z.B.\ Ortsfreiheitsgrad und Polarisationsfreiheitsgrad - 
geben kann) die Vorhersagem\"oglichkeiten f\"ur die m\"oglichen Messergebnisse an dem
zweiten Teilsystem (und damit den Zustand des zweiten Teilsystems) \"andern. Die Sache mit
den Erhaltungss\"atzen bedeutet ja eigentlich: Man kann Eigenschaften des Gesamtsystems
festlegen (pr\"aparieren), die f\"ur die Einzelsysteme mehrere korrelierte M\"oglichkeiten zulassen.
\item[43]
Der Zustand $\psi_{1r}\psi_{2r}$ hat noch weniger Kombinationen, trotzdem ist er nicht
verschr\"ankt. Irgendwie scheinen die Beispiele so gew\"ahlt, dass zwar alles hinkommt, aber
man darf sich keine anderen Beispiele \"uberlegen.
\item[47]
Die Verschr\"ankung gilt nur bez\"uglich der Photonen, die entlang des gelben und blauen
Strahls emittiert werden. Bei den anderen m\"oglichen Austrittsrichtungen auf den beiden Kreisen
sind die Photonen nicht verschr\"ankt. 
\item[59]
Die Tatsache, dass keine Korrelationen bez\"uglich einer bestimmten Basis auftreten, hei\ss t
nicht, dass die Teilchen nicht verschr\"ankt sind. Es k\"onnte ja Korrelationen bzgl.\ einer
anderen Basis geben.
\item[70]
Die Abbildung passt nicht ganz zum Text: In der Abbildung ist gezeigt, wie man verschr\"ankte
Photonen herstellen und in einen gro\ss en Abstand bringen kann. Die Abbildung hat eigentlich
nichts mit Quantenteleportation zu tun. Man kann allerdings mit Quantenteleportation erreichen,
dass zwei sehr weit entfernte Photonen verschr\"ankt sind, obwohl diese nie in engerem
Kontakt waren (bzw.\ aus einer Quelle erzeugt wurden). 
\end{itemize}

\end{document}

