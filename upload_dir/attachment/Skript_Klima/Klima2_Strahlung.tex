\documentclass[german,10pt]{book}      
\usepackage{makeidx}
\usepackage{babel}            % Sprachunterstuetzung
\usepackage{amsmath}          % AMS "Grundpaket"
\usepackage{amssymb,amsfonts,amsthm,amscd} 
\usepackage{mathrsfs}
\usepackage{rotating}
\usepackage{sidecap}
\usepackage{graphicx}
\usepackage{color}
\usepackage{fancybox}
\usepackage{tikz}
\usetikzlibrary{arrows,snakes,backgrounds}
\usepackage{hyperref}
\hypersetup{colorlinks=true,
                    linkcolor=blue,
                    filecolor=magenta,
                    urlcolor=cyan,
                    pdftitle={Overleaf Example},
                    pdfpagemode=FullScreen,}
%\newcommand{\hyperref}[1]{\ref{#1}}
%
\definecolor{Gray}{gray}{0.80}
\DeclareMathSymbol{,}{\mathord}{letters}{"3B}
%
\newcounter{num}
\renewcommand{\thenum}{\arabic{num}}
\newenvironment{anmerkungen}
   {\begin{list}{(\thenum)}{%
   \usecounter{num}%
   \leftmargin0pt
   \itemindent5pt
   \topsep0pt
   \labelwidth0pt}%
   }{\end{list}}
%
\renewcommand{\arraystretch}{1.15}                % in Formeln und Tabellen   
\renewcommand{\baselinestretch}{1.15}                 % 1.15 facher
                                                      % Zeilenabst.
\newcommand{\Anmerkung}[1]{{\begin{footnotesize}#1 \end{footnotesize}}\\[0.2cm]}
\newcommand{\comment}[1]{}
\setlength{\parindent}{0em}           % Nicht einruecken am Anfang der Zeile 

\setlength{\textwidth}{15.4cm}
\setlength{\textheight}{23.0cm}
\setlength{\oddsidemargin}{1.0mm} 
\setlength{\evensidemargin}{-6.5mm}
\setlength{\topmargin}{-10mm} 
\setlength{\headheight}{0mm}
\newcommand{\identity}{{\bf 1}}
%
\newcommand{\vs}{\vspace{0.3cm}}
\newcommand{\noi}{\noindent}
\newcommand{\leer}{}

\newcommand{\engl}[1]{[\textit{#1}]}
\parindent 1.2cm
\sloppy

         \begin{document}  \setcounter{chapter}{0}


\chapter{Physik des Klimas II\\Strahlungsgesetze}
% Kap x
\label{chap_Klima2}

In die Klimamodelle der Erde gehen verschiedene Strahlungsgesetze ein: das Stefan-Boltzmann-Gesetz,
das Wien'sche Verschiebungsgesetz, sowie das Planck'sche Strahlungsgesetz, aus dem sich die
ersten beiden Gesetze ableiten lassen. Man findet jedoch unterschiedliche Versionen des
Planck'schen Strahlungsgesetzes: Zum Einen wird das Gesetz gelegentlich als Funktion der
Strahlungsfrequenz oder auch als Funktion der Wellenl\"ange ausgedr\"uckt, was zu unterschiedlichen
Vorfaktoren f\"uhrt. Zum Anderen sollte man aber auch zwischen dem Strahlungsgesetz in Form
einer Energiedichte in einem Hohlk\"orper und in Form einer Flussdichte, die von einem Schwarzen
K\"orper abgestrahlt wird, unterscheiden.

Der erste Abschnitt beschreibt diese Unterschiede. Im zweiten Abschnitt wird das Planck'sche
Strahlungsgesetz zun\"achst in Form einer Energiedichte f\"ur einen Hohlk\"orper abgeleitet und dann
auf die anderen Formen verallgemeinert. Schlie\ss lich werden das Stefan-Boltzmann-Gesetz und
das Wien'sche Verschiebungsgesetz abgeleitet.

\section{Verschiedene Strahlungsgesetze - \"Uberblick}

\section{Herleitung der Planck'schen Formel}
\label{sec_Planck}

Bestimmt werden soll die spektrale
Verteilungsfunktion 
\begin{equation}
     u(T,\nu) = \frac{1}{V} 
         \frac{{\rm d} E(T,\nu)}{{\rm d}  \nu} \, ,
\end{equation}     
wobei $E(T,\nu)$ die Gesamtenergie der 
elektromagnetischen Strahlung des Systems
ist, die bei einer absoluten Temperatur $T$ auf 
Frequenzen kleiner oder gleich
$\nu$ entf\"allt. $V$ ist das Volumen des
Systems, man ist also an einer Energiedichte
interessiert. Man kann die Gleichung auch in folgender Form 
schreiben, die eine andere Sichtweise erlaubt:
\begin{equation}
     \frac{1}{V} {\rm d} E(T,\nu) = u(T,\nu)  {\rm d}  \nu   
\end{equation}     
${\rm d} E(T,\nu)$ ist der Anteil der Energie, der bei  einer Temperatur $T$
von den Moden im Bereich $[\nu, \nu + {\rm d} \nu]$ herr\"uhrt.

Die Verteilungsfunktion $u(T,\nu)$ setzt sich aus
zwei Anteilen zusammen:
\begin{itemize}
\item[-]
der spektralen Dichte $N(\nu)$ der thermodynamischen Freiheitsgrade
bei $\nu$ bzw.\ der Anzahl $N(\nu){\rm d} \nu$
der thermodynamischen
Freiheitsgrade in einem Frequenzbereich zwischen $\nu$
und $\nu + {\rm d}  \nu$. Etwas genauer definiert man diese Gr\"o\ss e, indem man
zun\"achst die kumulative Spektralverteilung
\begin{equation}
      g(\nu) = \sum_{\nu_i\leq \nu} 1
\end{equation}
betrachtet. $g(\nu)$ ist also gleich der Anzahl der Moden, deren Frequenz $\nu_i$ kleiner oder
gleich $\nu$ ist. Die spektrale Dichte ist dann definiert als die Ableitung dieser Funktion:
\begin{equation}
      N(\nu) = \frac{{\rm d} g(\nu)}{{\rm d} \nu}  \, .
\end{equation}
Da es sich bei $g(\nu)$ um eine Stufenfunktion handelt, ist diese Ableitung streng genommen
im distributiven Sinne zu verstehen. Diese Dichte ist unabh\"angig von der Temperatur und ergibt sich aus 
einem einfachen Abz\"ahlen der m\"oglichen Schwingungsmoden:
\begin{equation}
\label{eq_PlanckDichte}
      \frac{N(\nu)}{V} = \frac{4 \pi \nu^2}{c^3} 
\end{equation} 
Diese Beziehung war schon zu Plancks Zeiten bekannt
und unstrittig.

Zur Herleitung dieser Formel: Zun\"achst l\"osen wir ein einfaches mathematisches
Problem: Wie viele nat\"urliche Zahlentripel $(n_1,n_2,n_3)$ gibt es, sodass 
$\sqrt{n_1^2+n_2^2+n_3^2}\leq R$? Wir sind dabei nicht an der genauen Anzahl $N(R)$ dieser
Zahlentripel interessiert, sondern nur an der f\"uhrenden Ordnung f\"ur gro\ss e Werte von $R$. 
Wir denken uns das Tripel $(n_1,n_2,n_3)$ als Koordinaten von Punkten im
$\mathbb{R}^3$ und stellen zun\"achst die Frage: Wie viele Punkte im $\mathbb{R}^3$
gibt es, die ganzzahlige Koordinaten haben und innerhalb (oder auf dem Rand) einer
Kugel vom Radius $R$ liegen. Wir k\"onnen und um jeden dieser Punkte einen W\"urfel
der Kantenl\"ange 1 - und damit vom Volumen 1 - legen, und diese W\"urfel
\"uberdecken den Raum vollst\"andig ohne zu \"uberlappen. Die Kugel vom Radius $R$
hat ein Volumen von $V=\frac{4\pi}{3}R^3$, und damit ist dies n\"aherungsweise auch gleich der
Anzahl an Punkten mit ganzzahligen Koordinaten, die innerhalb der Kugel liegen.
Die Korrekturen beziehen sich auf Fl\"achen und sind somit von der Ordnung $O(1/R)$
relativ zum f\"uhrenden Term. 

Da wir nur an nat\"urlichen Zahlen interessiert sind, d\"urfen wir nur einen der acht
3-dim.\ Quadranten betrachten (f\"ur den die Vorzeichen aller drei Koordinaten nicht
negativ sind). Somit ist die gesuchte f\"uhrende Ordnung $N(R)=\frac{\pi}{6}R^3$.

In einem n\"achsten Schritt bestimmen wir die Anzahl der Zust\"ande $N(E)$, f\"ur die die
Energie eines einzelnen Photons in einem Kasten kleiner als ein vorgegebener Wert $E$ ist. 


Wir verwenden nun das obige Ergebnis und bestimmen die Anzahl der elektromagnetischen
Zust\"ande mit einer Energie $\epsilon \leq E$ in einem kubischen Kasten der Kantenl\"ange $L$
und damit dem Volumen $V=L^3$. Die m\"oglichen Wellenl\"angen in jede der drei Richtungen 
sind durch $\lambda= 2 L/n$ gegeben: In jede Richtung
muss ein Vielfaches einer halben Wellenl\"ange passen.
F\"ur die m\"oglichen Zust\"ande im $\vec{k}$-Raum ergibt
sich damit $\vec{k}=\frac{\pi}{L}(n_1,n_2,n_3)$ mit
beliebigen nat\"urlichen Zahlen $n_i$. Jedem
Zustand im $\vec{k}$-Raum kann man daher
ein Volumen $|\Delta k|^3=\pi^3/V$ zuschreiben. 
Eine Kugel vom Radius $k=|\vec{k}|$
hat ein Volumen von $\frac{4\pi}{3} k^3$
und enth\"alt somit 
\begin{equation}
   \frac{4\pi k^3}{3 |\Delta k|^3} =  V\frac{4}{3\pi^2}k^3 
\end{equation}    
Zust\"ande.
Da aber nur der Teil der Kugelschale
von Interesse ist, der sich im positiven Quadranten
befindet (die $n_i$ sind nie negativ), m\"ussen
wir noch durch 8 dividieren; andererseits kann
jeder Zustand wegen der beiden Polarisationsm\"oglichkeiten
zweimal auftreten, sodass die Anzahl der Zust\"ande 
durch $\frac{V}{3\pi^2}k^3$ gegeben ist.
Mit $|\vec{k}|= \frac{2 \pi}{\lambda}=\frac{2\pi \nu}{c}$
ergeben sich schlie\ss lich
\begin{equation}
        N(\nu)  = V
         \frac{4 \pi \nu^2}{c^3}{\rm d}  \nu
\end{equation}
Zust\"ande in diesem $\nu$-Bereich.

\item[-]
der mittleren Energie $\langle \epsilon (\nu, T)\rangle$, die 
ein gegebener Freiheitsgrad mit der Frequenz $\nu$ bei
einer Temperatur $T$ hat. Das Ergebnis ist dann
\begin{equation}
        u(T,\nu) = \frac{N(\nu)}{V} \langle \epsilon (\nu, T)\rangle \, , 
\end{equation}
also das Produkt aus der Anzahl der Zust\"ande zu einer gegebenen
Frequenz $\nu$ und der mittleren Energie eines Zustands mit
dieser Frequenz. 
\end{itemize}

F\"ur eine quantenmechanische Herleitung
der gesuchten Strahlungsdichte \"andert sich
an der Abz\"ahlung der m\"oglichen Moden
nichts. Jede Mode $\nu$ kann mit $n$
Photonen mit einer Gesamtenergie von $E_n(\nu)=n h \nu$
besetzt sein, und die Wahrscheinlichkeit einer
solchen Besetzung
ist proportional zum 
Boltzmann-Faktor\index{Boltzmann-Faktor} 
$\exp(-E_n(\nu)/k_{\rm B}T)=\exp(-n h \nu/k_{\rm B}T)$. Als
Mittelwert f\"ur die Energie zu einer solchen
Mode erhalten wir somit:
\begin{equation}
\label{eq_PlanckSummen}
  \langle \epsilon (\nu, T)\rangle = \frac{1}{Z} \sum_{n=0}^\infty
    (n h \nu) \exp \left( - \frac{n h \nu}{k_{\rm B}T}
     \right)  ~~~~ {\rm mit} ~~
     Z= \sum_{n=0}^\infty \exp\left( - \frac{n h \nu}{k_{\rm B}T}
     \right) 
\end{equation}
Bei den auftretenden Summen handelt es sich um
geometrische Reihen und deren Ableitungen,
die sich geschlossen ausf\"uhren lassen:
\begin{equation}
   \langle \epsilon(\nu,T)\rangle = \frac{ h \nu}{\exp \left(
    \frac{ h \nu}{k_{\rm B} T} \right)  - 1} 
\end{equation}
Als Ergebnis dieser \"Uberlegungen erhalten wir
die Planck'sche Formel:\index{Planck'sche Strahlungsformel}
\begin{equation}
\label{eq_PlanckFormel}
      u(\nu,T) = \frac{4\pi}{c^3} \frac{h \nu^3}{\exp \left(
    \frac{\hbar \nu}{k_{\rm B} T} \right)  - 1}
\end{equation}

Oft findet man die Strahlungsformeln als
Funktion der Frequenz $\nu$ oder der Wellenl\"ange $\lambda$. Bei der Umrechnung
muss man jedoch darauf achten, dass es sich bei $u(\nu,T)$ um
eine {\em Dichte} handelt, d.h., auch das Integrationsma\ss\
${\rm d} \nu$ muss transformiert werden.

\subsection{Anmerkungen}

\begin{anmerkungen}
\item
\label{Anm-1}%
Der Grund f\"ur den Faktor 2 zwischen der Schwankung im Abstand und
der Schwankung in der Intensit\"at der Sonneneinstrahlung liegt in dem $1/r^2$-Gesetz
der Intensit\"at als Funktion des Abstands:
\begin{equation}
        \frac{1}{(r\pm \Delta r)^2} \approx \frac{1}{r^2} \mp 2 \frac{\Delta r}{r} \, .
\end{equation}

\item
\label{Anm-2}
Das Stefan-Boltzmann-Gesetz gibt die Intensit\"at $L$ einer Schwarzk\"orper-Strahlung
als Funktion der absoluten Temperatur $T$ des K\"orpers an. 
Intensit\"at ist dabei wieder Energie pro Zeit und Fl\"ache, d.h.,
es handelt sich um die in einer Zeiteinheit (Sekunde) von einer Fl\"ache (Quadratmeter)
abgestrahlte Energie eines idealen schwarzen K\"orpers, integriert \"uber alle Frequenzen bzw.\
Wellenl\"angen. Das Gesetz lautet:
\begin{equation}
        L = \sigma T^4  \hspace{1.5cm} \sigma \equiv \mbox{Stefan-Boltzmann-Konstante.}
\end{equation}
Die Stefan-Boltzmann-Konstante l\"asst sich exakt aus dem Planck'schen Strahlungsgesetz
bestimmen:
\begin{equation}
        \sigma = \frac{2 \pi^5 k_{\rm B}^4}{15 h^3 c^2} = 5,670\,374\,42... \cdot 10^{-8} \frac{\rm W}{\rm m^2 K^4} \, .
\end{equation}
\end{anmerkungen}



\begin{thebibliography}{99}
\bibitem{Filk} Thomas Filk; \textit{Quantenmechanik (nicht nur) f\"ur Lehramtsstudierende};
         Springer-Verlag, 2019.  
\bibitem{Solar} Solarkonstante; Uni Kassel,\\
       {\small \verb+https://www.greenrhinoenergy.com/solar/radiation/images/solar-constant.jpg+}         
\bibitem{NASA_World} NASA Image and Video Library, 
       {\small \verb+https://images.nasa.gov/details/as17-148-22727+}
\bibitem{NASA_facts} NASA Earth Fact Sheet; 
      {\small \verb+ https://nssdc.gsfc.nasa.gov/planetary/factsheet/earthfact.html+}   
\bibitem{Petty} Grant W.\ Petty; \textit{A First Course in Atmospheric Radiation}, 2nd Ed.; Sundog
         Publishing, Madison, Wisconsin; 2006.       
\bibitem{Wikipedia_Faint} Wikipedia \glqq Faint young Sun paradox\grqq\ 
       {\small \verb+https://en.wikipedia.org/wiki/Faint_young_Sun_paradox+}.                 
\bibitem{Wikipedia_Milankovic} Wikipedia \glqq Milankovic-Zyklen\grqq.   
       {\small \verb+https://de.wikipedia.org/wiki/Milankovi+}{\small\tt \'c}{\small \verb+-Zyklen+}.        
\bibitem{Wikipedia_Solarkonstante} Wikipedia \glqq Solarkonstante\grqq.   
       {\small \verb+https://de.wikipedia.org/wiki/Solarkonstante+}.               
\end{thebibliography}

\end{document}

