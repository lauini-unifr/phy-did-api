\documentclass[german,10pt]{book}      
\usepackage{makeidx}
\usepackage{babel}            % Sprachunterstuetzung
\usepackage{amsmath}          % AMS "Grundpaket"
\usepackage{amssymb,amsfonts,amsthm,amscd} 
\usepackage{mathrsfs}
\usepackage{rotating}
\usepackage{sidecap}
\usepackage{graphicx}
\usepackage{color}
\usepackage{fancybox}
\usepackage{tikz}
\usetikzlibrary{arrows,snakes,backgrounds}
\usepackage{hyperref}
\hypersetup{colorlinks=true,
                    linkcolor=blue,
                    filecolor=magenta,
                    urlcolor=cyan,
                    pdftitle={Overleaf Example},
                    pdfpagemode=FullScreen,}
%\newcommand{\hyperref}[1]{\ref{#1}}
%
\definecolor{Gray}{gray}{0.80}
\DeclareMathSymbol{,}{\mathord}{letters}{"3B}
%
\newcounter{num}
\renewcommand{\thenum}{\arabic{num}}
\newenvironment{anmerkungen}
   {\begin{list}{(\thenum)}{%
   \usecounter{num}%
   \leftmargin0pt
   \itemindent5pt
   \topsep0pt
   \labelwidth0pt}%
   }{\end{list}}
%
\renewcommand{\arraystretch}{1.15}                % in Formeln und Tabellen   
\renewcommand{\baselinestretch}{1.15}                 % 1.15 facher
                                                      % Zeilenabst.
\newcommand{\Anmerkung}[1]{{\begin{footnotesize}#1 \end{footnotesize}}\\[0.2cm]}
\newcommand{\comment}[1]{}
\setlength{\parindent}{0em}           % Nicht einruecken am Anfang der Zeile 

\setlength{\textwidth}{15.4cm}
\setlength{\textheight}{23.0cm}
\setlength{\oddsidemargin}{1.0mm} 
\setlength{\evensidemargin}{-6.5mm}
\setlength{\topmargin}{-10mm} 
\setlength{\headheight}{0mm}
\newcommand{\identity}{{\bf 1}}
%
\newcommand{\vs}{\vspace{0.3cm}}
\newcommand{\noi}{\noindent}
\newcommand{\leer}{}

\newcommand{\engl}[1]{[\textit{#1}]}
\parindent 1.2cm
\sloppy

         \begin{document}  \setcounter{chapter}{1}


\chapter{Physik des Klimas II\\Strahlungsgesetze}
% Kap x
\label{chap_Klima2}

\info{Thomas Filk}{30.03.2024}%
In die Klimamodelle der\index{Strahlungsgesetze|(} 
Erde gehen verschiedene Strahlungsgesetze ein: das Stefan-Boltzmann-Gesetz,
das Wien'sche Verschiebungsgesetz und das Planck'sche Strahlungsgesetz, aus dem sich die
ersten beiden Gesetze ableiten lassen. Man findet jedoch unterschiedliche Versionen des
Planck'schen Strahlungsgesetzes: Zum einen wird das Gesetz gelegentlich als Funktion der
Strahlungsfrequenz oder auch als Funktion der Wellenl\"ange ausgedr\"uckt, was zu unterschiedlichen
Vorfaktoren f\"uhrt. Zum anderen sollte man aber auch zwischen dem Strahlungsgesetz in Form
einer Energiedichte in einem Hohlk\"orper und in Form einer Flussdichte, die von einem Schwarzen
K\"orper abgestrahlt wird, unterscheiden.

Der erste Abschnitt gibt einen \"Uberblick \"uber diese Strahlungsgesetze. 
Die weiteren Abschnitte enthalten Details zu ihrer Ableitung und k\"onnen \"ubersprungen werden. 
Im zweiten Abschnitt wird das Planck'sche
Strahlungsgesetz zun\"achst in Form einer Energiedichte f\"ur einen Hohlk\"orper abgeleitet und dann
auf die anderen Formen verallgemeinert. Schlie\ss lich werden das Wien'sche Verschiebungsgesetz 
und das Stefan-Boltzmann-Gesetz abgeleitet.

\section{Verschiedene Strahlungsgesetze -- \"Uberblick}

Das Planck'sche Strahlungsgesetz in der Form, wie es in Abschnitt \ref{sec_Planck} abgeleitet wird,
lautet:\index{Strahlungsgesetze!Planck}\index{Planck'sches Gesetz}
\begin{equation}
\label{eq_PlanckFormel_a}
      u(\nu,T) = \frac{8\pi}{c^3} \frac{h \nu^3}{\exp \left(
    \frac{h \nu}{k_{\rm B} T} \right)  - 1}  \, .
\end{equation}
Hierbei ist $u(\nu,T)$ der Anteil der Energiedichte (also Energie/Volumen) in einem
Hohlk\"orper bei der absoluten Temperatur $T$, der auf die elektromagnetische Strahlung in diesem
Hohlk\"orper mit der Frequenz $\nu$ entf\"allt. $c$ ist die Lichtgeschwindigkeit im Vakuum, $h$
die Planck'sche Konstante und $k_{\rm B}$ die Boltzmann-Konstante. Man spricht auch manchmal
von spektraler, d.h.\ nach der Frequenz oder Wellenl\"ange aufgeteilter Energiedichte.
Von den W\"anden des
Hohlk\"orpers wird angenommen, dass sie alle Frequenzen elektromagnetischer Strahlung
absorbieren bzw.\ emittieren k\"onnen und dass sie sich im thermischen Gleichgewicht mit der Strahlung im
Inneren befinden. 

Bei der spektralen Energiedichte sollte man betonen, dass es sich auch bez\"uglich der Frequenz
um eine Dichte bzw.\ Verteilung (Distribution) handelt, d.h., erst das Integral \"uber ein Frequenzintervall 
liefert eine interpretierbare Gr\"o\ss e. Insbesondere ist $\int_0^\nu u(\nu',T)\,{\rm d}\nu'$ der Anteil der
Energiedichte, der auf die Frequenzen zwischen $0$ und $\nu$ entf\"allt. 

Oft findet man diese Formel auch als Funktion der Wellenl\"ange $\lambda= c/\nu$. 
Hierbei ist zu beachten, dass man in Gl.\ \ref{eq_PlanckFormel_a} nicht nur $\nu$ entsprechend
durch $\lambda$ ersetzen muss, sondern es muss auch, da es sich bei $u(\nu,T)$ um eine Dichte handelt, 
das Integrationsma\ss\ ${\rm d}\nu$ transformiert werden:
\begin{equation}
            {\rm d}\nu = \frac{{\rm d}\nu}{{\rm d}\lambda} {\rm d}\lambda = - \frac{c}{\lambda^2} {\rm d}\lambda \, .
\end{equation}
Insgesamt erhalten wir somit
\begin{equation}
     \tilde{u}(\lambda, T) =  \frac{8\pi}{\lambda^5} \frac{h c}{\exp \left(
    \frac{h c}{\lambda k_{\rm B} T} \right)  - 1}  \, .
\end{equation}

Bei $u(\nu,T)$ und $\tilde{u}(\lambda,T)$ handelt es sich um spektrale Energiedichten, die
man beispielsweise mit einer Sonde im Inneren eines Hohlk\"orpers messen kann. In vielen
F\"allen ist man aber an der Abstrahlung eines Schwarzk\"orpers interessiert. Dazu stellt
man sich vor, die W\"ande des Hohlk\"orpers w\"urden pl\"otzlich weggenommen. Dann breitet
sich f\"ur einen kurzen Moment
die Energie mit Lichtgeschwindigkeit in alle Richtungen, d.h.\ einen vollen Raumwinkel
$4\pi$ aus. Die abgestrahlte Energie pro Fl\"ache, Zeit und Raumwinkel, d.h.\ die spektrale
Strahldichte $L(\nu,T)$, erh\"alt man somit, indem 
man Gl.\ \ref{eq_PlanckFormel_a} mit $c/4\pi$ multipliziert:\index{Planck'sches Strahlungsgesetz}
\begin{equation}
\label{eq_PlanckFormel_b}
      L(\nu,T) = \frac{2}{c^2} \frac{h \nu^3}{\exp \left(
    \frac{h \nu}{k_{\rm B} T} \right)  - 1}  \, .
\end{equation}

Das Wien'sche Verschiebungsgesetz\index{Wien'sches Verschiebungsgesetz} 
macht eine Aussage \"uber die Frequenz $\nu_{\rm max}(T)$
bzw.\ Wellenl\"ange $\lambda_{\rm max}(T)$, bei der die Intensit\"at der Energiedichte bzw.\
Strahldichte am gr\"o\ss ten ist. Die Frequenz nimmt dabei linear mit der Temperatur $T$ zu und
die Wellenl\"ange nimmt invers mit der Temperatur ab:
\begin{equation}
\label{eq_K_Wien}
          \nu_{\rm max}(T) = 5,8789 \cdot 10^{10} \cdot T \, \frac{1}{\rm Ks}  \hspace{1cm} {\rm und} \hspace{1cm}
          \lambda_{\rm max}(T) = \frac{2,8978 \cdot 10^{-3}}{T} \cdot {\rm Km} \, .      
\end{equation}
F\"ur eine effektive Oberfl\"achentemperatur von 5\,772\,K, wie sie f\"ur die Sonne angegeben wird \cite{NASA_sun}, 
folgt:
\begin{equation}
             \nu_{\rm max} = 3,4 \cdot 10^{14} \, {\rm Hz}   \hspace{1cm} {\rm und} \hspace{1cm}
             \lambda_{\rm max} = 500 \,{\rm nm} \, .
\end{equation}
Dies entspricht sichtbarem Licht im gr\"un-blauen Bereich. 
F\"ur die Erde mit einer Durchschnittstemperatur von 297\,K ergibt sich:
\begin{equation}
             \nu_{\rm max} = 17,5 \cdot 10^{12} \, {\rm Hz}   \hspace{1cm} {\rm und} \hspace{1cm}
             \lambda_{\rm max} = 9,72 \,{\rm \mu m} \, .
\end{equation}
Diese Strahlung liegt im infraroten Bereich.

Schlie\ss lich\index{Strahlungsgesetze!Stefan-Boltzmann-Gesetz}\index{Stefan-Boltzmann-Gesetz} 
ben\"otigen wir in der Klimaphysik noch das Stefan-Boltzmann-Gesetz. Es gibt die
Gesamtintensit\"at der Strahlung an, also das Integral der spektralen Strahldichte \"uber alle
Frequenzen und Richtungen, multipliziert mit der abstrahlenden Fl\"ache $A$. Man erh\"alt dann
\begin{equation}
                   I(T) =  A  \int L(\nu,T)\cos \theta \, {\rm d}\nu \, {\rm d}\Omega 
                   = \sigma A  T^4  \hspace{1cm} {\rm mit} \hspace{0.7cm}
                   \sigma = 5,67 \cdot 10^{-8} \frac{\rm W}{\rm m^2 K^4}  \, .
\end{equation}
Der Faktor $\cos \theta$ gibt die Strahlintensit\"at in eine Richtung $\theta$ zur Normalen
an und wird in Abschnitt \ref{sec_K_Abstr} n\"aher begr\"undet. 
Die Konstante $\sigma$ bezeichnet man als Stefan-Boltzmann-Konstante.\index{Stefan-Boltzmann-Konstante} 

In den folgenden Abschnitten werden diese Formeln hergeleitet. 

\section{Herleitung der Planck'schen Formel}
\label{sec_Planck}

Bestimmt werden soll die spektrale Verteilungsfunktion 
\begin{equation}
     u(\nu,T) = \frac{1}{V} 
         \frac{{\rm d} E(\nu,T)}{{\rm d}  \nu} \, ,
\end{equation}     
wobei $E(\nu,T)$ die Gesamtenergie der 
elektromagnetischen Strahlung des Systems
ist, die bei einer absoluten Temperatur $T$ auf 
Frequenzen kleiner oder gleich
$\nu$ entf\"allt. $V$ ist das Volumen des
Systems, man ist also an einer Energiedichte
interessiert. Man kann die Gleichung auch in folgender Form 
schreiben, die eine andere Sichtweise erlaubt:
\begin{equation}
     \frac{1}{V} {\rm d} E(\nu,T) = u(\nu,T)  {\rm d}  \nu   
\end{equation}     
${\rm d} E(\nu,T)$ ist der Anteil der Energie, der bei  einer Temperatur $T$
von den Moden im Bereich $[\nu, \nu + {\rm d} \nu]$ herr\"uhrt.

Die Verteilungsfunktion $u(\nu,T)$ setzt sich aus
zwei Anteilen zusammen: (1) der spektralen Dichte $n(\nu)$ der thermodynamischen Freiheitsgrade
mit einer Frequenz $\nu$ und (2) der mittleren Energie $\langle \epsilon (\nu, T)\rangle$, die 
ein gegebener Freiheitsgrad mit der Frequenz $\nu$ bei
einer Temperatur $T$ hat. Das Ergebnis ist dann
\begin{equation}
        u(\nu,T) = \frac{n(\nu)}{V} \langle \epsilon (\nu, T)\rangle \, , 
\end{equation}
also das Produkt aus der Anzahl der Zust\"ande zu einer gegebenen
Frequenz $\nu$ und der mittleren Energie eines Zustands mit
dieser Frequenz. 

\subsection{Die spektrale Dichte $n(\nu)$} 

Zur Bestimmung der spektralen Dichte $n(\nu)$ der Eigenzust\"ande zu
einer Frequenz $\nu$ definiert man zun\"achst die kumulative Spektralverteilung
\begin{equation}
      N(\nu) = \sum_{\nu_i\leq \nu} 1  \, .
\end{equation}
$N(\nu)$ ist also gleich der Anzahl der Moden, deren Frequenz $\nu_i$ kleiner oder
gleich $\nu$ ist. Die spektrale Dichte ist dann definiert als die Ableitung dieser Funktion:
\begin{equation}
      n(\nu) = \frac{{\rm d} N(\nu)}{{\rm d} \nu}  \, .
\end{equation}
Da es sich bei $N(\nu)$ streng genommen um eine Stufenfunktion handelt, ist diese Ableitung
eigentlich im distributiven Sinne zu verstehen, doch da $N(\nu)$ in f\"uhrender Ordnung glatt ist (also
keine Stufen hat), k\"onnen wir den distributiven Charakter vernachl\"assigen. 
Diese Dichte ist unabh\"angig von der Temperatur und ergibt sich aus 
einem einfachen Abz\"ahlen der m\"oglichen Schwingungsmoden:
\begin{equation}
\label{eq_PlanckDichte}
      \frac{n(\nu)}{V} = \frac{8 \pi \nu^2}{c^3}  \, .
\end{equation} 

Zur Herleitung dieser Formel bestimmen wir zun\"achst $N(\nu)$. Die Frequenz $\nu$ h\"angt
mit den Wellenzahlen \"uber $\nu = c \sqrt{k_1^2+k_2^2+k_3^3}/2\pi$ zusammen. Wir m\"ussen
also abz\"ahlen, wie viele Moden $(k_1,k_2,k_3)$ es gibt, sodass 
\begin{equation}
\label{eq_K_3d}
          \sqrt{k_1^2+k_2^2+k_3^3} \leq 2 \pi \nu/c
\end{equation}
gilt. 

Zur L\"osung dieses Problems betrachten wir einen
kubischen Kasten der Kantenl\"ange $L$
und damit dem Volumen $V=L^3$. Die m\"oglichen Wellenl\"angen in jede der drei Richtungen 
sind durch $\lambda= 2 L/n$ gegeben: In jede Richtung
muss ein Vielfaches einer halben Wellenl\"ange passen.
F\"ur die m\"oglichen Zust\"ande im $\vec{k}$-Raum ergibt
sich damit $\vec{k}=\frac{\pi}{L}(n_1,n_2,n_3)$ mit
beliebigen nat\"urlichen Zahlen $n_i$. Jedem
Zustand im $\vec{k}$-Raum kann man daher
ein Volumen $|\Delta k|^3=\pi^3/V$ zuschreiben. 
Eine Kugel vom Radius $k=|\vec{k}|$
hat ein Volumen von $\frac{4\pi}{3} k^3$
und enth\"alt somit 
\begin{equation}
       V_{\rm Kugel}/|\Delta k|^3 = \frac{4\pi k^3}{3 |\Delta k|^3} =  V\frac{4}{3\pi^2}k^3 
\end{equation}    
Zust\"ande.
Da aber nur der Teil der Kugelschale
von Interesse ist, der sich im positiven Quadranten
befindet (die $n_i$ sind nie negativ), m\"ussen
wir noch durch 8 dividieren; andererseits kann
jeder Zustand wegen der beiden Polarisationsm\"oglichkeiten
zweimal auftreten, sodass die Anzahl der Zust\"ande 
durch $\tilde{g}(k)  = \frac{V}{3\pi^2}k^3$ gegeben ist.
Mit $|\vec{k}|= \frac{2 \pi}{\lambda}=\frac{2\pi \nu}{c}$
erhalten wir schlie\ss lich
\begin{equation}
        N(\nu)  = V
         \frac{8 \pi \nu^3}{3c^3}  \, .
\end{equation}
F\"ur die gesuchte Dichte ergibt sich damit:
\begin{equation}
        n(\nu)  =  \frac{{\rm d} N(\nu)}{{\rm d}\nu} = V
         \frac{8 \pi \nu^2}{c^3}  \, .
\end{equation}

\subsection{Die mittlere Energie einer Frequenz $\nu$ bei einer
Temperatur $T$}

In diesem Abschnitt bestimmen wir die
mittlere Energie $\langle \epsilon (\nu, T)\rangle$, die 
ein gegebener Schwingungsmod mit der Frequenz $\nu$ bei
einer Temperatur $T$ hat. Jede Mode zu $\nu$ kann mit $n$
Photonen mit einer Gesamtenergie von $E_n(\nu)=n h \nu$
besetzt sein, und die Wahrscheinlichkeit einer
solchen Besetzung
ist proportional zum 
Boltzmann-Faktor\index{Boltzmann-Faktor} 
$\exp(-E_n(\nu)/k_{\rm B}T)=\exp(-n h \nu/k_{\rm B}T)$. Als
Mittelwert f\"ur die Energie zu einer solchen
Mode erhalten wir somit:
\begin{equation}
\label{eq_PlanckSummen}
  \langle \epsilon (\nu, T)\rangle = \frac{1}{Z} \sum_{n=0}^\infty
    (n h \nu) \exp \left( - \frac{n h \nu}{k_{\rm B}T}
     \right)  ~~~~ {\rm mit} ~~
     Z= \sum_{n=0}^\infty \exp\left( - \frac{n h \nu}{k_{\rm B}T}
     \right) 
\end{equation}
Hier bietet sich ein kleiner Trick an: Mit
\begin{equation}
     Z= \sum_{n=0}^\infty {\rm e}^{ - \beta n h \nu } \hspace{1.5cm} \beta = \frac{1}{k_{\rm B}T}
\end{equation}
folgt
\begin{equation}
      \langle \epsilon \rangle = \frac{1}{Z} \sum_{n=0}^\infty (nh\nu)  {\rm e}^{ - \beta n h \nu } =
                - \frac{\partial}{\partial \beta} \ln Z \, .
\end{equation}
$Z$ ist aber eine geometrische Reihe, f\"ur die gilt:
\begin{equation}
              Z= \sum_{n=0}^\infty {\rm e}^{ - \beta n h \nu } =  \frac{1}{1-{\rm e}^{- \beta h\nu}}
                = \frac{{\rm e}^{ \beta h\nu}}{{\rm e}^{ \beta h\nu}-1}  \, .
\end{equation}
Damit folgt:
\begin{equation}
             \langle \epsilon \rangle  = - \frac{\partial}{\partial \beta} \ln Z =
             - \frac{\partial}{\partial \beta} \Big( \beta h \nu - \ln \big( {\rm e}^{ \beta h\nu}-1 \big) \Big) 
             = -  h\nu + \frac{h\nu ~{\rm e}^{\beta h \nu} }{\big( {\rm e}^{ \beta h\nu}-1 \big) } 
\end{equation}
oder
\begin{equation}
             \langle \epsilon \rangle  = \frac{h\nu}{\big( {\rm e}^{ \beta h\nu}-1 \big) } 
\end{equation}
Die mittlere Energie pro Frequenz $\nu$ in einem Gas von Photonen ist somit
\begin{equation}
   \langle \epsilon(\nu,T)\rangle = \frac{ h \nu}{\exp \left(
    \frac{ h \nu}{k_{\rm B} T} \right)  - 1} 
\end{equation}
Als Ergebnis dieser \"Uberlegungen erhalten wir
die Planck'sche Formel:\index{Planck'sche Strahlungsformel}
\begin{equation}
\label{eq_PlanckFormel}
      u(\nu,T) = \frac{8\pi}{c^3} \frac{h \nu^3}{\exp \left(
    \frac{h \nu}{k_{\rm B} T} \right)  - 1}
\end{equation}

\section{Vom Hohlk\"orper zur Abstrahlung}
\label{sec_K_Abstr}

Es gibt sehr viele Formen von Strahlungsgesetzen, die sich oftmals nur
in Details, Faktoren $\pi$ oder $2\pi$, oder $c$ etc.\ unterscheiden. Wir beginnen
mit einer Einteilung in verschiedene Gruppen.

Aus der Mechanik ist die Energie bekannt. Ihr Einheit ist das Joule (J). 
Betrachtet man Prozesse, bei denen
Energie umgewandelt wird, interessiert meist die Leistung\index{Leistung}
\begin{equation}
            P =  \frac{\Delta E}{\Delta t} \, .
\end{equation}
Hierbei ist $\Delta E$ die Menge an Energie, die in einer bestimmten Zeitdauer $\Delta t$
umgewandelt, abgestrahlt oder 
transportiert wurde. Dabei kann es sich beispielsweise um die Umwandlung 
von elektrischer Energie in mechanische Energie bei einer Maschine oder um die Umwandlung
von elektrischer Energie in W\"arme bzw.\ Strahlung bei einer Gl\"uhbirne handeln. 
Beispiele f\"ur einen Energietransport sind die \"Ubertragung von Energie von einem warmen
zu einem kalten Gegenstand in Form von W\"arme oder die Strahlung von der Sonne.
Die Einheit der Leistung ist das Watt (W=J/s) bzw.\ Joule
durch Sekunde. 

Von beiden Gr\"o\ss en kann man auch (r\"aumliche) Dichten betrachten: die Energiedichte
(Energie pro Volumen) oder auch eine Leistungsdichte.
Oft hat man es dabei auch mit einem Energiefluss zu tun, d.h.\ der Menge an Energie, die
pro Zeiteinheit durch eine vorgegebene Fl\"ache tritt. Ihre Einheit ist Energie pro Zeiteinheit
pro Fl\"acheneinheit oder auch Leistung pro Fl\"acheneinheit. Eine solche Gr\"o\ss e bezeichnet
man auch als Intensit\"at.\index{Intensit\"aat} 
Eine Intensit\"at erh\"alt man ebenfalls, wenn man eine Energiedichte
mit einer Geschwindigkeit multipliziert. Es handelt sich in diesem Fall also um einen
Energiedichtestrom. 

Bei elektromagnetischer Strahlung kommt zu den genannten Begriffen noch die Abh\"angigkeit
der Energie von der Frequenz bzw.\ der Wellenl\"ange hinzu. Man verwendet in diesem Fall
das Adjektiv \glqq spektral\grqq. Eine spektrale Energiedichte\index{spektrale Verteilung} 
ist eine Energiedichte, die einer
bestimmten Frequenz (bzw.\ einem kleinen Frequenzintervall $[\nu, \nu + {\rm d}\nu]$) oder auch
einer bestimmten Wellenl\"ange (bzw.\ einem kleinen Wellenl\"angenintervall 
$[\lambda,\lambda + {\rm d}\lambda]$) zugeordnet werden kann. 

Und schlie\ss lich kann man bei der Strahlung noch eine Richtungsabh\"angigkeit ber\"ucksichtigen.
Die meisten Fl\"achen strahlen selbst bei einer diffusen Strahlung in bestimmte Richtungen
intensiver ab als in andere, insbesondere gibt es auch bei diffuser Strahlung (sogenannter
Lambert-Strahlung)\index{Lambert-Strahlung} 
eine Abh\"angigkeit vom Winkel $\theta$ relativ zur Fl\"achennormalen, die
durch $\cos \theta$ gegeben ist und die Projektion der abstrahlenden Fl\"ache auf die 
Richtung der Strahlung ber\"ucksichtigt. 

Bei dem Strahlungsgesetz Gl.\ \ref{eq_PlanckFormel_b} handelt es sich um eine
solche spektrale Strahlungsdichte, bei der ein Fl\"achenelement ${\rm d}A$ in einer
Richtung $\theta$ relativ zu Fl\"achennormalen eine Intensit\"at proportional zu 
${\rm d}A \cos \theta$ abstrahlt. Dies wird beim Stefan-Boltzmann-Gesetz relevant.

\section{Das Wien'sche Verschiebungsgesetz}

Ausgehend von der Planck'schen Strahlungsformel leiten wir nun das
Wien'sche Verschiebungsgesetz ab.\index{Strahlungsgesetze!Wien}\index{Wien'sches Verschiebungsgesetz} 
Dazu bestimmen wir das Maximum der
Verteilung $u(\nu,T)$ bez\"uglich der Frequenz:
\begin{equation}
     \frac{{\rm d} u(\nu,T)}{{\rm d}\nu} = \frac{8\pi}{c^3} \frac{3 h \nu^2}{\exp \left(
    \frac{h \nu}{k_{\rm B} T} \right)  - 1} - 
    \frac{8\pi}{c^3} \frac{h \nu^3}{\left( \exp \left(
    \frac{h \nu}{k_{\rm B} T} \right)  - 1 \right)^2}\exp \left(
    \frac{h \nu}{k_{\rm B} T} \right) \frac{h}{k_{\rm B}T} = 0 \, .
\end{equation}
Daraus erhalten wir:
\begin{equation}
       3 \left( \exp \left(
    \frac{h \nu}{k_{\rm B} T} \right)  - 1 \right) =  \frac{h\nu}{k_{\rm B}T} \exp \left(
    \frac{h \nu}{k_{\rm B} T} \right)  \, .
\end{equation}
Zu l\"osen ist somit die Gleichung:
\begin{equation}
           (3-x)  {\rm e}^x =  3 
\end{equation}
mit
\begin{equation}
            x =  \frac{h\nu}{k_{\rm B}T} \, .
\end{equation}
Eine L\"osung ist offensichtlich $x=0$, eine zweite L\"osung muss man nummerisch
finden. Sie liegt bei $x_3=2,82143937$.  F\"ur die
Frequenz folgt damit:
\begin{equation}
         \nu_{\rm max} = \frac{x_3 k_{\rm B}}{h} T 
         = \frac{x_3 \cdot  1,38065 \cdot 10^{-23}}{6,62607 \cdot 10^{-34}} \cdot T 
              \, \frac{\rm J}{\rm K \cdot J s } \approx 5,87893 \cdot 10^{10} \cdot T  \, \frac{1}{\rm K s }
\end{equation} 
Dies ist das Wien'sche Verschiebungsgesetz f\"ur die Frequenz und entspricht
Gl.\ \ref{eq_K_Wien} (links).

Wir wiederholen nun die gleiche Rechnung f\"ur $\tilde{u}(\lambda,T)$. 
Wegen des zus\"atzlichen Faktors vom Integrationsma\ss\ erh\"alt man die
L\"osung nicht einfach aus der L\"osung von $\nu$. Wir bilden die Ableitung
von $\tilde{u}(\lambda,T)$ nach $\lambda$ und erhalten:
\begin{equation}
     \frac{{\rm d} \tilde{u}(\lambda,T)}{{\rm d}\lambda} = - 5 \frac{8\pi}{\lambda^6} \frac{h c}{\exp \left(
    \frac{h c}{\lambda k_{\rm B} T} \right)  - 1} +  
    \frac{8\pi}{\lambda^5} \frac{h c }{\left( \exp \left(
    \frac{hc }{\lambda k_{\rm B} T} \right)  - 1 \right)^2}\exp \left(
    \frac{h c}{\lambda k_{\rm B} T} \right) \frac{hc}{\lambda^2 k_{\rm B}T} = 0 \, .
\end{equation}
Dies f\"uhrt auf:
\begin{equation}
       5 \left( \exp \left(
    \frac{h}{\lambda k_{\rm B} T} \right)  - 1 \right) =  \frac{hc}{\lambda k_{\rm B}T} \exp \left(
    \frac{h }{\lambda k_{\rm B} T} \right)  \, ,
\end{equation}
und somit auf die Gleichung
\begin{equation}
           (5-x)  {\rm e}^x =  5 
\end{equation}
mit
\begin{equation}
            x =  \frac{hc}{\lambda k_{\rm B}T} \, .
\end{equation}
Neben der L\"osung $x=0$ gibt es nun die L\"osung $x_5=4,96511423$. F\"ur die
Wellenl\"ange zur maximalen Intensit\"at folgt damit Gl.\ \ref{eq_K_Wien} (rechts):
\begin{equation}
         \lambda_{\rm max} = \frac{hc}{x_5 k_{\rm B}} \frac{1}{T} = 
           \frac{ 6,62607 \cdot 10^{-34} \cdot 2,99792458 \cdot 10^8}{x_5 \cdot 1,38065 \cdot 10^{-23}} \cdot \frac{1}{T}
              \, \frac{\rm J s m}{\rm Js/K } \approx 2,89777 \cdot 10^{-3} \cdot \frac{1}{T}  ~ {\rm Km} 
\end{equation} 

\section{Das Stefan-Boltzmann-Gesetz}

Das\index{Stefan-Boltzmann-Gesetz}\index{Strahlungsgesetze!Stefan-Boltzmann} 
Stefan-Boltzmann-Gesetz gibt uns eine Beziehung zwischen der Temperatur eines
Strahlers und der Gesamtenergie bzw.\ Gesamtintensit\"at (also Energie pro
Zeiteinheit und Fl\"ache) der abgestrahlten Leistung. Dazu betrachten wir ein Fl\"achenelement
${\rm d}A$ auf der Oberfl\"ache des Strahlers und integrieren \"uber die Hemisph\"are, in 
welche dieses Fl\"achenelement abstrahlt. Au\ss erdem 
integrieren wir Gl.\ \ref{eq_PlanckFormel_b}
\"uber die Frequenz. Damit ergibt sich f\"ur die Intensit\"at der Strahlung, die von ${\rm d}A$
abgestrahlt wird:
\begin{equation}
     I = \int_0^\infty  L(\nu,T)\, {\rm d} \nu \, \cos \theta\, {\rm d}A\, {\rm d}\Omega  
     = \frac{2}{c^2} {\rm d}A \int_0^\infty \frac{h \nu^3}{\exp \left(
    \frac{h \nu}{k_{\rm B} T} \right)  - 1} {\rm d}\nu \, \int_0^{\frac{\pi}{2}} \cos \theta \sin \theta \, {\rm d} \theta \int_0^{2\pi}{\rm d}\varphi  \, .
\end{equation}
Die Integration \"uber $\theta$ erfolgt nur \"uber eine Halbkugel, also nur von $\theta=0$ (Normalenrichtung zur
Fl\"ache ${\rm d}A$) bis $\pi/2$. F\"ur die Winkelintegration ergibt sich:
\begin{equation}
   \int_0^{\frac{\pi}{2}} \cos \theta \sin \theta \, {\rm d} \theta \int_0^{2\pi}{\rm d}\varphi = 
   \int_0^1 \cos \theta\, {\rm d}(\cos \theta) \, \int_0^{2\pi} {\rm d}\varphi = \frac{1}{2} \cdot 2 \pi = \pi \, .
\end{equation}

Wir f\"uhren eine neue Integrationsvariable $x=\frac{h \nu}{k_{\rm B}T}$ ein und erhalten
mit  ${\rm d}x = \frac{h}{k_{\rm B}T} {\rm d}\nu$ die Beziehung:
\begin{equation}
     I =  \frac{2 \pi (k_{\rm B}T)^3}{h^2 c^2}{\rm d}A \int_0^\infty \frac{h^3 \nu^3}{(k_{\rm B}T)^3 \exp \left(
    \frac{h \nu}{k_{\rm B} T} \right)  - 1} \frac{k_{\rm B}T}{h}\, {\rm d} x =
      \frac{2\pi k_{\rm B}^4}{h^3c^2}  T^4\, {\rm d}A   \int_0^\infty  \frac{x^3}{{\rm e}^x - 1}\,{\rm d}x  \, .
\end{equation}
Das Integral ist bekannt und hat den Wert:
\begin{equation}
                     \int_0^\infty  \frac{x^3}{{\rm e}^x - 1}\,{\rm d}x  =  \frac{\pi^4}{15}  \, .
\end{equation}
Damit folgt f\"ur die Intensit\"at, die von einer Fl\"ache $A$ eines schwarzen Strahlers
abgestrahlt wird:
\begin{equation}
     I =  \sigma \, A \, T^4 \hspace{1cm} {\rm mit} \hspace{1cm}
      \sigma =    \frac{2k_{\rm B}^4 \pi^4}{15 h^3c^2}   \, .
\end{equation}
Die Konstante $\sigma$, die sogenannte Stefan-Boltzmann-Konstante,\index{Stefan-Boltzmann-Konstante}
hat in den neuen SI-Einheiten, in denen die Planck'sche Konstante, die
Boltzmann-Konstante und die Lichtgeschwindigkeit fest definierte Werte haben, den Wert
$\sigma = 5,670\,374\,419 ... \cdot 10^{-8}\, {\rm W/(m^2 K^4)}$. 

F\"ur die Abstrahlung der Sonne ist die abstrahlende Fl\"ache $A= 4\pi R_\odot^2$ (mit dem
Sonnenradius $R_\odot=695\,700\,{\rm km} \approx 7\cdot 10^8\,{\rm m}$ \cite{NASA_sun}). Andererseits
kennen wir die Intensit\"at der Sonnenstrahlung bei der Erde -- dies ist die\index{Solarkonstante}
Solarkonstante $S=1361\,{\rm J\cdot s^{-1}\cdot m^{-2}}$. Aus dem Abstand Sonne-Erde
$R=1,496\cdot 10^{11}\, {\rm m}$ k\"onnen wir somit die effektive Oberfl\"achentemperatur
der Sonne bestimmen:\index{effektive Oberfl\"achentemperatur!Sonne}\index{Sonne!effektive Oberfl\"achentemperatur} 
\begin{equation}
           T^4 = \frac{4\pi R^2}{4 \pi R_\odot^2} \frac{S}{\sigma} 
           = \frac{R^2}{R_\odot^2} \frac{S}{\sigma} 
           = \frac{1,496^2 \cdot 10^{22}}{6,957^2 \cdot 10^{16}} \frac{1361}{5,67 \cdot 10^{-8}}
           \frac{\rm J\cdot s\cdot m^2 \cdot K^4}{\rm m^2\cdot s \cdot J}     
           \approx 1110 \cdot 10^{12}\, {\rm K}^4     
\end{equation}
oder
\begin{equation}
            T =  5772\,{\rm K}  \, .
\end{equation}

Auf diese Weise hat Josef Stefan (1835--1893)\index{Stefan, Josef} 
um 1890 zum ersten Mal einen vergleichsweise
genauen Wert f\"ur die Oberfl\"achentemperatur der Sonne erhalten. Die Hauptunsicherheit war
der Wert der Solarkonstanten, denn das Verh\"altnis $R/R_\odot$ verlangt nur den scheinbaren
Sonnendurchmesser (als \"Offnungswinkel). Die Stefan-Boltzmann-Konstante war zwar ebenfalls
nicht genau bekannt, aber Stefan verglich die Strahlungsintensit\"at eines bekannten Objekts
bei einer bekannten Temperatur unter vergleichbarem Raumwinkel mit der Intensit\"at der Sonne. 
Er berechnete so einen Wert von ungef\"ahr 5700\,K, was dem tats\"achlichen Wert recht nahe kommt.

\section{Weshalb $T^4$?}

Das Stefan-Boltzmann-Gesetz spielt eine wesentliche Rolle in der Klimaphysik. Andererseits
ist die Abh\"angigkeit der abgestrahlten Intensit\"at eines Schwarzk\"orpers von der 4.\ Potenz von
$T$ sehr ungew\"ohnlich. Daher soll hier versucht werden, diese Abh\"angigkeit zu \glqq veranschaulichen\grqq.

Es gibt eine Herleitung des Stefan-Boltzmann-Gesetzes im Rahmen der Thermodynamik. Diese
setzt aber bestimmte Beziehungen f\"ur den Strahlungsdruck voraus, die in der Schule meist
nicht bekannt sind. Vor diesem Hintergrund soll die hier gegebene Ableitung nochmals auf ihre
wesentlichen Anteile \glqq runtergebrochen\grqq\ werden. 

Die Anzahl der Moden $N(\nu)$ mit einer Frequenz, die kleiner ist als ein vorgegebenes 
$\nu$, verh\"alt sich wie das Volumen einer dreidimensionalen Kugel: $N(\nu) \propto \nu^3$. 
Dies l\"asst sich grob aus Gl.\ \ref{eq_K_3d} verstehen: Zu jeder Dimension ist die Anzahl der
Moden mit einer Frequenz kleiner als $\nu$ proportional zu $\nu$ und damit ist in drei Dimensionen
die Anzahl der Moden mit einer Frequenz kleiner als $\nu$ proportional zu $\nu^3$.

Jede Mode zu einer Frequenz $\nu$ hat eine Energie $h\nu$. Das bedeutet, die Gesamtenergie aller
Moden mit einer Frequenz kleiner als $\nu$ ist proportional zu $\nu^4$. Nun gilt
\glqq sehr grob\grqq\ folgende Regel: Alle Frequenzen $\nu$, deren
zugeh\"orige Energie $h\nu$ kleiner ist als die thermische Energie $k_{\rm B}T$, tragen zur 
abgestrahlten Intensit\"at bei, wohingegen alle Frequenzen $\nu$ mit einer Energie 
$h\nu > k_{\rm B}T$ nicht mehr beitragen. Das erkl\"art die Abh\"angigkeit der Intensit\"at 
von $T^4$.%
\index{Strahlungsgesetze|)}


\begin{thebibliography}{99}
\bibitem{NASA_sun} NASA Sun Fact Sheet; Williams, D.R.;
      \url{https://nssdc.gsfc.nasa.gov/planetary/factsheet/sunfact.html}      
\end{thebibliography}

\end{document}

