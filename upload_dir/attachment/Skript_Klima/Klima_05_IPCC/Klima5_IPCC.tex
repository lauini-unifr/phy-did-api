\documentclass[german,10pt]{book}      
\usepackage{makeidx}
\usepackage{babel}            % Sprachunterstuetzung
\usepackage{amsmath}          % AMS "Grundpaket"
\usepackage{amssymb,amsfonts,amsthm,amscd} 
\usepackage{mathrsfs}
\usepackage{rotating}
\usepackage{sidecap}
\usepackage{graphicx}
\usepackage{color}
\usepackage{fancybox}
\usepackage{tikz}
\usetikzlibrary{arrows,snakes,backgrounds}
\usepackage{hyperref}
\hypersetup{colorlinks=true,
                    linkcolor=blue,
                    filecolor=magenta,
                    urlcolor=cyan,
                    pdftitle={Overleaf Example},
                    pdfpagemode=FullScreen,}
%\newcommand{\hyperref}[1]{\ref{#1}}
%
\definecolor{Gray}{gray}{0.80}
\DeclareMathSymbol{,}{\mathord}{letters}{"3B}
%
\newcounter{num}
\renewcommand{\thenum}{\arabic{num}}
\newenvironment{anmerkungen}
   {\begin{list}{(\thenum)}{%
   \usecounter{num}%
   \leftmargin0pt
   \itemindent5pt
   \topsep0pt
   \labelwidth0pt}%
   }{\end{list}}
%
\renewcommand{\arraystretch}{1.15}                % in Formeln und Tabellen   
\renewcommand{\baselinestretch}{1.15}                 % 1.15 facher
                                                      % Zeilenabst.
\newcommand{\Anmerkung}[1]{{\begin{footnotesize}#1 \end{footnotesize}}\\[0.2cm]}
\newcommand{\comment}[1]{}
\setlength{\parindent}{0em}           % Nicht einruecken am Anfang der Zeile 

\setlength{\textwidth}{15.4cm}
\setlength{\textheight}{23.0cm}
\setlength{\oddsidemargin}{1.0mm} 
\setlength{\evensidemargin}{-6.5mm}
\setlength{\topmargin}{-10mm} 
\setlength{\headheight}{0mm}
\newcommand{\identity}{{\bf 1}}
%
\newcommand{\vs}{\vspace{0.3cm}}
\newcommand{\noi}{\noindent}
\newcommand{\leer}{}

\newcommand{\engl}[1]{[\textit{#1}]}
\parindent 1.2cm
\sloppy

         \begin{document}  \setcounter{chapter}{0}


\chapter{Physik des Klimas V\\IPCC - der Weltklimarat}
% Kap x
\label{chap_Klima5}

Alle f\"unf bis sechs Jahre erscheint der sogenannte IPCC-Bericht zum Stand des Weltklimas.
IPCC steht dabei f\"ur \glqq International Panel on Climate Change\grqq\ oder 
\glqq Internationaler Fachbeirat zum Klimawandel\grqq, kurz \glqq Weltklimarat\grqq. 


\section{Aufgaben des IPCC}

Die Aufgabe des IPCC ist, alle f\"unf Jahre einen Bericht \"uber aktuelle Ergebnisse aus der
klimarelevanten Forschung vorzulegen. Dieser Bericht soll objektiv und politisch neutral sein, 
kann aber f\"ur politische Entscheidungen relevante wissenschaftliche Ergebnisse enthalten.
Er richtet sich speziell an Politiker bzw.\ politische Entscheidungstr\"ager.

\section{Zur Geschichte des IPCC}

In den 1970er Jahren wurde immer deutlicher, dass die zunehmende ${\rm CO}_2$-Konzentration
in der Luft, die sich allein in der Zeit von 1958 bis 1975 von 315\,ppm (parts per Million) auf 330\,ppm
erh\"oht hatte (heute, 2023, liegt dieser Wert bei 425\,ppm), sowie die Zunahme anderer Treibhausgase
einen einschneidenden Einfluss auf unser Klima haben k\"onnte. Nach mehreren Hearings
(Anh\"orungen) vor amerikanischen Senats- und Kongressaussch\"ussen in den 80er Jahren - unter
anderem mit Al Gore (Nobelpreis 2007), Carl Sagan und Syukuro Manabe (Nobelpreis 2021) -  beschlossen
die Vereinten Nationen die Gr\"undung des IPCC im Jahre 1988. 



\section{Aufbau des IPCC-Berichts und Sonderberichte}

\section{Konzepte und wichtige Terminologie des IPCC}

\section{Anmerkungen}

\begin{anmerkungen}
\item
\label{Anm-1}%
Der Grund f\"ur den Faktor 2 zwischen der Schwankung im Abstand und
der Schwankung in der Intensit\"at der Sonneneinstrahlung liegt in dem $1/r^2$-Gesetz
der Intensit\"at als Funktion des Abstands:
\begin{equation}
        \frac{1}{(r\pm \Delta r)^2} \approx \frac{1}{r^2} \mp 2 \frac{\Delta r}{r} \, .
\end{equation}

\end{anmerkungen}



\begin{thebibliography}{99}
\bibitem{AR6} IPCC Assessment Report 6 ???,\\
       {\small \verb+https://www.ipcc.ch/assessment-report/ar6/+}                
\end{thebibliography}

\end{document}

