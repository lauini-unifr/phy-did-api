\documentclass[german,10pt]{book}     
\usepackage{makeidx}
\usepackage{babel}            % Sprachunterstuetzung
\usepackage{amsmath}          % AMS "Grundpaket"
\usepackage{amssymb,amsfonts,amsthm,amscd} 
\usepackage{mathrsfs}
\usepackage{rotating}
\usepackage{sidecap}
\usepackage{graphicx}
\usepackage{color}
\usepackage{fancybox}
\usepackage{tikz}
\usetikzlibrary{arrows,snakes,backgrounds}
\usepackage{hyperref}
\hypersetup{colorlinks=true,
                    linkcolor=blue,
                    filecolor=magenta,
                    urlcolor=cyan,
                    pdftitle={Overleaf Example},
                    pdfpagemode=FullScreen,}
%\newcommand{\hyperref}[1]{\ref{#1}}
%
\definecolor{Gray}{gray}{0.80}
\DeclareMathSymbol{,}{\mathord}{letters}{"3B}
%
\newcounter{num}
\renewcommand{\thenum}{\arabic{num}}
\newenvironment{anmerkungen}
   {\begin{list}{(\thenum)}{%
   \usecounter{num}%
   \leftmargin0pt
   \itemindent5pt
   \topsep0pt
   \labelwidth0pt}%
   }{\end{list}}
%
\renewcommand{\arraystretch}{1.15}                % in Formeln und Tabellen   
\renewcommand{\baselinestretch}{1.15}                 % 1.15 facher
                                                      % Zeilenabst.
\newcommand{\Anmerkung}[1]{{\begin{footnotesize}#1 \end{footnotesize}}\\[0.2cm]}
\newcommand{\comment}[1]{}
\setlength{\parindent}{0em}           % Nicht einruecken am Anfang der Zeile 

\setlength{\textwidth}{15.4cm}
\setlength{\textheight}{23.0cm}
\setlength{\oddsidemargin}{1.0mm} 
\setlength{\evensidemargin}{-6.5mm}
\setlength{\topmargin}{-10mm} 
\setlength{\headheight}{0mm}
\newcommand{\identity}{{\bf 1}}
%
\newcommand{\vs}{\vspace{0.3cm}}
\newcommand{\noi}{\noindent}
\newcommand{\leer}{}

\newcommand{\engl}[1]{[\textit{#1}]}
\parindent 1.2cm
\sloppy

             \begin{document}  \setcounter{chapter}{0}

\chapter{SRT - Grundlagen}   %  Kap. 1
\label{chap_Grundlagen}

Gegen Ende des 19.\ Jahrhunderts mehrten sich die
Anzeichen, dass irgendetwas im Weltbild der Physik
nicht stimmen konnte. Die Newton'sche Mechanik war
sehr erfolgreich bei der Beschreibung der Bewegungen
von materiellen K\"orpern, insbesondere den Bewegungen der Planeten. Andererseits war die
Theorie Maxwell's ebenso erfolgreich bei der 
Beschreibung der Ph\"anomene im Zusammenhang mit
elektrischen und magnetischen Feldern. Doch die beiden Theorien passten nicht zusammen:
Die Newton'sche Theorie ist fundamental Galilei-invariant, d.h.\ mit jeder Bahnkurve $x(t)$, 
die eine L\"osung der Bewegungsgleichungen ist, ist auch eine Galilei-transformierte Bahnkurve, 
$\hat{x}(t)= x(t)+vt+a$, eine L\"osung der Bewegungsgleichungen. Hierbei ist $v$ eine konstante
Geschwindigkeit und $a$ eine konstante Verschiebung. $x(t)$ kann sich auch auf mehrere Komponenten
und mehrere Objekte beziehen. 
  
Die Maxwell'sche Theorie enth\"alt jedoch als Parameter eine Geschwindigkeit (die 
Lichtgeschwindigkeit im Vakuum) und scheint somit ein
spezielles Ruhesystem auszuzeichnen. Dieses Ruhesystem dachte man sich gleichzeitig als
Ruhesystem eines \"Athers, dessen oszillatorische Anregungen den elektromagnetischen Wellen
entsprechen sollten, \"ahnlich wie die Schwingungen von Luft den Schallwellen entsprechen. Das
Experiment von Michelson und Morley (siehe Abschnitt \ref{sec_Michelson}) war dazu gedacht, 
dieses Ruhesystem experimentell zu bestimmen. 

Oftmals wird der Ausgang des Michelson-Morley-Experiments
als Beweis daf\"ur gewertet, dass es den \"Ather bzw.\
das ausgezeichnete Ruhesystem nicht gibt. Das ist
streng genommen nicht richtig: Die dynamischen
Erkl\"arungen von Lorentz und Fitzgerald (wonach
sich Gegenst\"ande bei einer Bewegung relativ zum
\"Ather verk\"urzen und alle Zeitabl\"aufe entsprechend
verlangsamen) kann s\"amtliche Ph\"anomene
ebenso erkl\"aren wie die heute vorherrschende
Interpretation von Einstein und Minkowski, bei der kein Ruhesystem ausgezeichnet ist und bei der
die geometrischen Eigenschaften der Raumzeit f\"ur die beobachteten Effekte verantwortlich
gemacht werden.
Man kann sogar sagen, dass jede
Lorentz-invariante Theorie \textit{per definitionem}
auch die Interpretation von Lorentz und Fitzgerald
zul\"asst. Experimentell l\"asst sich zwischen diesen
beiden Interpretationen nicht unterscheiden 
(N\"aheres siehe Kapitel \ref{chap_Philosophie}).
Die Einstein'sche Interpretation hat lediglich den
Vorteil, auf unbeobachtbare Entit\"aten wie das Ruhesystem eines \"Athers und den \"Ather selbst
verzichten zu k\"onnen.

Heute verbindet man die spezielle Relativit\"atstheorie in erster Linie
mit dem Namen Albert Einsteins, doch man sollte
nicht vergessen, dass insbesondere Hendrik Antoon
Lorentz (1853--1928) und Jules Henri Poincar\'e 
\index{Poincare@Poincar\'e, Jules Henri}\index{Einstein, Albert}
(1854--1912) wesentliche Vorarbeiten geliefert haben. Was die 
entscheidenden Schritte zur speziellen Relativit\"atstheorie betrifft, so werden
heute meist drei Arbeiten zitiert (aus \cite{Pauli}, S.~2 und 
\cite{Simonyi}, S.~408):
\begin{enumerate}
\item
H.A.\ Lorentz; {\it Electromagnetic phenomena in a system moving with
any velocity smaller than that of light} (\cite{Lorentz}). 
Eingereicht hatte er diese Arbeit am 27.5.1904.
\item
J.H.\ Poincar\'e; {\it Sur la dynamique de l'\'electron} (\cite{Poincare}).
Diese Arbeit wurde bei 
der Franz\"osischen Akademie der Wissenschaften am 5.6.1905 eingereicht.
\item 
A.\ Einstein; {\it Zur Elektrodynamik bewegter K\"orper} (\cite{Einstein1}).
Diese Arbeit wurde am 30.6.1905 eingereicht. 
\end{enumerate}

Eine Diskussion, welchem Autor was zuzuschreiben ist, findet man 
bei Pauli \cite{Pauli} (S.~2/3).  


\section{Der \"Ather}
\index{Aether@\"Ather}

Den Begriff des \"Athers gab es in unterschiedlichen Bedeutungen und Bezeichnungen
schon im Altertum. Bei den Griechen bezog sich dieser Begriff 
je nach Autor auf eine \glqq leuchtende
Substanz\grqq, \glqq Sitz der G\"otter\grqq, \glqq Urmaterie und Quintessence 
(f\"unftes
Element neben den vier bekannten Elementen)\grqq\ etc.\ [Brockhaus].  
Eine konkretere Wiederbelebung erfuhr der \"Ather bei Descartes zur
Erkl\"arung der Planetenbahnen (allgemeiner zur Erkl\"arung der
Gravitation) \cite{Descartes} und bei Huygens als Tr\"ager der Lichtwellen.
Newton setzte sich in seiner Optik (\cite{Newton2}, Frage 18ff, besonders
Frage 22: \glqq ... \"Ather (denn so will ich ihn nennen) ...\grqq) mit der
\"Atherhypothese auseinander. 

Eine klare Definition von \"Ather bzw.\ der \"Atherhypothese zu
geben f\"allt schwer, da sich die Bedeutung des Wortes wie auch die
ihm zugesprochenen Eigenschaften oft gewandelt haben. Meist verstand
man aber unter \"Ather eine \glqq schwerelose, durchsichtige, reibungslose, 
chemisch oder physikalisch nicht nachweisbare und alle Materie und den 
gesamten Raum durchdringende Substanz\grqq\ (\cite{Britannica}; Stichwort 
`Ether'). Manchmal schienen die oben genannten Eigenschaften
jedoch auch im Widerspruch zu den Beobachtungen zu stehen.
Um den gro\ss en Wert der Lichtgeschwindigkeit erkl\"aren
zu k\"onnen, musste man beispielsweise eine
sehr hohe Dichte des \"Athers annehmen. 1816 zeigten 
Augustin Jean Fresnel (1788--1827) und Fran\c{c}ois Arago
(1786--1853), dass zwei senkrecht zueinander polarisierte Strahlen
nicht interferieren und 1817 erkl\"arte Thomas Young 
(1773--1829) diese Erscheinung durch
die Annahme transversaler Schwingungen. 

Transversale Schwingungen setzen jedoch voraus, dass es in
dem Tr\"agermedium Scherkr\"afte gibt, die einer transversalen
Auslenkung entgegenwirken. Damit schied aber ein
\"Ather mit den Eigenschaften von Fl\"ussigkeiten oder
Gasen (in denen nur
longitudinale Wellen existieren) aus (vgl.\ Born \cite{Born}, S.~3). 
Die fehlende longitudinale Polarisation konnte sogar nur 
erkl\"art werden, wenn man dem  
\"Ather die Eigenschaften eines unendlich dichten Festk\"orpers 
zuschrieb. Andererseits sollten sich aber die Planeten 
nahezu reibungslos durch dieses Medium bewegen k\"onnen.

Eine Theorie von George Gabriel Stoke (1819--1903) zur Erkl\"arung dieser
\index{Stoke, George Gabriel}
scheinbaren Widerspr\"uche erscheint uns heute eher absurd: 
Er schrieb dem
\"Ather die Eigenschaften bestimmter nicht-newtonscher Fluide zu, wie
sie beispielsweise bei Pech, Siegellack oder nassem Sand beobachtet 
wurden. Von diesen Stoffen war bekannt, dass sie einerseits recht 
schneller Schwingungen f\"ahig sind (also die hohe Lichtgeschwindigkeit 
und die fehlende longitudinale Polarisation erkl\"arbar wurde), andererseits
aber auch gegen\"uber langsamen Beanspruchungen v\"ollig nachgiebig
sind (und dadurch die vergleichsweise langsame, nahezu reibungslose 
Planetenbewegung m\"oglich war). 

Im 19.\ Jahrhundert wurden viele Experimente unternommen, den
\"Ather nachzuweisen. Als Beweis f\"ur die Existenz des
\"Athers wurde oft ein Experiment von 
\index{Fizeau, Armand}
Armand Hypolit Louis Fizeau (1819--1896) gewertet, der die 
Lichtgeschwindigkeit $c'$ in einer bewegten Fl\"ussigkeit gemessen
und festgestellt hatte, dass sich die Geschwindigkeit von Licht
in der ruhenden Fl\"ussigkeit (d.h.\ $c/n$, wobei $n$ der Brechungsindex
der Fl\"ussigkeit ist) und die Geschwindigkeit der Fl"ussigkeit $v$
nicht addieren, sondern $v$ um einen vom Brechungsindex abh\"angigen
Faktor verringert werden muss  (\cite{Simonyi}; S.~400):
\index{Lichgeschwindigkeit in bewegten Fl\"ussigkeiten}
\begin{equation}
    c' ~=~ \frac{c}{n} + \left( 1 - \frac{1}{n^2} \right) v \;. 
\end{equation}
Diese Ergebnis konnte unter der Annahme einer partiellen,
von der optischen Dichte $n$ abh\"angigen Mitf\"uhrung
des \"Athers durch die Fl\"ussigkeit erkl\"art werden (\cite{Simonyi},
S.~400).
Erst das verallgemeinerte Additionstheorem
f\"ur Geschwindigkeiten in der Relativit\"atstheorie konnte
diese Erscheinung auch ohne \"Atherhypothese erkl\"aren. Danach erh\"alt
man (vgl.\ Pauli \cite{Pauli}, S.~114):
\begin{equation}
\label{eq_Fizeau}
      c' ~=~ \frac{\frac{c}{n} + v}{1 + \frac{cv}{nc^2}}  ~=~
       \frac{c}{n} + v \left( 1 - \frac{1}{n^2} \right) 
                      \frac{1}{1 + \frac{v}{nc}}   \;. 
\end{equation}                      
In f\"uhrender Ordnung von $v/c$ stimmt dieses Ergebnis mit dem alten 
Resultat \"uberein. In seiner bekannten \glqq Geschichte der Physik\grqq\
schreibt Max von Laue (\cite{Laue}, S.~63) in diesem Zusammenhang:
\vspace{0.3cm}

\small
\index{Zitat!Laue}%
Der Fizeausche Versuch galt
lange als der schlagende Beweis f\"ur die Existenz eines \"Athers, der
alle K\"orper durchdringen sollte, ohne an ihrer Bewegung teilzunehmen.
Denn nur so konnte man diesen verkleinerten Faktor verstehen. ... So ist
die Geschichte des Fizeau-Versuchs ein lehrreiches Beispiel daf\"ur, wie
weit in die Deutung jedes Versuchs schon theoretische Elemente 
hineinspielen; man kann sie gar nicht ausschalten. Und wenn dann die 
Theorien wechseln, so wird aus einem schlagenden Beweise f\"ur die
eine leicht ein ebenso starkes Argument f\"ur eine ganz 
entgegengesetzte.
\vspace{0.3cm}

\normalsize
Im 19.\ Jahrundert war die \"Atherhypothese auch Grundlage vieler
Modelle von Raum, Zeit und Materie, die weit \"uber die einfache
Erkl\"arung der Wellennatur von Licht hinausgingen. Ein interessantes
Modell stammt beispielsweise von\index{Thomson, William (Kelvin)} 
William Thomson (1824--1907), 
dem sp\"ateren Lord Kelvin of Largs. 1866 hatte er unter dem Eindruck 
der bahnbrechenden Arbeiten von\index{Helmholtz@ von Helmholtz, Hermann}  
Hermann von Helmholtz (1821--1894) zur Theorie der Vortizes in einem 
idealen Fluid (1858, \cite{Helmholtz2}) -- insbesondere ihrer
erstaunlichen Stabilit\"at, der M\"oglichkeit elastischer Sto\ss prozesse
zwischen Vortizes und der Komplexit\"at ihrer Strukturen -- eine Theorie
aufgestellt, wonach der \"Ather in unserem Kosmos nicht nur f\"ur die 
optischen, elektrischen und magnetischen Ph\"anomene verantwortlich ist, 
\index{Knotentheorie der Materie}
sondern dar\"uberhinaus auch die Atome -- die Bausteine der Materie -- 
als Verknotungen von Vortizes in diesem \"Ather beschreibt. Die 
einzelnen Atomarten entsprechen dabei topologisch verschiedenen 
Knotentypyen. S\"amtliche Naturgesetze sollten sich somit aus den 
statischen und dynamischen Eigenschaften des \"Athers als einem idealen 
Fluid ableiten lassen. Dieses Modell w\"urde sogar erkl\"aren, weshalb der
Raum eines \glqq nicht leeren\grqq\ Universums dreidimensional sein muss,
denn nur in drei Dimensionen sind Knoten topologisch stabil.
(Lit.: Encyclopaedia Britannica \cite{Britannica}, 
Macropaedia, Stichwort `Helmholtz', Bd.~20, S.~564-2b.)

\section{Das Experiment von Michelson und Morley}
\label{sec_Michelson}

Wenn der \"Ather tats\"achlich existierte und wenn er, wie das
Experiment von Fizeau andeutete, die K\"orper durchdringt, ohne
unmittelbar an ihrer Bewegung teilzuhaben, 
dann sollte die Geschwindigkeit der
Erde relativ zum \"Ather -- und damit relativ zum absoluten Raum --
bestimmbar sein. Auf diese M\"oglichkeit hatte auch bereits Maxwell 
hingewiesen. Da die von Maxwell, Hertz und Lorentz
entwickelte Theorie des Elektromagnetismus die Lichtgeschwindigkeit
$c$ als Konstante enthielt, galt es als sicher, dass die
Maxwell'schen Gleichungen nur in dem Bezugssystem gelten, in dem Licht
diese Geschwindigkeit hat, d.h.\ dem System, in dem der \"Ather als
Tr\"ager der Lichtwellen ruht.

Das Schl\"usselexperiment zum Nachweis des \"Athers sollte das
\index{Michelson, Albert}\index{Morley, Edward}
\index{Michelson-Morley-Experiment}
Experiment von Albert Abraham Michelson (1852--1931) und Edward Williams
Morley (1838--1923) werden. Der entsprechende Versuch wurde 1881
von Michelson, dann 1887 nochmals von ihm gemeinsam mit Morley 
durchgef\"uhrt. Mit Hilfe eines Interferometers (Abb.\ \ref{fig_Michelson}(a))
wurde die
Laufzeit von Licht entlang zweier aufeinander senkrecht stehender
Richtungen $l_{\rm l}$ und $l_{\rm t}$ verglichen. $l_{\rm l}$ bezeichnet
dabei die Distanz in longitudinaler Richtung, d.h.\ der Richtung der
vermuteten Erdbewegung relativ zum \"Ather, und $l_{\rm t}$ eine dazu
senkrechte Distanz. 

\begin{figure}[htb]
\begin{picture}(240,150)(-10,0)
\put(0,50){\line(1,0){150}}
\put(40,40){\line(1,1){20}}
\put(50,20){\line(0,1){120}}
\put(35,10){\line(1,0){30}}
\put(35,10){\line(0,1){10}}
\put(35,20){\line(1,0){30}}
\put(65,10){\line(0,1){10}}
\put(35,50){\vector(1,0){0}}
\put(110,50){\vector(1,0){0}}
\put(80,50){\vector(-1,0){0}}
\put(50,100){\vector(0,1){0}}
\put(50,70){\vector(0,-1){0}}
\put(50,30){\vector(0,-1){0}}
\put(35,20){\vector(1,0){0}}
\put(35,35){\makebox(0,0){{\footnotesize ST}}}
\put(32,140){\makebox(0,0){{\footnotesize Sp}}}
\put(150,66){\makebox(0,0){{\footnotesize Sp}}}
\put(70,15){\makebox(0,0){{\footnotesize S}}}
\put(120,104){\makebox(0,0){${\scriptstyle v}$}}
\put(100,40){\makebox(0,0){${\scriptstyle l_l}$}}
\put(43,90){\makebox(0,0){${\scriptstyle l_t}$}}
\put(120,10){\makebox(0,0){(a)}}
\thicklines
\put(120,110){\vector(1,0){30}}
\put(150,40){\line(0,1){20}}
\put(150.5,40){\line(0,1){20}}
\put(40,140){\line(1,0){20}}
\put(40,140.5){\line(1,0){20}}
\multiput(37.5,10)(4,0){7}{\line(0,1){10}}
\multiput(38,10)(4,0){7}{\line(0,1){10}}
\end{picture}
%
\begin{picture}(150,110)(0,-20)
\put(30,20){\line(1,0){90}}
\put(55,20){\vector(1,0){0}}
\put(100,20){\vector(1,0){0}}
\put(75,20){\line(0,1){90}}
\put(30,20){\line(1,2){45}}
\put(120,20){\line(-1,2){45}}
\put(54,68){\vector(1,2){0}}
\put(98,64){\vector(1,-2){0}}
\put(45,70){\makebox(0,0){$ct$}}
\put(82,50){\makebox(0,0){$l_{\rm t}$}}
\put(55,12){\makebox(0,0){$vt$}}
\put(100,-10){\makebox(0,0){(b)}}
\end{picture}
\caption{\label{fig_Michelson}%
Das Michelson-Morley-Interferometer.
(a) Ein Lichtstrahl trifft auf einen Strahlteiler
(ST) und die beiden Anteile breiten sich
in orthogonale Richtungen entlang der
Strecken $l_l$ und $l_t$ aus. Sie werden
an Spiegeln (Sp) reflektiert, treffen wieder auf
den Strahlteiler und k\"onnen auf
dem Schirm (S) interferieren. (b) Der
Strahl transversal zur Bewegungsrichtung
relativ zum \"Ather legt die Strecke $2ct$
zur\"uck, w\"ahrend sich die Apparatur
um die Strecke $2vt$ weiterbewegt hat.}
\end{figure}

Relativ zum \"Ather hat Licht immer die Geschwindigkeit $c$.
F\"ur die longitudinale Richtung berechnen wir die Laufzeit am einfachsten
im Laborsystem. Je nachdem, ob sich die experimentelle Anordnung 
in oder entgegen
der Ausbreitungsrichtung des Lichts bewegt, hat das Licht im Labor\-sys\-tem 
die Geschwindigkeit $c+v$ bzw.\ $c-v$. Die Summe der Zeiten zur 
Durchquerung der Strecke $l_{\rm l}$ in beide Richtungen ist somit
\begin{equation}
\label{MM1}
      t_{\rm l} ~=~ \frac{l_{\rm l}}{c+v} + \frac{l_{\rm l}}{c-v} 
       ~=~ \frac{2l_{\rm l}}{c} \frac{1}{1-\frac{v^2}{c^2}}  \;. 
\end{equation}
F\"ur die transversale Richtung berechnen wir die Laufzeit im Ruhesystem
des \"Athers (vgl.\ Abb. \ref{fig_Michelson}(b)). 
Das Labor bewegt sich in diesem System mit der Geschwindigkeit
$v$ und das Licht \glqq schr\"ag\grqq\ dazu mit der Geschwindigkeit $c$, sodass
die Geschwindigkeitskomponente von Licht parallel zum Laborsystem ebenfalls
gleich $v$ ist. Wir berechnen zun\"achst die Zeit $t$,
die das Licht bis zum Umkehrpunkt ben\"otigt, also die H\"alfte der 
Zeit $t_{\rm t}$ zum Durchlaufen der gesamten Strecke.
F\"ur die vom Licht und vom Bezugssys\-tem (Erde) zur\"uckgelegten 
Strecken, bis das Licht am Umkehrpunkt ist, gilt: 
\[      (vt)^2 + l_{\rm t}^2 ~=~ (ct)^2   \]
d.h.\
\[      t ~=~ \frac{l_{\rm t}}{\sqrt{c^2- v^2}}  \]
und damit insgesamt
\begin{equation}
\label{MM2} 
    t_{\rm t} ~=~  
    \frac{ 2 l_{\rm t}}{c} \frac{1}{\sqrt{1-\frac{v^2}{c^2}}}  \;. 
\end{equation}    

Durch Drehung der Apparatur um $90^\circ$ konnten die Rollen von
$l_{\rm t}$ und $l_{\rm l}$ vertauscht werden. Au\ss erdem wurde das 
Experiment zu verschiedenen Jahreszeiten wiederholt, falls zu einem 
Zeitpunkt des Experiments die Erde zuf\"allig relativ zum \"Ather ruhen 
sollte. 

W\"are die \"Atherhypothese richtig gewesen, h\"atte man
im Rahmen einer Newton'schen Beschreibung
eine Differenz zwischen der longitudinalen und der transversalen 
Richtung finden m\"ussen. Das Experiment zeigte aber keine solche 
Differenz. 

Zun\"achst war man derart von der Richtigkeit der \"Atherhypothese
\index{Aetherhypothese@\"Atherhypothese}
\"uberzeugt, dass man nach anderen Erkl\"arungen f\"ur den negativen 
Ausgang des Michelson-Morley-Experiments suchte. Eine naheliegende
Erkl\"arung war, dass die Erde den \"Ather in ihrer Umgebung gleichsam
mitschleppt, sodass an der Erdoberfl\"ache die Geschwindigkeit
des \"Athers relativ zur Erde immer Null ist. Eine solche Erkl\"arung
widersprach aber nicht nur dem Fizeau'schen Experiment (wonach
der Mitf\"uhrungsterm von der optischen Dichte abh\"angen
sollte und somit f\"ur Luft nahezu verschwindet), sondern auch
\index{Bradley, James}\index{Aberration}
der 1728 von James Bradley (1692--1762) entdeckten Aberration des Lichtes. 
Darunter versteht man den Effekt, dass ein Fernrohr relativ zur Richtung
zu einem Stern etwas vor bzw.\ nachgestellt werden muss, je nach der
senkrechten Geschwindigkeit der Erde relativ zu dieser Richtung 
(\cite{Simonyi}, S.~400). Der Effekt beruht darauf, dass das Licht
wegen der Endlichkeit der Lichtgeschwindigkeit auch eine endliche
Zeit ben\"otigt, um das Fernrohr zu durchqueren. Die Aberration lie\ss\ 
sich am einfachsten durch die Annahme erkl\"aren, dass
die Erde den \"Ather nicht mitf\"uhrt. 

Ein interessanter Vorschlag kam 1892 von Hendrik Antoon Lorentz 
\index{Lorentz, Hendrik Antoon}\index{Fitzgerald, George Francis}
(1853--1928) und gleichzeitig von George Francis Fitzgerald (1851--1901).
Nach ihrer Hypothese sollte jeder Ma\ss stab als Folge der 
Wechselwirkung mit dem \"Ather in Richtung der relativen Bewegung zum
\index{Lorentz-Fitzgerald-Kontraktion}
\"Ather eine sogenannte Lorentz-Fitzgerald-Kontraktion erfahren.
Diese Kontraktion bzw.\ Verk\"urzung von L\"angenma\ss st\"aben 
sollte gerade einem Faktor $\sqrt{1-\beta^2}$ (mit $\beta=v/c$) entsprechen. 
Wie ein Vergleich der Gleichungen \ref{MM1} und \ref{MM2} zeigt, werden
die beiden Laufzeiten $t_{\rm l}$ und $t_{\rm t}$ gleich, wenn man $l_{\rm l}$
mit diesem Faktor multipliziert. Lorentz konnte in den folgenden Jahren
seine Theorie soweit ausbauen, dass er nicht nur die bekannten
Ph\"anomene beschreiben sondern sogar die Transformationsgesetze
formulieren konnte, die sich sp\"ater aus der speziellen 
Relativit\"atstheorie ergeben sollten. F\"ur eine widerspruchsfreie
Theorie musste neben der Kontraktion von L\"angen auch noch
angenommen werden, dass die Zeitskalen s\"amtlicher physikalischer
Ph\"anomene bei einer Bewegung relativ zum \"Ather um einen
entsprechenden Faktor gr\"o\ss er werden.
Seine \"Uberlegungen basierten
jedoch immer noch auf der Annahme eines ausgezeichneten Bezugssystems,
in welchem der \"Ather ruhte. Diese Annahme hat er auch nachdem die
Relativit\"atstheorie ihre Triumpfe feierte nur langsam und z\"ogerlich
aufgegeben.

\section{Axiomatische Formulierung der\\speziellen Relativit\"atstheorie}

Wie schon aus den Titeln der drei in der Einleitung zu diesem
Kapitel zitierten Arbeiten deutlich wird, nahm die Relativit\"atstheorie
ihren Ausgang von der Elektrodynamik. Auch Lorentz hat sich
die Frage gestellt, wie sich physikalische Systeme (z.B.\ solche,
die wir als Uhren und Ma\ss st\"abe verwenden) verhalten, wenn
ihre elementaren Bestandteile durch elektromagnetische
Kr\"afte zusammengehalten werden. Auf diese Weise konnte er
den Faktor f\"ur die L\"angenkontraktion aus den Maxwell'schen
Gleichungen ableiten. In der Folgezeit wurde jedoch versucht,
die Annahmen, die zur Herleitung der speziellen 
Relativit\"atstheorie f\"uhren, auf ein Minimum zu reduzieren. 
Man kann zeigen, dass die folgenden drei Axiome bereits die 
Struktur der speziellen Relativit\"atstheorie -- und damit auch die
Lorentz-Invarianz der fundamentalen Gleichungen -- implizieren:

\begin{enumerate}
\item
Der Raum ist homogen (\"uberall gleich) und isotrop (es ist keine Richtung ausgezeichnet).
\item
Es gilt das Relativit\"atsprinzip.
\item
Die Konstanz der Lichtgeschwindigkeit ist unabh\"angig vom
Bewegungszustand der Lichtquelle.
\end{enumerate}

Axiom 1 wird zun\"achst als Erfahrungstatsache angesehen.
Axiom 2 war f\"ur Einstein eine 
Konsequenz der fehlgeschlagenen Versuche,
den \"Ather bzw.\ Bewegungen relativ zu dem ausgezeichneten Ruhe\-sys\-tem
des Universums nachzuweisen. Wenn sich experimentell kein ausgezeichnetes 
Ruhe\-sys\-tem nachweisen l\"asst, dann sollte die Annahme eines absoluten
Raumes oder einer absoluten Zeit auch aus der Theorie verschwinden.

Diese ersten beiden Axiome gelten auch f\"ur die
Newton'sche Theorie. Es gibt also kein \glqq Relativit\"atsprinzip der
\index{Relativit\"atsprinzip}
Relativit\"atstheorie\grqq\ oder relativistisches 
Relativit\"ats\-prinzip. 
Inertialsysteme sind solche Bezugssysteme,
in denen die kr\"aftefreie Bewegung geradlinig und gleichf\"ormig
verl\"auft. Das Relativit\"atsprinzip besagt, dass die Physik in allen
Inertialsystemen gleich ist.

Axiom 3 ist das Minimum, auf das sich die Aussagen der Maxwell-Gleichungen
reduzieren lassen, so dass zusammen mit den ersten
beiden Axiomen die 
spezielle Relativit\"atstheorie eindeutig wird. Die Konstanz der
\index{Konstanz der Lichtgeschwindigkeit}
Lichtgeschwindigkeit bedeutet, dass jeder inertiale Beobachter
die Wellenfronten einer punktf\"ormigen Lichtquelle als 
konzentrische (gleichzentrierte) Sph\"aren beobachtet. 

Streng genommen gelten die ersten beiden
Axiome nur in einem lokalen Sinne: Die Mikrowellenhintergrundstrahlung
bzw.\ die sichtbare Masse im Universum zeichnen
ein Ruhesystem aus. Solange wir aber nicht
das Universum als Ganzes bzw.\ kosmologische
Probleme betrachten, sind die ersten beiden
Axiome hinreichend gut erf\"ullt.

Nicht alle Schritte zur Herleitung der
Lorentz-Transformationen werden in voller
mathematischer Strenge durchgef\"uhrt (siehe
beispielsweise Pauli \cite{Pauli} 
oder Sexl und 
Urbantke \cite{Sexl}). Die folgende Herleitung
umgeht die meisten mathematischen
Feinheiten.

\begin{enumerate}
\item
Aus dem Relativit\"atsprinzip (Axiom 2) 
folgt insbesondere, dass geradlinige
Bewegungen wieder in geradelinige 
Bewegungen \"ubergehen m\"ussen,
also Geraden in der Raum-Zeit in Geraden 
transformiert werden. Diese Aussage
bedeutet, dass der \"Ubergang von einem
Koordinaten\-sys\-tem zu einem anderen nur
durch eine lineare Transformation gegeben
sein kann:
\[   \left( \begin{array}{c} ct' \\ \pmb{x}^{\, \prime} 
         \end{array} \right) = \Lambda(\pmb{v}) 
         \left( \begin{array}{c} ct \\ \pmb{x} 
         \end{array} \right)  \, ,   \]
wobei $\Lambda(\pmb{v})$ eine $4\times4$
Matrix ist, die von der relativen Geschwindigkeit
$\pmb{v}$ zwischen den beiden Koordinatensystemen
abh\"angen kann. An dieser Stelle
haben wir au\ss erdem angenommen, dass
beide Koordinatensys\-teme denselben
Ursprung haben, d.h., dass die Koordinaten
$(0,0,0,0)$ f\"ur den Urspung in beiden Systemen dasselbe
Ereignis O beschreiben. Allgemeiner sind
es die (bijektiven) affinen Transformationen, die
s\"amtliche Geraden wieder in Geraden
\"uberf\"uhren. Hier beschr\"anken wir uns auf Geraden durch den
Koordinatenursprung und somit auf Transformationen, die
diesen Ursprung invariant lassen.

Dieser Schritt ist \"ubrigens mathematisch
am schwierigsten zu beweisen.
\item
Da die Lichtegschwindigkeit in jedem
Intertialsystem dieselbe sein soll (Axiom 3),
folgt aus $|\pmb{x}|/t=\pm c$ auch $|\pmb{x}^{\,\prime}|/t'=\pm c$, 
bzw.\
\begin{equation}
\label{eq_lc}
   (ct)^2 - (\pmb{x})^2 = 0 ~~ \Longleftrightarrow
   ~~ (ct')^2 - (\pmb{x}^{\,\prime})^2=0   \, .
\end{equation}  
Hierbei stellen wir uns vor, dass bei dem
Ereigniss O  (das f\"ur beide Koordinatensysteme
dasselbe ist) ein Lichtsignal ausgesandt wurde.
In Gl.\ \ref{eq_lc} sollen sich 
$(t,\pmb{x})$ bzw.\ $(t',\pmb{x}^{\,\prime})$ auf
ein Ereignis A beziehen, das von diesem
Lichtsignal \glqq getroffen\grqq\ 
wird (siehe Abb.\ \ref{fig_Lorentz}).\footnote{Man beachte, dass es sich bei
$t,t'$ bzw.\ $\pmb{x}$ und $\pmb{x}'$ um
{\em Differenzen} handelt, die sich auf 
zwei Ereignisse beziehen: das Ereignis
A und das
Referenzereignis O. Trotzdem werde ich
die umst\"andlichere Notation $\Delta t$,
$\Delta \pmb{x}$ etc.\ vermeiden.}

\begin{SCfigure}[30][htb]
\begin{picture}(170,150)(0,0)
\put(10,40){\vector(1,0){160}}
\put(80,0){\vector(0,1){140}}
\put(67,0){\vector(1,3){45}}
\put(20,20){\vector(3,1){145}}
\put(40,0){\line(1,1){120}}
\put(120,0){\line(-1,1){110}}
\put(80,40){\makebox(0,0){$\bullet$}}
\put(140,100){\makebox(0,0){$\bullet$}}
\put(71,43.5){\makebox(0,0){${\scriptstyle O}$}}
\put(140,107){\makebox(0,0){${\scriptstyle A}$}}
\multiput(140,40)(0,3){20}{\makebox(0,0){$\cdot$}}
\multiput(80,100)(3,0){20}{\makebox(0,0){$\cdot$}}
\multiput(125,56)(1,3){15}{\makebox(0,0){$\cdot$}}
\multiput(96,85)(3,1){15}{\makebox(0,0){$\cdot$}}
\put(110,36.0){\makebox(0,0){${\scriptstyle \pmb{x}}$}}
\put(107,54){\makebox(0,0){${\scriptstyle \pmb{x}'}$}}
\put(76,70){\makebox(0,0){${\scriptstyle t}$}}
\put(93,68){\makebox(0,0){${\scriptstyle t'}$}}

\end{picture}
\caption{\label{fig_Lorentz}%
Die beiden Ereignisse $O$ und $A$ werden durch einen Lichtstrahl
verbunden (sie sind lichtartig). F\"ur die beiden Koordinatensysteme sei $O$ 
ein Ereignis im Ursprung. $\pmb{x},t$ und $\pmb{x}',t'$ sind jeweils die Koordinaten
von Ereignis $A$ in den beiden Koordinatensystemen.}
\end{SCfigure}

Betrachten wir nun ein beliebiges
Ereignis (nicht notwendigerweise auf
dem Lichtkegel) mit den Koordinaten
$(t,\pmb{x})$ bzw.\ $(t',\pmb{x}^{\,\prime})$,
so folgt zusammen mit der Linearit\"at der
Transformation, dass sich die beiden
Ausdr\"ucke nur um einen Faktor 
unterscheiden k\"onnen:
\[   (ct)^2 - (\pmb{x})^2 = \alpha 
            \Big( (ct')^2 - (\pmb{x}^{\,\prime})^2 \Big) \, . \]
\item
Der Faktor $\alpha$ kann noch von 
der Geschwindigkeit $\pmb{v}$ abh\"angen, mit
der sich das eine System gegen das andere
bewegt: $\alpha=\alpha(\pmb{v})$. 
Wegen der Isotropie des Raumes
(Axiom 1) sollte $\alpha(-\pmb{v})=\alpha(\pmb{v})$
gelten, und aus der Tatsache, dass
die zu $\pmb{v}$ inverse Transformation durch
$-\pmb{v}$ gegeben ist,
folgt $\alpha(-\pmb{v})\alpha(\pmb{v})=1$, 
insgesamt also
$\alpha(\pmb{v})=\pm 1$. Aus
der Stetigkeit als Funktion von $\pmb{v}$ 
(sowie $\alpha(0)=1$) k\"onnen wir
schlie\ss en: $\alpha(\pmb{v})=1$ oder
\begin{equation}
   (ct)^2 - (\pmb{x})^2 = (ct')^2 - (\pmb{x}^{\,\prime})^2  \, .
\end{equation}
\item
Gesucht sind also alle linearen 
Transformationen $\Lambda$,
welche die Kombination $(ct)^2 - \pmb{x}^{\,2}$
invariant lassen. 
\end{enumerate}

\section{Lorentz-Transformationen}
\label{sec_Lorentz}

Wie allgemein \"ublich f\"uhren wir nun 
die 4-Koordinaten $x^0=ct$ und $x^i$
ein und bezeichnen mit $x$ (ohne Vektorpfeil)
einen 4-Vektor: $x=(x^0,x^1,x^2,x^3)$, wobei
$\pmb{x}=(x^1,x^2,x^3)$.
Die Komponenten eines 4-Vektors
bezeichnen wir mit griechischen
Buchstaben ($\mu$, $\nu$, etc.) und sie
nehmen die Werte $0,1,2,3$ an. F\"ur die Indizes von
r\"aumlichen Komponenten verwenden wir
weiterhin lateinische Buchstaben. 

Dass die Indizes f\"ur die Komponenten
von Vektoren hochgestellt sind, ist eine
Konvention. Wir bezeichnen damit die
Koordinaten von Vektoren. Wie aus der
Linearen Algebra bekannt, gibt es zu
jedem Vektorraum auch einen
Dualraum (der Raum der linearen 
Abbildungen von dem Vektorraum in
den jeweiligen Zahlenk\"orper). Die
Komponenten von Elementen des
Dualraums kennzeichnen wir durch
untenstehende Indizes. Au\ss erdem
verwenden wir im Folgenden noch
die {\em Einstein'sche Summenkonvention}:
\"Uber doppelt auftretende Indizes,
einmal oben und einmal unten, auf einer
Seite einer Gleichung wird summiert.
Diese Konvention macht viele
Formeln wesentlich \"ubersichtlicher.

Wir definieren nun ein symmetrisches,
bilineares Produkt\footnote{Die Vorzeichen sind Konvention und werden
insbesondere in der relativistischen Feldtheorie im Vergleich zur Relativit\"atstheorie
oft unterschiedlich gew\"ahlt. Bei der hier angegebene Konvention ist das Skalarprodukt
von zeitartigen Vektoren mit sich selbst positiv.}
\begin{equation}
      (x,y) :=  \eta_{\mu \nu} x^\mu y^\nu
      = x^0 y^0 - \sum_{i,j=1}^3 x^i y^i  \, .
\end{equation}
Wir bezeichnen dieses Produkt manchmal
als Skalarprodukt, obwohl es nicht
positiv definit und damit in der \"ublichen 
mathematischen Sprechweise
kein Skalarprodukt ist. Oft nennen wir es auch
Minkowski-Produkt. Dieses Produkt ist 
nicht-entartet, d.h., es gibt keine nicht-verschwindenden
Vektoren $y$, sodass $(x,y)$ f\"ur alle Vektoren
$x$ gleich null ist. Die Matrix
\begin{equation}
       \eta = {\rm diag}(1,-1,-1,-1) =
       \left( \begin{array}{cccc}
       1 & 0 & 0 &  0 \\
       0 & -1 & 0 &  0 \\
       0 & 0 & -1 &  0 \\
       0 & 0 & 0 &  -1  \end{array} \right) 
\end{equation}
bezeichnen wir manchmal als
Minkoswki-Metrik. Auch dieser Ausdruck
ist strenggenommen irref\"uhrend, da
von einer Metrik \"ublicherweise verlangt wird,
dass Abst\"ande nie negativ werden
k\"onnen, was hier aber nicht der Fall ist.
Daher spricht man manchmal auch von
einer {\em Pseudo-Metrik}. 

Durch die Bilinearform $\eta$ k\"onnen wir
jedem Vektor mit Komponenten $\{x^\mu\}$ einen dualen Vektor
mit den Komponenten $\{x_\mu\}$ zuordnen:
\begin{equation}
            x_\mu = \eta_{\mu \nu} x^\nu  \, .
\end{equation}
Beim dualen Vektoren kehren sich also
alle Vorzeichen der r\"aumlichen Komponenten
um.

Die Lorentz-Transformationen $\Lambda$
bestehen aus allen linearen Transformationen, welche die
Minkow\-ski-Metrik invariant lassen.
Das bedeutet
\begin{equation}
      (x, y) = (\Lambda x, \Lambda y) 
\end{equation}
f\"ur alle 4-Vektoren $x$ und $y$. 
Ausgedr\"uckt in Komponenten bedeutet
diese Bedingung
\begin{equation}
    \Lambda^\alpha_{~ \mu} \eta_{\alpha \beta} 
    \Lambda^\beta_{~ \nu}  = \eta_{\mu \nu}      
\end{equation}
oder komponentenunabh\"angig
\begin{equation}
    \Lambda^T \eta \Lambda = \eta  \, .      
\end{equation}

Wir l\"osen diese Gleichungen f\"ur
eine Raumdimension, also f\"ur $2\times 2$
Matrizen. Die Matrixgleichung
\begin{equation}
 \left( \begin{array}{cc} A & C \\ B & D
 \end{array} \right)
 \left( \begin{array}{cc} 1 & 0 \\ 0 & -1
 \end{array} \right)
 \left( \begin{array}{cc} A & B \\ C & D
 \end{array} \right)
 =
  \left( \begin{array}{cc} 1 & 0 \\ 0 & -1
 \end{array} \right)
\end{equation} 
f\"uhrt auf drei algebraische Gleichungen,
\begin{equation}
    A^2-C^2=D^2 - B^2 = 1 \hspace{1cm} AB=CD  \, , 
\end{equation}
die (bis auf Vorzeichen) eine einparametrige 
Schar an L\"osungen zulassen. Eine m\"ogliche
Parametrisierung dieser L\"osungen ist:
\begin{equation}
      A = D = \cosh \phi ~~~~  B = C = \sinh \phi \, .
\end{equation}
Man bezeichnet $\phi$ auch manchmal als
Rapidit\"at. 

Wir k\"onnen aber auch eine
anschaulichere Parametrisierung w\"ahlen,
die sich aus folgender \"Uberlegung ergibt:
Die Weltlinie des r\"aumlichen Ursprungs 
des $(t',x')$-Systems,
also die Gerade zu $x'=0$, soll sich f\"ur den anderen 
Beobachter als die Gerade $x=vt$ darstellen. 
Damit erh\"alt die Geschwindigkeit $v$
erst ihre Bedeutung. Das bedeutet aber, dass
\begin{equation}
         x' = \gamma(v) (x - v t) = \gamma(v) 
         \left( x^1 - \frac{v}{c} x^0 \right)  
\end{equation}
sein muss, mit einem noch zu bestimmenden ($v$-abh\"angigen)
Faktor $\gamma(v)$. Durch Vergleich mit den obigen
Transformationen folgt $A=D=\gamma$ und
$B=C=- \gamma \frac{v}{c}$, und aus $A^2 - C^2=1$ ergibt
sich
\begin{equation}
            \gamma(v) = \frac{1}{\sqrt{1 - \frac{v^2}{c^2}}}   \, .
\end{equation} 
Damit erhalten wir f\"ur die Lorentz-Transformationen
in einer Raumdimension:
\begin{equation}
           \Lambda(v) = \frac{1}{\sqrt{1 - \frac{v^2}{c^2}}} \left( \begin{array}{cc}
           1 & - \frac{v}{c} \\ - \frac{v}{c} & 1 \end{array} \right) \, .
\end{equation}
F\"ur die weiteren \"Uberlegungen wird meist diese
Form der Lorentz-Transformation ausreichen.
Man bezeichnet sie auch als \glqq Boost\grqq.
Die Matrixdarstellung eines allgemeinen Boosts 
(f\"ur eine beliebige dreidimensionale 
Geschwindigkeit $\pmb{v}$) erh\"alt man am
einfachsten, indem man die Raumkoordinaten 
in zur Geschwindigkeit $\pmb{v}$ parallele und
senkrechte Komponenten aufspaltet und
ber\"ucksichtigt, dass sich die senkrechen
Komponenten nicht \"andern (vgl.\ z.B.\
die englische Wikipedia-Seite 
\glqq Lorentz transformation\grqq):
\begin{equation}
    \Lambda(\pmb{v}) = \left( \begin{array}{cc} 
  \gamma ~~&~~ - \gamma \pmb{\beta}^{\,\rm T} \\
  - \gamma \pmb{\beta} ~~&~~ \mathbb{I} +
   (\gamma - 1) \pmb{\beta}\pmb{\beta}^{\,\rm T}/\beta^2  
    \end{array} \right)  \, ,
\end{equation}
wobei $\pmb{\beta}=\pmb{v}/c$ ist, $\pmb{\beta}^{\,\rm T}$
der zugeh\"orige transponierte (Zeilen)-Vektor und 
$\pmb{\beta}\pmb{\beta}^{\,\rm T}$ die $3\times 3$ Matrix 
mit den Komponenen $\beta_i \beta_j$. 
$\mathbb{I}$ ist die $3\times 3$ Identit\"atsmatrix
und $\beta^2=v^2/c^2$. 

Diese Matrizen bilden noch keine Gruppe. 
Die gew\"ohnlichen dreidimensionalen Drehungen
$R\in {\rm SO}(3)$ lassen die Minkowski-Metrik
ebenfalls invariant. Erst die Boosts zusammen mit
den Drehungen bilden eine (sechsparametrige) Gruppe, die 
Lorentz-Gruppe SO(1,3).\footnote{Zur Notation: Die
Gruppe ${\rm SO}(n,m)$ ist die Gruppe aller
reellen linearen Transformationen mit Determinante 1,
welche die $(n+m)\times (n+m)$ Matrix 
$\eta={\rm diag}(1,...,1,-1,...,-1)$
invariant lassen, wobei die ersten $n$ 
Eintr\"age $+1$ und die letzten $m$ Eintr\"age
$-1$ sind. F\"ur die \"ublichen speziellen 
orthogonalen Gruppen ist $m=0$ und man
schreibt einfach SO($n$).} Man kann die Gruppe noch
um r\"aumliche Spiegelungen (Parit\"atstransformationen)
und die zeitliche Umkehr $t\rightarrow -t$ erweitern.
Da der Minkowski-Raum ein affiner Raum ohne
ausgezeichneten Raum-Zeit-Ursprung ist, erh\"alt man
insgesamt als Invarianzgruppe der Speziellen
Relativit\"atstheorie die
Poincar\'{e}-Gruppe, bestehend aus den
Transformationen $\tilde{\Lambda}+\pmb{a}$
wobei $\tilde{\Lambda}$ eine Lorentz-Transformation
(eventuell plus Parit\"atstransformation oder
Zeitumkehr) ist und $\pmb{a}$ ein beliebiger
4-dimensionaler Translationsvektor.   

Abschlie\ss end soll noch das
Geschwindigkeitadditionstheorem in seiner
einfachsten Form (parallele Geschwindigkeiten)
abgeleitet werden. Dazu betrachten wir einfach
das Produkt zweier Lorentz-Boosts, f\"ur
die gelten soll:
\begin{equation}
   \gamma (v_1) \left( \begin{array}{cc}
     1 & - v_1/c \\ - v_1/c & 1 
   \end{array} \right) 
   \gamma (v_2) \left( \begin{array}{cc}
     1 & - v_2/c \\ - v_2/c & 1 
   \end{array} \right)  =
   \gamma (v_{\rm ges}) \left( \begin{array}{cc}
     1 & - v_{\rm ges} /c \\ - v_{\rm ges}/c & 1 
   \end{array} \right) 
\end{equation}
Durch direktes Nachrechnen (am einfachsten
bildet man das Produkt auf der linken Seite
und erh\"alt $-v_{\rm ges}/c$ aus dem
Verh\"altnis eines Nebendiagonalelements
mit einem Diagonalelement) findet man:
\begin{equation}
\label{eq_vadd}
     v_{\rm ges} = \frac{ v_1 + v_2}{1+ \frac{v_1 v_2}{c^2}} \, .
\end{equation}
F\"ur nicht-relativistische Geschwindigkeiten
$v_i\ll c$ erh\"alt man das klassische
Ergebnis der Newton'schen Mechanik --
die Gesamtgeschwindigkeit ist die Summe
der Einzelgeschwindigkeiten. $v_{\rm ges}$
kann jedoch nie gr\"o\ss er als $c$ werden
und setzt man z.B.\ $v_1=c$ so erh\"alt man
auch $v_{\rm ges}=c$. Gleichung \ref{eq_vadd}
hatten wir schon bei der Herleitung des
Fresnel'schen Mitf\"uhrungsfaktors in
Gl.\ \ref{eq_Fizeau} verwendet.

\section{Die Minkowski-Raumzeit}
\index{Minkowski, Hermann}

1908 hatte Hermann Minkowski (1864--1909) die 4-dimensionale Raumzeit 
eingef\"uhrt und damit den Formalismus der speziellen Relativit\"atstheorie 
wesentlich vereinfacht. Ber\"uhmt geworden sind die Anfangsworte zu
einem seiner Vortr\"age,  
gehalten auf der \glqq 80.\ Versammlung Deutscher 
Naturforscher und \"Arzte zu C\"oln\grqq\ am 21.\ September 1908 (aus
\cite{Aichelburg}, S.~123):
\vspace{0.3cm}

{\small
Meine Herren! Die Anschauungen \"uber Raum und Zeit, die ich Ihnen
entwickeln m\"ochte, sind auf experimentell-physikalischem Boden
erwachsen. Darin liegt ihre St\"arke. Ihre Tendenz ist eine radikale.
Von Stund an sollen Raum f\"ur sich und Zeit f\"ur sich v\"ollig zu
Schatten herabsinken und nur noch eine Art Union der beiden soll
Selbst\"andigkeit bewahren.}
\vspace{0.1cm}

Doch worin bestand das eigentlich Neue?

\subsection{Die Geometrie des Minkowski-Raums}

Die Besonderheit der 4-dimensionalen Minkowski-Raumzeit
ergibt sich nicht einfach aus der Zusammenfassung des
3-dimensionalen gew\"ohnlichen Raums mit einer 1-dimensionalen
Zeitkoordinate. Dies ist auch in der gew\"ohnlichen
Newton'schen Mechanik m\"oglich. Die Besonderheit
ergibt sich aus den Invarianzen dieses Raumes bzw.\ der
Art von Geometrie, welche durch die Minkowski-Metrik
auf ihm definiert wird. 

Abgesehen von r\"aumlichen und zeitlichen Translationen
ist der Newton'sche Raum
invariant unter Galilei-Transformationen: dreidimensionale
Rotationen sowie die speziellen Galilei-Transformationen
\begin{equation}
     t \longrightarrow t'=t ~~~ {\rm und} ~~~ 
     \pmb{x} \longrightarrow  \pmb{x}^{\,\prime}= \pmb{x} + \pmb{v} t  \, .
\end{equation}
Dass beide Koordinatensysteme \"uber eine affine
Transformation zusammenh\"angen, folgt
wiederum aus dem Relativit\"atsprinzip,
das auch in der Newton'schen Mechanik gilt (Geraden in
einem Inertialsystem sind Geraden in allen Inertialsystemen).
W\"ahrend die ersten beiden Axiome in unver\"anderter
Form auch in der Newton'schen Mechanik gelten,
kann man dort das 3.\
Axiom ersetzen durch: Zwei Ereignisse haben in
allen Inertialsystemen denselben zeitlichen Abstand
(daraus folgt $\Delta t=\Delta t'$) und zwei gleichzeitige
Ereignisse haben in allen Inertialsystemen denselben
r\"aumlichen Abstand (daraus 
folgt $\Delta \pmb{x}^{\,2} = (\Delta \pmb{x}^{\,\prime})^2$). Damit liegen
die Galilei-Transformationen als Invarianzgruppe
fest. 

Demgegen\"uber ist die Geometrie des Minkowski-Raums 
durch die Invarianz von $(ct)^2 - \pmb{x}^{\,2}$ bestimmt.
Genauer bedeutet dies Folgendes: Zwei Ereignisse A
und B werden von zwei Inertial\-sys\-temen durch die
Koordinaten $(t_A,\pmb{x}_A)$ bzw.\ $(t_B,\pmb{x}_B)$
sowie $(t'_A,\pmb{x}^{\,\prime}_A)$ und $(t'_B,\pmb{x}^{\,\prime}_B)$
beschrieben. Dann gilt
\begin{equation}
       c^2(t_A-t_B)^2 - (\pmb{x}_A - \pmb{x}_B)^2 =
       c^2(t'_A-t'_B)^2 - (\pmb{x}^{\,\prime}_A - \pmb{x}^{\,\prime}_B)^2 \, . 
\end{equation}
Durch diese Invariante wird zwei Ereignissen ein
\glqq Abstand\grqq\ zugeschrieben, der nicht vom
Koordinatensystem abh\"angt. Diesen Abstand werden
wir sp\"ater nutzen, um Linien (insbesondere 
Weltlinien) eine \glqq L\"ange\grqq\ zuzuschreiben
und Geometrie zu betreiben.
Diese Minkowski-Geometrie ist zun\"achst etwas
ungewohnt, sodass ich einige Aspekte betonen m\"ochte.

\subsection{Darstellung des Minkowski-Raums}

Im Folgenden werden wir sehr oft Gebrauch von
geometrischen Konstruktionen machen. Wir stellen
dabei die Raumzeit meist vereinfacht durch eine
Ebene dar, die einer
Raum- und einer Zeitdimension entspricht.
Die Punkte dieser Ebene repr\"asentieren 
Ereignisse und damit physikalische Tatsachen,
die nicht von irgendeinem Koordinatensystem oder 
einer anderen Wahl der Beschreibung 
abh\"angen (vgl.\ Abb.\ \ref{fig_events}). 
Ob sich zwei Personen am selben Ort treffen, oder
der Zeiger einer bestimmten Uhr auf die 12 springt,
oder eine Lampe an- und wieder ausgeknippst
wird, oder eine Rakete dicht an einem bestimmten
Satelliten vorbeifliegt -- das sind Tatsachen, die 
f\"ur alle Beobachter gleicherma\ss en existent 
sind. 

\begin{SCfigure}[50][htb]
\begin{picture}(200,205)(-30,0)
\qbezier(30,10)(35,15)(30,20)
\qbezier(30,20)(25,25)(30,30)
\qbezier(30,30)(35,35)(30,40)
\qbezier(30,40)(25,45)(30,50)
\qbezier(30,50)(35,55)(30,60)
\qbezier(30,60)(25,65)(30,70)
\qbezier(30,70)(35,75)(30,80)
\qbezier(30,80)(25,85)(30,90)
\qbezier(30,90)(35,95)(30,100)
\qbezier(30,100)(25,105)(30,110)
\qbezier(30,110)(35,115)(30,120)
\qbezier(30,120)(25,125)(30,130)
\qbezier(30,130)(35,135)(30,140)
\qbezier(30,140)(25,145)(30,150)
\qbezier(30,150)(35,155)(30,160)
\qbezier(30,160)(25,165)(30,170)
\qbezier(30,170)(35,175)(30,180)
\qbezier(30,180)(25,185)(30,190)
%
\qbezier(30,130)(15,150)(15,190)
%
\put(90,10){\line(-1,1){60}}
\put(30,70){\makebox(0,0){{\footnotesize $\bullet$}}}
\put(30,130){\line(1,3){20}}
\put(30,130){\makebox(0,0){{\footnotesize $\bullet$}}}
\put(40,10){\line(1,2){80}}
\put(56.5,43){\makebox(0,0){{\footnotesize $\bullet$}}}
\put(110,150){\makebox(0,0){{\footnotesize $\bullet$}}}
\put(29,109){\makebox(0,0){{\footnotesize $\bullet$}}}
\put(69,68.5){\makebox(0,0){{\footnotesize $\bullet$}}}
\put(108,30){\makebox(0,0){{\footnotesize $\bullet$}}}
%
\qbezier(100,10)(115,30)(100,50)
\qbezier(100,50)(85,70)(100,90)
\qbezier(100,90)(130,120)(110,150)
\qbezier(110,150)(100,170)(110,190)
\multiput(108,30)(12,12){5}{\line(1,1){10}}
\multiput(108,30)(-12,12){9}{\line(-1,1){10}}
\end{picture}
\caption{\label{fig_events}
Die Raumzeit ist die Menge aller Ereignisse.
Klassische Weltlinien sind kontinuierliche Folgen von
Ereignissen (nicht zu verwechseln mit den
\glqq Weltlinien\grqq\ in Feynman-Graphen, hierbei
handelt es sich um Repr\"asentationen von
Propagatoren bzw.\ Green'schen Funktionen, 
nicht um reale, \glqq faktische\grqq\ Weltlinien). }
\end{SCfigure} 

Sehr oft handelt es sich bei Ereignissen um
den Schnittpunkt von Weltlinien, wobei
Weltlinien von Objekten bestimmte kontinuierliche 
Folgen von Ereignissen und somit ebenfalls
unabh\"angig von einem Koordinantensystem
sind. Auch wenn der Zeiger einer Uhr auf
12 zeigt, schneiden sich im Prinzip zwei
Weltlinien: die Weltlinie der Zeigerspitze und
die Weltlinie der Markierung f\"ur die 12.
Die Weltlinien von Objekten, auf die keine
Kr\"afte wirken, werden in Minkowski-Diagrammen
durch Geraden dargestellt. Insbesondere
verl\"auft der r\"aumliche Ursprung eines Inertialsystems
entlang einer geraden Weltlinie. (Dies wird
bei allgemeinen Raumzeit-Diagrammen 
und in
der Allgemeinen Relativit\"atstheorie nicht
immer der Fall sein.)

\subsection{Die kausale Struktur}

Zu jedem Ereignis k\"onnen wir den 
Zukunfts- und den Vergangenheitslichtkegel 
angeben. Der Zukunftslichtkegel besteht 
aus allen Ereignissen, die von einem 
Lichtblitz, der bei dem betreffenden
Ereignis ausgesandt wird, \"uberstrichen
wird. Dabei stellen wir uns vor, dass
sich das Licht von diesem Ereignis aus
kugelf\"ormig in alle Richtungen
ausbreitet. Die Zeitdauer des Lichtblitzes
sei vernachl\"assigbar kurz. Der
Vergangenheitslichtkegel besteht aus
allen Lichtstrahlen, die das betreffende
Ereignis treffen. 

\begin{figure}[htb]
\begin{picture}(180,150)(0,0)
\thicklines
\put(10,10){\line(1,1){100}}
\put(110,10){\line(-1,1){100}}
\put(60,60){\makebox(0,0){{\footnotesize $\bullet$}}}
\put(67,60){\makebox(0,0){{\footnotesize O}}}
\put(90,90){\makebox(0,0){{\footnotesize $\bullet$}}}
\put(96,90){\makebox(0,0){{\footnotesize L}}}
\put(70,80){\makebox(0,0){{\footnotesize $\bullet$}}}
\put(64,80){\makebox(0,0){{\footnotesize A}}}
\put(60,30){\makebox(0,0){{\footnotesize $\bullet$}}}
\put(66,30){\makebox(0,0){{\footnotesize B}}}

\put(30,55){\makebox(0,0){{\footnotesize $\bullet$}}}
\put(24,55){\makebox(0,0){{\footnotesize C}}}
\put(100,70){\makebox(0,0){{\footnotesize $\bullet$}}}
\put(106,70){\makebox(0,0){{\footnotesize D}}}

\put(60,100){\makebox(0,0){{\footnotesize Zukunft}}}
\put(60,20){\makebox(0,0){{\footnotesize Vergangenheit}}}
\put(110,50){\makebox(0,0){{\footnotesize kausales}}}
\put(120,40){\makebox(0,0){{\footnotesize Komplement}}}
\end{picture}
\begin{picture}(150,150)(0,0)
\put(40,40){\line(1,3){15}}
\put(25,55){\line(3,1){45}}
\thicklines
\put(10,10){\line(1,1){90}}
\put(70,10){\line(-1,1){60}}
\put(40,40){\makebox(0,0){{\footnotesize $\bullet$}}}
\put(48,40){\makebox(0,0){$x$}}
\put(10,40){\line(1,1){70}}
\put(90,50){\line(-1,1){60}}
\put(55,85){\makebox(0,0){{\footnotesize $\bullet$}}}
\put(25,55){\makebox(0,0){{\footnotesize $\bullet$}}}
\put(70,70){\makebox(0,0){{\footnotesize $\bullet$}}}
\put(65,85){\makebox(0,0){$y$}}
\put(15,55){\makebox(0,0){$a$}}
\put(80,70){\makebox(0,0){$b$}}
\end{picture}
\caption{\label{fig_lightcone}
(links) Der Zukunfts- und Vergangenheitslichtkegel
zu einem Ereignis O und sein kausales Komplement.
(rechts) Ein \glqq Diamant\grqq\ zu zwei Ereignissen
$x$ und $y$ besteht aus
allen Ereignissen, die sowohl in der Zukunft von $x$ als
auch in der Vergangenheit von $y$ liegen. Er definiert 
zwei raumartige Ereignisse $a$ und $b$, die
\glqq gerade eben noch\grqq\ im Diamanten 
liegen. Bis auf ein Vorzeichen ist der Abstand
$\overline{xy}$ gleich dem Abstand 
$\overline{ab}$ (f\"ur $c=1$);
damit lassen sich die Messungen r\"aumlicher Abst\"ande auf die
Messungen von zeitlichen Abst\"anden reduzieren.
}
\end{figure} 

F\"ur alle Ereignisse auf dem Lichtkegel
gilt (in jedem Inertialsystem)
$(ct)^2 - \pmb{x}^{\,2}=0$ (wobei wir
das Ereignis O als Ursprung $(0,0)$
des Koordinatensystems gew\"ahlt
haben, ansonsten sind $t$ und
$\pmb{x}$ entsprechend durch
$\Delta t$ und $\Delta \pmb{x}$ zu
ersetzen). Zwei Ereignisse, die direkt
durch einen Lichtstrahl verbunden
werden k\"onnen, also auf
dem Lichtkegel des jeweils anderen
Ereignisses liegen, bezeichnet man als
{\em lichtartig}. In Abb.\ \ref{fig_lightcone}
sind O und L lichtartig. 

F\"ur Ereignisse innerhalb des
Zukunfts- oder Vergangenheitslichtkegels 
gilt $(ct)^2 - \pmb{x}^{\,2}>0$. Solche
Ereignispaare bezeichnet man als
{\em zeitartig}. Die Ereignisse A und
B sind zeitartig zu O. Ereignisse
au\ss erhalb des Lichtkegels (z.B.\
die Ereignisse C und D) bezeichnet
man als relativ zu O {\em raumartig}. 
F\"ur solche Ereignisse gilt
$(ct)^2 - \pmb{x}^{\,2} <0$. 

F\"ur zeit- und lichtartige Ereignispaare
kann man eindeutig angeben, welches
der beiden Ereignisse in der Zukunft
relativ zu dem anderen Ereignis liegt.
(Ist $c|t| \geq |\pmb{x}|$ und $t>0$, so
gibt es keine Lorentz-Transformation,
f\"ur die $t'$ negativ wird.)
Diese Relation -- \glqq A liegt in der Zukunft von
B\grqq, geschrieben als A$>$B -- 
ist antisymmetrisch (wenn A$>$B,
gilt nicht B$>$A). Au\ss erdem ist
diese Relation transitiv: Aus A$>$B
und B$>$C folgt A$>$C. Bei dieser
Relation handelt es sich also um 
eine Teilordnung. 

F\"ur raumartige Ereignisse ist
eine allgemein g\"ultige zeitliche Ordnung 
nicht m\"oglich. Man kann immer
Bezugssysteme finden, in denen
$\Delta t>0$ ist, und andere Bezugssysteme, f\"ur die $\Delta t'< 0$ ist,
d.h., w\"ahrend in dem einen System
$C$ scheinbar sp\"ater als $O$ liegt,
findet es in dem anderen System 
fr\"uher statt. Solche Ereignisse
k\"onnen sich gegenseitig nicht
kausal beeinflussen. 

Die beiden Lichtkegel (meist spricht
man einfach von {\em dem} Lichtkegel)
zu einem
Ereignis O unterteilen also die Menge
aller Ereignisse in drei Klassen: (1) die
Menge der zuk\"unftigen Ereignisse,
{\em die von} O theoretisch kausal beeinflusst
werden k\"onnen, (2) die Menge der
Ereignisse in der Vergangenheit, {\em von
denen} O theoretisch kausal beeinflusst
werden kann, sowie (3) die raumartigen
Ereignisse, die in {\em keinem} kausalen
Zusammenhang zu O stehen. Die
f\"ur die Newton'sche Raumzeit noch
sinnvolle Relation \glqq gleichzeitig\grqq\ gibt
es in der Relativit\"atstheorie nicht mehr
in einem absoluten Sinne.

\subsection{Inertialsysteme}

In einem Inertialsystem werden 
alle kr\"aftefreien Bewegungen (im Sinne
der speziellen Relativit\"atstheorie --
wir werden in der allgemeinen Relativit\"atstheorie
auch die Geod\"aten
in gekr\"ummten Raumzeiten als
kr\"aftefreie Bahnkurven ansehen) durch
Geraden dargestellt. Damit bewegt
sich auch der r\"aumliche Ursprung
eines Intertialsys\-tems entlang einer 
Geraden. Die Koordinate entlang
dieser Geraden bezeichnen wir als
Zeitkoordinate. Sie wird realisiert
durch die Weltline einer idealen Uhr 
(n\"aherungsweise
z.B.\ durch eine Cs-Uhr), die sich im
Ursprung des Systems befindet. 

Wenn wir einem Ereignis A in einem 
solchen Inertialsystem die Koordinaten
$(t,\pmb{x})$ zuordnen, ist damit operational 
Folgendes gemeint: Wir denken uns den gesamten
Raum des Inertialsystems mit Uhren
ausgepflastert, die alle synchronisiert
sind. (Auf die Problematik der Synchronisation
von Uhren werden wir in Abschnitt
\ref{sec_Synch} n\"aher eingehen, in einer
speziellen Form schon im n\"achsten Abschnitt.  
An dieser Stelle soll gen\"ugen, dass eine
solche globale Synchronisation m\"oglich
ist und dass sie beispielsweise durch den
langsamen Transport von Uhren --
alle Uhren wurden in der fernen Vergangenheit
im Ursprung auf dieselbe Zeit eingestellt und dann
langsam an ihren Platz gebracht --
realisiert werden kann.\footnote{Sp\"ater 
werden wir zeigen, dass die so genannte
Einstein-Synchronisation, bei der Uhren durch
Austausch von Lichtsignalen synchronisiert
werden, zu dieser Vorschrift identisch ist.}) 
Au\ss erdem k\"onnen wir 
die r\"aumliche Lage von jeder Uhr in diesem Sys\-tem
durch einen Vektor $\pmb{x}$ kennzeichnen,
dessen Komponenten wir durch Anlegen
eines geeichten L\"angenma\ss stabes bestimmen
k\"onnen.\footnote{Auch hier ist der Austausch
von Lichtsignalen und die Messung des
r\"aumlichen Abstands durch die Messung der
Zeit f\"ur Hin- plus R\"uckweg ein praktikableres
Verfahren. Das Anlegen eines L\"angenma\ss stabs
ist \"aquivalent und dient hier nur der 
Veranschaulichung.} Alle Uhren bewegen sich
im Raumzeit-Diagramm auf parallelen Weltlinien
und halten untereinander ihren Abstand. 
Die $\pmb{x}$-Koordinate
eines Ereignisses A ist dann gleich der
r\"aumlichen Koordinate der Uhr, bei der
das Ereignis A stattfindet. Die $t$-Koordinate
des Ereignisses A ist gleich der Zeitanzeige
dieser Uhr. 
 
Zwei Ereignisse ereignen sich
{\em in diesem Inertialsystem} 
gleichzeitig, wenn die Uhren an den
jeweiligen Punkten, an denen die
Ereignisse stattfinden, dieselbe Zeit anzeigen.
F\"ur ein gegebenes Inertialsystem
ist es also sinnvoll, von der Gleichzeitigkeit
zweier Ereignisse zu sprechen. Allerdings
muss betont werden, dass die physikalische
Realisation dieser Gleichzeitigkeit
nur durch die Synchronisation von Uhren
m\"oglich ist, die sich bei den jeweiligen
Ereignissen befinden, und eine solche
globale Synchronisation ist nur in einem
Inertialsystem realisierbar. Insbesondere
wird der Gleichzeitigkeitsbegriff problematisch
f\"ur allgemeine Bezugssysteme (die keine
Inertialsysteme sind, also beispielsweise
beschleunigt werden) oder auch in 
der allgemeinen Relativit\"atstheorie.

Abschlie\ss end noch eine Anmerkung
zur Sprechweise. Sehr oft liest man, dass
ein Beobachter ein Ereignis A zu einem
bestimmten Zeitpunkt $t$ an einem bestimmten
Ort $\pmb{x}$ \glqq sieht\grqq\ oder \glqq wahrnimmt\grqq. 
Dieses \glqq sehen\grqq\ oder \glqq wahrnehmen\grqq\
hat jedoch meist nichts mit einem
physiologischen Sehen oder Wahrnehmen
zu tun, sondern bedeutet, dass das 
Ereignis in dem Inertialsystem des 
Beobachters von einer entsprechenden
Uhr (im oben beschriebenen Sinne)
zum Zeitpunkt $t$ am Ort $\pmb{x}$
registriert wird. Der Beobachter im 
Ursprung des Systems erf\"ahrt
m\"oglicherweise erst sehr viel
sp\"ater durch einen Datenaustausch
von dem Ereignis und seinen Koordinaten.
F\"ur ein tats\"achliches Sehen muss erst
ein Lichtsignal von dem Ereignis A zu
dem Beobachter gelangen. Dieses
Sehen findet also im Allgemeinen sp\"ater
statt, au\ss erdem kann es sein, dass
unterschiedliche Ereignisse, die in dem
Inertialsystem zwar gleichzeitig stattfinden
(und damit dieselbe $t$-Koordinate
haben) von dem Beobachter im Ursprung
zu unterschiedlichen Zeiten gesehen
werden und umgekehrt. 

\begin{thebibliography}{99}
\addcontentsline{toc}{chapter}{Literaturangaben}
\bibitem{Aichelburg} Peter C.\ Aichelburg (Hrsg.); {\it Zeit im 
       Wandel der Zeit}; Verlag Vieweg, Braunschweig, Wiesbaden, 1988.
\bibitem{Born} Max Born; {\it Optik}; Springer-Verlag, Berlin, Heidelberg,
        1972.
\bibitem{Britannica} Encyclopaedia Britannica; 15.th edition, 1988.
\bibitem{Descartes} Ren\'e Descartes; {\it Die Prinzipien der
        Philosophie}; Felix Meiner Verlag, Hamburg, 1992; \"ubersetzt
        von Artur Buchenau.
\bibitem{Einstein1} Albert Einstein; {\it Zur Elektrodynamik bewegter 
        K\"orper}; Annalen der Physik, Leipzig, 17 (1905) 891. 
\bibitem{Helmholtz2} Hermann von Helmholtz; {\em \"Uber Wirbelbewegungen,
        \"Uber Fl\"ussigkeitsbewegungen}, 1858; in Ostwalds Klassiker der 
       exakten Wissenschaften Bd.\ 1; Verlag Harri Deutsch, Frankfurt, 
       1996.                   
\bibitem{Laue} Max von Laue; {\it Geschichte der Physik}; 
         Universit\"ats-Verlag Bonn, 1947.
\bibitem{Lorentz} Hendrik Antoon Lorentz; {\it Electromagnetic phenomena 
         in a system moving with any velocity smaller than that of light}; 
         Proc.\ Acad.\ Sci., Amsterdam, 6 [1904], S.\ 809.
\bibitem{Mittelstaedt} Peter Mittelstaedt; {\it Der Zeitbegriff in der
        Physik}; BI-Wissenschaftsverlag, 1989.        
\bibitem{Mittelstaedt2} Peter Mittelstaedt; {\it Philosophische Probleme
        der modernen Physik}; BI-Wissenschaftsverlag, 1989.        
\bibitem{Newton2} Isaac Newton; {\it \"Uber die Gravitation...};
       Klostermann Texte Philosophie; Vittorio Klostermann, Frankfurt,
      1988; \"ubersetzt von Gernot B\"ohme.
\bibitem{Newton3} Isaac Newton; {\it Optik oder Abhandlung \"uber
      Spiegelungen, Brechungen, Beugungen und Farben des Lichts};
      I., II.\ und III.\ Buch (1704); aus dem Englischen \"ubersetzt
      von W.\ Abendroth; Ostwalds Klassiker der exakten Wissenschaften,
      Verlag Harri Deutsch 1998.   
\bibitem{Pauli} Wolfgang Pauli; {\it Theory of Relativity}; Dover
      Publications, New York, 1981.      
\bibitem{Poincare} Jules Henri Poincar\'e; {\it Sur la dynamique de 
     l'\'electron}, C.R.\ Acad.\ Sci., Paris, 140 (1905) S.~1504; und 
      Rendiconti del Circolo Matematico di Palermo, Bd.~21 (1906) S.~129.
\bibitem{Sexl} Roman U.\ Sexl, Helmuth K.\ Urbantke; {\it Relativit\"at,
      Gruppen, Teilchen}; Springer-Verlag, Wien, New York, 1992.
\bibitem{Simonyi}
      K\'aroly Simonyi; {\it Kulturgeschichte der Physik}; Verlag
       Harri Deutsch, Thun, Frankfurt am Main, 1990.
\end{thebibliography}

\end{document}
