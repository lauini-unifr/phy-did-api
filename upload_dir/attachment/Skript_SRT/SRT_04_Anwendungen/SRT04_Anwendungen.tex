\documentclass[german,10pt]{book}    
\usepackage{makeidx}
\usepackage{babel}            % Sprachunterstuetzung
\usepackage{amsmath}          % AMS "Grundpaket"
\usepackage{amssymb,amsfonts,amsthm,amscd} 
\usepackage{mathrsfs}
\usepackage{rotating}
\usepackage{sidecap}
\usepackage{graphicx}
\usepackage{color}
\usepackage{fancybox}
\usepackage{tikz}
\usetikzlibrary{arrows,snakes,backgrounds}
\usepackage{hyperref}
\hypersetup{colorlinks=true,
                    linkcolor=blue,
                    filecolor=magenta,
                    urlcolor=cyan,
                    pdftitle={Overleaf Example},
                    pdfpagemode=FullScreen,}
%\newcommand{\hyperref}[1]{\ref{#1}}
%
\definecolor{Gray}{gray}{0.80}
\DeclareMathSymbol{,}{\mathord}{letters}{"3B}
%
\newcounter{num}
\renewcommand{\thenum}{\arabic{num}}
\newenvironment{anmerkungen}
   {\begin{list}{(\thenum)}{%
   \usecounter{num}%
   \leftmargin0pt
   \itemindent5pt
   \topsep0pt
   \labelwidth0pt}%
   }{\end{list}}
%
\renewcommand{\arraystretch}{1.15}                % in Formeln und Tabellen   
\renewcommand{\baselinestretch}{1.15}                 % 1.15 facher
                                                      % Zeilenabst.
\newcommand{\Anmerkung}[1]{{\begin{footnotesize}#1 \end{footnotesize}}\\[0.2cm]}
\newcommand{\comment}[1]{}
\setlength{\parindent}{0em}           % Nicht einruecken am Anfang der Zeile 

\setlength{\textwidth}{15.4cm}
\setlength{\textheight}{23.0cm}
\setlength{\oddsidemargin}{1.0mm} 
\setlength{\evensidemargin}{-6.5mm}
\setlength{\topmargin}{-10mm} 
\setlength{\headheight}{0mm}
\newcommand{\identity}{{\bf 1}}
%
\newcommand{\vs}{\vspace{0.3cm}}
\newcommand{\noi}{\noindent}
\newcommand{\leer}{}

\newcommand{\engl}[1]{[\textit{#1}]}
\parindent 1.2cm
\sloppy

    \begin{document} \setcounter{chapter}{3}

\chapter{SRT - Effekte}   %  Kap. 4
\label{chap_SRT-Effekte}

In diesem Kapitel sollen einige Effekte beschrieben werden, die sich
aus der Lorentz-Invarianz der Minkowski-Raumzeit ergeben.
Dazu z\"ahlen die Zeitdilatation, die Lorentz-Kontraktion und der relativistische
Doppler-Effekt (longitudinal und transversal). Dem sogenannten 
Zwillings-Paradoxon ist ein eigenes Kapitel gewidmet (Kap.\ \ref{chap_Zwilling}).

Anschlie\ss end deuten wir an, wie sich der Lagrange-Formalismus auf die
Relativit\"atstheorie verallgemeinern l\"asst und was der zum Ortsvektor
kanonisch konjugierte Impuls ist. In diesem Zusammenhang definieren wir
auch den Begriff der Eigenzeit. Den Abschluss bildet ein von Einstein
erdachtes Gedankenexeriment, das eine elegante Herleitung der 
bekannten Beziehung $E=mc^2$ erlaubt.

\section{Zeitdilatation}

Wir behandeln zun\"achst das Ph\"anomen
der Zeitdilatation. Schon bei der Pendelkette
hatten wir gesehen, dass aus der Lorentz-Invarianz
der Feldgleichungen folgt, dass eine
relativ zum \"Ather bewegte Breather-L\"osung
langsamer schwingt bzw.\ eine gr\"o\ss ere
Schwingungsperiode hat, als eine Breather-L\"osung, die relativ zum \"Ather ruht.  
Doch aus dem Relativit\"atsprinzip sollte folgen,
dass ein Beobachter in dem bewegten System
umgekehrt ebenfalls den Eindruck hat, dass
die Uhren in dem ruhenden System langsamer
gehen. Wir wollen nun untersuchen, was
genau damit gemeint ist und weshalb dieses
Ph\"anomen tats\"achlich auftritt. 

\begin{SCfigure}[50][htb]
\setlength{\unitlength}{2.1pt}
\begin{picture}(125,110)(0,0)
\put(20,35){\vector(1,0){95}}
\put(50,5){\vector(0,1){90}}
\put(35,5){\vector(1,2){45}}
%\qbezier(19,102.5)(85,35)(151,102.5)
\qbezier(50,68.5)(82,68.5)(116,102.5)
\qbezier(0,88)(28,68.5)(50,68.5)
\put(68,71){\makebox(0,0){{\footnotesize $\bullet$}}}
\put(50,68.5){\makebox(0,0){{\footnotesize $\bullet$}}}
\put(66.5,68.3){\makebox(0,0){{\footnotesize $\bullet$}}}
\put(50,62.4){\makebox(0,0){{\footnotesize $\bullet$}}}
\put(50,35){\makebox(0,0){{\footnotesize $\bullet$}}}
\put(69,65.5){\makebox(0,0){$B'$}}
\put(64,73){\makebox(0,0){$A'$}}
\put(47.5,71.5){\makebox(0,0){$A$}}
\put(56,33){\makebox(0,0){$O$}}
\put(52,60){\makebox(0,0){$B$}}
\put(15,45){\line(2,1){105}}
\put(20,68.5){\line(1,0){90}}
\put(16,18){\vector(2,1){100}}
\put(110,33){\makebox(0,0){$x$}}
\put(110,62){\makebox(0,0){$x'$}}
\put(48,90){\makebox(0,0){$t$}}
\put(81,90){\makebox(0,0){$t'$}}
\put(50,100){\makebox(0,0){$1$}}
\put(80,100){\makebox(0,0){$2$}}
\multiput(66.5,5)(0,8){12}{\line(0,1){6}}
\multiput(21.3,5)(4,8){12}{\line(1,2){3}}
\thicklines
\put(20,5){\line(1,1){95}}
\put(80,5){\line(-1,1){80}}
\end{picture}
\caption{\label{fig_zeitdilat}%
F\"ur den ruhenden Beobachter 1 scheint
die bewegte Uhr von Beobachter 2 langsamer zu gehen
($B'$ liegt vor $A'$). Denselben
Eindruck hat umgekehrt auch der 
Beobachter 2 von der Uhr von Beobachter 1 ($B$ liegt vor $A$).
Die gestrichelten Linien beziehen sich auf jeweils einen
weiteren Beobachter in System 1 bzw.\ System 2. Diese
Beobachter lesen ihre (mit 1 bzw.\ 2 synchronosierten) Uhren bei den
Ereignissen $B'$ und $B$ ab.}
\end{SCfigure}

Wir betrachten wieder zwei Inertialsysteme
mit den jeweiligen Koordinaten $(x,t)$ und
$(x',t')$ (vgl.\ Abb.\ \ref{fig_zeitdilat}). 
Bei dem Ereignis $O$ treffen sich die
beiden Beobachter im Ursprung ihrer jeweiligen
Systeme und synchronosieren ihre Uhren jeweils auf
$t_0=0$. Wir betrachten nun die Situation
zun\"achst aus der Perspektive des Inertialsystems
von Beobachter 1. F\"ur diesen Beobachter 
zeigt die Uhr bei Ereignis $A$ auf die Zeit $t$.
In seinem Inertialsystem ist das Ereignis
$B'$ gleichzeitig zu $A$, hat also ebenfalls
die Koordinate $t$, denn bei diesem Ereignis $B'$
schneidet die Weltlinie von Beobachter 2
die (in der Zeichnung waagerechte) \glqq Gleichzeitigkeitslinie\grqq\ von 
Beobachter 1 zur Zeitkoordinate $t$. 
Wir wissen jedoch, dass f\"ur Beobachter 2
erst im Ereignis $A'$ dieselbe Zeit
vergangen ist wie f\"ur Beobachter 1 zum
Ereignis $A$. Das Ereignis $A'$ ist aber sp\"ater als
$B'$. Das bedeutet, dass die Uhr von Beobachter
2 bei Ereignis $B'$ noch nicht so viele
Zeittakte anzeigt, wie zum Zeitpunkt $A'$
(n\"amlich $t$) und damit die Uhr von Beobachter 1 zum
Zeitpunkt $t$. Beobachter 1 hat also den
Eindruck, die Uhr von Beobachter 2 gehe 
langsamer.

Ich betone hier nochmals, dass Beobachter 1
die Uhr von Beobachter 2 nicht \glqq sieht\grqq\
(au\ss er vielleicht im Augenblick, wo sich 
beide Beobachter treffen, also bei Ereignis
$O$). F\"ur den Vergleich der Uhren
w\"ahlt er f\"ur sein Inertialsystem zwei
gleichzeitige Ereignisse (z.B.\ $A$ und $B'$)
und die Zeit auf der Uhr von Beobachter 2
wird von einem anderen Beobachter
(dessen Weltlinie parallel zu der von 1 ist
aber durch das Ereignis $B'$ verl\"auft, dessen
Uhr aber mit der von 1 synchronisiert wurde - in Abb.\ \ref{fig_zeitdilat} 
die gestrichelt gezeichnete vertikale Weltlinie)
abgelesen. Der Uhrenvergleich erfolgt
in Inertialsystem 1 zu vollkommen anderen
Ereignissen als in Inertialsystem 2 und
der Grund daf\"ur liegt in der 
unterschiedlichen Zuordnung von
Gleichzeitigkeit f\"ur Ereignisse.

\section{Lorentz-Kontraktion}

\"Ahnlich wie im letzten Abschnitt
die Zeitdilatation untersuchen wir nun
das Ph\"anomen der Lorentz-Kontraktion
aus der Sichtweise der verschiedenen
Inertialsysteme. Die L\"ange, beispielsweise eines 
Lineals, wird dabei als der
r\"aumliche Abstand von Anfangs- und
Endpunkt des Lineals bestimmt, wobei die
Augenblicke der Messung dieses Abstands in den
jeweiligen Inertialsystemen gleichzeitig
sein sollen.

Wir betrachten zun\"achst einen 
Beobachter 1 in dessen Inertialsystem
das Lineal ruht (Abb.\ref{fig_LorentzKon} (links)). 
Die von dem Lineal \"uberstrichene Weltfl\"ache
ist in der Abbildung grau unterlegt. Wichtig
sind f\"ur uns die Weltlinien der beiden
Endpunkte des Lineals. Zu einem bestimmten
Zeitpunkt in Inertialsystem 1 (beispielsweise
$t=0$) befinden sich die beiden Entpunkte
bei den Ereignissen $O$ und $A$. Der
r\"aumliche Abstand $l=OA$ dieser Ereignisse f\"ur
Beobachter 1 definiert die L\"ange des
Lineals. Im Inertialsystem von Beobachter 2
sind aber die Ereignisse $O$ (linkes Ende
des Lineals) und $B$ (rechtes Ende) 
gleichzeitig, in seinem System ist die
L\"ange also durch $l'=OB$ gegeben. 
Diese L\"ange ist aber k\"urzer als $l$,
da das Ereignis $A'$, das f\"ur Inertialsystem 2
von $O$ denselben r\"aumlichen Abstand 
hat wie $A$ f\"ur Inertialsystem 1, au\ss erhalb
des Lineals liegt. 

\begin{figure}[htb]
\setlength{\unitlength}{2.0pt}
\begin{picture}(110,110)(0,0)
\put(40,5){\colorbox{Gray}{\makebox(30,80){}}}
\put(10,35){\vector(1,0){90}}
\put(40,5){\vector(0,1){90}}
\put(25,5){\vector(1,2){45}}
\put(6,18){\vector(2,1){100}}
\qbezier(73.5,35)(73.5,67)(107.5,101)
\qbezier(73.5,35)(73.5,15)(82,0)
\put(73.5,35){\makebox(0,0){{\footnotesize $\bullet$}}}
\put(76.5,53){\makebox(0,0){{\footnotesize $\bullet$}}}
\put(73.5,52){\makebox(0,0){{\footnotesize $\bullet$}}}
\put(40,35){\makebox(0,0){{\footnotesize $\bullet$}}}
\put(76.5,32.5){\makebox(0,0){$A$}}
\put(79,52){\makebox(0,0){$A'$}}
\put(46,32.5){\makebox(0,0){$O$}}
\put(70,54){\makebox(0,0){$B$}}
\put(60,38){\makebox(0,0){$l$}}
\put(59,47){\makebox(0,0){$l'$}}
\put(5,45){\line(2,1){105}}
\put(100,32){\makebox(0,0){$x$}}
\put(100,62){\makebox(0,0){$x'$}}
\put(38,90){\makebox(0,0){$t$}}
\put(65,90){\makebox(0,0){$t'$}}
\put(40,100){\makebox(0,0){$1$}}
\put(70,100){\makebox(0,0){$2$}}
\thicklines
\put(40,0){\line(0,1){90}}
\put(40.3,0){\line(0,1){90}}
\put(73.2,0){\line(0,1){90}}
\put(73.5,0){\line(0,1){90}}
\put(10,5){\line(1,1){95}}
\put(70,5){\line(-1,1){65}}
\end{picture}
%
\begin{picture}(110,110)(0,0)
\put(30,17){\rotatebox{-27}{\colorbox{Gray}{\makebox(21.5,80){}}}}

%\put(195,245){\rotatebox{90}{\makebox(0,0){{\footnotesize Detektor}}}}

\put(40,5){\vector(0,1){90}}
\put(10,35){\vector(1,0){90}}
\put(25,5){\vector(1,2){45}}
\put(6,18){\vector(2,1){100}}
\qbezier(73.5,35)(73.5,67)(107.5,101)
\qbezier(73.5,35)(73.5,15)(82,0)
\put(73.5,35){\makebox(0,0){{\footnotesize $\bullet$}}}
\put(76.5,53){\makebox(0,0){{\footnotesize $\bullet$}}}
\put(67.5,35){\makebox(0,0){{\footnotesize $\bullet$}}}
%\put(50,62.4){\makebox(0,0){{\footnotesize $\bullet$}}}
\put(40,35){\makebox(0,0){{\footnotesize $\bullet$}}}
\put(76.5,32.5){\makebox(0,0){$A$}}
\put(79,52){\makebox(0,0){$A'$}}
\put(62,32){\makebox(0,0){$B'$}}
%\put(65,75){\makebox(0,0){$A'$}}
%\put(47,73){\makebox(0,0){$A$}}
\put(46,32.5){\makebox(0,0){$O$}}
%\put(53,60){\makebox(0,0){$B$}}
\put(5,45){\line(2,1){105}}
\put(100,32){\makebox(0,0){$x$}}
\put(100,62){\makebox(0,0){$x'$}}
\put(38,90){\makebox(0,0){$t$}}
\put(65,90){\makebox(0,0){$t'$}}
\put(40,100){\makebox(0,0){$1$}}
\put(70,100){\makebox(0,0){$2$}}
\thicklines
\put(25,5){\line(1,2){43}}
\put(52.5,5){\line(1,2){43}}
\put(10,5){\line(1,1){95}}
\put(70,5){\line(-1,1){65}}
\end{picture}
\caption{\label{fig_LorentzKon}%
Das von einem Lineal \"uberstrichene
Raumzeitgebiet (grau unterlegt) erscheint in den
verschiedenen Inertialsystemen unterschiedlich
breit. (links) F\"ur den bewegten Beobachter 2 
scheint das in Inertialsystem 1 ruhende
Lineal k\"urzer. (rechts) Umgekehrt
erscheint ein bewegtes Lineal f\"ur den
ruhenden Beobachter k\"urzer als f\"ur
einen Beobachter, der sich mit dem Lineal
bewegt.}
\end{figure}

Wir betrachten nun die umgekehrte Situation:
Das Lineal ist in System 2 in Ruhe, seine
Anfangs- und Endpunkte bewegen sich also
in System 1 mit einer bestimmten 
Geschwindigkeit (Abb.\ \ref{fig_LorentzKon}(rechts)).
In System 2 wird die L\"ange des Lineals
beispielsweise bei den gleichzeitigen
Ereignissen $O$ und $A'$ bestimmt, und
der zugeh\"orige r\"aumliche Abstand 
ist $l=OA'$. In Inertialsystem 1 wird der
Abstand bei den gleichzeitigen Ereignissen
$O$ und $B'$ gemessen, und deren
Abstand ist offensichtlich kleiner als $l$. 

In beiden F\"allen finden wir somit, dass
die L\"ange des Lineals, gemessen von einem
bewegten System aus, immer kleiner ist
als seine L\"ange in seinem eigenen
Ruhesystem. 

\section{Doppler-Effekte}

Der Doppler-Effekt ist schon aus der 
nicht-relativistischen Mechanik bekannt:
Ein Martinshorn klingt h\"oher, wenn das
Auto auf uns zukommt, und tiefer, wenn es
sich von uns entfernt. Durch die Bewegung
des Autos werden die Wellenberge in 
Fahrtrichtung gestaucht -- treffen daher
in k\"urzeren Zeitabst\"anden beim Empf\"anger
ein und klingen h\"oher -- und entgegen der
Fahrtrichtung gestreckt -- sie treffen in 
gr\"o\ss eren Zeitabst\"anden beim Empf\"anger
ein und klingen daher tiefer. In der klassischen
Mechanik gibt es nur einen {\em longitudinalen}
Doppler-Effekt, d.h.\ dieser Effekt tritt nur auf,
wenn sich der radiale Abstand eines Senders
relativ zu einem Empf\"anger \"andert.

In der Relativit\"atstheorie kommen wegen
der Zeitdilatationen bzw.\ der Lorentz-Kontraktionen
in relativ zueinander bewegten Systemen
noch weitere Einfl\"usse 
hinzu, insbesondere gibt es nun auch den
so genannten {\em transversalen} Doppler-Effekt.

\subsection{Der Doppler-Effekt in der klassischen Mechanik}

Abbildung \ref{fig_Doppler1} zeigt eine nicht-relativistische
Raumzeit, d.h., die Gleichzeitigkeitslinien sind f\"ur alle
Beobachter parallel zur $x$-Achse und die zeitlichen
Abst\"ande zwischen zwei Ereignissen entsprechen
den Projektionen auf die $t$-Achse. Beobachter
E (der Empf\"anger) sei in Ruhe (bei Schallwellen 
bedeutet dies im Ruhesystem des Schalltr\"agers, also der Luft), 
Beobachter S (der Sender) bewegt
sich mit konstanter Geschwindigkeit auf E zu, trifft ihn bei 
Ereignis $C$ und bewegt sich ab dann von E weg. 
In gleichen Zeitabst\"anden $\Delta t_{\rm S}$ sendet Beobachter S
bei den Ereignissen $A$, $B$, $C$ und $D$ Signale an E.
Solange sich der Abstand zwischen dem Sender S 
und dem Empf\"anger E mit konstanter Rate verk\"urt, 
empf\"angt E die Signale
in gleichen Zeitabst\"anden $\Delta t_{\rm E}$; bewegt
sich S von E weg, sind die Zeitabst\"ande $\Delta t_{\rm E}'$.

\begin{SCfigure}[50][htb]
\setlength{\unitlength}{1pt}
\begin{picture}(180,200)(0,0)
\put(20,100){\vector(1,0){160}}
\put(100,0){\vector(0,1){200}}
\put(50,0){\vector(1,2){100}}
\put(100,100){\makebox(0,0){{\footnotesize $\bullet$}}}
\put(80,60){\makebox(0,0){{\footnotesize $\bullet$}}}
\put(120,140){\makebox(0,0){{\footnotesize $\bullet$}}}
\put(60,20){\makebox(0,0){{\footnotesize $\bullet$}}}
\put(100,60){\makebox(0,0){{\footnotesize $\bullet$}}}
\put(100,80){\makebox(0,0){{\footnotesize $\bullet$}}}
\put(100,160){\makebox(0,0){{\footnotesize $\bullet$}}}
\put(94,106){\makebox(0,0){$C$}}
\put(54,26){\makebox(0,0){$A$}}
\put(74,66){\makebox(0,0){$B$}}
\put(123,134){\makebox(0,0){$D$}}
\put(175,106){\makebox(0,0){$x$}}
\put(105,195){\makebox(0,0){$t$}}
\put(105,5){\makebox(0,0){E}}
\put(58,5){\makebox(0,0){S}}
\put(108,90){\makebox(0,0){${\scriptstyle \Delta t_{\rm E}}$}}
\put(92,130){\makebox(0,0){${\scriptstyle \Delta t_{\rm E}'}$}}
\put(83,84){\makebox(0,0){${\scriptstyle \Delta t_{\rm S}}$}}
\put(108,70){\makebox(0,0){${\scriptstyle \Delta t_{\rm E}}$}}
\put(115,115){\makebox(0,0){${\scriptstyle \Delta t_{\rm S}}$}}
\put(63,44){\makebox(0,0){${\scriptstyle \Delta t_{\rm S}}$}}
\put(60,20){\line(1,1){40}}
\put(80,60){\line(1,1){20}}
\put(120,140){\line(-1,1){20}}
\end{picture}
\caption{\label{fig_Doppler1}%
Doppler-Effekt in der nicht-relativistischen
Mechanik. Die Gleichzeitigkeitslinien sind
alle parallel zur $x$-Achse. In gleichen
Zeitabst\"anden $\Delta t_{\rm S}$ 
(bei den Ereignissen $A$, $B$, $C$, $D$) 
sendet Beobachter S Signale an Beobachter E.
Solang sich S auf E zubewegt, empf\"angt E
die Signale im Abstand $\Delta t_{\rm E}$,
bewegt sich S von E weg, ist der zeitliche
Abstand zwischen dem Empfang zweier
Signale $\Delta t_{\rm E}'$. Offensichtlich ist
$\Delta t_{\rm E}$ k\"urzer als $\Delta t_{\rm S}$, aber
$\Delta t_{\rm E}'$ l\"anger als $\Delta t_{\rm S}$.}
\end{SCfigure}

$\Delta t_{\rm E}$ ist k\"urzer als $\Delta t_{\rm S}$ und
zwar um die Zeitdauer, 
die das Signal braucht, um eine Strecke zur\"uckzulegen,
die Beobachter S in der Zeit $\Delta t_{\rm S}$ zur\"ucklegt.
Um diese Strecke bewegt sich S in der Zeit $\Delta t_{\rm S}$
auf E zu, und um diese Strecke ist der Weg f\"ur ein
Signal k\"urzer als beim letzten Signal. Die Strecke
ist $\Delta l = v \Delta t_{\rm S}$, die Zeit, die das Signal
f\"ur diese Strecke ben\"otigt, ist $\Delta T=\Delta l/c$,
also folgt:
\begin{equation}
          \Delta t_{\rm E} = \Delta t_{\rm S} - \frac{v}{c} \Delta t_{\rm S} =
           \left( 1 - \frac{v}{c} \right) \Delta t_{\rm S} \, .
\end{equation}
Also ist die Frequenz $\nu_{\rm E}$, mit der E Signale 
empf\"angt, um den Faktor $(1-\frac{v}{c})^{-1}$ 
gr\"o\ss er als die Frequenz $\nu_{\rm S}$, mit der S 
die Signale abschickt:
\begin{equation}
      \nu_{\rm E} = \frac{1}{(1 - \frac{v}{c})} \, \nu_{\rm S} \, .
\end{equation} 
Entsprechend ist die zugeh\"orige Wellenl\"ange
k\"urzer:
\begin{equation}
       \lambda_{\rm E} = \left( 1 - \frac{v}{c} \right) \lambda_{\rm S} \, .
\end{equation}
Wenn sich S von E entfernt, nimmt der Abstand
zwischen den beiden Beobachtern zu und das
sp\"atere Signal braucht die Zeit $\Delta T$ l\"anger
als das vorhergehende, um 
von S zu E zu gelangen:
\begin{equation}
          \Delta t_{\rm E}' = \Delta t_{\rm S} + \frac{v}{c} \Delta t_{\rm S} =
           \left( 1 + \frac{v}{c} \right) \Delta t_{\rm S} \, .
\end{equation}

\subsection{Der longitudinale relativistische Doppler-Effekt}

Wie zuvor emittiert  ein Sender S in regelm\"a\ss igen
Zeitabst\"anden $\Delta \tau_{\rm S}$ Signale 
(vgl.\ Abb.\ \ref{fig_Doppler2}). Diese Zeitabst\"ande
$\Delta \tau_{\rm S}$ beziehen sich nun auf die Eigenzeit des
Beobachters. Handelt es sich z.B.\ dabei um Licht einer 
bestimmten Frequenz $\nu_{\rm S}$, so bezieht sich diese
Frequenz nat\"urlich auf die Eigenzeit in dem System
des Senders. Zwei Ereignisse im zeitlichen Abstand
$\Delta \tau_{\rm S}$ f\"ur den Sender haben jedoch im
Inertialsystem E des Empf\"angers eine Zeitdifferenz
$\Delta t_{\rm S}$, die um einen Faktor $\gamma$ gr\"o\ss er
ist als die Eigenzeit (der \glqq ruhende\grqq\ Beobachter
sieht die Zeit in einem relativ zu ihm bewegten System
langsamer verstreichen):
\begin{equation}
         \Delta t_{\rm S} = \frac{1}{\sqrt{1 - \frac{v^2}{c^2}}} 
         \, \Delta \tau_{\rm S} \, .
\end{equation}

\begin{SCfigure}[50][htb]
\setlength{\unitlength}{1pt}
\begin{picture}(180,200)(20,0)
\put(20,100){\vector(1,0){160}}
\put(100,0){\vector(0,1){200}}
\put(50,0){\vector(1,2){100}}
\put(100,100){\makebox(0,0){{\footnotesize $\bullet$}}}
\put(80,60){\makebox(0,0){{\footnotesize $\bullet$}}}
\put(120,140){\makebox(0,0){{\footnotesize $\bullet$}}}
\put(60,20){\makebox(0,0){{\footnotesize $\bullet$}}}
\put(100,60){\makebox(0,0){{\footnotesize $\bullet$}}}
\put(100,80){\makebox(0,0){{\footnotesize $\bullet$}}}
\put(100,160){\makebox(0,0){{\footnotesize $\bullet$}}}
\put(94,106){\makebox(0,0){$C$}}
\put(54,26){\makebox(0,0){$A$}}
\put(74,66){\makebox(0,0){$B$}}
\put(123,134){\makebox(0,0){$D$}}
\put(175,106){\makebox(0,0){$x$}}
\put(105,195){\makebox(0,0){$t$}}
\put(105,5){\makebox(0,0){E}}
\put(58,5){\makebox(0,0){S}}
\put(108,90){\makebox(0,0){${\scriptstyle \Delta t_{\rm E}}$}}
\put(92,130){\makebox(0,0){${\scriptstyle \Delta t_{\rm E}'}$}}
\put(83,84){\makebox(0,0){${\scriptstyle \Delta \tau_{\rm S}}$}}
\put(108,70){\makebox(0,0){${\scriptstyle \Delta t_{\rm E}}$}}
\put(115,115){\makebox(0,0){${\scriptstyle \Delta \tau_{\rm S}}$}}
\put(63,44){\makebox(0,0){${\scriptstyle \Delta \tau_{\rm S}}$}}
\put(60,20){\line(1,1){40}}
\put(80,60){\line(1,1){20}}
\put(120,140){\line(-1,1){20}}
\end{picture}
\caption{\label{fig_Doppler2}%
Longitudinaler Doppler-Effekt in der relativistischen
Mechanik. Die Eigenzeiten $\Delta \tau_{\rm S}$ 
im System des Senders
sind nun um einen Faktor 
$1/\gamma=\sqrt{1-\frac{v^2}{c^2}}$ kleiner
als der zeitliche Abstand $\Delta t_{\rm S}$ derselben
Ereignisse im System von Beobachter E. Man beachte,
dass die Beziehung der Ereignisse identisch ist,
wie im nicht-relativistischen Fall. Ge\"andert hat sich
lediglich die Beziehung zwischen der Eigenzeit $\Delta \tau$
und der entsprechenden Zeit im System des
Signalempf\"angers.}
\end{SCfigure}

Abgesehen von diesem Unterschied bleibt die
Argumentation dieselbe: In dem System E bewegt
sich der Sender im Zeitraum $\Delta t_{\rm S}$ um die
Strecke $v\, \Delta t_{\rm S}$ und das Lichtsignal 
ben\"otigt daher bei zwei aufeinanderfolgenden
Signalen f\"ur das zweite Signal die Zeit
$\Delta T=(\frac{v}{c})\, \Delta t_{\rm S}$ weniger. Insgesamt
ergibt sich damit folgende Beziehung:
\begin{equation}
     \Delta t_{\rm E} = \left( 1-\frac{v}{c} \right) \Delta t_{\rm S}
       = \frac{\Big(1 - \frac{v}{c}\Big)}{\sqrt{1 - \frac{v^2}{c^2}}}
      \,  \Delta \tau_{\rm S} 
       = \sqrt{\frac{1 - \frac{v}{c}}{1 + \frac{v}{c}}} \, \Delta \tau_{\rm S}       
       \, .
\end{equation}
Die Vorzeichen f\"ur $v$ drehen sich entsprechend um,
wenn sich der Sender vom Empf\"anger entfernt. 

Bewegt sich der Sender auf den Empf\"anger
zu und handelt es sich bei dem ausgetauschten
Signal um Licht (was wir beim relativistischen
Effekt angenommen haben), so erscheint das
Licht f\"ur den Empf\"anger mit einer h\"oheren
Frequenz als f\"ur den Sender, daher spricht
man auch von einer {\em Blauverschiebung}.
Entfernt sich der Sender vom Empf\"anger, kommt
es entsprechend zu einer {\em Rotverschiebung}. 

\subsection{Der transversale Doppler-Effekt}

In der nicht-relativistischen Mechanik gibt es keinen
transversalen Doppler-Effekt, da sich der Abstand
zwischen Sender und Empf\"anger nicht \"andert
und die Zeitdifferenzen f\"ur beide Beobachter gleich
sind. In der relativistischen Mechanik bleibt bei 
einer transversalen Bewegung (d.h.\ der Sender
bewegt sich in einem gewissen Abstand senkrecht
zum Abstandsvektor) der Abstand ebenfalls 
konstant, es bleibt aber noch der Faktor
der Zeitdilatation. Dieser bewirkt, dass es
nun auch zwischen Sender (S) und Empf\"anger (E)
eine Frequenzverschiebung des Lichts gibt:
\begin{equation}
      \Delta \nu_{\rm S} = \sqrt{1 - \frac{v^2}{c^2}}\; \Delta \nu_{\rm S} \, . 
\end{equation}

\section{\glqq Einparken\grqq} 
\label{sec_Parken}

Die L\"angenkontraktion und die Zeitdilatation
geben Anlass zu einer Vielfalt an Scheinparadoxa,
die mit der speziellen Relativit\"atstheorie
assoziiert werden. In fast allen F\"allen liegt die Ursache der scheinbaren 
Widerspr\"uche in den unterschiedliche Ereignismengen, die von den beiden
Beobachtern als gleichzeitig empfunden werden.

Das folgende Beispiel bezieht sich auf die Lorentz-Kontraktion und wird in verschiedenen
Varianten in der Literatur behandelt. Wir formulieren es hier als ein Problem des
Einparkens.

Gegeben sei eine Garage der L\"ange $L$ und ein Auto der L\"ange $l>L$. Offensichtlich passt
das Auto nicht in die Garage, d.h., wenn die Frontsto\ss stange die Garagenhinterwand
(gegen\"uber dem Garagentor)
ber\"uhrt, l\"asst sich das Garagentor nicht
schlie\ss en. Wenn das Auto aber mit
einer gen\"ugend gro\ss en Geschwindigkeit
in die Garage f\"ahrt, ist seine L\"ange 
k\"urzer als die L\"ange der Garage und man
sollte das Tor schlie\ss en k\"onnen. 
Andererseits k\"onnte man sich aber auch
in das Inertialsystem des Fahrers versetzen, in 
dem das Auto in Ruhe ist und
sich die Garage mit gro\ss er
Geschwindigkeit auf das Auto zubewegt.
Nun ist die Garage verk\"urzt, die Situation
ist noch ung\"unstiger und das Tor sollte sich
erst Recht nicht schlie\ss en lassen. Ob
aber ein Garagentor geschlossen werden
kann oder nicht ist eine physikalische Tatsache
und kann nicht vom Inertialsystem eines
Beobachters abh\"angen.

Was passiert im Ruhesystem
der Garage, wenn das Auto mit hoher
Geschwindigkeit hereinf\"ahrt? Tats\"achlich
ist das Auto im Augenblick der Einfahrt
k\"urzer und passt in die Garage -- das
Garagentor kann geschlossen werden. Doch 
nun wird das Auto abgebremst und dehnt sich
aus, dabei st\"o\ss t es vorne und hinten
gegen die Garagenwand bzw.\ das
Garagentor und wird physikalisch gestaucht.

Wie erf\"ahrt der Autofahrer dieselbe
Situation? F\"ur ihn ist die Garage
wesentlich k\"urzer als das Auto. Wenn er
mit seiner Frontstange gegen die Garagenwand
f\"ahrt, ist der hintere Teil des Wagens noch
weit au\ss erhalb der Garage. Doch wenn
der Wagen bez\"uglich des Systems der
Garage vorne und hinten gleichzeitig
abgebremst wird, wird er im Ruhesystem
des Autofahrers von vorne abgebremst.
D.h., der Wagen f\"ahrt vorne gegen die
Garagenwand und wird dadurch gestaucht,
w\"ahrend der hintere Teil des Wagens
sich weiter nach vorne bewegt und
schlie\ss lich ebenfalls ganz in der
Garage ist, sodass das Tor geschlossen
werden kann. Erst dann erreicht die
Stauchung des Wagens auch den hinteren
Teil.

In beiden F\"allen f\"ahrt der Wagen 
in die Garage und das Tor kann geschlossen
werden, aber der Wagen wurde durch das
Abbremsen bzw.\ die Garagenw\"ande
derart gestaucht, dass er nicht mehr seine
urspr\"ungliche L\"ange hat. 

Eine wichtige Erkenntnis k\"onnen
wir aus diesem Beispiel festhalten:
Es gibt keinen idealen starren K\"orper!
Darunter w\"urde man einen K\"orper
verstehen, der jede Beeinflussung
(z.B.\ Verschiebung) an seinem einen
Ende instantan auf den gesamten 
K\"orper \"ubertr\"agt. Aufgrund der
endlichen Ausbreitungsgeschwindigkeit
von Licht kann sich auch in einem
K\"orper kein Signal mit einer gr\"o\ss eren
Geschwindigkeit ausbreiten. Ein Sto\ss\
auf der einen Seite f\"uhrt notwendigerweise
zu einer Sto\ss welle, die sich nicht
schneller als mit Lichtgeschwindigkeit
durch den K\"orper ausbreitet und
daher jede Einwirkung an einem
Ende verz\"ogert zu dem anderen Ende \"ubertr\"agt.

\section{Das Zwillingsparadoxon}
\label{sec_twin}

Dem Zwillingsparadoxon ist ein eigenes Kapitel gewidmet (siehe Kap.\ \ref{chap_Zwilling}),
daher wird es hier nur sehr kurz behandelt. Das Zwillingsparadoxon bezeichnet das
folgende Ph\"anomen: Wenn sich zwei Zwillinge
(die nach unserer Vorstellung immer
dasselbe Alter haben) treffen, und einer
von beiden auf der Erde verbleibt w\"ahrend
der andere mit gro\ss er Geschwindigkeit
in den Weltraum hinausfliegt und nach
vielen Jahren zur\"uckkommt, dann
haben die beiden nicht mehr dasselbe
biologische Alter. Der auf der Erde
verbliebene Zwilling ist \"alter als
derjenige, der durch den Weltraum
gereist ist. 

\begin{SCfigure}[50][htb]
\begin{picture}(120,200)(0,0)
\put(50,0){\line(0,1){180}}
\put(50,30){\line(2,3){40}}
\put(90,90){\line(-2,3){40}}
\put(50,30){\makebox(0,0){{\footnotesize $\bullet$}}}
\put(50,150){\makebox(0,0){{\footnotesize $\bullet$}}}
\put(50,90){\makebox(0,0){{\footnotesize $\bullet$}}}
\put(90,90){\makebox(0,0){{\footnotesize $\bullet$}}}
\put(50,75){\makebox(0,0){{\footnotesize $\bullet$}}}
\put(50,105){\makebox(0,0){{\footnotesize $\bullet$}}}
\put(42,30){\makebox(0,0){${\scriptstyle A_0}$}}
\put(42,70){\makebox(0,0){${\scriptstyle A_1}$}}
\put(42,90){\makebox(0,0){${\scriptstyle A_2}$}}
\put(42,110){\makebox(0,0){${\scriptstyle A_3}$}}
\put(42,150){\makebox(0,0){${\scriptstyle A_4}$}}
\put(57,30){\makebox(0,0){${\scriptstyle B_0}$}}
\put(100,90){\makebox(0,0){${\scriptstyle B_2}$}}
\put(57,150){\makebox(0,0){${\scriptstyle B_4}$}}
\qbezier(0,97.5)(50,52.5)(100,97.5)
\qbezier(0,82.3)(50,127.3)(100,82.3)
\end{picture}
\caption{\label{fig_Twin}%
Zum Zwillingsparadoxon: Die Weltlinie
von Zwilling 1 verl\"auft entlang der
Ereignisse $A_0=B_0$, $A_1,A_2,A_3$, $A_4=B_4$,
die von Zwilling 2 entlang $B_0,B_2,B_4$.
Die Weltlinie von Zwilling 1 ist l\"anger
als die von Zwilling 2, d.h., Zwilling 1
ist bei der Wiedervereinigung in 
Ereignis $A_4=B_4$ \"alter als sein
Bruder. Bei $B_2$ hat Zwilling 2 dasselbe
Alter wie Zwilling 1 bei $A_1$. Insgesamt ist
Zwilling 1 um die Zeitspanne zwischen $A_1$ und
$A_3$ \"alter.}
\end{SCfigure}

Betrachten wir dazu die Weltlinien der
beiden Zwillinge. Bis Ereignis $A_0=B_0$ 
haben beide dieselbe Weltlinie. Dann 
kommt es zur Trennung. W\"ahrend Zwilling
1 seinen bisherigen Bewegungszustand
beibeh\"alt (und die Ereignisse $A_1,A_2,A_3$
durchl\"auft), bewegt sich Zwilling 2 sehr
rasch zu Ereignis $B_2$, dort bremst er ab
und beschleunigt in die umgekehrte Richtung,
fliegt also wieder auf seinen Zwillingspartner zu.
Bei $A_4=B_4$ treffen sich die beiden
Zwillinge wieder.

Die Zeitdauern lassen sich leicht berechnen:
Die Zeitdauer f\"ur Zwilling 1 von $A_0$ bis
$A_1$ ist genauso lang wie die f\"ur Zwilling
2 von $A_0$ bis $B_2$. Entsprechend ist
die Zeitdauer $A_3 A_4$ dieselbe wie die
von Ereignis $B_2$ bis $B_4$ f\"ur Zwilling
2. Insgesamt hat die Weltlinie von Zwilling
2 von Ereignis $B_0$ bis $B_4$ also dieselbe
L\"ange wie die Summe der beiden Abschnitte
$A_0 A_1$ und $A_3 A_4$ f\"ur Zwilling 1.
Die Zwischenzeit - von $A_1$ bis  $A_3$ -
ist die Zeitdauer, um die Zwilling 1 \"alter ist.

Man k\"onnte auf die Idee kommen, dass
das unterschiedliche Alter der beiden
Zwillinge darauf zur\"uckzuf\"uhren ist,
dass Zwilling 2 mehrfach beschleunigt
wurde, insbesondere auch bei Ereignis
$B_2$. Um zu verdeutlichen, dass dieses
Argument nicht richtig ist, betrachten
wir Abb.\ \ref{fig_Drill}. Die Weltlinien der
drei Drillinge (der erste bleibt in Ruhe,
der zweite macht eine kurze Reise
\"uber Ereignis $C_2$, der dritte
macht die gro\ss e Reise \"uber $B_2$)
sind unterschiedlich lang. Drilling 1 ist
am meisten gealtert, Drilling 2 etwas weniger
und Drilling 3 noch weniger. Insbesondere
haben Drilling 2 und Drilling 3 dieselben
Beschleunigungsphasen erlebt, sind
aber trotzdem bei $A_4=B_4$ 
unterschiedlich alt. Es handelt sich um
einen Effekt, der auf der Geometrie der
Minkowski-Raumzeit beruht und nicht
auf unterschiedlichen Beschleunigungsphasen.

\begin{SCfigure}[50][htb]
\begin{picture}(120,200)(0,0)
\put(50,0){\line(0,1){180}}
\put(50,30){\line(2,3){40}}
\put(90,90){\line(-2,3){40}}
\put(50,75){\line(2,3){10}}
\put(60,90){\line(-2,3){10}}
\put(50,30){\makebox(0,0){{\footnotesize $\bullet$}}}
\put(50,150){\makebox(0,0){{\footnotesize $\bullet$}}}
\put(50,90){\makebox(0,0){{\footnotesize $\bullet$}}}
\put(90,90){\makebox(0,0){{\footnotesize $\bullet$}}}
\put(50,75){\makebox(0,0){{\footnotesize $\bullet$}}}
\put(50,105){\makebox(0,0){{\footnotesize $\bullet$}}}
\put(60,90){\makebox(0,0){{\footnotesize $\bullet$}}}
\put(42,30){\makebox(0,0){${\scriptstyle A_0}$}}
\put(42,70){\makebox(0,0){${\scriptstyle A_1}$}}
\put(42,90){\makebox(0,0){${\scriptstyle A_2}$}}
\put(42,110){\makebox(0,0){${\scriptstyle A_3}$}}
\put(42,150){\makebox(0,0){${\scriptstyle A_4}$}}
\put(57,30){\makebox(0,0){${\scriptstyle B_0}$}}
\put(100,90){\makebox(0,0){${\scriptstyle B_2}$}}
\put(57,150){\makebox(0,0){${\scriptstyle B_4}$}}
\put(67,90){\makebox(0,0){${\scriptstyle C_2}$}}
%\qbezier(0,97.5)(50,52.5)(100,97.5)
%\qbezier(0,82.3)(50,127.3)(100,82.3)
\end{picture}
\caption{\label{fig_Drill}%
Erweiterung des Zwillingsparadoxons f\"ur
Drillinge. Die drei Weltlinien -- ($A_0 \rightarrow
A_1 \rightarrow A_2 \rightarrow A_3 \rightarrow A_4$)
f\"ur Drilling 1, ($A_0 \rightarrow A_1 \rightarrow 
C_2 \rightarrow A_3 \rightarrow A_4$) f\"ur
Drilling 2 und ($B_0 \rightarrow B_2 \rightarrow B_4$)
f\"ur Drilling 3 -- sind unterschiedlich
lang. Insbesondere ist die Weltlinie von 
Drilling 3  k\"urzer als die von Drilling 2, obwohl
beide dieselben Beschleunigungsphasen
erlebt haben.}
\end{SCfigure}

\section{Die Eigenzeit}

Wie wir gesehen haben, k\"onnen ganz allgemein
Uhren, die entlang unterschiedlicher Weltlinien
transportiert wurden, unterschiedliche Zeitdauern
anzeigen, selbst wenn die Weltlinien dieselben Ereignisse verbinden.
Im letzten Abschnitt hat es sich um st\"uckweise gerade
Weltlinien gehandelt, bei denen wir zur
Bestimmen der Zeitdauer die Anteile der
einzelnen Teilst\"ucke addiert haben. Dies
k\"onnen wir f\"ur stetige Weltlinien verallgemeinern.

F\"ur zwei zeitartige Ereignisse $B$ und $A$
($B$ zeitlich nach $A$),
mit den Differenzkoordinaten $\Delta t=t_B-t_A$
und $\Delta \vec{x}=\vec{x}_B - \vec{x}_A$ 
in einem beliebigen Inertialsystem,
definieren wir die {\em Eigenzeit} $\tau_{AB}$
durch
\begin{equation}
\label{eq_eigen1}
      \tau_{AB}^2 = (\Delta t)^2 - \frac{1}{c^2} (\Delta \vec{x})^2\, .
\end{equation}
$\tau_{AB}$ ist die Zeitdifferenz zwischen den beiden
Ereignissen, die von einer Uhr angezeigt
wird, die mit konstanter Geschwindigkeit (also in einem
Inertialsystem) von Ereignis $A$ zu Ereignis $B$
transportiert wird. In dem Inertialsystem dieser
Uhr finden beide Ereignisse im r\"aumlichen
Koordinantenursprung statt. F\"ur zwei Ereignisse
ist $\tau$ eine Invariante und es spielt keine
Rolle, in welchem Inertialsystem $\tau$ nach
Gl.\ \ref{eq_eigen1} berechnet wird, da die rechte Seite dieser Gleichung
invariant unter Lorentz-Transformationen ist (siehe Kap.\ \ref{chap_Grundlagen}). 

Handelt es sich um
infinitesimal benachbarte Ereignisse,
deren Raum- und Zeitkoordinaten sich in einem
beliebigen Inertialsystem um
${\rm d}\vec{x}$ und ${\rm d}t$ unterscheiden,
erhalten wir
\begin{equation}
     {\rm d}\tau
      = \sqrt{ 1 - \frac{1}{c^2} \left(\frac{{\rm d}\vec{x}}{{\rm d}t} \right)^2} {\rm d}t
      = \sqrt{1 - \frac{v(t)^2}{c^2}} {\rm d}t
\end{equation}
als die Eigenzeit zwischen den beiden
Ereignissen. Wir k\"onnen nun einer beliebigen
Weltlinie $\gamma$ (die nat\"urlich an jedem ihrer Punkte
eine zeitartige Tangente haben muss) zwischen
zwei Ereignissen $A$ und $B$
eine Eigenzeit zuordnen:
\begin{equation}
     \tau(\gamma) = \int_{t_B;\gamma}^{t_A} \sqrt{1 - \frac{v(t)^2}{c^2}}
     \, {\rm d} t  \, .
\end{equation}
Diese Zeit wird von einer Uhr angezeigt, die entlang der
Weltlinie $\gamma$ von $A$ nach $B$ transportiert
wird. Wie schon erw\"ahnt spielt es dabei keine
Rolle, in welchem Inertialsystem die momentane
Geschwindigkeit $v(t)$ gemessen und das Integral
ausgewertet wird. Die Eigenzeit zwischen zwei
Ereignissen entlang eines Weges $\gamma$ ist
am gr\"o\ss ten, wenn es sich bei $\gamma$ um
eine gerade Verbindungslinie handelt. Wir werden
sp\"ater in der Allgemeinen Relativit\"atstheorie
nicht mehr von Geraden sprechen k\"onnen,
wohl aber von geod\"atischen Verbindungswegen.
Diese zeichnen sich durch eine maximale
Eigenzeit aus.  

\section{Der kanonische Formalismus}

In der klassischen, Newton'schen Mechanik hat sich
der Lagrange-Formalismus als sehr n\"utzlich erwiesen. Auch
in der speziellen Relativit\"atstheorie lassen sich
eine Lagrange-Funktion und eine zugeh\"orige Wirkung 
angeben, aus der nicht nur die Bewegungsgleichungen 
sondern auch die kanonisch konjugierten Variablen folgen.

Die Wirkung ordnet jeder Trajektorie $t \mapsto \vec{x}(t)$ 
eine Zahl zu. Ein gro\ss er Vorteil des Lagrange-Formalismus 
ist, dass diese Zahl nicht von der Wahl
der Koordinaten abh\"angt. Daher sollte diese Zahl auch
f\"ur die relativistische Wirkung eine Invariante sein. 
Statt diese Invariante abzuleiten, geben wir die
Lagrange-Funktion und ihre Wirkung einfach an und
zeigen, dass sie im klassischen Grenzfall mit den
bekannten Gr\"o\ss en \"ubereinstimmen, aber relativistisch
invariant sind. (Durch diese beiden Forderungen sind
die Gr\"o\ss en im Allgemeinen bereits festgelegt.)

Als Lagrange-Funktion eines freien Teilchens
definieren wir f\"ur ein beliebiges Inertialsystem
mit Koordinaten $(t,\vec{x})$ (und der Geschwindigkeit
$\vec{v}(t) = \dot{\vec{x}}= \frac{{\rm d}\vec{x}(t)}{{\rm d}t}$):
\begin{equation}
           L = - mc^2 \sqrt{1 - \frac{\vec{v}(t)^{\,2}}{c^2}} \, ,
\end{equation} 
und die zugeh\"orige Wirkung ist
\begin{equation}
        S =  - mc^2 \int_{t_0}^{t_1}
        \sqrt{1 - \frac{\vec{v}(t)^{\,2}}{c^2}} \, {\rm d}t \, . 
\end{equation}
Ausgedr\"uckt durch die Eigenzeit entlang einer
Bahnkurve ergibt sich f\"ur die Wirkung:
\begin{equation}
        S = -mc^2 \int_{\tau_0}^{\tau_1} \!\! {\rm d}\tau \, .
\end{equation}
Zu integrieren ist jeweils 
entlang der Bahnkurve. Aus der letzteren Darstellung
wird die Invarianz der Wirkung offensichtlich, d.h.\ $S$ ist
f\"ur jedes Inertialsystem gleich. 

Entwickeln wir in einem festen Inertialsystem die
Quadratwurzel, so erhalten wir f\"ur die Lagrange-Funktion 
in f\"uhrender Ordnung:
\begin{equation}
    L = - mc^2 \left( 1 - \frac{1}{2} \frac{\vec{v}(t)^{\,2}}{c^2} + ... \right)
       \approx - mc^2 + \frac{1}{2} m \vec{v}(t)^2  \, ,    
\end{equation}
also bis auf den konstanten Term $-mc^2$ (der, wie wir gleich
sehen werden, der Ruheenergie eines Teilchens entspricht)
die klassische Wirkung.

Die zu $\vec{x}$ kanonisch
konjugierte Variable $\vec{p}$ ist definiert als
\begin{equation}
\label{eq_kankon}
         p_i(t) = \frac{\partial L}{\partial v_i(t)} =
         \frac{mv_i(t)}{\sqrt{1- \frac{\vec{v}(t)^2}{c^2}}}
         \hspace{0.5cm} {\rm bzw.} \hspace{0.5cm}
         \vec{p}(t) =  \frac{m\vec{v}(t)}{\sqrt{1- \frac{\vec{v}(t)^2}{c^2}}} \, .
\end{equation} 
Entsprechend ist die Energie (die eine Erhaltungsgr\"o\ss e
ist, da $L$ nicht explizit von der Zeit abh\"angt) definiert als:
\begin{equation}
    E = \vec{v}\cdot \vec{p} - L(\vec{v})  \, ,
\end{equation}
woraus wir erhalten:
\begin{equation}
   E = \frac{m \vec{v}(t)^2}{\sqrt{1 - \frac{\vec{v}^{\,2}}{c^2}}} +
     \frac{mc^2}{\sqrt{1 - \frac{\vec{v}^{\,2}}{c^2}}} \left(1 -
     \frac{\vec{v}^{\,2}}{c^2} \right)  ~= ~ 
     \frac{mc^2}{\sqrt{1 - \frac{\vec{v}^{\,2}}{c^2}}}  \, . 
\end{equation}
In einem kanonischen Formalismus sollte die Energie
allerdings als Funktion von Ort und kanonisch 
konjugiertem Impuls aufgefasst werden. Wir l\"osen
daher Gl.\ \ref{eq_kankon} nach $\vec{v}^{\,2}$ auf
und erhalten schlie\ss lich:
\begin{equation}
        E = \sqrt{m^2c^4 + \vec{p}^{\; 2}c^2}   \, .
\end{equation}


\section{Die \"Aquivalenz von Masse und Energie}
\index{Aequivalenz@\"Aquivalenz von Masse und Energie}

Eine weitere Folgerung aus der speziellen
Relativit\"atstheorie ist nicht nur besonders bekannt
geworden sondern auch von fundamentaler 
Bedeutung, n\"amlich die \"Aquivalenz von tr\"ager Masse und Energie.

Allgemein schreibt man die Formel
\begin{equation}
\label{Emc}
           E ~=~ m c^2        
\end{equation}
Einstein\index{Einstein, Albert} 
zu, der sie 1905 in der Arbeit {\it Ist die Tr\"agheit eines
K\"orpers von seinem Energieinhalt abh\"angig?} (\cite{Einstein2}) 
f\"ur den Sonderfall eines strahlenden K\"orpers hergeleitet 
und ihre allgemeine Richtigkeit vermutet hatte. Abgesehen davon, dass
die Herleitung dieser Formel als falsch gilt (siehe \cite{Simonyi};
Farbtafel XXIV, Zitat aus Jammer), findet man \"Uberlegungen \"uber 
eine Tr\"agheit von Energie auch schon bei Poincar\'e 1900. 

Die folgende \glqq Elementare Ableitung der \"Aquivalenz von Masse und
Energie\grqq\ stammt ebenfalls von Einstein aus dem Jahre 1946, und ist
in seinem Buch {\it Aus meinen sp\"aten Jahren} (\cite{Einstein3}, S.~121) 
entnommen.\index{Zitat!Einstein}

\small
Die vorstehende Ableitung des \"Aquivalenzgesetzes, die bisher nicht 
publiziert ist, hat zwei Vorteile. Sie bedient sich zwar des speziellen
Relativit\"atsprinzips, setzt aber die technisch-formalen Hilfsmittel
der Theorie nicht voraus, sondern bedient sich nur dreier vorbekannter
Gesetzm\"a\ss igkeiten:
\begin{enumerate}
\item
des Satzes von der Erhaltung des Impulses,
\item
des Ausdrucks f\"ur den Strahlungsdruck beziehungsweise f\"ur den
Impuls eines in bestimmter Richtung sich ausbreitenden
Strahlungs-Komplexes,
\item
der wohlbekannte Ausdruck f\"ur die Aberration des Lichts (Einflu\ss\
der Bewegung der Erde auf den scheinbaren Ort der Fixsterne (Bradley)).
\end{enumerate}

Wir fassen nun folgendes System ins Auge. Bez\"uglich eines 
Koordinatensystems $K_0$ (Abb.\ \ref{fig_Emc2a}) schwebe der K\"orper B frei im Raum.
Zwei Strahlungskomplexe S, ${\rm S}'$, je von der Energie
$\frac{E}{2}$ breiten sich l\"angs der positiven bzw.\ negativen 
$x_0$-Richtung
aus und werden dann von B absorbiert. Bei der Absorption w\"achst die
Energie von B um $E$. Der K\"orper B bleibt bei diesem Prozess aus
Symmetrie-Gr\"unden in Ruhe.


\begin{figure}[htb]
\begin{picture}(360,190)(0,0)
\put(30,30){\vector(1,0){220}}
\put(30,30){\vector(0,1){140}}
\put(140,50){\vector(1,0){200}}
\put(140,50){\vector(0,1){120}}
\put(250,20){\makebox(0,0){$x$}}
\put(340,40){\makebox(0,0){$x'$}}
\put(10,35){\makebox(0,0){$K$}}
\put(130,55){\makebox(0,0){$K_0$}}
\put(23,170){\makebox(0,0){$z$}}
\put(130,170){\makebox(0,0){$z_0$}}
\put(75,120){\circle{16}}
\put(75,135){\makebox(0,0){$S$}}
\put(83,120){\vector(1,0){20}}
\put(300,120){\circle{16}}
\put(300,135){\makebox(0,0){$S'$}}
\put(292,120){\vector(-1,0){20}}
\put(172,95){\framebox(30,50){B}}
\put(20,40){\vector(0,-1){20}}
\end{picture}
\caption{\label{fig_Emc2a}%
Das System $B$ absorbiert die Energie von $S$ und $S'$ - aus dem Inertialsystem von $B$ aus
betrachtet.}
\end{figure}

Nun betrachten wir diesen selben Prozess von einem System $K$ aus,
welches sich gegen\"uber $K_0$ mit der konstanten Geschwindigkeit $v$ in
der negativen $z_0$-Richtung bewegt. 
In bezug auf $K$ ist dann die
Beschreibung des Vorganges [folgende] (siehe Abb.\ \ref{fig_Emc2b}).

\begin{figure}[htb]
\begin{picture}(360,160)(0,0)
\put(60,30){\vector(1,0){220}}
\put(60,30){\vector(0,1){110}}
\put(280,20){\makebox(0,0){$x$}}
\put(53,140){\makebox(0,0){$z$}}
\put(96,98){\makebox(0,0){$S$}}
\put(184,98){\makebox(0,0){$S'$}}
\put(112,90){\makebox(0,0){$\alpha$}}
\put(168,90){\makebox(0,0){$\alpha$}}
\put(90,85){\vector(2,1){20}}
\put(190,85){\vector(-2,1){20}}
\put(140,110){\vector(0,1){20}}
\put(133,130){\makebox(0,0){$v$}}
\put(125,60){\framebox(30,50){B}}
\multiput(90,85)(3,0){7}{\makebox(0,0){$\cdot$}}
\multiput(172,85)(3,0){7}{\makebox(0,0){$\cdot$}}
\end{picture}
\caption{\label{fig_Emc2b}%
Dieselbe Situation wie in Abb.\ \ref{fig_Emc2a}, allerdings aus dem
bewegten System $K$ heraus betrachet.}
\end{figure}

Der K\"orper B bewegt sich in der positiven $z$-Richtung mit der
Geschwindigkeit $v$. Die beiden Lichtkomplexe haben in bezug auf $K$
eine Fortpflanzungsrichtung, welche einen Winkel $\alpha$ mit der
$x$-Achse bildet. Das Aberrationsgesetz besagt, dass in erster 
N\"aherung $\alpha=\frac{v}{c}$ ist, wobei $c$ die Lichtgeschwindigkeit
bedeutet. Aus der Betrachtung in bezug auf $K_0$ wissen wir, dass die
Geschwindigkeit $v$ von $B$ durch die Absorption von $S$ und $S'$ keine
\"Anderung erf\"ahrt.

Nun wenden wir auf den Prozess in bezug auf $K$ das Gesetz von der
Erhaltung des Impulses in bezug auf die Richtung $z$ des
betrachteten Gesamtsystems an.

I.\ {\it Vor der Absorption} sei $M$ die Masse von $B$; $Mv$ ist dann der
Ausdruck des Impulses von $B$ (gem\"a\ss\ der klassischen Mechanik). Jeder
der Strahlungskomplexe hat die Energie $\frac{E}{2}$ und deshalb 
gem\"a\ss\ einer wohlbekannten Folgerung aus Maxwells Theorie den
Impuls 
\begin{equation}
        \frac{E}{2c}  \;.  
\end{equation}
        
Dies ist streng genommen zun\"achst der Impuls von $S$ in bezug auf
$K_0$. Wenn aber $v$ klein ist gegen $c$, so muss der Impuls in 
bezug auf $K$ bis auf die Gr\"o\ss e von zweiter Ordnung $\frac{v^2}{c^2}$
dieselbe sein. Von diesem Impuls f\"allt in die $z$-Richtung
die Komponente $\frac{E}{2c}\sin \alpha$, sind aber gen\"ugend genau
(bis auf Gr\"o\ss en h\"oherer Ordnung) $\frac{E}{2c}\alpha$ oder
$\frac{E}{2}\cdot \frac{v}{c^2}$. $S$ und $S'$ zusammen haben also in
der $z$-Richtung den Impuls $E \frac{v}{c^2}$. Der Gesamtimpuls des
Systems vor der Absorption ist also
\begin{equation}
          Mv + \frac{E}{c^2} v   \;.   
\end{equation}
          
II. {\it Nach der Absorption} sei $M'$ die Masse von $B$. Wir
antizipieren hier die M\"oglichkeit, dass die Masse bei der Aufnahme
der Energie $E$ eine Zunahme erfahren k\"onnte (dies ist n\"otig, damit
das Endresultat unserer \"Uberlegungen widerspruchsfrei sei). Der
Impuls des Systems nach der Absorption ist dann
\begin{equation}
      M' v  \;. 
\end{equation}      
Nun setzen wir den Satz von der Erhaltung des Impulses als richtig
voraus und wenden ihn in bezug auf die $z$-Richtung an. Dies ergibt
die Gleichung
\begin{equation}
       Mv + \frac{E}{c^2} v ~=~ M' v   
\end{equation}
oder 
\begin{equation}
       M' - M ~=~ \frac{E}{c^2}   \;.   
\end{equation}      
Diese Gleichung dr\"uckt den Satz der \"Aquivalenz von Energie und
Masse aus. Der Energiezuwachs $E$ ist mit dem Massenzuwachs 
$\frac{E}{c^2}$ verbunden. Da die Energie ihrer \"ublichen Definition
gem\"a\ss\ eine additive Konstante freil\"asst, so k\"onnen wir nach
der Wahl der letzteren stattdessen auch k\"urzer schreiben
\begin{equation}
        E ~=~ M c^2    \;.    
\end{equation}

\normalsize

\begin{thebibliography}{99}
\addcontentsline{toc}{chapter}{Literaturangaben}
\bibitem{Einstein2} Albert Einstein; {\it Ist die Tr\"agheit eines
        K\"orpers von seinem Energieinhalt abh\"angig?} (Ann.\ Phys., 
        Leipzig, 18 (1905) 639.
\bibitem{Einstein3} Albert Einstein; {\it Aus meinen sp\"aten Jahren};
         Ullstein Sachbuch, Verlag Ullstein, Frankfurt, Berlin, 1993.                 
\bibitem{Simonyi}
      K\'aroly Simonyi; {\it Kulturgeschichte der Physik}; Verlag
       Harri Deutsch, Thun, Frankfurt am Main, 1990.
\end{thebibliography}

\end{document}
