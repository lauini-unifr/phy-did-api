\documentclass[german,10pt]{book}      
\usepackage{makeidx}
\usepackage{babel}            % Sprachunterstuetzung
\usepackage{amsmath}          % AMS "Grundpaket"
\usepackage{amssymb,amsfonts,amsthm,amscd} 
\usepackage{mathrsfs}
\usepackage{rotating}
\usepackage{sidecap}
\usepackage{graphicx}
\usepackage{color}
\usepackage{fancybox}
\usepackage{tikz}
\usetikzlibrary{arrows,snakes,backgrounds}
\usepackage{hyperref}
\hypersetup{colorlinks=true,
                    linkcolor=blue,
                    filecolor=magenta,
                    urlcolor=cyan,
                    pdftitle={Overleaf Example},
                    pdfpagemode=FullScreen,}
%\newcommand{\hyperref}[1]{\ref{#1}}
%
\definecolor{Gray}{gray}{0.80}
\DeclareMathSymbol{,}{\mathord}{letters}{"3B}
%
\newcounter{num}
\renewcommand{\thenum}{\arabic{num}}
\newenvironment{anmerkungen}
   {\begin{list}{(\thenum)}{%
   \usecounter{num}%
   \leftmargin0pt
   \itemindent5pt
   \topsep0pt
   \labelwidth0pt}%
   }{\end{list}}
%
\renewcommand{\arraystretch}{1.15}                % in Formeln und Tabellen   
\renewcommand{\baselinestretch}{1.15}                 % 1.15 facher
                                                      % Zeilenabst.
\newcommand{\Anmerkung}[1]{{\begin{footnotesize}#1 \end{footnotesize}}\\[0.2cm]}
\newcommand{\comment}[1]{}
\setlength{\parindent}{0em}           % Nicht einruecken am Anfang der Zeile 

\setlength{\textwidth}{15.4cm}
\setlength{\textheight}{23.0cm}
\setlength{\oddsidemargin}{1.0mm} 
\setlength{\evensidemargin}{-6.5mm}
\setlength{\topmargin}{-10mm} 
\setlength{\headheight}{0mm}
\newcommand{\identity}{{\bf 1}}
%
\newcommand{\vs}{\vspace{0.3cm}}
\newcommand{\noi}{\noindent}
\newcommand{\leer}{}

\newcommand{\engl}[1]{[\textit{#1}]}
\parindent 1.2cm
\sloppy

       \begin{document} \setcounter{chapter}{5}

\chapter{Beschleunigte Systeme und das Rindler-Universum}   %  Kap. 6
\label{SRT_Beschleunigung}

Das \"Aquivalenzprinzip besagt im Wesentlichen,
dass wir in einem lokalen Bezugssys\-tem nicht 
zwischen einer konstanten Beschleunigung 
und einem konstanten Gravitationsfeld 
unterscheiden k\"onnen. Wir werden
dieses Prinzip in den n\"achsten Kapiteln 
ausgiebig nutzen, um den Einfluss von
Gravitationsfeldern zu untersuchen und damit
ersten Schritte zur Allgemeinen Relativit\"atstheorie
zu gehen.

In diesem Kapitel geht es konkret um 
einen konstant
beschleunigten Beobachter in der Speziellen
Relativit\"atstheorie. Viele der Effekte
lassen sich dann \"uber das \"Aquivalenzprinzip
auf die Allgemeine Relativit\"atstheorie
\"ubertragen.

\section{Die konstante Beschleunigung}

Schon allein die Frage,  was genau unter einer
konstanten Beschleunigung zu verstehen ist, bedarf
in der Speziellen Relativit\"atstheorie einer 
eingehenderen Analyse. Wir k\"onnen an einem
ausgedehnten K\"orper nicht einfach eine
Kraft angreifen lassen, da kein K\"orper wirklich
starr ist - dies w\"urde der Speziellen Relativit\"atstheorie
widersprechen - und sich die Wirkung jeder
angreifenden Kraft erst \"uber eine Sto\ss welle
auf den K\"orper ausdehnt. Der Einfachheit
halber betrachten wir daher zun\"achst nur
einen idealisierten Massepunkt, der
konstant beschleunigt werden soll. Doch auch
hier ist das Konzept einer konstanten
Beschleunigung nicht trivial.

Einerseits befindet sich der 
beschleunigte Gegenstand zu jedem Zeitpunkt in 
einem anderen Inertialsystem, andererseits
h\"angt die naheliegende Antwort --
eine konstante Beschleunigung bedeutet einen pro 
Zeiteinheit konstanten Geschwindigkeitszuwachs --
vom Bezugssystem ab. Eine invariante Definition
des Konzepts einer konstanten Beschleunigung,
die wir auch in diesem Kapitel verwenden 
werden, ist folgende: 
{\em Im jeweiligen momentanten Inertialsystem
des beschleunigten Massepunktes ist die 
Beschleunigung konstant.} Damit ist gemeint,
dass es in jedem Augenblick ein Inertialsystem
gibt, das dieselbe Geschwindigkeit wie der
beschleunigte Beobachter hat (nat\"urlich 
\"andert sich dieses Inertialsystem st\"andig); 
von diesem \glqq momentanen Inertialsystem\grqq\
aus betrachtet erf\"ahrt der beschleunigte 
Beobachter einen konstanten Geschwindigkeitszuwachs.

In diesem Abschnitt betrachten wir die Bewegung eines 
konstant beschleunigten Massepunktes von einem
Inertialsystem aus, in dem
der Massepunkt zum Zeitpunkt $t_0=0$
ruht. Zun\"achst leiten wir die 
Differentialgleichung f\"ur die 
Geschwindigkeit --
gemessen in diesem Inertialsystem -- 
her, anschlie\ss end l\"osen wir
diese Gleichung und bestimmten die
Bahnkurve $x(t)$ f\"ur diesen Massepunkt.

\subsection{Herleitung der Differentialgleichung
f\"ur die Geschwindigkeit}

Die Differentialgleichung f\"ur die
Geschwindigkeit im Inertialsystem des zu
Beginn ruhenden Teilchens lautet:
\begin{equation}
\label{eq_Diffkonst}
      \frac{{\rm d}v(t)}{{\rm d}t} =
          g \left( 1 - \frac{v(t)^2}{c^2} \right)^{3/2} \, .
\end{equation}
Diese Gleichung werden wir im n\"achsten
Abschnitt l\"osen, doch zun\"achst wollen wir
sie auf zwei verschiedene Weisen ableiten.

In dem Ruhesystem des Massepunktes
zu einem bestimmten Zeitpunkt ($t$ im
Inertialsystem des anf\"anglichen Ruhezustands,
$\tau$ im Bezugssytem des beschleunigten
Massepunktes) soll
eine Kraft wirken, die ihn in der infinitesimalen
Eigenzeit ${\rm d}\tau$ immer auf dieselbe 
infinitesimale Geschwindigkeit ${\rm d}v$ 
beschleunigt:
\begin{equation}
          {\rm d} v = g \, {\rm d} \tau  \, ,
\end{equation}
wobei $g$ ein Ma\ss\ f\"ur die konstante
Beschleunigung ist. 

Im anf\"anglichen Ruhesstem, d.h.\ bez\"uglich
der Zeit $t$, nimmt die Geschwindigkeit also
zu einem Zeitpunkt $t$ um
\begin{equation}
      {\rm d} v = g \sqrt{1 - \frac{v(t)^2}{c^2} } \,  {\rm d}t
\end{equation}
zu. Hierbei wurde die Beziehung ${\rm d}\tau = \sqrt{1-v^2/c^2}\,{\rm dt}$
zwischen der Eigenzeit $\tau$ und der Zeit $t$ in einem (momentanen)
Inertialsystem verwendet. 

Hat f\"ur den ruhenden Beobachter das beschleunigte
System zum Zeitpunkt $t$ die Geschwindigkeit $v(t)$,
so hat es zum Zeitpunkt $t+ {\rm d}t$ nach dem
Geschwindigkeitsadditionstheorem
(vgl.\ Gl.\ \ref{eq_vadd}) die Geschwindigkeit:
\begin{eqnarray}
   v(t+{\rm d}t) &=& \frac{v(t) + {\rm d}v}{1 + \frac{v(t) \, {\rm d}v}{c^2}}
    \approx v(t) + \left( 1 - \frac{v(t)^2}{c^2} \right) {\rm d} v +
    ... \\\
    & = & v(t) + g \left( 1 - \frac{v(t)^2}{c^2} \right)^{3/2} {\rm d} t + ...  \, .
\end{eqnarray}
Daraus erhalten wir unmittelbar Gl.\ \ref{eq_Diffkonst}.

Die zweite Herleitung der Differentialgleichung
geht von der allgemeinen Beziehung f\"ur das
Transformationsgesetz der Beschleunigung zwischen
zwei Bezugssystemen aus. Wenn die Beschleunigung
in dieselbe Richtung wie die Transformation erfolgt,
gilt:
\begin{equation}
        a' = \gamma^3 a  \, .
\end{equation} 
Mit $a'=g$ und $a=\frac{{\rm d}v}{{\rm d}t}$ folgt obige
Differentialgleichung.

\subsection{Bestimmung der Bahnkurve}
\label{sec_Konst}

Wir k\"onnen die Differentialgleichung \ref{eq_Diffkonst}
beispielsweise durch Trennung der Variablen
l\"osen:
\begin{equation}
     \frac{{\rm d}v}{\left( 1 - \frac{v^2}{c^2} \right)^{3/2}} = g\, {\rm d}t \, .
\end{equation}
Da
\begin{equation}
           \frac{\rm d}{{\rm d} v} \left( \frac{v}{\sqrt{1 - \frac{v^2}{c^2}}} \right)
           =      \frac{1}{\left( 1 - \frac{v^2}{c^2} \right)^{3/2}} 
\end{equation}
und wir f\"ur $t_0=0$ die Anfangsbedingung $v=0$ gesetzt
haben, erhalten wir
\begin{equation}
\label{eq_Diff2}
         \frac{v(t)}{\sqrt{1-\frac{v(t)^2}{c^2}}} = g t \, .
\end{equation}
Diese Gleichung k\"onnen wir nach $v(t)$ aufl\"osen und
finden:
\begin{equation}
\label{eq_relGesch}
          v(t) = \frac{gt}{\sqrt{1 + \frac{(gt)^2}{c^2}}}  \, .
\end{equation}
F\"ur $t \ll c/g$ nimmt $v(t)$ offensichtlich linear mit $t$
zu, wie es in der nicht-relativistischen Mechanik f\"ur
die konstante Beschleunigung gelten muss, f\"ur
$t \gg c/g$ n\"ahert sich $v(t)$ der Lichtgeschwindigkeit
$c$ als der Grenzgeschwindigkeit.\footnote{F\"ur die
Erdbeschleunigung $g=9,81\,{\rm m/s}^2$ entspricht
die Zeitskala $c/g\approx 354$ Tage bzw.\ fast einem Jahr.
Ab dieser zeitlichen Gr\"o\ss enordnung lohnt sich eine 
Weltraumreise mit konstanter Beschleunigung.}

Wir k\"onnen diese Gleichung nochmals integrieren
und erhalten die Trajektorie $x(t)$:
\begin{eqnarray}
     x(t) &=&  \int_0^t \frac{gt'}{\sqrt{1 + \frac{(gt')^2}{c^2}}} {\rm d}t' ~=~  
     \left. \frac{c^2}{g} \sqrt{1+\frac{(g t')^2}{c^2}} ~ \right|_0^t  \\
     &=&
\label{eq_konstBeschl}
     \frac{c^2}{g} \left( \sqrt{1+\frac{(gt)^2}{c^2}} -1 \right) \, .
\end{eqnarray}
Im n\"achsten Abschnitt werden wir die L\"osungen
f\"ur $v(t)$ und $x(t)$ etwas genauer untersuchen.
Zum Abschluss dieses Abschnitts m\"ochte ich
noch eine kurze Anmerkung zu der relativistischen
Bewegungsgleichung machen:

Multiplizieren wir Gleichung \ref{eq_Diff2} auf beiden
Seiten mit der Konstanten $m$ (der Ruhemasse des Teilchens)
und bilden die
Ableitung nach $t$, so erhalten wir:
\begin{equation}
      \frac{\rm d}{{\rm d}t} \left( \frac{m v(t)}{\sqrt{1-\frac{v(t)^2}{c^2}}}
      \right)  = mg  \, ,
\end{equation}
was wir mit dem relativistischen Impuls 
\begin{equation}
         p = \frac{mv}{\sqrt{1 - \frac{v^2}{c^2}}} 
\end{equation}
auch in der Form
\begin{equation}
          \frac{{\rm d}p}{{\rm d}t} = mg = F 
\end{equation} 
schreiben k\"onnen. Dies ist die relativistische
Bewegungsgleichung (auch f\"ur eine allgemeine Kraft $F$)
und aus ihr h\"atten wir die Bewegungsgleichung
f\"ur die konstante Beschleunigung durch Umkehrung der
obigen Schritte sofort ableiten k\"onnen. Es mag allerdings
zun\"achst \"uberraschen, dass auf der linken Seite
der Gleichung die Ableitung nach $t$ und nicht nach
der Eigenzeit $\tau$ im beschleunigten System steht.
Es sieht daher zun\"achst so aus, als ob diese Gleichung nicht
invariant sei. Doch die relativistische Kraft ist eigentlich
nicht $F$ sondern $\gamma F$ und die invariante
Gleichung lautet
\begin{equation}
          \frac{{\rm d}p}{{\rm d}\tau} = \gamma F  \, .
\end{equation}
Da
\begin{equation} 
      \frac{{\rm d}p}{{\rm d}\tau} =    \frac{{\rm d}p}{{\rm d}t}
      \frac{{\rm d}t}{{\rm d}\tau} =    \frac{{\rm d}p}{{\rm d}t} \gamma \, ,
\end{equation}
hebt sich der $\gamma$-Faktor auf beiden Seiten weg.

\subsection{Erste Analyse der konstant
beschleunigten Bewegung}
  
Wir untersuchen zun\"achst die Trajektorie aus der
Sichtweise des ruhenden Beobachters (im Ruhesystem
der Ausgangslage des beschleunigten Systems).
Anschlie\ss end betrachten wir die Situation aus
der Sichtweise eines Beobachters in dem 
konstant beschleunigten System (beispielsweise
einer konstant beschleunigten Rakete).    
  
Die Trajektorie der konstanten
relativistischen Beschleunigung im Inertialsys\-tem
der anf\"anglichen Ruhelage beschreibt einen 
Hyperbelast (siehe Abb.\ \ref{fig_KonstBeschl}).
Dies sieht man leicht, wenn man Gl.\ \ref{eq_konstBeschl}  
in folgende Form bringt:
\begin{equation}
   \left(  x(t) + \frac{c^2}{g} \right)^2 - \frac{(gt)^2}{c^2} = 1 \, . 
\end{equation}
    
\begin{SCfigure}[50][htb]
\setlength{\unitlength}{2.0pt}
\begin{picture}(110,90)(20,0)
\put(20,15){\vector(1,0){80}}
\put(73.5,55){\vector(0,1){20}}
\put(24,7){\line(2,1){90}}
\put(73.5,15){\makebox(0,0){{\footnotesize $\bullet$}}}
\put(76.5,33){\makebox(0,0){{\footnotesize $\bullet$}}}
\put(40,15){\makebox(0,0){{\footnotesize $\bullet$}}}
\put(80,12){\makebox(0,0){${\scriptstyle x_0=0}$}}
\put(79,32){\makebox(0,0){$A$}}
\put(43,10){\makebox(0,0){$-\frac{c^2}{g}$}}
\put(100,12){\makebox(0,0){$x$}}
\put(100,42){\makebox(0,0){$x'$}}
\put(70,70){\makebox(0,0){$t$}}
\put(38,19){\makebox(0,0){$O$}}
\put(90,55){\makebox(0,0){$1$}}
\put(70,60){\makebox(0,0){$2$}}
\thicklines
\qbezier(73.5,15)(73.5,47)(107.5,81)
\put(73.2,0){\line(0,1){70}}
\put(73.5,0){\line(0,1){70}}
\put(25,0){\line(1,1){90}}
\end{picture}
\caption{\label{fig_KonstBeschl}%
Die konstante Beschleunigung. Die
Trajektorie eines konstant beschleunigten
Massepunktes (1) beschreibt im Inertialsystem
eines ruhenden Beobachters (2) eine
Hyperbel. Bei dem Ereignis $O$ mit den
Koordinaten $x=-\frac{c^2}{g}$ 
und $t_0=0$ schneiden
sich alle Gleichzeitigkeitslinien der
Trajektorie einschlie\ss lich der Weltlinie
des Lichtstrahls, dem sich die Trajektorie 
asymptotisch n\"ahert.}
\end{SCfigure}

F\"ur kleine Werte von $t$ (genauer $t \ll \frac{c}{g}$)
beschreibt die Trajektorie die zu erwartende Parabel
der Newton'schen Mechanik. Dazu entwickeln wir
die L\"osung nach kleinen Werten von $(tg)/c$:
\begin{equation}
         x(t) \approx \frac{c^2}{g} \left(  1 + \frac{1}{2}
          \frac{g^2 t^2}{c^2} - \frac{1}{8} \frac{g^4t^4}{c^4} + ...
          - 1 \right)  = \frac{1}{2} g t^2 + ...         
\end{equation} 
F\"ur sehr gro\ss e Werte von $t$ n\"ahert sich
die Trajektorie immer mehr dem Lichtstrahl
$x(t) = c t $. Dieses Verhalten zeigt sich auch in
der Geschwindigkeit (Gl.\ \ref{eq_relGesch}), die
f\"ur kleine Werte von $t$ linear zunimmt,
$v(t) \approx gt +...$, und sich f\"ur gro\ss e
Werte von $t$ der Lichtgeschwindigkeit n\"ahert.

Interessant ist, dass sich alle 
Gleichzeitigkeitslinien zu der Trajektorie
in einem Ereignis $O$ bei $t=0$ und $x=-\frac{c^2}{g}$
schneiden (in Abb.\ \ref{fig_KonstBeschl} ist die
Gleichzeitigkeitslinie zum Ereignis $A$ angegeben).
Dies ist gleichzeitig das Ereignis, bei dem
ein ausgesandter Lichtstrahl die Asymptote
der Hyperbel bildet. Der Abstand, gemessen in 
einem augenblicklichen Inertialsystem, 
zwischen einem Punkt auf der Hyperbelbahn 
(z.B.\ dem Ereignis $A$)
und diesem Ereignis $O$ bleibt konstant. 

Wir betrachten nun dieselbe Situation, allerdings
aus der Sichtweise eines
Beobachters in dem konstant beschleunigten
System (man denke beispielsweise an eine
Rakete, die konstant beschleunigt wird).
Zun\"achst m\"ussen wir die Zeit $t$
in die Eigenzeit $\tau$ des beschleunigten Beobachters
umrechnen. Dazu verwenden wir die
allgemeine Beziehung
\begin{equation} 
   {\rm d} \tau = \sqrt{1-\frac{v(t)^2}{c^2}} \, {\rm d}t
\end{equation}
und nutzen nun aus, dass
\begin{equation} 
   \sqrt{1-\frac{v(t)^2}{c^2}} = \frac{1}{\sqrt{1+\frac{(gt)^2}{c^2}}} \, .
\end{equation}
Diese Beziehung folgt unmittelbar aus den
beiden Gleichungen \ref{eq_Diff2} und \ref{eq_relGesch}.
Wir erhalten somit
\begin{equation}
   \tau = \int_0^t \frac{1}{\sqrt{1+\frac{(gt')^2}{c^2}}} {\rm d}t'
    = \frac{c}{g} \sinh^{-1} \frac{gt}{c} 
\end{equation}
oder, aufgel\"ost nach $t$:
\begin{equation}
      t = \frac{c}{g} \sinh \frac{g\tau}{c} \, .
\end{equation}
Zwischen der verstrichenen Zeit $t$ eines Beobachters
im anf\"anglichen Ruhesystem (beispielsweise auf
der Erde) und der Zeit $\tau$ f\"ur einen Beobachter in dem konstant
beschleunigten System besteht also f\"ur gro\ss e
Zeiten eine exponentielle Beziehung. F\"ur die
relativ zum anf\"anglichen Ruhesystem zur\"uckgelegte
Strecke als Funktion der Eigenzeit einer Person in
 dem beschleunigten System erhalten wir
 \begin{equation}
       x(\tau) = \frac{c^2}{g} \left( \cosh \frac{g \tau}{c} - 1 \right) \, .
 \end{equation}
Diese Beziehung scheint zun\"achst den
M\"oglichkeiten eines bemannten Raumflugs
sehr entgegen zu kommen: F\"ur die Reise zum 
rund 2 Millionen Lichtjahre entfernten Andro\-meda-Nebel
(der n\"achsten gro\ss en Galaxie au\ss erhalb der
Milchstra\ss e) w\"urden bei einer konstanten
Beschleunigung von $g=9,81\,{\rm m/s}^2$ (der 
Erdbeschleunigung) f\"ur einen Astronauten
in seinem Bezugssystem nur rund 28 Jahre
vergehen.\footnote{Auf der Internetseite
von John Baez \cite{Baez}
findet man eine sehr sch\"one Beschreibung der
seltsamen Effekte einer konstant beschleunigten
relativistischen Rakete.} 

Dies widerspricht nicht der Relativit\"atstheorie:
Die 2 Millionen Lichtjahre erfahren f\"ur den
Beobachter in der Rakete eine (Lorentz-)Kontraktion auf
weniger als 28 Lichtjahre. Das bedeutet aber, dieser
Beobachter \glqq sieht\grqq\ den Andromeda-Nebel
mit \"Uberlichtge\-schwindigkeit auf sich zukommen.

Einen \"ahnlich erstaunlichen Effekt sieht
der Beobachter auch, wenn er zur\"uck\-blickt.
Das Ereignis $O$ bei $(t=0,x=-\frac{c^2}{g})$ bleibt
f\"ur immer auf seiner Gleichzeitigkeitslinie.
Es wird zu einem Augenblick, der nie vergeht.
Auch der Abstand zwischen ihm
und diesem Ereignis bleibt in seinem Bezugssystem 
immer konstant. Allerdings sollte nochmals
betont werden, dass sich diese Effekte auf ein
augenblickliches globales Inertialsystem 
beziehen, und dies l\"asst sich f\"ur einen
beschleunigten Beobachter nicht
operational realisieren.

In Abschnitt \ref{sec_Rindler} kommen wir
nochmals auf den konstant beschleunigten
Beobachter zu sprechen und beschreiben
dort, was der Beobachter wirklich \glqq sieht\grqq.


\section{Zwei konstant beschleunigte Systeme}

Die bekannte Schriftensammlung 
\glqq Speakable and unspeakable in quantum mechanics\grqq\ 
von John Bell (\cite{Bell}, Kapitel 9) enth\"alt auch einen Artikel,
der nichts mit Quantentheorie zu tun hat. Er tr\"agt
den Titel \glqq How to teach special relativity\grqq, und
hier pl\"adiert Bell daf\"ur, Studierende der Physik auch
mit der \glqq alten\grqq\ Version der Speziellen Relativit\"atstheorie
vertraut zu machen, wie sie von Larmor, Lorentz und
Poincar\'{e} vertreten wurde und wie sie in
Kapitel \ref{chap_SpezRel} angedeutet wurde. Er beschreibt
dort eine einfache Situation aus der Speziellen Relativit\"atstheorie,
die seiner Ansicht nach in der \glqq alten\grqq\ Sichtweise
leichter nachvollzogen werden kann als in der Form, in der die
Relativit\"atstheorie heute meist gelehrt wird.

\begin{SCfigure}[50][htb]
\setlength{\unitlength}{2.0pt}
\begin{picture}(125,100)(0,0)
\put(0,15){\vector(1,0){100}}
\put(53.5,55){\vector(0,1){20}}
\put(53.5,15.2){\line(1,0){25}}
\put(53.5,15){\makebox(0,0){{\footnotesize $\bullet$}}}
\put(79,15){\makebox(0,0){{\footnotesize $\bullet$}}}
\put(59,12){\makebox(0,0){${\scriptstyle x_1=0}$}}
\put(84,12){\makebox(0,0){${\scriptstyle x_2=L}$}}
\put(67,19){\makebox(0,0){$L$}}
\put(100,12){\makebox(0,0){$x$}}
\put(50,70){\makebox(0,0){$t$}}
\put(87,76){\makebox(0,0){$1$}}
\put(113,76){\makebox(0,0){$2$}}
\multiput(5,0)(8,8){11}{\line(1,1){5}}
\multiput(30.5,0)(8,8){11}{\line(1,1){5}}
\thicklines
\qbezier(53.5,15)(53.5,47)(87.5,81)
\qbezier(79,15)(79,47)(113,81)
\put(53.5,0){\line(0,1){70}}
\put(53.5,15.2){\line(1,0){25}}
\end{picture}
\caption{\label{fig_Bell}%
Zwei Raketen (1 und 2) erfahren dieselbe konstante
Beschleunigung. Sie seien durch ein Seil miteinander
verbunden, dessen L\"ange gerade dem
anf\"anglichen Abstand $L$ der Raketen entspricht.
Wird das Seil rei\ss en oder nicht?}
\end{SCfigure}

Zwei gleichartige Raketen befinden sich zun\"achst in Ruhe 
und haben einen Abstand $L$ (in diesem Ruhesystem haben
sie die Koordinaten $x_1=0$ und $x_2=L$).
Sie seien durch ein Seil der L\"ange $L$ miteinander
verbunden. Ab dem Zeitpunkt $t=0$ erfahren beide Raketen 
dieselbe konstante Beschleunigung $g$ in Richtung
ihres Abstandsvektors (also die $x$-Richtung). Ihre 
Weltlinien sind somit Hyperbeln, deren Abstand
im anf\"anglichen Ruhesystem (mit den Koordinaten
$(t,x)$) konstant bleibt (vgl.\ Abb.\ \ref{fig_Bell}).  

Bell stellt nun die Frage, ob das Seil zwischen den
beiden Raketen
irgendwann rei\ss t. Ein solches Ereignis ist eine
physikalische Tatsache und h\"angt daher nicht
vom Bezugssystem ab. Anscheinend hat er diese
Frage in den 70er Jahren mehreren Physikern am
CERN gestellt und sehr unterschiedliche Antworten
erhalten (viele scheinen behauptet zu haben,
das Seil rei\ss e nicht). Tats\"achlich rei\ss t das Seil.
Wir betrachten nun diese
Situation aus allen drei Bezugssystemen -- dem
Inertialsystem, in dem die Raketen anf\"anglich
in Ruhe sind, sowie den beiden Bezugssystemen
der Raketen. 

\begin{SCfigure}[50][htb]
\setlength{\unitlength}{2.0pt}
\begin{picture}(105,90)(20,0)
\put(20,15){\vector(1,0){80}}
\put(53.5,55){\vector(0,1){20}}
\put(63,49.5){\line(1,0){25}}
%
\put(63,49.5){\vector(1,2){10}}
\put(88.6,49.5){\vector(1,2){10}}
\put(43,39.5){\line(2,1){40}}
\put(68.6,39.5){\line(2,1){40}}
%
\put(63,49.5){\makebox(0,0){{\footnotesize $\bullet$}}}
\put(53.5,15){\makebox(0,0){{\footnotesize $\bullet$}}}
\put(79,15){\makebox(0,0){{\footnotesize $\bullet$}}}
\put(88.6,49.5){\makebox(0,0){{\footnotesize $\bullet$}}}
\put(60,12){\makebox(0,0){${\scriptstyle x_1=0}$}}
\put(85,12){\makebox(0,0){${\scriptstyle x_2=L}$}}
\put(64,46){\makebox(0,0){$A$}}
\put(90,46){\makebox(0,0){$B$}}
\put(100,12){\makebox(0,0){$x$}}
\put(50,70){\makebox(0,0){$t$}}
\put(87,76){\makebox(0,0){$1$}}
\put(113,76){\makebox(0,0){$2$}}
\multiput(21,16)(8,8){9}{\line(1,1){5}}
\multiput(46.5,16)(8,8){9}{\line(1,1){5}}
\thicklines
\qbezier(53.5,15)(53.5,47)(87.5,81)
\qbezier(79,15)(79,47)(113,81)
\put(53.5,0){\line(0,1){70}}
\end{picture}
\caption{\label{fig_Bell2}%
Die beiden Ereignisse $A$ und $B$ sind
im ruhenden Inertialsystem gleichzeitig. 
Auch die Eigenzeiten der beiden
Raketen sind bei diesen Ereignissen gleich.
Der r\"aumliche Abstand der Ereignisse ist $L$.}
\end{SCfigure}

Wir beginnen unsere Betrachtungen im 
ruhenden Inertialsystem mit den Koordinaten
$(t,x)$. Zwei in diesem System gleichzeitige
Positionen der Raketen (z.B.\ $A$ und $B$) haben
immer noch den r\"aumlichen Abstand $L$ 
(vgl.\ Abb.\ \ref{fig_Bell2}). Trotzdem bewegt
sich das Seil mit einer bestimmten 
Geschwindigkeit relativ zu dem Ruhesystem
(angedeutet durch die Vektorpfeile in
Abb.\ \ref{fig_Bell2}).
Damit sind die Ereignisse $A$ und $B$ 
f\"ur das Seil auch nicht gleichzeitig
(auch die Gleichzeitigkeitslinien zu den
beiden Ereignissen sind in der Abbildung
angedeutet). Da sich das Seil bewegt,
kommt es zu einer Lorentz-Kontraktion und
das Seil wird rei\ss en.

Bell bemerkt, dass damals viele Physiker in der
Lorentz-Kontraktion nur eine scheinbare
Verk\"urzung von Abst\"anden sahen, weil
man eine L\"ange aus verschiedenen Systemen mit
unterschiedlichen Gleichzeitigkeitsvorstellungen
ausmisst. Doch das Rei\ss en des Seils ist
eine Tatsache, kein \glqq Schein\-effekt\grqq. 
Hier, so argumentiert er, gibt die Lorentz'sche
Vorstellung einer tats\"achlichen Verk\"urzung
ein besseres Bild: Die Reichweite der
elektromagnetischen Kr\"afte, die das Seil
zusammenhalten, wird (\"ahnlich wie die
Solitonen bei unserer linearen, gekoppelten
Kette) k\"urzer, doch die Atome m\"ussen,
da sie zwischen den Raketen eingespannt
sind, ihren Abstand behalten. Irgendwann
wird die Reichweite der Kr\"afte so klein, dass
die Atome nicht mehr zusammengehalten werden
k\"onnen und das Seil rei\ss t. 

\begin{figure}[htb]
\setlength{\unitlength}{2.0pt}
\begin{picture}(110,90)(-8,0)
\put(0,15){\vector(1,0){80}}
\put(33.5,55){\vector(0,1){20}}
\put(-4,13){\line(2,1){90}}
\put(33.5,15){\makebox(0,0){{\footnotesize $\bullet$}}}
\put(36.5,33){\makebox(0,0){{\footnotesize $\bullet$}}}
\put(59,15){\makebox(0,0){{\footnotesize $\bullet$}}}
\put(68.4,49){\makebox(0,0){{\footnotesize $\bullet$}}}
\put(0,15){\makebox(0,0){{\footnotesize $\bullet$}}}
\put(40,12){\makebox(0,0){${\scriptstyle x_1=0}$}}
\put(65,12){\makebox(0,0){${\scriptstyle x_2=L}$}}
\put(-3,20){\makebox(0,0){${\scriptstyle -\frac{c^2}{g}}$}}
\put(39,32){\makebox(0,0){$A$}}
\put(70,46){\makebox(0,0){$B$}}
\put(80,12){\makebox(0,0){$x$}}
\put(30,70){\makebox(0,0){$t$}}
\put(67,76){\makebox(0,0){$1$}}
\put(93,76){\makebox(0,0){$2$}}
\multiput(1,16)(8,8){9}{\line(1,1){5}}
\multiput(26.5,16)(8,8){9}{\line(1,1){5}}
\thicklines
\qbezier(33.5,15)(33.5,47)(67.5,81)
\qbezier(59,15)(59,47)(93,81)
\put(33.5,0){\line(0,1){70}}
\end{picture}
%
\begin{picture}(110,90)(-8,0)
\put(0,15){\vector(1,0){80}}
\put(33.5,55){\vector(0,1){20}}
\put(15.3,7){\line(5,4){70}}
\put(33.5,15){\makebox(0,0){{\footnotesize $\bullet$}}}
\put(59,15){\makebox(0,0){{\footnotesize $\bullet$}}}
\put(68.6,49.5){\makebox(0,0){{\footnotesize $\bullet$}}}
\put(34,21.5){\makebox(0,0){{\footnotesize $\bullet$}}}
\put(25,15){\makebox(0,0){{\footnotesize $\bullet$}}}
\put(34.4,24){\makebox(0,0){{\footnotesize $\bullet$}}}
\put(40,12){\makebox(0,0){${\scriptstyle x_1=0}$}}
\put(65,12){\makebox(0,0){${\scriptstyle x_2=L}$}}
\put(20,20){\makebox(0,0){${\scriptstyle L-\frac{c^2}{g}}$}}
\put(36.5,19){\makebox(0,0){$C$}}
\put(70,46){\makebox(0,0){$D$}}
\put(30,27){\makebox(0,0){$E$}}
\put(80,12){\makebox(0,0){$x$}}
\put(30,70){\makebox(0,0){$t$}}
\put(67,76){\makebox(0,0){$1$}}
\put(93,76){\makebox(0,0){$2$}}
\multiput(1,16)(8,8){9}{\line(1,1){5}}
\multiput(26.5,16)(8,8){9}{\line(1,1){5}}
\thicklines
\qbezier(33.5,15)(33.5,47)(67.5,81)
\qbezier(59,15)(59,47)(93,81)
\put(33.5,0){\line(0,1){70}}
\end{picture}
\caption{\label{fig_Bell3}%
Die Gleichzeitigkeitslinien f\"ur die beiden Raketen.
(Links) Ereignis $A$ befindet sich auf der Weltlinie 
von Rakete 1. In dem augenblicklichen Inertialsystem
ist Ereignis $B$ auf der Weltline von Rakete 2 
gleichzeitig zu $A$. Doch bei $B$ hat Rakete 2 bereits eine 
gr\"o\ss ere Geschwindigkeit als Rakete 1 bei Ereignis
$A$, sodass der Abstand von Rakete 2 aus Sicht
von Rakete 1 zunimmt. (Rechts) $D$ ist ein Ereignis
auf der Weltlinie von Rakete 2 und in dem augenblicklichen
Inertialsystem ist Ereignis $C$ auf der Weltlinie von
Rakete 1 gleichzeitig zu $D$. Doch nicht nur ist die Geschwindigkeit
von Rakete 1 bei $C$ sehr viel langsamer als die von
Rakete 2 bei $D$, sodass der Abstand zwischen den
beiden Raketen aus der Sicht von Rakete 2 zunimmt,
sondern Rakete 2 sieht Rakete 1 auch nie zu
dem Ereignis $E$ gelangen. F\"ur Rakete 2 endet
die Weltlinie von Rakete 1 an diesem Punkt.}
\end{figure}

Wir betrachten nun die Situation aus
dem Bezugssystem von Rakete 1, also der
hinteren Rakete bez\"uglich der
Beschleunigungsrichtung (Abb.\ \ref{fig_Bell3}
(links)). Zu einem beliebigen Ereignis $A$ auf
der Weltlinie dieser Rakete kann der Beobachter
in der Rakete zumindest mathematisch seine augenblickliche 
Gleichzeitigkeitslinie konstruieren (er kann sie nicht im Sinne
einer Einstein-Synchronisation operational realisieren): Das sind alle
Ereignisse zu Vektoren, die bez\"uglich der
Minkowski-Metrik senkrecht auf dem augenblicklichen
4-Vektor der Geschwindigkeit bzw.\ der
augenblicklichen Tangente an die Weltlinie stehen. 
Danach ist Ereignis $B$ auf der Weltlinie von
Rakete 2 gleichzeitig zu Ereignis $A$ (f\"ur 
Rakete 1). Doch bei $B$ bewegt sich Rakete
2 bereits wesentlich schneller als Rakete 1 bei $A$.
Das bedeutet, f\"ur Rakete 1 nimmt der Abstand
zu Rakete 2 st\"andig zu. Somit wird das Seil auch
irgendwann rei\ss en.

Abschlie\ss end betrachten wir dieselbe Situation
noch aus dem Bezugssystem von Rakete 2
(Abb.\ \ref{fig_Bell3} (rechts)).
$D$ sei ein beliebiges Ereignis auf dieser
Weltlinie, und \"ahnlich wie zuvor konstruiert der
Beobachter in dieser Rakete die Gleichzeitigkeitslinie
zu diesem Ereignis. Er findet, dass in diesem
augenblicklichen Inertialsystem Ereignis $C$
auf der Weltlinie von Rakete 1 gleichzeitig zu $D$
ist. Doch bei $C$ bewegt sich Rakete 1 wesentlich
langsamer als Rakete 2 bei $D$, daher nimmt 
auch in seinem Bezugssystem der Abstand zu
Rakete 1 zu und das Seil wird rei\ss en.

Wir beobachten hier aber noch etwas Weiteres:
Alle Gleichzeitigkeitslinien zur Weltlinie von Rakete
2 liegen (bei $x$-Koordinaten gr\"o\ss er als
$L-\frac{c^2}{g}$) unterhalb des Lichtstrahls,
der zur Asymptote der Bahnkurve von 2 wird. 
Das bedeutet, f\"ur Rakete 2 wird Rakete 1
niemals weiter als bis zu dem Ereignis $E$
auf diesem Lichtstrahl
gelangen, Rakete 1 wird dieses Ereignis {\em aus
der Sichtweise von Rakete 2} noch nicht einmal 
erreichen. 

Im n\"achsten Abschnitt gehen wir auf diesen
letztgenannten Punkt nochmals ein. Jedenfalls
sind sich alle drei Beobachter darin einig,
dass das Seil nach den physikalischen Gesetzen
in ihrem Bezugssystem rei\ss en muss.
Man sollte aber in jedem Fall ber\"ucksichtigen,
dass die Konstruktion einer \glqq augenblicklichen
Gleichzeitigkeitsfl\"ache\grqq\ bei beschleunigten
Weltlinien rein mathematische ist und sich 
physikalisch nicht realisieren l\"asst. Bei einem
\glqq ewigen\grqq\ Inertialsystem ist eine
solche Konstruktion zumindest im Prinzip
operational m\"oglich. 

\section{Das Rindler-Universum}
\label{sec_Rindler}

Wir haben gesehen, wie sich im Rahmen der
Speziellen Relativit\"atstheorie bereits einige
sehr interessante Effekte in der Physik eines konstant 
beschleunigten Beobachters untersuchen lassen. 
Dabei haben wir allerdings von globalen 
Gleichzeitigkeitslinien Gebrauch gemacht,
die f\"ur einen Beobachter entlang einer
Weltlinie nicht unbedingt von Relevanz sind.
Beispielsweise kann bei beschleunigten
Systemen die Folge solcher Gleichzeitigkeitslinien,
selbst wenn sie entlang der Weltlinie in kausaler 
Reihenfolge konstruiert werden, Ereignisse in 
gro\ss em Abstand (hier definiert 
$d=\frac{c^2}{g}$ die Skala) 
in kausal r\"uckl\"aufiger Reihenfolge 
\"uberstreichen (man betrachte beispielsweise
Ereignisse, die in Abb.\ \ref{fig_KonstBeschl}
links von Ereignis $O$ liegen).
Aus diesem Grunde 
verwendet man auch in der Allgemeinen
Relativit\"atstheorie solche globalen
Gleichzeitigkeitslinien -- in $2+1$ Raumzeit-Dimensionen
sind es nat\"urlich Fl\"achen und in $3+1$
Raumzeit-Dimensionen Volumina -- nur 
selten.

In diesem Abschnitt soll nochmals der 
konstant beschleunigter Beobachter betrachtet
werden, allerdings mit den Methoden, die
wir sp\"ater bei verallgemeinerten Geometrien 
in der Allgemeinen Relativit\"atstheorie 
verwenden werden: (1) durch Angabe der 
kausalen Beziehungen und (2) durch 
die Untersuchung von Signalen, die zwischen
Beobachtern auf verschiedenen Weltlinien
ausgetauscht werden k\"onnen.
Die Raumzeit eines solchen konstant 
beschleunigten Beobachters bezeichnet 
man auch als Rindler-Universum\index{Rindler-Universum}.
Aufgrund des \"Aquivalenzprinzips lassen sich
viele der beobachteten Effekte qualitativ (und in manchen 
Einzelheiten sogar quantitativ) auf einen Beobachter 
in der N\"ahe eines schwarzen Loches \"ubertragen. 
Von besonderer Bedeutung ist in diesem Zusammenhang 
das Konzept eines \glqq Horizonts\grqq.

\subsection{Kausale Beziehungen}

\begin{SCfigure}[50][ht]
\begin{picture}(270,250)(40,0)
\put(160,20){\vector(0,1){200}}
\put(60,120){\vector(1,0){240}}
\thicklines
\put(195,20){\line(0,1){200}}
\put(195.3,20){\line(0,1){200}}
\put(60,20){\line(1,1){200}}
\put(60,220){\line(1,-1){200}}
\qbezier(270,20)(170,120)(270,220)
\thinlines
\put(155,220){\makebox(0,0){$t$}}
\put(300,115){\makebox(0,0){$x$}}
\put(250,130){\makebox(0,0){{\Large I}}}
\put(170,30){\makebox(0,0){{\Large IV}}}
\put(60,130){\makebox(0,0){{\Large III}}}
\put(175,200){\makebox(0,0){{\Large II}}}
\put(160,128){\makebox(0,0){$O$}}
\put(275,20){\makebox(0,0){$A$}}
\put(190,20){\makebox(0,0){$B$}}
\end{picture}
\caption{\label{figrindler}%
Rindler-Univer\-sum. Dargestellt sind die Weltlinien eines konstant
beschleunigten Beobachters $A$ und eines inertialen Beobachters $B$.
Die Hyperbelbahn von Beobachter $A$ definiert die 
angegebenen Lichtstrahlen sowie das Ereignis $O$.
Die Quadranten I---IV stehen jeweils in einer besonderen
kausalen Beziehung zu Beobachter $A$.}
\end{SCfigure}

Wie wir in Abschnitt \ref{sec_Konst} gezeigt haben, l\"asst sich im 
Raumzeit-Diagramm eines inertialen Beobachters $B$ die 
Weltlinie eines konstant beschleunigten Beobachters $A$ als Hyperbel 
darstellen. In diesem Abschnitt betrachten wir eine
Weltline zu einem System, das seit \glqq ewigen Zeiten\grqq\
einer konstanten Beschleunigung unterlag und auch f\"ur
ewige Zeiten dieser Beschleunigung unterliegen wird 
(vgl.\ Abb.~\ref{figrindler}). Das System kommt also aus dem
Unendlichen auf den inertialen Beobachter $B$ zu und wird
dabei konstant abgebremst bis es schlie\ss lich f\"ur einen
Augenblick relativ zu dem inertialen Beobachter in Ruhe ist und
sich nun mit derselben Beschleunigung wieder von dem
inertialen Beobachter entfernt. 

Der Minkowski-Raum des inertialen Beobachters $B$ 
l\"asst sich durch die kausale Relationen der Ereignisse 
zu dem beschleunigten Beobachter $A$ in vier Klassen 
einteilen:
\begin{itemize}
\item[I]
Dieser Bereich enth\"alt alle Ereignisse, die irgendwann 
einmal in der kausalen Zukunft des Beoachters $A$ lagen 
und gleichzeitig irgendwann einmal in der kausalen 
Vergangenheit von $A$ sein werden. Jedes der Ereignisse
konnte von $A$ einmal beeinflusst werden und kann
umgekehrt einmal einen Einfluss auf $A$ haben.
Dieser Bereich entspricht also im \"ublichen Sinne der 
kausalen Raumzeit f\"ur Beobachter $A$.
\item[II]
Dieser Bereich enth\"alt alle Ereignisse, die in der kausalen Zukunft
von Ereignissen auf der Weltlinie von $A$ liegen, aber nicht 
in der kausalen Vergangenheit irgendeines Ereignisses auf der Weltlinie 
von $A$. Der Beobachter $A$ kann diesen Bereich also nicht 
\glqq einsehen\grqq\ bzw.\ er kann von den Ereignissen in
diesem Bereich nie
kausal beeinflusst werden, er kann aber umgekehrt die 
Ereignisse in diesem Bereich kausal beeinflussen.
\item[III]
Die Ereignisse in diesem Bereich haben keinen kausalen Zusammenhang --
weder in der Zukunft noch in der Vergangenheit --
zu irgendeinem Ereignis auf der Weltlinie von Beobachter $A$. 
\item[IV]
Alle Ereignisse in diesem Bereich liegen irgendwann einmal in der kausalen
Vergangenheit von $A$, waren aber niemals in seiner kausalen Zukunft.
$A$ kann von diesen Ereignissen also kausal beeinflusst
werden, hat aber umgekehrt keinen Einfluss auf sie.
\end{itemize}
Hinsichtlich der Kausalbeziehungen ist f\"ur den
beschleunigten Beobachter $A$ die Ereignismenge in 
Bereich I so, wie f\"ur einen inertialen Beobachter 
die Ereignismenge der
Minkowski-Raum-Zeit: Zu jedem Ereignis gibt es 
auf seiner Weltlinie
einen Zeitpunkt in der Vergangenheit, 
{\em vor} dem dieses Ereignis in seiner kausalen Zukunft
lag, es kann also durch diesen Beobachter
beeinflusst werden. Ebenso gibt es zu jedem Ereignis einen
Zeitpunkt, {\em ab} dem der Beobachter in der kausalen Zukunft des
Ereignisses liegt, d.h.\ dieses Ereignis wahrnehmen bzw.\ von 
ihm Kenntnis erlangen kann.

Alle anderen Bereiche haben f\"ur einen inertialen Beobachter in
einer Minkowski-Raum-Zeit kein Gegenst\"uck. Die Ereignisse in den
Bereichen III und IV liegen beispielsweise niemals in der kausalen
Zukunft von $A$. Der beschleunigte Beobachter hat somit auch keine
M\"oglichkeit, diese Ereignisse jemals zu beeinflussen. 
Allerdings kann er die Ereignisse aus Bereich IV
wahrnehmen bzw.\ kausal von ihnen beeinflusst werden, da
er sich irgendwann in der kausalen Zukunft von diesen 
Ereignissen befinden wird.
Der Bereich III geh\"ort zu einem Teil des Universums, der mit $A$
\"uberhaupt keine kausale Verbindung hat, weder in der Zukunft noch
in der Vergangenheit. In gewisser Hinsicht existiert dieser Bereich f\"ur
den beschleunigten Beobachter $A$ gar nicht. Die Ereignisse von Bereich II
k\"onnen zwar von $A$ beeinflusst werden, allerdings kann $A$ diesen 
Bereich ebenfalls nie einsehen.

\subsection{Was \glqq sehen\grqq\ die Beobachter voneinander?}

\begin{SCfigure}[50][ht]
\begin{picture}(240,260)(20,0)
\put(110,20){\vector(0,1){220}}
\put(50,120){\vector(1,0){200}}
\thicklines
\put(145,20){\line(0,1){230}}
\put(145.3,20){\line(0,1){230}}
\put(50,60){\line(1,1){160}}
\put(50,180){\line(1,-1){160}}
\qbezier(220,18)(120,120)(220,222)
%
\put(225,226.5){\makebox(0,0){{\footnotesize $\bullet$}}}
\put(209,210){\makebox(0,0){{\footnotesize $\bullet$}}}
\put(197.3,195.5){\makebox(0,0){{\footnotesize $\bullet$}}}
\put(189,182.7){\makebox(0,0){{\footnotesize $\bullet$}}}
\put(183,171.3){\makebox(0,0){{\footnotesize $\bullet$}}}
\put(178.3,161.1){\makebox(0,0){{\footnotesize $\bullet$}}}
\put(175,151.8){\makebox(0,0){{\footnotesize $\bullet$}}}
\put(172.6,143.3){\makebox(0,0){{\footnotesize $\bullet$}}}
\put(171,135.2){\makebox(0,0){{\footnotesize $\bullet$}}}
\put(170.2,127.5){\makebox(0,0){{\footnotesize $\bullet$}}}
\put(170,120){\makebox(0,0){{\footnotesize $\bullet$}}}
\put(170.2,112.5){\makebox(0,0){{\footnotesize $\bullet$}}}
\put(171,104.8){\makebox(0,0){{\footnotesize $\bullet$}}}
\put(172.6,96.7){\makebox(0,0){{\footnotesize $\bullet$}}}
\put(175,88.2){\makebox(0,0){{\footnotesize $\bullet$}}}
\put(178.3,78.9){\makebox(0,0){{\footnotesize $\bullet$}}}
\put(183,68.7){\makebox(0,0){{\footnotesize $\bullet$}}}
\put(189,57.3){\makebox(0,0){{\footnotesize $\bullet$}}}
\put(197.3,44.5){\makebox(0,0){{\footnotesize $\bullet$}}}
\put(209,30){\makebox(0,0){{\footnotesize $\bullet$}}}
\put(225,13.5){\makebox(0,0){{\footnotesize $\bullet$}}}
%
\thinlines
\put(225,226.5){\line(-1,1){20}}
\put(209,210){\line(-1,1){40}}
\put(197.3,195.5){\line(-1,1){52.3}}
\put(189,182.7){\line(-1,1){44}}
\put(183,171.3){\line(-1,1){38}}
\put(178.3,161.1){\line(-1,1){33.3}}
\put(175,151.8){\line(-1,1){30}}
\put(172.6,143.3){\line(-1,1){27.6}}
\put(171,135.2){\line(-1,1){26}}
\put(170.2,127.5){\line(-1,1){25.2}}
\put(170,120){\line(-1,1){25}}
\put(170.2,112.5){\line(-1,1){25.2}}
\put(171,104.8){\line(-1,1){26}}
\put(172.6,96.7){\line(-1,1){27.6}}
\put(175,88.2){\line(-1,1){30}}
\put(178.3,78.9){\line(-1,1){33.3}}
\put(183,68.7){\line(-1,1){38}}
\put(189,57.3){\line(-1,1){44}}
\put(197.3,44.5){\line(-1,1){52.3}}
\put(209,30){\line(-1,1){64}}
\put(225,13.5){\line(-1,1){80}}
%
\put(145,91.5){\line(1,-1){90}}
\put(145,89.5){\line(1,-1){90}}
\put(145,88){\line(1,-1){90}}
\put(145,87){\line(1,-1){90}}
\put(145,86){\line(1,-1){90}}
\put(145,85.5){\line(1,-1){90}}
%
\put(105,240){\makebox(0,0){$t$}}
\put(250,115){\makebox(0,0){$x$}}
\put(200,130){\makebox(0,0){{\Large I}}}
\put(120,30){\makebox(0,0){{\Large IV}}}
\put(50,130){\makebox(0,0){{\Large III}}}
\put(125,200){\makebox(0,0){{\Large II}}}
\put(110,128){\makebox(0,0){$O$}}
\put(225,25){\makebox(0,0){$A$}}
\put(139,20){\makebox(0,0){$B$}}
\end{picture}
\caption{\label{fig_Rindler1}%
Was sieht der inertiale Beobachter $B$ von dem
beschleunigten Beobachter $A$? In regelm\"a\ss igen
Eigenzeitabst\"anden (Ereignisse auf der Weltlinie
von $A$, gekennzeichnet durch 
\glqq {\footnotesize $\bullet$}\grqq) 
sendet der beschleunigte Beobachter
Signale aus. $B$ empf\"angt diese Signale in
seinem System in unterschiedlichen
Zeitabst\"anden.}
\end{SCfigure}

Wir \"uberlegen uns zun\"achst, was der inertiale Beobachter
$B$ von dem beschleunigten Beobachter $A$ \glqq sieht\grqq.
In Abbildung \ref{fig_Rindler1} sind beide Weltlinien 
dargestellt, zust\"atzlich sind in gleichm\"a\ss igen 
Eigenzeitabst\"anden von $A$ Ereignisse markiert,
bei denen $A$ ein Lichtsignal zu Beobachter $B$
aussendet.

Solange $B$ sich in Bereich IV befindet, hat er 
keinerlei Kenntnisse von $A$. Erst beim \"Uberscheiten
der Grenze zwischen Bereich IV zu Bereich I 
\glqq erf\"ahrt\grqq\ Beobachter $B$ von $A$. 
Das geschieht allerdings gleich sehr heftig: Innerhalb einer
beliebig kurzen Zeit nimmt Beobachter $B$ eine unendliche Zeitspanne
in der Vergangenheit von Beobachter $A$ wahr. 

Hinsichtlich seiner Wahrnehmung sieht Beobachter $B$
den beschleunigten Beobachter $A$ in einer beliebig
kurzen Zeit eine unendliche Strecke auf ihn zukommen.
Diese scheinbare Wahrnehmung widerspricht nat\"urlich
nicht der Aussage, dass die Lichtegeschwindigkeit  
eine Grenzgeschwindigkeit darstellt. Da 
Beobachter $B$ in beliebig kurzer Zeit eine unendliche
Vergangenheit von $A$ wahrnimmt, sind die eintreffenden
Lichtwellen auch unendlich blau-verschoben. 
In gewisser Hinsicht ist dieses Ereignis f\"ur Beobachter $B$ 
wie eine Singularit\"at.

Solange sich der inertiale Beobachter $B$ im Bereich I befindet, 
kann er mit dem beschleunigten Beobachter $A$ Information austauschen.
F\"ur Beobachter $B$ \"andert sich auch nicht viel,
wenn er in den Bereich II tritt. F\"ur ihn hat die Grenze
zwischen Bereich I und II keinerlei Bedeutung und
er hat an dieser Grenze auch keine besondere
Wahrnehmung. Er kann den beschleunigten
Beobachter $A$ bis in eine beliebige Zukunft
weiter beobachten. Allerdings werden die 
Zeitabst\"ande
zwischen Signalen, die $A$ in gleichen Eigenzeitabst\"anden
aussendet, f\"ur $B$ immer gr\"o\ss er. Der intertiale 
Beobachter $B$ sieht also den beschleunigten Beobachter
$A$ immer st\"arker rot-verschoben. Diese Rotverschiebung
entspricht im Wesentlichen dem Doppler-Effekt eines 
sich zunehmend rasch entfernenden Senders.

\begin{SCfigure}[50][ht]
\begin{picture}(240,260)(35,0)
\put(110,20){\vector(0,1){220}}
\put(50,120){\vector(1,0){200}}
\thicklines
\put(145,10){\line(0,1){230}}
\put(145.3,10){\line(0,1){230}}
\put(50,60){\line(1,1){190}}
\put(50,180){\line(1,-1){160}}
\qbezier(220,18)(120,120)(220,222)
%
\put(225,226.5){\makebox(0,0){{\footnotesize $\bullet$}}}
\put(209,210){\makebox(0,0){{\footnotesize $\bullet$}}}
\put(197.3,195.5){\makebox(0,0){{\footnotesize $\bullet$}}}
\put(189,182.7){\makebox(0,0){{\footnotesize $\bullet$}}}
\put(183,171.3){\makebox(0,0){{\footnotesize $\bullet$}}}
\put(178.3,161.1){\makebox(0,0){{\footnotesize $\bullet$}}}
\put(175,151.8){\makebox(0,0){{\footnotesize $\bullet$}}}
\put(172.6,143.3){\makebox(0,0){{\footnotesize $\bullet$}}}
\put(171,135.2){\makebox(0,0){{\footnotesize $\bullet$}}}
\put(170.2,127.5){\makebox(0,0){{\footnotesize $\bullet$}}}
\put(170,120){\makebox(0,0){{\footnotesize $\bullet$}}}
\put(170.2,112.5){\makebox(0,0){{\footnotesize $\bullet$}}}
\put(171,104.8){\makebox(0,0){{\footnotesize $\bullet$}}}
\put(172.6,96.7){\makebox(0,0){{\footnotesize $\bullet$}}}
\put(175,88.2){\makebox(0,0){{\footnotesize $\bullet$}}}
\put(178.3,78.9){\makebox(0,0){{\footnotesize $\bullet$}}}
\put(183,68.7){\makebox(0,0){{\footnotesize $\bullet$}}}
\put(189,57.3){\makebox(0,0){{\footnotesize $\bullet$}}}
\put(197.3,44.5){\makebox(0,0){{\footnotesize $\bullet$}}}
\put(209,30){\makebox(0,0){{\footnotesize $\bullet$}}}
\put(225,13.5){\makebox(0,0){{\footnotesize $\bullet$}}}
%
\put(145,155){\makebox(0,0){{\footnotesize $\bullet$}}}
%
\thinlines
\put(190,10){\line(1,1){30}}
\put(182.5,10){\line(1,1){40}}
\put(175,10){\line(1,1){50}}
\put(167.5,10){\line(1,1){60}}
\put(160,10){\line(1,1){70}}
\put(152.5,10){\line(1,1){80}}
\put(145,10){\line(1,1){90}}
\put(145,17.5){\line(1,1){90}}
\put(145,25){\line(1,1){90}}
\put(145,32.5){\line(1,1){90}}
\put(145,40){\line(1,1){90}}
\put(145,47.5){\line(1,1){90}}
\put(145,55){\line(1,1){90}}
\put(145,62.5){\line(1,1){90}}
\put(145,70){\line(1,1){90}}
\put(145,77.5){\line(1,1){90}}
\put(145,85){\line(1,1){90}}
\put(145,92.5){\line(1,1){90}}
\put(145,100){\line(1,1){90}}
\put(145,107.5){\line(1,1){90}}
\put(145,115){\line(1,1){90}}
\put(145,122.5){\line(1,1){90}}
\put(145,130){\line(1,1){90}}
\put(145,137.5){\line(1,1){90}}
\put(145,145){\line(1,1){90}}
\put(145,152.5){\line(1,1){90}}
\put(145,160){\line(1,1){80}}
\put(145,167.5){\line(1,1){70}}
\put(145,175){\line(1,1){60}}
\put(145,182.5){\line(1,1){50}}
%
\put(105,240){\makebox(0,0){$t$}}
\put(250,115){\makebox(0,0){$x$}}
\put(200,130){\makebox(0,0){{\Large I}}}
\put(120,30){\makebox(0,0){{\Large IV}}}
\put(50,130){\makebox(0,0){{\Large III}}}
\put(125,200){\makebox(0,0){{\Large II}}}
\put(110,128){\makebox(0,0){$O$}}
\put(137,157){\makebox(0,0){$E$}}
\put(225,25){\makebox(0,0){$A$}}
\put(139,20){\makebox(0,0){$B$}}
\end{picture}
\caption{\label{fig_Rindler2}%
Was sieht der beschleunigte Beobachter $A$ von dem
inertialen Beobachter $B$? Nun sendet $B$
in regelm\"a\ss igen Eigenzeitabst\"anden 
Signale aus, die $A$ empf\"angt. Wegen der
unterschiedlichen relativen Geschwindigkeit
sowie zus\"atzlich den unterschiedlichen Skalen
f\"ur die Eigenzeiten von $A$ relativ zu $B$ 
(Ereignisse in gleichen Eigenzeitabst\"anden
wurden auf der Weltlinie
von $A$ wieder gekennzeichnet) 
empf\"angt auch $A$ die Signale in 
unterschiedlichen Zeitabst\"anden.}
\end{SCfigure}

Nun untersuchen wir, was der beschleunigte
Beobachter $A$ von dem inertialen Beobachter
$B$ sieht. In Abb.\ \ref{fig_Rindler2} sind wieder
die beiden Weltlinien dargestellt, diesmal sendet
aber $B$ in regelm\"a\ss igen Abst\"anden
Lichtisgnale aus. Auf der Weltlinie von $A$
sind in \"aquidistanten Eigenzeitabst\"anden
Ereignispunkte eingetragen -- der Eigenzeitabstand
entspricht dem zeitlichen Abstand, mit dem $B$
die Lichtsignale abschickt.

In beliebig ferner Vergangenheit kann $A$ den
Beobachter $B$ schon wahrnehmen, allerdings
treffen die Lichtsignale bez\"uglich seiner
Eigenzeit sehr viel rascher bei ihm ein, sodass
er Beobachter $B$ blau-verschoben wahrnimmt,
wiederum wie bei einem Doppler-Effekt.
Je weiter man in die Vergangenheit von $A$
zur\"uckgeht, umso blauverschobener sieht
$A$ den intertialen Beobachter $B$. 
Sobald $B$ auch in den Bereich $I$ eingedrungen
ist, k\"onnen $A$ und $B$ Signale austauschen
und sich gegenseitig verst\"andigen. 

Wenn die beiden Weltlinien von $A$ und $B$
f\"ur einen Augenblick parallel sind, sieht $A$
die Signale von $B$ mit derselben Frequenz,
wie $B$ sie aussendet. Ab dann empf\"angt
$A$ die Signale seltener, d.h.\ $A$ sieht
$B$ rotverschoben.

Offenbar kann $A$ keinerlei Signale mehr
aus dem Bereich II empfangen. Das
bedeutet, kein Signal, das $B$ nach dem
Ereignis $E$ verschickt, wird $A$ jemals
erreichen. F\"ur $A$ endet die Weltlinie
von $B$ an dem Ereignis $E$. Der Lichtstrahl,
der die Bereiche I und II trennt, ist f\"ur
Beobachter $A$ ein Horizont, hinter den
er nicht blicken kann. Man bezeichnet diesen
Horizont auch als {\em Ereignishorizont}.

Der Beobachter $A$ sieht den Beobachter $B$ aber nicht einfach hinter
dem Horizont verschwinden. Im Gegenteil: Er kann f\"ur alle Zukunft den
Beobachter $B$ wahrnehmen, wie er sich immer mehr dem Horizont 
bzw.\ dem Ereignis $E$ n\"ahert.
Die Abst\"ande, mit denen $A$ aber von $B$ die in gleichen 
Zeitabst\"anden ausgesandten 
Signale erh\"alt, werden immer gr\"o\ss er.
$A$ nimmt die Zeit im $B$-System immer langsamer wahr.
Damit ist eine Rotverschiebung der Strahlung
verbunden. $B$ verschwindet
also nicht hinter dem Horizont, sondern $B$ verschwindet an der 
Oberfl\"ache des Horizonts im langwelligen Bereich des Spektrums. 

Die Grenzen zwischen den Bereichen I und IV einerseits und den
Bereichen I und II andererseits verhalten sich also in gewisser
Hinsicht symmetrisch: $A$ kann die Ereignisse in Bereich IV wahrnehmen,
nicht aber die Ereignisse in Bereich II. Umgekehrt hat $B$ keine
Kenntnis von $A$, solange er sich in Bereich IV befindet, er nimmt
das Schicksal von $A$ aber durchaus wahr, wenn er sich in Bereich II
befindet.

Der Bereich III geh\"ort zu einem Teil des Universums, der mit $A$
\"uberhaupt keine kausale Verbindung hat, weder in der Zukunft noch
in der Vergangenheit. In gewisser Hinsicht existiert dieser Bereich f\"ur
den beschleunigten Beobachter $A$ gar nicht. Trotzdem ist dieser Bereich 
f\"ur den inertialen Beobachter $B$ ein ganz normaler Teil seines
Universums. Da andererseits $B$ von diesem Bereich auch erst erf\"ahrt,
nachdem er den Horizont zwischen I und II durchschritten hat, kann er 
$A$ keine Mitteilung davon machen.

\begin{thebibliography}{99}
\addcontentsline{toc}{chapter}{Literaturangaben}
\bibitem{Baez} Baez, John; \url{math.ucr.edu/home/baez}, und speziell f\"ur Physik 
                 \url{math.ucr.edu/home/baez/physics}
\bibitem{Bell} John Bell;  {\em Speakable and Unspeakable in 
        Quantum Physics}, 2.\ edition, Cambridge University Press (2004).       
\end{thebibliography}

\end{document}
