\documentclass[german,10pt]{book}     
\usepackage{makeidx}
\usepackage{babel}            % Sprachunterstuetzung
\usepackage{amsmath}          % AMS "Grundpaket"
\usepackage{amssymb,amsfonts,amsthm,amscd} 
\usepackage{mathrsfs}
\usepackage{rotating}
\usepackage{sidecap}
\usepackage{graphicx}
\usepackage{color}
\usepackage{fancybox}
\usepackage{tikz}
\usetikzlibrary{arrows,snakes,backgrounds}
\usepackage{hyperref}
\hypersetup{colorlinks=true,
                    linkcolor=blue,
                    filecolor=magenta,
                    urlcolor=cyan,
                    pdftitle={Overleaf Example},
                    pdfpagemode=FullScreen,}
%\newcommand{\hyperref}[1]{\ref{#1}}
%
\definecolor{Gray}{gray}{0.80}
\DeclareMathSymbol{,}{\mathord}{letters}{"3B}
%
\newcounter{num}
\renewcommand{\thenum}{\arabic{num}}
\newenvironment{anmerkungen}
   {\begin{list}{(\thenum)}{%
   \usecounter{num}%
   \leftmargin0pt
   \itemindent5pt
   \topsep0pt
   \labelwidth0pt}%
   }{\end{list}}
%
\renewcommand{\arraystretch}{1.15}                % in Formeln und Tabellen   
\renewcommand{\baselinestretch}{1.15}                 % 1.15 facher
                                                      % Zeilenabst.
\newcommand{\Anmerkung}[1]{{\begin{footnotesize}#1 \end{footnotesize}}\\[0.2cm]}
\newcommand{\comment}[1]{}
\setlength{\parindent}{0em}           % Nicht einruecken am Anfang der Zeile 

\setlength{\textwidth}{15.4cm}
\setlength{\textheight}{23.0cm}
\setlength{\oddsidemargin}{1.0mm} 
\setlength{\evensidemargin}{-6.5mm}
\setlength{\topmargin}{-10mm} 
\setlength{\headheight}{0mm}
\newcommand{\identity}{{\bf 1}}
%
\newcommand{\vs}{\vspace{0.3cm}}
\newcommand{\noi}{\noindent}
\newcommand{\leer}{}

\newcommand{\engl}[1]{[\textit{#1}]}
\parindent 1.2cm
\sloppy

             \begin{document}     \setcounter{chapter}{2}

\chapter{Die drei klassischen Effekte der ART}

Nachdem wir die mathematischen Grundlagen 
behandelt und die Einstein'schen Gleichungen
angegeben haben, betrachten wir nun erste
Anwendungen bzw.\ Konsequenzen der 
Allgemeinen Relativit\"atstheorie. 

Sehr viele Effekte der Allgemeinen Relativit\"atstheorie --
dazu geh\"oren alle Effekte, die sich innerhalb
unseres Sonnensystems beobachten lassen -- 
erh\"alt man im Rahmen einer st\"orungstheoretischen
Behandlung um die flache Minkowski-Raumzeit
bzw.\ um die Newton'sche Theorie. Das GPS --
Global Positioning System -- ber\"ucksichtigt
gleich mehrere Ordnungen in einer solchen
St\"orungstheorie. Ohne die Ber\"ucksichtigung
dieser Einfl\"usse w\"are eine Positionierung,
wie sie heute m\"oglich ist (im Bereich von
wenigen Metern, unter bestimmten Umst\"anden
sogar wenigen Zentimetern), nicht m\"oglich.

Wir k\"onnen im Rahmen dieses \"Uberblicks
auf die meisten dieser Effekte leider nicht eingehen, 
sondern beschr\"anken uns hier auf die drei 
\glqq klassischen Effekte\grqq\ der ART. Diese
beziehen sich auf die Ablenkung von Licht im 
Gravitationsfeld, die Rotverschiebung von Licht, das
ein Gravitationsfeld verl\"asst, und die
Erkl\"arung der Periheldrehung des Merkur.  
Ebenfalls kurz ansprechen werde ich die Beschreibung 
von Gravitationswellen, den Newton'schen
Grenzfall sowie das Verhalten
von Drehimpuls bzw.\ Spin im Gravitationsfeld.

\section{Die Lichtablenkung im Gravitationsfeld}
\label{sec_Lichtablenkung}

Der erste wichtige Test der Relativit\"atstheorie
bezog sich auf die\index{Lichtablenkung an der Sonne} 
Lichtablenkung an der Sonne.
Die Einstein'sche Relativit\"atstheorie sagt
vorher, dass ein Lichtstrahl von einem Stern hinter 
der Sonne in der gekr\"ummten Raumzeit
in der N\"ahe der Sonnenoberfl\"ache abgelenkt
werden muss. Diese Sprechweise bezieht sich
auf den Vergleich zwischen der Situation, wo
die Sonne nahezu zwischen Erde und dem
betreffenden Stern steht, und der Situation,
wo diese Verbindungslinie frei von gravitativen
Einfl\"ussen ist. Eine tats\"achliche \glqq Ablenkung\grqq\
des Lichtstrahls an der Sonne findet nicht statt,
denn der Lichtstrahl folgt immer
einer geod\"atischen Linie.

Wir betrachten zun\"achst die Lichtablenkung
in einem konstanten Gravitationsfeld. Nach
dem \"Aquivalenzprinzip\index{Aequivalenzprinzip@\"Aquivalenzprinzip} 
sollte die Physik in
diesem Fall nicht unterscheidbar von der
Physik in einem konstant beschleunigten
Bezugssystem sein. Dazu stellen wir uns zwei
Systeme vor (vgl.\ Abb.\ \ref{fig_aufzug}): das eine
System werde konstant beschleunigt (Beschleunigung
$a$), das zweite System befinde sich in einem
Gravitationsfeld der St\"arke $g=a$.  

\begin{SCfigure}[50][htb]
\begin{picture}(110,110)(-10,0)
\qbezier(30,60)(60,60)(90,50)
\put(30,20){\line(1,0){60}}
\put(30,20){\line(0,1){80}}
\put(30,100){\line(1,0){60}}
\put(90,20){\line(0,1){80}}
\put(15,40){\vector(0,1){40}}
\put(20,60){\makebox(0,0){$a$}}
\put(60,12){\makebox(0,0){$L$}}
\put(95,55){\makebox(0,0){$s$}}
\multiput(33,60)(3,0){20}{\makebox(0,0){${\scriptstyle \cdot}$}}
\end{picture}
%
\begin{picture}(100,110)(0,0)
\qbezier(30,60)(60,60)(90,50)
\put(30,20){\line(1,0){60}}
\put(30,20){\line(0,1){80}}
\put(30,100){\line(1,0){60}}
\put(90,20){\line(0,1){80}}
\put(15,80){\vector(0,-1){40}}
\put(20,60){\makebox(0,0){$g$}}
\put(60,12){\makebox(0,0){$L$}}
\put(95,55){\makebox(0,0){$s$}}
\multiput(33,60)(3,0){20}{\makebox(0,0){${\scriptstyle \cdot}$}}
\end{picture}
\caption{\label{fig_aufzug}%
Nach dem \"Aquivalenz\-prinzip sollte
die Lichtablenkung $s$ in einem Labor, das
mit der Beschleunigung $a$ nach oben
beschleunigt wird (links) dieselbe sein
wie die in einem Labor, das sich in einem
konstanten Gravitationsfeld der St\"arke
$g=a$ befindet (rechts).}
\end{SCfigure}

N\"aherungsweise ben\"otigt ein Lichtstrahl
zum Durchlaufen des Labors (L\"ange $L$)
die Zeit $t=L/c$. In dieser Zeit hat sich
das Laborsystem aufgrund der Beschleunigung
$a$ um die Strecke $s=at^2/2$ weiterbewegt.
Um dieselbe Strecke sollte daher ein
Lichtstrahl in dem Laborsystem, das in
einem Gravitationsfeld mit $g=a$ ruht, abgelenkt
werden.

Wir betrachten nun die Ablenkung des
Lichts an einem Stern wie der Sonne.
Wenn das \"Aquivalenzprinzip gilt,
sollte der Winkel der Ablenkung nur von
der Anfangsgeschwindigkeit und dem
Sto\ss parameter abh\"angen, nicht aber von
der Masse eines Teilchens. F\"ur sehr
kleine Ablenkungswinkel (bei denen alle
Korrekturterme h\"oherer 
Ordnung\index{Streuung am Kepler-Potenzial}
vernachl\"assigt werden) ergibt sich in der klassischen
Mechanik f\"ur die Streuung am Kepler-Potenzial:
\begin{equation}
     \Delta \phi = 2 \frac{GM}{v^2 r_0} \, ,
\end{equation}
wobei $r_0$ der Minimalabstand der Bahnkurve
vom Sonnenmittelpunkt ist (vgl.\ Abb.\ \ref{fig_sonne}),
was bei den Beobachtungen w\"ahrend einer
Sonnenfinsternis dem Sonnenradius entspricht.

Wenn wir (etwas naiv, da nicht-relativistisch
gerechnet) f\"ur die Geschwindigkeit
$c$ einsetzen, erhalten wir:
\begin{equation}
     \Delta \phi = 2 \frac{GM}{c^2 r_0} \, ,
\end{equation}
Dies war das Ergebnis, das Einstein zun\"achst
vermutet hatte.

\begin{SCfigure}[50][htb]
\begin{picture}(240,90)(-20,0)
\qbezier(65,25)(75,40)(95,40)
\qbezier(95,40)(115,40)(125,25)
\qbezier(0,40)(95,60)(190,40)
\put(95,10){\line(0,1){40}}
\put(194,40){\circle{10}}
\put(83,30){\makebox(0,0){$M$}}
\put(194,52){\makebox(0,0){$E$}}
\put(0,32){\makebox(0,0){$S$}}
\put(0,90){\makebox(0,0){$S'$}}
\put(0,40){\makebox(0,0){$\bullet$}}
\put(0,78){\makebox(0,0){$\bullet$}}
\put(125,59){\makebox(0,0){${\scriptstyle \Delta \phi}$}}
\put(101,25){\makebox(0,0){${\scriptstyle r_0}$}}
\multiput(0,78.5)(15,-3){13}{\line(5,-1){10}}
\multiput(0,40)(15,3){13}{\line(5,1){10}}
%
\end{picture}
\caption{\label{fig_sonne}%
(Links) Ein Lichtstrahl eines entfernten Sterns 
$S$ wird an 
der Sonne abgelenkt. F\"ur einen Beobachter auf der
Erde $E$ erscheint der Stern scheinbar an einer
anderen Position $S'$. Als Ablenkungswinkel
bezeichnet man den Winkel $\Delta \phi$.}
\end{SCfigure}

Wie schon erw\"ahnt (S.\ \pageref{p_solareclipse})
wurde der Effekt der Lichtablenkung an der Sonne
1919 gemessen und damit die Vorhersage der
ART best\"atigt. Einstein hatte urspr\"unglich nur
die oben skizzierte Form des \"Aquivalenzprinzips
f\"ur seine Vorhersage verwendet. Eine genauere
Rechnung, die m\"oglich wurde, nachdem 1916 die
Schwarz\-schild-L\"osung bekannt war, ergab diese f\"ur
den Ablenkungswinkel
\begin{equation}
     \Delta \phi =  4 \frac{GM}{c^2 r_0} \frac{1 + \gamma}{2} \, .
\end{equation}
Hierbei ist $\gamma$ einer der Parameter der 
Robertson-Entwicklung\index{Robertson-Entwicklung}
(siehe Abschnitt \ref{sec_IsoStat}) und f\"ur die 
Schwarzschild-L\"osung gilt 
$\gamma = 1$ (siehe Gl.\ \ref{eq_betagammaART}), 
wohingegen in der Newton'schen N\"aherung $\gamma=0$
folgt. Die tats\"achliche Ablenkung eines Lichtstrahls 
an der Sonne f\"allt also im Rahmen der ART doppelt
so stark aus, wie urspr\"unglich vermutet.

\section{Die Rotverschiebung von Licht}

Einer der klassischen Effekte der Relativit\"atstheorie,
der mittlerweile auch mit einer gro\ss en Genauigkeit
auf der Erde gemessen wurde, ist die Rotverschiebung
von Licht, das ein Gravitationsfeld verl\"asst.
Diese Rotverschiebung hat mit dem Gang von
Uhren -- genauer mit der L\"ange von Weltlinien --
 im Gravitationsfeld zu tun. 

\subsection{Uhren im Gravitationsfeld}

Wir haben gesehen, dass wir f\"ur raumartige Abst\"ande die Metrik
durch starre K\"orper ausmessen k\"onnen. F\"ur zeitartige
Wege k\"onnen wir die L\"ange immer durch die Eigenzeit
einer mittransportieren Uhr\index{Eigenzeit!im Gravitationsfeld} ausmessen. 

Grunds\"atzlich gilt in diesem Fall das Gleiche, wie schon bei der 
Geometrisierung des Raumes. Es ist eine Definition, die Eigenzeit und 
damit die zeitartige Metrik
\"uber den Lauf guter Uhren zu bestimmen. Universelle Einfl\"usse sind
solche, die den Lauf aller Uhren -- d.h.\ aller physikalischen Systeme
mit einer charakteristischen Zeitskala --
gleicherma\ss en beeinflussen, sodass 
es sinnvoll ist, die Eigenzeit der Geometrie und nicht einer universellen
Kraft zuzuschreiben. 

\begin{figure}[ht]
\begin{picture}(360,200)(0,0)
\put(30,20){\line(0,1){160}}
\put(5,80){\vector(0,1){40}}
\multiput(30,30)(0,10){16}{\line(-1,-1){10}}
\multiput(60,60)(0,40){3}{\vector(-1,0){20}}
\thicklines
\put(90,20){\line(0,1){160}}
\qbezier(70,20)(200,100)(70,180)
\thinlines
\put(30,10){\makebox(0,0){Erdoberfl\"ache}}
\put(50,105){\makebox(0,0){$g$}}
\put(10,125){\makebox(0,0){$t$}}
\put(85,100){\makebox(0,0){1}}
\put(145,100){\makebox(0,0){2}}
\put(97,32){\makebox(0,0){$A$}}
\put(96,170){\makebox(0,0){$B$}}
\put(130,20){\makebox(0,0){(a)}}
%
\thicklines
\put(290,20){\line(0,1){160}}
\qbezier(310,20)(180,100)(310,180)
\thinlines
\put(260,70){\vector(1,0){20}}
\put(250,100){\vector(1,0){20}}
\put(260,130){\vector(1,0){20}}
\put(260,105){\makebox(0,0){$g$}}
\put(295,100){\makebox(0,0){2}}
\put(235,100){\makebox(0,0){1}}
\put(283,32){\makebox(0,0){$A$}}
\put(283,170){\makebox(0,0){$B$}}
\put(330,20){\makebox(0,0){(b)}}
\end{picture}
\caption{\label{figaeq}%
Das \"Aquivalenzprinzip f\"ur Beobachter im Gravitationsfeld. In Teil
(a) befinden sich beide Beobachter in einem Gravitationsfeld $g$. 
Beobachter 1 h\"alt seinen Abstand zur Erdoberfl\"ache konstant, sp\"urt
also das Feld. Beobachter 2 bewegt sich in einem freien Inertialsystem.
Die Situation ist \"aquivalent zu der Darstellung in Teil (b). Beobachter
2 bewegt sich frei entlang einer Geod\"aten. Beobachter 1 erf\"ahrt eine
konstante Beschleunigung. Die Eigenzeit zwischen Ereignis $A$ und $B$
ist daher f\"ur Beobachter 2 l\"anger.}
\end{figure}

Wir wollen nun unter Ausnutzung des \"Aquivalenzprinzips 
qualitativ herleiten, wie
sich Uhren in einem Gravitationsfeld verhalten. Dazu
vergleichen wir zun\"achst zwei Beobachter 1 und 2 
(siehe Abb.\ref{figaeq}). Beobachter 1
sei in einem konstanten Gravitationsfeld \glqq in Ruhe\grqq; er h\"alt
beispielsweise seinen Abstand zur Quelle des Gravitationsfeldes 
(der Erdoberfl\"ache) unter Ausnutzung einer anderen Kraft 
(beispielsweise der elektromagnetischen Kraft, die in einem 
Raketenantrieb wirksam ist) konstant. Beobachter 2 hingegeben bewegt sich 
auf einer inertialen Bahnkurve, zum Beispiel in einem mit gro\ss er
Geschwindigkeit abgeschossenen und dann frei fallenden Satelliten.
Der Moment der Trennung der beiden Beobachter sei Ereignis $A$, der
Moment des Zusammentreffens sei Ereignis $B$. Beide Beobachter haben
auf ihren Uhren die Zeit zwischen Ereignis $A$ und $B$ gemessen.
Was stellen sie fest?

Dieses Problem l\"asst sich mit Hilfe des \"Aquivalenzprinzips
\index{Aequivalenzprinzip@\"Aquivalenzprinzip}
besonders einfach auf ein Problem der speziellen Relativit\"atstheorie
zur\"uckf\"uhren. Wir k\"onnen n\"amlich auch sagen, dass sich
Beobachter 2 entlang einer Geod\"aten bewegt hat (in seinem System gelten
die physikalischen Gesetze eines Inertialsystems), im Minkowski-Raum
also entlang einer geraden Linie, und Beobachter 1 wurde konstant
beschleunigt. Er entfernt sich zun\"achst bei Ereignis $A$ mit gro\ss er 
Geschwindigkeit von Beobachter 1, aber seine Beschleunigung lie\ss\ 
ihn immer langsamer werden, bis sich seine Geschwindigkeit umkehrte  
und er schlie\ss lich bei Ereignis $B$ wieder mit Beobachter 1 
zusammentraf. Nun wissen wir in der speziellen Relativit\"atstheorie,
dass  die Eigenzeit von Beobachter 2 l\"anger ist als die Eigenzeit 
von Beobachter 1. 
Die Situation entspricht genau dem Beispiel des Zwillingsparadoxons,
wobei 1 der \glqq reisende Astronaut\grqq\ ist und 2 der rascher alternde
Zwilling, der zu Hause zur\"uckbleibt. 

Da die physikalische Situation in beiden F\"allen nach dem 
\"Aquivalenzprinzip die gleiche ist, bedeutet das, dass f\"ur den
Beobachter 1 im Gravitationsfeld die Uhr langsamer geht als f\"ur
den Beobachter 2, der sich scheinbar entlang einer l\"angeren Linie
bewegt, allerdings in einem Inertialsystem.
Die Allgemeinen Relativit\"atstheorie besagt somit, dass Uhren im
Gravitationsfeld langsamer gehen.

Diese letzte Bemerkung erfordert noch eine Erkl\"arung: Was genau
bedeutet \glqq Uhren gehen im Gravitationsfeld langsamer?\grqq\
Zun\"achst einmal verlangt \glqq langsamer\grqq\ nach einem
Vergleich bzw.\ einer Referenz -- langsamer im Vergleich zu
was? Gemeint ist hier der Vergleich zu der Uhr, die sich
nicht im Gravitationsfeld befindet (bzw.\ einer Beschleunigung
unterliegt). Wir k\"onnen jedoch nicht einfach das Gravitationsfeld
abschalten (das war eine der Forderungen an eine
\glqq universelle Kraft\grqq). In dem oben diskutierten 
Beispiel ist offensichtlich, welche zwei Uhren wir miteinander
vergleichen. Diese Uhren haben aber unterschiedliche
Weltlinien durchlaufen. Statt zu sagen, Uhren gehen im
Gravitationsfeld langsamer, sollten wir besser sagen:
Die Weltlinie entlang eines Weges, bei dem die
Uhr den Einfluss der Gravitation sp\"urt, ist k\"urzer.
Eine gute Uhr setzt den Ma\ss stab und geht nicht
langsamer. Wenn wir trotzdem manchmal sagen
\glqq Uhren gehen im Gravitationsfeld langsamer\grqq,
ist diese Bedeutung gemeint, obwohl diese Sprechweise
streng genommen irref\"uhrend ist.

\subsection{Rotverschiebung}

An einem ganz \"ahnlichen Gedankenexperiment l\"asst sich auch
\index{Rotverschiebung im Gravitationsfeld}
die Rotverschiebung von Licht im Gravitationsfeld verstehen. 
Beobachter $S$ (der Sender) befinde sich 
in einem Gravitationsfeld, Beobachter
$E$ (Empf\"anger) sei au\ss erhalb dieses 
Gravitationsfeldes. Beide Beobachter halten
konstanten Abstand. Beobachter $S$ benutzt 
nun eine Referenzfrequenz, beispielsweise 
die Frequenz einer bestimmten Spektrallinie eines
Atoms, und sendet in entsprechendem Takt Signale 
zu Beobachter $E$. Zwischen der Ankunft zweier Signale 
vergeht f\"ur Beobachter $E$ aber mehr Zeit, als zwischen 
den Absendezeiten f\"ur Beobachter $S$.
Da es sich um Frequenzen handelt, sieht Beobachter 
$E$ die entsprechende Spektrallinie also rotverschoben.

\begin{SCfigure}[50][ht]
\begin{picture}(200,160)(0,0)
\put(10,10){\vector(1,0){150}}
\put(10,10){\vector(0,1){150}}
\qbezier(60,20)(80,50)(140,90)
\qbezier(60,40)(80,70)(140,110)
\qbezier(60,60)(80,90)(140,130)
\qbezier(60,80)(80,110)(140,150)
\multiput(10,14)(0,4){9}{\line(1,0){50}}
\multiput(25,50)(0,4){2}{\line(1,0){20}}
\multiput(10,58)(0,4){25}{\line(1,0){50}}
\thicklines
\put(60,10){\line(0,1){150}}
\put(140,10){\line(0,1){150}}
\put(160,5){\makebox(0,0){$r$}}
\put(15,160){\makebox(0,0){$t$}}
\put(18,52){\makebox(0,0){$O$}}

\put(60,5){\makebox(0,0){$S$}}
\put(140,5){\makebox(0,0){$E$}}
\put(53,52){\makebox(0,0){${\scriptstyle \Delta \tau}$}}
\put(146,120){\makebox(0,0){${\scriptstyle \Delta t}$}}
\end{picture}
\caption{\label{fig_rot}%
Rotverschiebung von Licht. Von der Weltlinie eines
Senders $S$ an der Oberfl\"ache
$O$ eines massiven K\"orpers wird in regelm\"a\ss igen
Abst\"anden ein Lichtsignal ausgesandt. Die
Signale verlaufen zwar in der angegebenen Karte
parallel, aber der Eigenzeitabstand $\Delta \tau$ 
zwischen zwei Ereignissen beim Sender ist k\"urzer 
als der Eigenzeitabstand $\Delta t$ beim Empf\"anger $E$.
Daher sieht $E$ die Farbe von Licht rotverschoben.}
\end{SCfigure}

Wir werden im n\"achsten Kapitel sehen, dass
die $g_{00}$-Komponente der Metrik
von der Form
\begin{equation}
        g_{00} = 1 - 2 \frac{\phi(x)}{c^2}
\end{equation}
ist (vgl.\ Gleichung \ref{eq_gnnNewton}), 
wobei $\phi(x)=GM/r$ das klassische Newton'sche
Potenzial einer kugelsymmetrischen 
Gravitationsquelle der Masse $M$ (noch
geteilt durch die Masse eines Probek\"orpers) ist. 
Damit folgt als Beziehung zwischen den Eigenzeiten:
\begin{equation}
       \Delta \tau = \sqrt{1-2\frac{\phi(x)}{c^2}} \Delta t \, .
\end{equation}

\section{Periheldrehung des Merkur}
\label{sec_Perihel}

Nach der Newton'schen Gravitationstheorie
ist die Ellipsenbahn eines einzelnen um eine
schwere Masse (z.B.\ die Sonne) rotierenden 
K\"orpers in einem
Inertialsystem konstant, d.h., die Lage der
gro\ss en Halbachse oder auch der Vektor vom
Massenzentrum zum Perihel bleiben zeitlich
konstant. Dies ist die Folge einer zus\"atzlichen
Erhaltungsgr\"o\ss e, die als Lenz-Runge-Vektor bekannt
ist. Der\index{Lenz-Runge-Vektor} 
Lenz-Runge-Vektor zeigt vom Kraftzentrum
zum Perihel (dem Punkt mit dem k\"urzesten
Abstand der Bahnkurve zum Kraftzentrum)
und ist proportional zur Exzentrizit\"at der Ellipse.

F\"ur die meisten Planetenbahnen ist jedoch
bekannt, dass sich ihr Perihel im Laufe der
Zeit langsam auf einer Kreisbahn um das
Kraftzentrum dreht. Diese 
Periheldrehung ist\index{Periheldrehung des Merkur}
f\"ur den Planeten Merkur am gr\"o\ss ten
und betr\"agt in einem Inertialsystem rund
575,2 Bogensekunden pro Jahrhundert.
Der Gro\ss teil dieser Periheldrehung --
rund 532,1 Bogensekunden pro
Jahrhundert -- l\"asst
sich auf Einfl\"usse der anderen Planeten
sowie eine nicht kugelf\"ormige Massenverteilung
der Sonne zur\"uckf\"uhren. Die Differenz
zwischen dem gemessenen und dem
aus der Newton'schen Theorie berechneten
Wert bet\"agt rund 43,1 Bogensekunden 
pro Jahrhundert.

Dieser Wert ist zu gro\ss, als dass man ihn
als \glqq Dreckeffekt\grqq\ wegdiskutieren 
k\"onnte. Mitte des 19.\ Jahrhunderts gab es
daher Vorschl\"age, einen weiteren Planeten
-- Vulkan -- zu postulieren, der sich so nah
an der Sonne befinden sollte, dass man ihn
(mit damaligen Mitteln) nicht beobachten kann. 

Eine genauere Rechnung (siehe z.B.\ \cite{Fliessbach}, Kap.\ 27)
ergibt f\"ur die Periheldrehung pro Umlauf:
\begin{equation}
\label{eq_Perihel}
       \Delta \phi = \frac{6 \pi GM}{c^2 p} \frac{2 - \beta + 2\gamma}{3} \, .
\end{equation} 
Hierbei ist $p$ der harmonische
Mittelwert der beiden Halbachsen -- f\"ur die Merkurbahn gilt
$p=55 \cdot 10^6$\,km -- und somit ein fester Ellipsenparameter:
\begin{equation}
       \frac{2}{p} = \frac{1}{r_+} + \frac{1}{r_-}  \, , 
\end{equation}
und $\beta$ und $\gamma$ sind zwei Parameter der
Robertson-Entwicklung\index{Robertson-Entwicklung}
(siehe Abschnitt \ref{sec_IsoStat}), die sich f\"ur die
Schwarzschild-L\"osung zu $\beta=\gamma=1$ ergeben
(Gl.\ \ref{eq_betagammaART}).
Setzt man Werte ein, so erh\"alt man pro Umlauf
$\Delta \phi = 0,104$ Bogensekunden oder im Jahrhundert (der Merkur
hat in einem Jahrhundert 415 Sonnenuml\"aufe):
$\Delta \phi = 43,0$ Bogensekunden. 

Erstaunen mag zun\"achst, dass f\"ur $\beta = \gamma =0$,
also in der Newton'schen N\"aherung der Schwarzschild-L\"osung,
der Defektwinkel $\Delta \phi$ in Gl.\ \ref{eq_Perihel}
nicht verschwindet, sich also nicht
das klassische Ergebnis des Kepler-Problems ergibt.
Der Grund ist, dass zwar das Gravitationspotenzial in
der Newton'schen N\"aherung angesetzt wird, die Bewegungsgleichung
zur Bestimmung der Bahnkurven in diesem Potenzial aber
immer noch relativistisch ist. Die relativistische L\"osung des
Wasserstoffproblems (hier wird das Gravitationspotential durch
das elektromagnetische Potential ersetzt, was -- zumindest
nach der g\"angigen Theorie -- nicht zu einer Ver\"anderung
der Raumzeitgeometrie f\"uhrt) hat ebenfalls keine stabilen
Ellipsenkurven. 

\begin{thebibliography}{99}
%\addcontentsline{toc}{chapter}{Literaturangaben}
%\bibitem{Aichelburg} Peter C.\ Aichelburg (Hrsg.); {\it Zeit im 
%       Wandel der Zeit}; Verlag Vieweg, Braunschweig, Wiesbaden, 1988.
%\bibitem{Barbour3} {\it Mach's Principle -- From Newton's Bucket to
%        Quantum Gravity}; Julian Barbour \& Herbert Pfister (Hrsg.);
%        Birkh\"auser, Boston, Basel, Berlin, 1995.       
%\bibitem{Bekenstein} Jacob D.\ Bekenstein, \textit{Black holes
%          and entropy}, Phys.\ Rev.\ D\,7 (1973) 2333--2346.        
%\bibitem{Bell} John Bell;  {\em Speakable and Unspeakable in 
%        Quantum Physics}, 2.\ edition, Cambridge University Press (2004).       
%\bibitem{Born} Max Born; {\it Optik}; Springer-Verlag, Berlin, Heidelberg,
%        1972.
%\bibitem{Britannica} Encyclopaedia Britannica; 15.th edition, 1988.
%\bibitem{Descartes} Ren\'e Descartes; {\it Die Prinzipien der
%        Philosophie}; Felix Meiner Verlag, Hamburg, 1992; \"ubersetzt
%        von Artur Buchenau.
%\bibitem{EDM} Encyclopaedic Dictionary of Mathematics; Second Edition,
%        MIT Press, 1987.
%\bibitem{Einstein1} Albert Einstein; {\it Zur Elektrodynamik bewegter 
%        K\"orper}; Annalen der Physik, Leipzig, 17 (1905) 891. 
%\bibitem{Einstein2} Albert Einstein; {\it Ist die Tr\"agheit eines
%        K\"orpers von seinem Energieinhalt abh\"angig?} (Ann.\ Phys., 
%        Leipzig, 18 (1905) 639.
%\bibitem{Einstein3} Albert Einstein; {\it Aus meinen sp\"aten Jahren};
%         Ullstein Sachbuch, Verlag Ullstein, Frankfurt, Berlin, 1993.                 
%\bibitem{Einstein4} Albert Einstein; {\it Prinzipielles zur allgemeinen
%        Relativit\"atstheorie}; Annalen der Physik 55 (1918) 241.
%\bibitem{Einstein5} Albert Einstein; {\it \"Uber den Einflu\ss\ der
%        Schwerkraft auf die Ausbreitung des Lichtes}; Annalen der
%        Physik 35 (1911) 898.                 
%\bibitem{Feynman} Richard Feynman; {\it The Character of Physical Law};
%        The MIT Press, 1987.        
%\bibitem{Fierz} Markus Fierz; {\it \"Uber den Ursprung und die Bedeutung
%        der Lehre Isaac Newtons vom absoluten Raum}; Gesnerus, 
%        11.\ Jahrgang (1954), S.\,62--120.
%\bibitem{Fliessbach} Torsten Flie\ss bach; {\it Allgemeine 
%        Relativit\"atstheorie}; BI-Wissenschaftsverlag, Mannheim, Wien
%        Z\"urich, 1990. 
%\bibitem{Galilei} Galilei; {\it Dialog \"uber die beiden haupts\"achlichen
%        Weltsysteme, das ptolem\"aische und das kopernikanische}; 
%        Teubner Stuttgart, 1982; aus dem Italienischen \"ubersetzt von
%        Emil Strauss.   
%  \bibitem{Hawking} Stephen W.\ Hawking, \textit{Particle Creation by
%            black holes}, Comm.\ Math.\ Phys.\ 43 (1976) 199--220.      
%\bibitem{Helmholtz2} Hermann von Helmholtz; {\em \"Uber Wirbelbewegungen,
%        \"Uber Fl\"ussigkeitsbewegungen}, 1858; in Ostwalds Klassiker der 
%       exakten Wissenschaften Bd.\ 1; Verlag Harri Deutsch, Frankfurt, 
%       1996.                   
%\bibitem{Lamb} G.L.\ Lamb, Jr.; {\it Elements of Soliton Theory}; 
%         Pure \& Applied Mathematics, John Wiley \& Sons, 1980. 
%\bibitem{Laue} Max von Laue; {\it Geschichte der Physik}; 
%         Universit\"ats-Verlag Bonn, 1947.
%\bibitem{Lorentz} Hendrik Antoon Lorentz; {\it Electromagnetic phenomena 
 %        in a system moving with any velocity smaller than that of light}; 
%         Proc.\ Acad.\ Sci., Amsterdam, 6 [1904], S.\ 809.
%\bibitem{Mach} Ernst Mach; {\it Die Mechanik in ihrer Entwicklung
%      historisch kritisch dargestellt}; Akademie Verlag, Berlin, 1988.       
%\bibitem{Mainzer} Klaus Mainzer; {\it Philosophie und Geschichte von
%         Raum und Zeit}; in {\it Philosophie und Physik der Raum-Zeit};
%         J\"urgen Audretsch und Klaus Mainzer (Hrsg.); 
%         BI-Wissenschaftsverlag, 1994. 
%\bibitem{Microscope} Touboul, Pierre, et al. (MICROSCOPE Collaboration); \textit{MICROSCOPE Mission:
%         Final Results of the Test of the Equivalence Principle}; Phys.\ Rev.\ Lett.\ \textbf{129} (2022) 121102.
%\bibitem{Misner} C.W.\ Misner, K.S.\ Thorne, J.A.\ Wheeler; 
%        {\it Gravitation}; W.H.\ Freeman and Company, San Francisco,  1973.
%\bibitem{Mittelstaedt} Peter Mittelstaedt; {\it Der Zeitbegriff in der
%        Physik}; BI-Wissenschaftsverlag, 1989.        
%\bibitem{Mittelstaedt2} Peter Mittelstaedt; {\it Philosophische Probleme
%        der modernen Physik}; BI-Wissenschaftsverlag, 1989.        
%\bibitem{Newton}
%   Isaac Newton; {\it Mathematische Grundlagen der Naturphilosophie}; 
%   \"ubersetzt von Ed Dellian; Felix Meiner Verlag, 1988. 
%\bibitem{Newton2} Isaac Newton; {\it \"Uber die Gravitation...};
%       Klostermann Texte Philosophie; Vittorio Klostermann, Frankfurt,
%      1988; \"ubersetzt von Gernot B\"ohme.
%\bibitem{Newton3} Isaac Newton; {\it Optik oder Abhandlung \"uber
%      Spiegelungen, Brechungen, Beugungen und Farben des Lichts};
%      I., II.\ und III.\ Buch (1704); aus dem Englischen \"ubersetzt
%      von W.\ Abendroth; Ostwalds Klassiker der exakten Wissenschaften,
%      Verlag Harri Deutsch 1998.   
%\bibitem{Neumann} Carl Neumann; {\it \"Uber die Principien der
%         Galilei-Newtonschen Theorie}; Akademische Antrittsvorlesung,
%         gehalten in der Aula der Universit\"at Leipzig am 3.\ Nov.\
%         1869; Teubner (Leipzig) 1870.         
%\bibitem{Pauli} Wolfgang Pauli; {\it Theory of Relativity}; Dover
%      Publications, New York, 1981.      
%\bibitem{Poincare} Jules Henri Poincar\'e; {\it Sur la dynamique de 
%     l'\'electron}, C.R.\ Acad.\ Sci., Paris, 140 (1905) S.~1504; und 
%      Rendiconti del Circolo Matematico di Palermo, Bd.~21 (1906) S.~129.
%\bibitem{Reichenbach1} Hans Reichenbach; {\em Philosophie der 
%       Raum-Zeit-Lehre}; Hans Reichenbach - Gesammelte Werke Bd.\ 2;
%       Vieweg-Verlag, Braunschweig; 1977.
%\bibitem{Reichenbach2} Hans Reichenbach; {\em Axiomatik der
%       relativistischen Raum-Zeit-Lehre}; in {\em Die philosophische
%       Bedeutung der Relativit\"atstheorie}; Hans Reichenbach - Gesammelte
%       Werke Bd.\ 3; Vieweg-Verlag, Braunschweig, 1977. 
%\bibitem{Rovelli} Carlo Rovelli, \textit{Quantum Gravity}; Cambridge
%      University Press, 2007.       
%\bibitem{Schlamminger} Schlamminger, Choi, Wagner, Gundlach,
%         Adelberger; {\em Test of the Equivalence Principle using a
%         rotating torsion balance}; Phys.\ Rev.\ Lett.\ {\bf 100} (2008)
%         041101.     
%\bibitem{Sexl} Roman U.\ Sexl, Helmuth K.\ Urbantke; {\it Relativit\"at,
%      Gruppen, Teilchen}; Springer-Verlag, Wien, New York, 1992.
%\bibitem{Simonyi}
%       K\'aroly Simonyi; {\it Kulturgeschichte der Physik}; Verlag
%       Harri Deutsch, Thun, Frankfurt am Main, 1990.
%\bibitem{Weisberg} Weisberg, J.M., Taylor, J.H.; {\em Relativistic Binary Pulsar
%          B1913+16: Thirty Years of Observations and Analysis}; 
%          \verb+arXiv:astro-ph/0407149v1+; 2004. 
%\bibitem{Thomson} James Thomson; {\it On the Law of Inertia; the
%       Principle of Chronometry; and the Principle of Absolute Clinural
%       Rest, and of Absolute Rotation}; Proc.\ Roy.\ Soc.\ (Edinburgh),
%       Session 1883-84, Vol.\ XII, 568--578.       
%\bibitem{Weizsaecker} Carl Friedrich von Weizs\"acker; {\em Der zweite
%      Hauptsatz und der Unterschied von Vergangenheit und Zukunft};
%      Annalen der Physik 36 (1939) 275--283.       
%\bibitem{Zeh} Zeh, H.D.; {\em The Physical Basis of the Direction of Time},
%      Springer-Verlag, Berlin, 1989.       

%\bibitem{Einstein} Einstein, Albert; {\em ??}, .                   
\end{thebibliography}


\end{document}
