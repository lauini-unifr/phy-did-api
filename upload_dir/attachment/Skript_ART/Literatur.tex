%\documentclass[german,12pt]{book}
%\begin{document}


\begin{thebibliography}{99}
\addcontentsline{toc}{chapter}{Literaturangaben}
\bibitem{Aichelburg} Peter C.\ Aichelburg (Hrsg.); {\it Zeit im 
       Wandel der Zeit}; Verlag Vieweg, Braunschweig, Wiesbaden, 1988.
\bibitem{Barbour3} {\it Mach's Principle -- From Newton's Bucket to
        Quantum Gravity}; Julian Barbour \& Herbert Pfister (Hrsg.);
        Birkh\"auser, Boston, Basel, Berlin, 1995.       
\bibitem{Bekenstein} Jacob D.\ Bekenstein, \textit{Black holes
          and entropy}, Phys.\ Rev.\ D\,7 (1973) 2333--2346.        
\bibitem{Bell} John Bell;  {\em Speakable and Unspeakable in 
        Quantum Physics}, 2.\ edition, Cambridge University Press (2004).       
\bibitem{Born} Max Born; {\it Optik}; Springer-Verlag, Berlin, Heidelberg,
        1972.
\bibitem{Britannica} Encyclopaedia Britannica; 15.th edition, 1988.
%\bibitem{Descartes} Ren\'e Descartes; {\it Die Prinzipien der
%        Philosophie}; Felix Meiner Verlag, Hamburg, 1992; \"ubersetzt
%        von Artur Buchenau.
%\bibitem{EDM} Encyclopaedic Dictionary of Mathematics; Second Edition,
%        MIT Press, 1987.
\bibitem{Einstein1} Albert Einstein; {\it Zur Elektrodynamik bewegter 
        K\"orper}; Annalen der Physik, Leipzig, 17 (1905) 891. 
\bibitem{Einstein2} Albert Einstein; {\it Ist die Tr\"agheit eines
        K\"orpers von seinem Energieinhalt abh\"angig?} (Ann.\ Phys., 
        Leipzig, 18 (1905) 639.
\bibitem{Einstein3} Albert Einstein; {\it Aus meinen sp\"aten Jahren};
         Ullstein Sachbuch, Verlag Ullstein, Frankfurt, Berlin, 1993.                 
\bibitem{Einstein4} Albert Einstein; {\it Prinzipielles zur allgemeinen
        Relativit\"atstheorie}; Annalen der Physik 55 (1918) 241.
\bibitem{Einstein5} Albert Einstein; {\it \"Uber den Einflu\ss\ der
        Schwerkraft auf die Ausbreitung des Lichtes}; Annalen der
        Physik 35 (1911) 898.                 
%\bibitem{Feynman} Richard Feynman; {\it The Character of Physical Law};
%        The MIT Press, 1987.        
%\bibitem{Fierz} Markus Fierz; {\it \"Uber den Ursprung und die Bedeutung
%        der Lehre Isaac Newtons vom absoluten Raum}; Gesnerus, 
%        11.\ Jahrgang (1954), S.\,62--120.
\bibitem{Fliessbach} Torsten Flie\ss bach; {\it Allgemeine 
        Relativit\"atstheorie}; BI-Wissenschaftsverlag, Mannheim, Wien
        Z\"urich, 1990. 
\bibitem{Galilei} Galilei; {\it Dialog \"uber die beiden haupts\"achlichen
        Weltsysteme, das ptolem\"aische und das kopernikanische}; 
        Teubner Stuttgart, 1982; aus dem Italienischen \"ubersetzt von
        Emil Strauss.   
  \bibitem{Hawking} Stephen W.\ Hawking, \textit{Particle Creation by
            black holes}, Comm.\ Math.\ Phys.\ 43 (1976) 199--220.      
\bibitem{Helmholtz2} Hermann von Helmholtz; {\em \"Uber Wirbelbewegungen,
        \"Uber Fl\"ussigkeitsbewegungen}, 1858; in Ostwalds Klassiker der 
       exakten Wissenschaften Bd.\ 1; Verlag Harri Deutsch, Frankfurt, 
       1996.                   
\bibitem{Lamb} G.L.\ Lamb, Jr.; {\it Elements of Soliton Theory}; 
         Pure \& Applied Mathematics, John Wiley \& Sons, 1980. 
\bibitem{Laue} Max von Laue; {\it Geschichte der Physik}; 
         Universit\"ats-Verlag Bonn, 1947.
\bibitem{Lorentz} Hendrik Antoon Lorentz; {\it Electromagnetic phenomena 
         in a system moving with any velocity smaller than that of light}; 
         Proc.\ Acad.\ Sci., Amsterdam, 6 [1904], S.\ 809.
\bibitem{Mach} Ernst Mach; {\it Die Mechanik in ihrer Entwicklung
      historisch kritisch dargestellt}; Akademie Verlag, Berlin, 1988.       
%\bibitem{Mainzer} Klaus Mainzer; {\it Philosophie und Geschichte von
%         Raum und Zeit}; in {\it Philosophie und Physik der Raum-Zeit};
%         J\"urgen Audretsch und Klaus Mainzer (Hrsg.); 
%         BI-Wissenschaftsverlag, 1994. 
\bibitem{Misner} C.W.\ Misner, K.S.\ Thorne, J.A.\ Wheeler; 
        {\it Gravitation}; W.H.\ Freeman and Company, San Francisco,
        1973.
\bibitem{Mittelstaedt} Peter Mittelstaedt; {\it Der Zeitbegriff in der
        Physik}; BI-Wissenschaftsverlag, 1989.        
\bibitem{Mittelstaedt2} Peter Mittelstaedt; {\it Philosophische Probleme
        der modernen Physik}; BI-Wissenschaftsverlag, 1989.        
%\bibitem{Newton}
%   Isaac Newton; {\it Mathematische Grundlagen der Naturphilosophie}; 
%   \"ubersetzt von Ed Dellian; Felix Meiner Verlag, 1988. 
\bibitem{Newton2} Isaac Newton; {\it \"Uber die Gravitation...};
       Klostermann Texte Philosophie; Vittorio Klostermann, Frankfurt,
      1988; \"ubersetzt von Gernot B\"ohme.
%\bibitem{Newton3} Isaac Newton; {\it Optik oder Abhandlung \"uber
%      Spiegelungen, Brechungen, Beugungen und Farben des Lichts};
%      I., II.\ und III.\ Buch (1704); aus dem Englischen \"ubersetzt
%      von W.\ Abendroth; Ostwalds Klassiker der exakten Wissenschaften,
%      Verlag Harri Deutsch 1998.   
%\bibitem{Neumann} Carl Neumann; {\it \"Uber die Principien der
%         Galilei-Newtonschen Theorie}; Akademische Antrittsvorlesung,
%         gehalten in der Aula der Universit\"at Leipzig am 3.\ Nov.\
%         1869; Teubner (Leipzig) 1870.         
\bibitem{Pauli} Wolfgang Pauli; {\it Theory of Relativity}; Dover
      Publications, New York, 1981.      
\bibitem{Poincare} Jules Henri Poincar\'e; {\it Sur la dynamique de 
     l'\'electron}, C.R.\ Acad.\ Sci., Paris, 140 (1905) S.~1504; und 
      Rendiconti del Circolo Matematico di Palermo, Bd.~21 (1906) S.~129.
\bibitem{Reichenbach1} Hans Reichenbach; {\em Philosophie der 
       Raum-Zeit-Lehre}; Hans Reichenbach - Gesammelte Werke Bd.\ 2;
       Vieweg-Verlag, Braunschweig; 1977.
\bibitem{Reichenbach2} Hans Reichenbach; {\em Axiomatik der
       relativistischen Raum-Zeit-Lehre}; in {\em Die philosophische
       Bedeutung der Relativit\"atstheorie}; Hans Reichenbach - Gesammelte
       Werke Bd.\ 3; Vieweg-Verlag, Braunschweig, 1977. 
\bibitem{Rovelli} Carlo Rovelli, \textit{Quantum Gravity}; Cambridge
      University Press, 2007.       
\bibitem{Schlamminger} Schlamminger, Choi, Wagner, Gundlach,
         Adelberger; {\em Test of the Equivalence Principle using a
         rotating torsion balance}; Phys.\ Rev.\ Lett.\ {\bf 100} (2008)
         041101.     
\bibitem{Sexl} Roman U.\ Sexl, Helmuth K.\ Urbantke; {\it Relativit\"at,
      Gruppen, Teilchen}; Springer-Verlag, Wien, New York, 1992.
\bibitem{Simonyi}
      K\'aroly Simonyi; {\it Kulturgeschichte der Physik}; Verlag
       Harri Deutsch, Thun, Frankfurt am Main, 1990.
\bibitem{Weisberg} Weisberg, J.M., Taylor, J.H.; {\em Relativistic Binary Pulsar
          B1913+16: Thirty Years of Observations and Analysis}; 
          \verb+arXiv:astro-ph/0407149v1+; 2004. 
%\bibitem{Thomson} James Thomson; {\it On the Law of Inertia; the
%       Principle of Chronometry; and the Principle of Absolute Clinural
%       Rest, and of Absolute Rotation}; Proc.\ Roy.\ Soc.\ (Edinburgh),
%       Session 1883-84, Vol.\ XII, 568--578.       
%\bibitem{Weizsaecker} Carl Friedrich von Weizs\"acker; {\em Der zweite
%      Hauptsatz und der Unterschied von Vergangenheit und Zukunft};
%      Annalen der Physik 36 (1939) 275--283.       
%\bibitem{Zeh} Zeh, H.D.; {\em The Physical Basis of the Direction of Time},
%      Springer-Verlag, Berlin, 1989.       

%\bibitem{Einstein} Einstein, Albert; {\em ??}, .                   
\end{thebibliography}


%\end{document}