\documentclass[german,10pt]{book}         
\usepackage{makeidx}
\usepackage{babel}            % Sprachunterstuetzung
\usepackage{amsmath}          % AMS "Grundpaket"
\usepackage{amssymb,amsfonts,amsthm,amscd} 
\usepackage{mathrsfs}
\usepackage{rotating}
\usepackage{sidecap}
\usepackage{graphicx}
\usepackage{color}
\usepackage{fancybox}
\usepackage{tikz}
\usetikzlibrary{arrows,snakes,backgrounds}
\usepackage{hyperref}
\hypersetup{colorlinks=true,
                    linkcolor=blue,
                    filecolor=magenta,
                    urlcolor=cyan,
                    pdftitle={Overleaf Example},
                    pdfpagemode=FullScreen,}
%\newcommand{\hyperref}[1]{\ref{#1}}
%
\definecolor{Gray}{gray}{0.80}
\DeclareMathSymbol{,}{\mathord}{letters}{"3B}
%
\newcounter{num}
\renewcommand{\thenum}{\arabic{num}}
\newenvironment{anmerkungen}
   {\begin{list}{(\thenum)}{%
   \usecounter{num}%
   \leftmargin0pt
   \itemindent5pt
   \topsep0pt
   \labelwidth0pt}%
   }{\end{list}}
%
\renewcommand{\arraystretch}{1.15}                % in Formeln und Tabellen   
\renewcommand{\baselinestretch}{1.15}                 % 1.15 facher
                                                      % Zeilenabst.
\newcommand{\Anmerkung}[1]{{\begin{footnotesize}#1 \end{footnotesize}}\\[0.2cm]}
\newcommand{\comment}[1]{}
\setlength{\parindent}{0em}           % Nicht einruecken am Anfang der Zeile 

\setlength{\textwidth}{15.4cm}
\setlength{\textheight}{23.0cm}
\setlength{\oddsidemargin}{1.0mm} 
\setlength{\evensidemargin}{-6.5mm}
\setlength{\topmargin}{-10mm} 
\setlength{\headheight}{0mm}
\newcommand{\identity}{{\bf 1}}
%
\newcommand{\vs}{\vspace{0.3cm}}
\newcommand{\noi}{\noindent}
\newcommand{\leer}{}

\newcommand{\engl}[1]{[\textit{#1}]}
\parindent 1.2cm
\sloppy

         \begin{document}  \setcounter{chapter}{5}

\chapter{Thermodynamik Schwarzer L\"ocher}


\section{Quantentheorie und Gravitation}

Eine wirklich zufriedenstellende Quantentheorie der
Gravitation bzw.\ der Raumzeit gibt es derzeit 
trotz umfangreicher Versuche (es gibt die kanonische Quantengravitation,
die Loop-Quantengravitation, die String-Theorie, kausale
Mengen, kausale Triangulationen, nicht-kommutative Geometrie, ...)
noch nicht. Auch nur ein elementarer \"Uberblick zu diesen
Ans\"atzen w\"urde den Rahmen dieser Vorlesung mehr als sprengen.

Ein paar wichtige \"Uberlegungen scheinen jedoch so allgemein,
dass sie vermutlich in jeder sinnvollen Form der Quantengravitation
eine Rolle spielen werden. Dazu z\"ahlen beispielsweise
die so genannten Planck'schen Einheiten, als auch insbesondere
die Thermodynamik Schwarzer L\"ocher. Auf beides m\"ochte ich
kurz eingehen.

\subsubsection{Planck'sche Einheiten}

Die Quantentheorie, die Spezielle Relativit\"ats\-theorie und
die Allgemeine Relativit\"ats\-theorie liefern uns drei
fundamentale Naturkonstanten:\index{Planck-Einheiten}
\begin{eqnarray*}
      \hbar &=& 1,054572\times 10^{-34} \, {\rm J \cdot s}\\
      c &=& 299\,792\,458\,{\rm m} \\
      G &=& 6,675 \times 10^{-11}\, \frac{\rm m^3}{kg \cdot s^2} \, .
\end{eqnarray*}
Aus diesen Konstanten lassen sich fundamentale
Gr\"o\ss en zu allen physikalischen Einheiten
ableiten, z.B.:
\begin{eqnarray*}
  \mbox{Planck-L\"ange} \hspace{0.4cm}
        l_{\rm P} &=&  \sqrt{\frac{\hbar G}{c^3}} =
         1,6162 \times 10^{-35}\, {\rm m} \\
  \mbox{Planck-Zeit} \hspace{0.4cm}
        t_{\rm P} &=&  \sqrt{\frac{\hbar G}{c^5}} =
         5,391 \times 10^{-44}\, {\rm s} \\
  \mbox{Planck-Masse} \hspace{0.4cm}
        m_{\rm P} &=&  \sqrt{\frac{\hbar c}{G}} =
         2,1765 \times 10^{-8}\, {\rm kg} \\
  \mbox{Planck-Energie} \hspace{0.4cm}
        E_{\rm P} &=&  \sqrt{\frac{\hbar c^5}{G}} =
         1,9561 \times 10^{9}\, {\rm J}  \\
  \mbox{Planck-Dichte} \hspace{0.4cm}
        \rho_{\rm P} &=&  \frac{c^5}{\hbar G^2} =
         5,155 \times 10^{96}\, {\rm kg/m^3} \\
  \mbox{Planck-Volumen} \hspace{0.4cm}
        V_{\rm P} &=&  \sqrt{\frac{(\hbar G)^3}{c^9}} =
         4,2217 \times 10^{-105}\, {\rm m^3} 
\end{eqnarray*}
Nehmen wir noch die Boltzmann-Konstante
\begin{equation}
      k_{\rm B} = 1,38065 \times 10^{-23}\, {\rm J/K}
\end{equation}
hinzu, so erhalten wir auch eine\index{Planck-Einheiten!Planck-Temperatur}
\begin{equation}
  \mbox{Planck-Temperatur} \hspace{0.4cm}
        T_{\rm P} ~~ = ~~  \sqrt{\frac{\hbar c^5}{G k_{\rm B}^2}} =
         1,4168 \times 10^{32}\, {\rm K} \, . 
\end{equation}

Schon 1899 (also ein Jahr vor seinem ber\"uhmten Vortrag
am 14.\ Dezember 1900 vor der Physikalischen Gesellschaft,
der allgemein als die Geburtsstunde der Quantentheorie
angesehen wird) machte Max Planck\index{Planck, Max} 
darauf aufmerksam,
dass es diese fundamentalen Gr\"o\ss en geben m\"usse.
Er hatte zu dieser Zeit eine empirische Formel f\"ur die
Schwarzk\"orperstrahlung gefunden und erkannt, dass in
dieser Formel eine fundamentale Konstante (die wir heute
als Planck-Konstante bezeichnen) auftritt. Mit dieser Konstanten
und den damals schon bekannten Konstanten $c$ und $G$
gelangte er zu den Gr\"o\ss en, die wir heute als 
\textit{Planck-Einheiten} bezeichnen.

Die Planck-L\"ange\index{Planck-Einheiten!Planck-L\"ange} 
ist vermutlich die kleinste Skala, auf der
das Konzept einer L\"ange noch sinnvoll ist, \"ahnliches gilt
f\"ur die Planck-Zeit.\index{Planck-Einheiten!Planck-Zeit} 
Die Planck-Zeit entspricht der Zeit, die
das Licht braucht, um eine Planck-L\"ange zu durchqueren.

Die Planck-Masse ist die Masse\index{Planck-Einheiten!Planck-Masse}
eines Schwarzen Loches, dessen Schwarzschild-Radius gleich
der Planck-L\"ange ist. Mit rund 0,02\,mg hat sie fast einen
makroskopischen Wert. Ausgedr\"uckt in GeV/$c^2$,
eine bei Teilchenphysikern sehr beliebte Einheit, entspricht
dies $m_{\rm P}=1,22\times 10^{19}$\,GeV/$c^2$. Dies ist
vermutlich die h\"ochste Masse, die ein nachweisbares 
elementares Teilchen haben kann.   

Die Planck-Energie ist die Energie zu einer Planck-Masse
nach\index{Planck-Einheiten!Planck-Energie} 
der Einstein'schen Beziehung $E=mc^2$.
Gleichzeitig ist es die Unbestimmtheit in der Energie, die
nach der Quantentheorie innerhalb einer Planck-Zeit 
nicht unterschritten werden kann.

Au\ss erdem definieren die Planck-Einheiten die Grenzen,
bis zu denen wir nach dem Standardmodell der Teilchenphysik
vordringen k\"onnen, ohne die Quantengravitation mit 
einbeziehen zu m\"ussen: Das Universum hatte eine 
Planck-Zeit nach dem Urknall einen Radius von der
Planck-L\"ange bzw.\ ein Planck-Volumen.
Es hatte zu diesem Zeitpunkt eine Temperatur von der
Planck-Temperatur, usw. 

\subsection{Die Hawking-Bekenstein-Strahlung}

Als einzigen Quantenaspekt der Gravitation
m\"ochte ich in dieser Vorlesung kurz die Thermodynamik
Schwarzer L\"ocher ansprechen. Es geht dabei darum,
dass man einem Schwarzen Loch eine Entropie und
eine Temperatur zuschreiben kann, und dass mit dieser
Temperatur auch eine Strahlung verbunden ist,
die \textit{Bekenstein-Hawking-Strahlung}\index{Bekenstein-Hawking-Strahlung} 
Schwarzer 
L\"ocher \cite{Bekenstein,Hawking}.

Wie wir im Zusammenhang mit den L\"osungen der
Einstein'schen Gleichungen gesehen haben, gilt f\"ur
Schwarze L\"ocher das \glqq No Hair-Theorem\grqq. 
Neben ihrer Masse, ihrem Drehimpuls und ihrer Ladung
(sowie einigen Erhaltungsgr\"o\ss en aus der Teilchenphysik)
haben Schwarze L\"oche keine weiteren Freiheitsgrade.
Insbesondere gibt es keine Quadrupolmomente zur Massen-
oder Ladungsverteilung etc. Im Sinne der Boltzmann'schen
Beziehung\index{Entropie} 
$S=k_{\rm B} \ln \Omega$ zwischen der Entropie
$S$ und der Anzahl der Mikrozust\"ande $\Omega$ sollte
ein Schwarzes Loch daher praktisch keine Entropie besitzen.

Damit w\"urden Schwarze L\"ocher jedoch dem zweiten
Hauptsatz der Thermodynamik (zumindest in einer bestimmten
Form) widersprechen: Die Gesamtentropie in unserem Universum
kann niemals abnehmen. Wenn jedoch ein Schwarzes Loch
praktisch keine Entropie h\"atte, gleichzeitig aber beliebige
Materiemengen mit nahezu beliebig viel Entropie \glqq verschlucken\grqq\
kann, dann scheint bei diesem Prozess die Entropie in 
unserem Universum abzunehmen. 

Ein Ausweg ergibt sich, wenn man einem Schwarzen Loch
eine Entropie\index{Entropie!Schwarzes Loch}\index{Schwarzes Loch!Entropie} 
zuschreibt, die mit seiner Masse zusammenh\"angt.
Durch solche \"Uberlegungen gelangte Bekenstein zu
der Schlussfolgerung, dass ein Schwarzes Loch eine Entropie
\begin{equation}
          S = \frac{k_{\rm B}}{4} \frac{A}{l_{\rm P}^2}  
\end{equation}
haben muss. Hierbei ist $A$ die Oberfl\"ache des Schwarzen
Lochs (also die Fl\"ache einer Kugeloberfl\"ache vom Radius
$R_{\rm S}$) und $l_{\rm P}$ die oben eingef\"uhrte
Planck-L\"ange. Die Entropie eines Schwarzen Lochs entspricht
also seiner Oberfl\"ache in Einheiten der Planck-Fl\"ache.
Es zeigt sich, dass dies die h\"ochste Entropie ist, die sich
in dem Volumen eines Schwarzen Lochs unterbringen
l\"asst. Keine Materieform kann in einem so kleinen Bereich
eine gr\"o\ss ere Entropie haben.

Es mag erstaunen, dass ein Schwarzes Loch eine Entropie hat,
die proportional zu seiner Oberfl\"ache und nicht zu seinem
Volumen ist, denn gew\"ohnlich ist die Entropie eine extensive
Gr\"o\ss e, die proportional zum Volumen eines Systems ist. 
Hier hilft aber vielleicht die Vorstellung, dass f\"ur einen 
asymptotischen Beobachter Materie (und damit Entropie) niemals
hinter dem Horizont verschwindet, sondern in gewisser Hinsicht
an der Oberfl\"ache \glqq kleben\grqq\ bleibt. Die emittierte
Strahlung wird zwar immer langwelliger, aber die Wellenl\"ange
kann nie gr\"o\ss er als der Durchmesser des Schwarzen
Lochs werden. 

Dies gibt auch gleichzeitig eine Vorstellung von der Temperatur
des Schwarzen Lochs:\index{Temperatur eines Schwarzen Lochs} 
Ein Schwarzes Loch strahlt eine\index{Schwarzes Loch!Temperatur}
thermische Strahlung ab, deren Wellenl\"ange seinem
Durchmesser entspricht. Verwenden wir das Wien'sche
Verschiebungsgesetz, das eine Beziehung zwischen der
Temperatur und der Wellenl\"ange zur maximalen Intensit\"at
der abgestrahlten Strahlung angibt,
\begin{equation}
       \lambda = {\rm const} \frac{\hbar c}{k_{\rm B} T} \, ,
\end{equation} 
und setzen f\"ur $\lambda$ den Schwarzschild-Durchmesser
(doppelten Radius) eines Schwarzen Lochs ein, so erhalten wir
\begin{equation}
     \frac{4GM}{c^2} = {\rm const} \frac{\hbar c}{k_{\rm B}T}
\end{equation}
oder
\begin{equation}
      T = {\rm const} \frac{\hbar c^3}{k_{\rm B} GM}  \, .
\end{equation}
Eine genaue quantenfeldtheoretische Rechnung liefert
die \textit{Bekenstein-Hawking-Tem\-pe\-ratur}:\index{Bekenstein-Hawking-Temperatur}
\begin{equation}
      T = \frac{\hbar c^3}{8 \pi k_{\rm B} GM}  \, .
\end{equation}
Aus dieser Gleichung kann man auch wieder die
Entropie des Schwarzen Lochs ableiten.

In einem sehr vereinfachten Bild wird der Ursprung der
Bekenstein-Hawking-Strahlung in der Quantenfeldtheorie
gerne folgenderma\ss en beschrieben: Das Vakuum
ist in der Quantenfeldtheorie nicht leer, sondern es
gibt eine Grundzustandsenergie und Grundzustandsfluktuationen.
Diese werden oft durch die Erzeugung und Vernichtung
virtueller Teilchen-Antiteilchen-Paare beschrieben.
Wenn nun ein Teilchen-Antiteilchen-Paar nahe der
Oberfl\"ache eines Schwarzen Lochs entsteht, kann es
passieren, dass eines der Teilchen den Schwarzschild-Radius
\"uberquert, wohingegen das andere Teilchen dem
Schwarzen Loch entkommt. F\"ur einen entfernten 
Beobachter scheint dieses (nun reale) Teilchen von dem Schwarzen
Loch abgestrahlt worden zu sein. Das vom Schwarzen Loch
verschluckte Teilchen hatte eine negative Energie und
hat somit die Energie des Schwarzen Lochs um die
Energie des abgestrahlten Teilchens verringert. 
Man sollte jedoch nicht vergessen, dass dies nur
ein sehr vereinfachtes, anschauliches Bild ist.

Die Entropie zu einem Schwarzen Loch von der
Masse der Sonne ist ungef\"ahr\index{Entropie!Sonne} 
\begin{equation}
        \frac{S_{\rm Sonne}}{k_{\rm B}} \approx 1,07 \cdot 10^{76} \, .
\end{equation}
Die Masse unseres Universums wird auf rund $10^{24}$ Sonnenmassen
gesch\"atzt. Da die Entropie proportional zum Quadrat der Sonnenmasse
ist, folgt daraus f\"ur die maximal m\"ogliche Entropie, die unser
Universum haben kann (wenn n\"amlich die gesamte Masse in
einem einzige Schwarzen Loch verschwunden ist), ein 
Wert\index{Entropie!Uni@des Universums}
von rund $10^{124}$. Die gesch\"atzte Entropie des heutigen
Universums liegt um rund 20 Gr\"o\ss enordnungen niedriger. 

Derzeit entspricht die Mikrowellenhintergrundstrahlung
einer Temperatur von rund $2,7$\,K. Dies bezeichnet man
manchmal als die heutige Temperatur des Universums.
Die Temperatur eines Schwarzen Lochs von der Masse
unserer Sonne ist nach obiger Formel ungef\"ahr $10^{-6}$\,K,
schwerere Schwarze L\"ocher haben sogar noch niedrigere
Temperaturen. Somit w\"urde ein Schwarzes Loch derzeit
mehr Strahlung aufnehmen (und dadurch an Masse
zunehmen) als es abstrahlt. Da sich unser Universum aber
beschleunigt ausdehnt (siehe n\"achstes Kapitel) und somit 
seine Temperatur weiterhin abnimmt, k\"onnte irgendwann
der Zeitpunkt kommen, an dem Schwarze L\"ocher mehr
Strahlung abstrahlen als sie aufnehmen. Ist das der
Fall, k\"onnten langsam s\"amtliche Schwarzen L\"ocher
verstrahlen und zur\"uck bliebe ein sehr d\"unnes Gas
in einem ansonsten leeren Raum. 


\begin{thebibliography}{99}
%\addcontentsline{toc}{chapter}{Literaturangaben}
%\bibitem{Aichelburg} Peter C.\ Aichelburg (Hrsg.); {\it Zeit im 
%       Wandel der Zeit}; Verlag Vieweg, Braunschweig, Wiesbaden, 1988.
%\bibitem{Barbour3} {\it Mach's Principle -- From Newton's Bucket to
%        Quantum Gravity}; Julian Barbour \& Herbert Pfister (Hrsg.);
%        Birkh\"auser, Boston, Basel, Berlin, 1995.       
%\bibitem{Bekenstein} Jacob D.\ Bekenstein, \textit{Black holes
%          and entropy}, Phys.\ Rev.\ D\,7 (1973) 2333--2346.        
%\bibitem{Bell} John Bell;  {\em Speakable and Unspeakable in 
%        Quantum Physics}, 2.\ edition, Cambridge University Press (2004).       
%\bibitem{Born} Max Born; {\it Optik}; Springer-Verlag, Berlin, Heidelberg,
%        1972.
%\bibitem{Britannica} Encyclopaedia Britannica; 15.th edition, 1988.
%\bibitem{Descartes} Ren\'e Descartes; {\it Die Prinzipien der
%        Philosophie}; Felix Meiner Verlag, Hamburg, 1992; \"ubersetzt
%        von Artur Buchenau.
%\bibitem{EDM} Encyclopaedic Dictionary of Mathematics; Second Edition,
%        MIT Press, 1987.
%\bibitem{Einstein1} Albert Einstein; {\it Zur Elektrodynamik bewegter 
%        K\"orper}; Annalen der Physik, Leipzig, 17 (1905) 891. 
%\bibitem{Einstein2} Albert Einstein; {\it Ist die Tr\"agheit eines
%        K\"orpers von seinem Energieinhalt abh\"angig?} (Ann.\ Phys., 
%        Leipzig, 18 (1905) 639.
%\bibitem{Einstein3} Albert Einstein; {\it Aus meinen sp\"aten Jahren};
%         Ullstein Sachbuch, Verlag Ullstein, Frankfurt, Berlin, 1993.                 
%\bibitem{Einstein4} Albert Einstein; {\it Prinzipielles zur allgemeinen
%        Relativit\"atstheorie}; Annalen der Physik 55 (1918) 241.
%\bibitem{Einstein5} Albert Einstein; {\it \"Uber den Einflu\ss\ der
%        Schwerkraft auf die Ausbreitung des Lichtes}; Annalen der
%        Physik 35 (1911) 898.                 
%\bibitem{Feynman} Richard Feynman; {\it The Character of Physical Law};
%        The MIT Press, 1987.        
%\bibitem{Fierz} Markus Fierz; {\it \"Uber den Ursprung und die Bedeutung
%        der Lehre Isaac Newtons vom absoluten Raum}; Gesnerus, 
%        11.\ Jahrgang (1954), S.\,62--120.
%\bibitem{Fliessbach} Torsten Flie\ss bach; {\it Allgemeine 
%        Relativit\"atstheorie}; BI-Wissenschaftsverlag, Mannheim, Wien
%        Z\"urich, 1990. 
%\bibitem{Galilei} Galilei; {\it Dialog \"uber die beiden haupts\"achlichen
%        Weltsysteme, das ptolem\"aische und das kopernikanische}; 
%        Teubner Stuttgart, 1982; aus dem Italienischen \"ubersetzt von
%        Emil Strauss.   
%  \bibitem{Hawking} Stephen W.\ Hawking, \textit{Particle Creation by
%            black holes}, Comm.\ Math.\ Phys.\ 43 (1976) 199--220.      
%\bibitem{Helmholtz2} Hermann von Helmholtz; {\em \"Uber Wirbelbewegungen,
%        \"Uber Fl\"ussigkeitsbewegungen}, 1858; in Ostwalds Klassiker der 
%       exakten Wissenschaften Bd.\ 1; Verlag Harri Deutsch, Frankfurt, 
%       1996.                   
%\bibitem{Lamb} G.L.\ Lamb, Jr.; {\it Elements of Soliton Theory}; 
%         Pure \& Applied Mathematics, John Wiley \& Sons, 1980. 
%\bibitem{Laue} Max von Laue; {\it Geschichte der Physik}; 
%         Universit\"ats-Verlag Bonn, 1947.
%\bibitem{Lorentz} Hendrik Antoon Lorentz; {\it Electromagnetic phenomena 
 %        in a system moving with any velocity smaller than that of light}; 
%         Proc.\ Acad.\ Sci., Amsterdam, 6 [1904], S.\ 809.
%\bibitem{Mach} Ernst Mach; {\it Die Mechanik in ihrer Entwicklung
%      historisch kritisch dargestellt}; Akademie Verlag, Berlin, 1988.       
%\bibitem{Mainzer} Klaus Mainzer; {\it Philosophie und Geschichte von
%         Raum und Zeit}; in {\it Philosophie und Physik der Raum-Zeit};
%         J\"urgen Audretsch und Klaus Mainzer (Hrsg.); 
%         BI-Wissenschaftsverlag, 1994. 
%\bibitem{Microscope} Touboul, Pierre, et al. (MICROSCOPE Collaboration); \textit{MICROSCOPE Mission:
%         Final Results of the Test of the Equivalence Principle}; Phys.\ Rev.\ Lett.\ \textbf{129} (2022) 121102.
%\bibitem{Misner} C.W.\ Misner, K.S.\ Thorne, J.A.\ Wheeler; 
%        {\it Gravitation}; W.H.\ Freeman and Company, San Francisco,  1973.
%\bibitem{Mittelstaedt} Peter Mittelstaedt; {\it Der Zeitbegriff in der
%        Physik}; BI-Wissenschaftsverlag, 1989.        
%\bibitem{Mittelstaedt2} Peter Mittelstaedt; {\it Philosophische Probleme
%        der modernen Physik}; BI-Wissenschaftsverlag, 1989.        
%\bibitem{Newton}
%   Isaac Newton; {\it Mathematische Grundlagen der Naturphilosophie}; 
%   \"ubersetzt von Ed Dellian; Felix Meiner Verlag, 1988. 
%\bibitem{Newton2} Isaac Newton; {\it \"Uber die Gravitation...};
%       Klostermann Texte Philosophie; Vittorio Klostermann, Frankfurt,
%      1988; \"ubersetzt von Gernot B\"ohme.
%\bibitem{Newton3} Isaac Newton; {\it Optik oder Abhandlung \"uber
%      Spiegelungen, Brechungen, Beugungen und Farben des Lichts};
%      I., II.\ und III.\ Buch (1704); aus dem Englischen \"ubersetzt
%      von W.\ Abendroth; Ostwalds Klassiker der exakten Wissenschaften,
%      Verlag Harri Deutsch 1998.   
%\bibitem{Neumann} Carl Neumann; {\it \"Uber die Principien der
%         Galilei-Newtonschen Theorie}; Akademische Antrittsvorlesung,
%         gehalten in der Aula der Universit\"at Leipzig am 3.\ Nov.\
%         1869; Teubner (Leipzig) 1870.         
%\bibitem{Pauli} Wolfgang Pauli; {\it Theory of Relativity}; Dover
%      Publications, New York, 1981.      
%\bibitem{Poincare} Jules Henri Poincar\'e; {\it Sur la dynamique de 
%     l'\'electron}, C.R.\ Acad.\ Sci., Paris, 140 (1905) S.~1504; und 
%      Rendiconti del Circolo Matematico di Palermo, Bd.~21 (1906) S.~129.
%\bibitem{Reichenbach1} Hans Reichenbach; {\em Philosophie der 
%       Raum-Zeit-Lehre}; Hans Reichenbach - Gesammelte Werke Bd.\ 2;
%       Vieweg-Verlag, Braunschweig; 1977.
%\bibitem{Reichenbach2} Hans Reichenbach; {\em Axiomatik der
%       relativistischen Raum-Zeit-Lehre}; in {\em Die philosophische
%       Bedeutung der Relativit\"atstheorie}; Hans Reichenbach - Gesammelte
%       Werke Bd.\ 3; Vieweg-Verlag, Braunschweig, 1977. 
%\bibitem{Rovelli} Carlo Rovelli, \textit{Quantum Gravity}; Cambridge
%      University Press, 2007.       
%\bibitem{Schlamminger} Schlamminger, Choi, Wagner, Gundlach,
%         Adelberger; {\em Test of the Equivalence Principle using a
%         rotating torsion balance}; Phys.\ Rev.\ Lett.\ {\bf 100} (2008)
%         041101.     
%\bibitem{Sexl} Roman U.\ Sexl, Helmuth K.\ Urbantke; {\it Relativit\"at,
%      Gruppen, Teilchen}; Springer-Verlag, Wien, New York, 1992.
%\bibitem{Simonyi}
%       K\'aroly Simonyi; {\it Kulturgeschichte der Physik}; Verlag
%       Harri Deutsch, Thun, Frankfurt am Main, 1990.
%\bibitem{Weisberg} Weisberg, J.M., Taylor, J.H.; {\em Relativistic Binary Pulsar
%          B1913+16: Thirty Years of Observations and Analysis}; 
%          \verb+arXiv:astro-ph/0407149v1+; 2004. 
%\bibitem{Thomson} James Thomson; {\it On the Law of Inertia; the
%       Principle of Chronometry; and the Principle of Absolute Clinural
%       Rest, and of Absolute Rotation}; Proc.\ Roy.\ Soc.\ (Edinburgh),
%       Session 1883-84, Vol.\ XII, 568--578.       
%\bibitem{Weizsaecker} Carl Friedrich von Weizs\"acker; {\em Der zweite
%      Hauptsatz und der Unterschied von Vergangenheit und Zukunft};
%      Annalen der Physik 36 (1939) 275--283.       
%\bibitem{Zeh} Zeh, H.D.; {\em The Physical Basis of the Direction of Time},
%      Springer-Verlag, Berlin, 1989.       

%\bibitem{Einstein} Einstein, Albert; {\em ??}, .                   
\end{thebibliography}

\end{document}
