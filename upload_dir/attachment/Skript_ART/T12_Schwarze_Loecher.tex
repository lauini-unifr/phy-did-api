\documentclass[german,10pt]{book}      
\usepackage{makeidx}
\usepackage{babel}            % Sprachunterstuetzung
\usepackage{amsmath}          % AMS "Grundpaket"
\usepackage{amssymb,amsfonts,amsthm,amscd} 
\usepackage{mathrsfs}
\usepackage{rotating}
\usepackage{sidecap}
\usepackage{graphicx}
\usepackage{color}
\usepackage{fancybox}
\usepackage{tikz}
\usetikzlibrary{arrows,snakes,backgrounds}
\usepackage{hyperref}
\hypersetup{colorlinks=true,
                    linkcolor=blue,
                    filecolor=magenta,
                    urlcolor=cyan,
                    pdftitle={Overleaf Example},
                    pdfpagemode=FullScreen,}
%\newcommand{\hyperref}[1]{\ref{#1}}
%
\definecolor{Gray}{gray}{0.80}
\DeclareMathSymbol{,}{\mathord}{letters}{"3B}
%
\newcounter{num}
\renewcommand{\thenum}{\arabic{num}}
\newenvironment{anmerkungen}
   {\begin{list}{(\thenum)}{%
   \usecounter{num}%
   \leftmargin0pt
   \itemindent5pt
   \topsep0pt
   \labelwidth0pt}%
   }{\end{list}}
%
\renewcommand{\arraystretch}{1.15}                % in Formeln und Tabellen   
\renewcommand{\baselinestretch}{1.15}                 % 1.15 facher
                                                      % Zeilenabst.
\newcommand{\Anmerkung}[1]{{\begin{footnotesize}#1 \end{footnotesize}}\\[0.2cm]}
\newcommand{\comment}[1]{}
\setlength{\parindent}{0em}           % Nicht einruecken am Anfang der Zeile 

\setlength{\textwidth}{15.4cm}
\setlength{\textheight}{23.0cm}
\setlength{\oddsidemargin}{1.0mm} 
\setlength{\evensidemargin}{-6.5mm}
\setlength{\topmargin}{-10mm} 
\setlength{\headheight}{0mm}
\newcommand{\identity}{{\bf 1}}
%
\newcommand{\vs}{\vspace{0.3cm}}
\newcommand{\noi}{\noindent}
\newcommand{\leer}{}

\newcommand{\engl}[1]{[\textit{#1}]}
\parindent 1.2cm
\sloppy

         \begin{document}  \setcounter{chapter}{6}


\chapter{Schwarze L\"ocher}
% Kap x
\label{chap_BlackHole}

Kaum ein anderes physikalisches System hat die Phantasie der Menschen
mehr angeregt als Schwarze L\"ocher. Auf der einen Seite fl\"o\ss en sie Angst ein bei
der Vorstellung, dass man einem solchen Schwarzen Loch nie mehr entrinnen kann,
sobald der Horizont einmal \"uberschritten ist, andererseits sind manche von ihnen aber auch
ein potenzieller Eingang zu sogenannten Wurml\"ochern - Abk\"urzungen zu anderen
Bereichen des Universums oder m\"oglicherweise auch anderen Universen. 

Nachdem Albert Einstein 1914 seine Allgemeinen Relativit\"atstheorie in
wesentlichen Z\"ugen aufgestellt hatte,
fand Karl Schwarzschild sehr schnell eine L\"osung der Einstein'schen
Gleichungen, die f\"ur gro\ss e Abst\"ande die Raumzeitgeometrie einer konzentrierten Masse
(z.B.\ eines Sterns oder Planeten) beschreibt. Diese L\"osung hat zwei Singularit\"aten, d.h.\
Werte der Koordinaten, bei denen Freiheitsgrade der Theorie, die Komponenten des
metrischen Feldtenors $g_{\mu \nu}(x)$, unendlich werden. Dies betrifft einmal die
Koordinate $r$, die asymptotisch f\"ur gro\ss e Abst\"ande einen radialen Abstand vom Ursprung
beschreibt, sowie die Koordinate $t$, die asymptotisch f\"ur gro\ss e Abst\"ande der Eigenzeitkoordinate
entspricht. Neben einer Singularit\"at bei $r=0$ gibt es eine Singularit\"at bei einem endlichen
Abstand, dem sogenannten Schwarzschild-Radius oder auch Horizont des Schwarzen Lochs
Die Bedeutung dieser zweiten Singularit\"at blieb lange unklar.

Zun\"achst glaubte man nicht, dass diese rein formalen mathematischen L\"osungen
in irgendeiner Form in unserem Universum realisiert seien. Erst rund f\"unfzig Jahr sp\"ater
mehrten sich die Anzeichen, dass es tats\"achlich astrophysikalische Objekte geben k\"onnte,
die die Eigenschaften der Schwarzschild-L\"osung haben. Anf\"anglich handelte es sich jedoch
um reine Vermutungen. Im Jahre 2020 wurde der Nobelpreis f\"ur die Untersuchung von
Schwarzen L\"ochern vergeben: an Andrea Ghez und Reinhard Genzel f\"ur ihre Untersuchungen
des Schwarzen Lochs im Zentrum unserer Milchstra\ss e, und an Roger Penrose f\"ur seine
mathematischen Untersuchungen zu den Singularit\"aten in L\"osungen der Einstein'schen
Gleichungen. Heute sind wird \"uberzeugt, dass es sowohl stellare Schwarze L\"ocher gibt (mit
einer Masse von ein bis 10 Sonnenmassen), mittelschwere Schwarze L\"ocher (mit Massen von
rund einhundert Sonnenmassen) wie auch superschwere Schwarze L\"ocher (mit Millionen bis
Milliarden Sonnenmassen). 

Als Vorbereitung zu den mathematischen Er\"orterungen in diesem Kapitel empfiehlt sich das
Kapitel zur Einf\"uhrung in die Allgemeine Relativit\"atstheorie. Es wird vorausgesetzt, dass man
wei\ss, was der metrische Feldtensor ist und von welcher Struktur die Einstein'schen
Feldgleichungen sind.  

\section{Die Einstein-Gleichungen}

\begin{equation}
              {\rm d}s^2 = g(x)_{\mu \nu}\, {\rm d} x^\mu\, {\rm d} x^\nu  \hspace{2cm} \mbox{Einstein'sche Summenkonvention!}
\end{equation}
 
Die Einstein'schen Feldgleichungen stellen eine Beziehung her zwischen den geometrischen
Eigenschaften der Raumzeit, ausgedr\"uckt durch den metrischen Feldtensor $g_{\mu \nu}(x)$  
und dem Energie-Impuls-Tensor $T_{\mu \nu}$ der Materie. Die Einstein'schen Feldgleichungen lauten:
\begin{equation}
            G_{\mu \mu} + \Lambda g_{\mu \nu} = \kappa T_{\mu \nu}
\end{equation}
mit
\begin{equation}
            G_{\mu \mu}  = R_{\mu \nu} - \frac{1}{2} R g_{\mu \nu} \, ,
\end{equation}
wobei $R_{\mu \nu}$ der Ricci-Tensor und $R$ die skalare Kr\"ummung sind. Der Ricci-Tensor ist eine
Verk\"urzung (Spur) des Kr\"ummungstensors. 
Die Kopplungskonstante $\kappa$ enth\"alt Newtons Gravitationskonstante $G$:
\begin{equation}
                 \kappa = \frac{8 \pi G}{c^4} \approx  2,07665 \cdot 10^{-43} \,  {\rm N}^{-1} \, .
\end{equation}
$\Lambda$ bezeichnet man als kosmologische Konstante. 

Lange Zeit glaubte man, dass
$\Lambda=0$, doch nachdem seit der Mitte der 90er Jahre des letzten Jahrhunderts die Beobachtungen
an Supernovae nahe legen, dass sich unser Universum beschleunigt ausdehnt, erkl\"art man dies meist
mit einer positiven kosmologischen Konstanten. 

\section{Die Schwarzschild-L\"osung}

Die Schwarzschild-L\"osung der Einstein'schen Feldgleichungen ist eine sogenannte Vakuuml�sung,
d.h.\ man setzt $T_{\mu \nu}=0$. Au\ss erdem wird $\Lambda =0$ gesetzt. Damit werden die
Einstein-Gleichungen zu
\begin{equation}
            R_{\mu \nu}  = - \frac{1}{2}  R  g_{\mu \nu} \, .
\end{equation}
Da der Ricci-Tensor und die skalare Kr\"ummung quadratisch von der Metrik und ihren Ableitungen
abh\"angen, handelt es sich um ein nicht-lineares System gekoppelter Differentialgleichungen.
Allgemeine L\"osungen gibt es nicht. 

Die Schwarzschild-L\"osung beruht auf folgenden Annahmen: Die L\"osung soll rotationssymmetrisch
und statisch sein (also keine Zeitabh\"angigkeit haben).  





\section{L\"osungen mit Drehimpuls und Ladung}

\section{Die Suche nach Schwarzen L\"ochern}

\section{Die ersten Bilder von Schwarzen L\"ochern}

\end{document}

