\documentclass[german,10pt]{book}   
\usepackage{makeidx}
\usepackage{babel}            % Sprachunterstuetzung
\usepackage{amsmath}          % AMS "Grundpaket"
\usepackage{amssymb,amsfonts,amsthm,amscd} 
\usepackage{mathrsfs}
\usepackage{rotating}
\usepackage{sidecap}
\usepackage{graphicx}
\usepackage{color}
\usepackage{fancybox}
\usepackage{tikz}
\usetikzlibrary{arrows,snakes,backgrounds}
\usepackage{hyperref}
\hypersetup{colorlinks=true,
                    linkcolor=blue,
                    filecolor=magenta,
                    urlcolor=cyan,
                    pdftitle={Overleaf Example},
                    pdfpagemode=FullScreen,}
%\newcommand{\hyperref}[1]{\ref{#1}}
%
\definecolor{Gray}{gray}{0.80}
\DeclareMathSymbol{,}{\mathord}{letters}{"3B}
%
\newcounter{num}
\renewcommand{\thenum}{\arabic{num}}
\newenvironment{anmerkungen}
   {\begin{list}{(\thenum)}{%
   \usecounter{num}%
   \leftmargin0pt
   \itemindent5pt
   \topsep0pt
   \labelwidth0pt}%
   }{\end{list}}
%
\renewcommand{\arraystretch}{1.15}                % in Formeln und Tabellen   
\renewcommand{\baselinestretch}{1.15}                 % 1.15 facher
                                                      % Zeilenabst.
\newcommand{\Anmerkung}[1]{{\begin{footnotesize}#1 \end{footnotesize}}\\[0.2cm]}
\newcommand{\comment}[1]{}
\setlength{\parindent}{0em}           % Nicht einruecken am Anfang der Zeile 

\setlength{\textwidth}{15.4cm}
\setlength{\textheight}{23.0cm}
\setlength{\oddsidemargin}{1.0mm} 
\setlength{\evensidemargin}{-6.5mm}
\setlength{\topmargin}{-10mm} 
\setlength{\headheight}{0mm}
\newcommand{\identity}{{\bf 1}}
%
\newcommand{\vs}{\vspace{0.3cm}}
\newcommand{\noi}{\noindent}
\newcommand{\leer}{}

\newcommand{\engl}[1]{[\textit{#1}]}
\parindent 1.2cm
\sloppy

      \begin{document}   \setcounter{chapter}{1} 
 
\chapter{Differentialgeometrische\\ 
Grundlagen der ART}

In diesem Kapitel behandeln wir die mathematischen
Grundlagen der allgemeinen Relativit\"atstheorie. Damit
sind insbesondere die differentialgeometrischen
Konzepte gemeint, die in die Einstein'schen
Feldgleichungen eingehen bzw.\ zur Formulierung
der Theorie ben\"otigt werden. Es geht mir dabei weder
um mathematische Vollst\"andigkeit noch um
mathematische Strenge. Der Schwerpunkt
soll in einer Veranschaulichung der Konzepte
bestehen. Wie so oft, ist gerade der Einstieg
-- das bedeutet die Definition von Mannigfaltigkeiten
und ihren Tangentialr\"aumen, insbesondere wenn
man keinen Bezug auf eine Einbettung dieser
Mannigfaltigkeit nehmen m\"ochte --
am schwierigsten.

\section{Mannigfaltigkeiten}

Etwas vereinfacht ist eine Mannigfaltigkeit $M$ 
ein\index{Mannigfaltigkeit}
topologischer Raum (also eine Menge mit einer
Topologie, f\"ur die wir im Folgenden immer eine
gew\"ohnliche Hausdorff-Topologie nehmen wollen),
der lokal (also in offenen Umgebungen von jedem Punkt)
isomorph zu offenen Umgebungen im
$\mathbb{R}^n$ ist. Eine solche Isomorphie bezeichnet
man als {\em Karte} und die Menge aller
Karten als einen {\em Atlas}. Dies wird im
Folgenden etwas genauer behandelt.

Zu einer Mannigfaltigkeit k\"onnen wir
an jedem Punkt ihren Tangentialraum
definieren als \glqq Raum der
Geschwindigkeiten von parametrisierten
Bahnkurven durch diesen Punkt\grqq. 

\subsection{Karten und Atlanten}

Die Metapher von Karten in einem Atlas werden wir oft
zur Veranschaulichung verwenden, da sie den
Sachverhalt sehr gut trifft. S\"amtliche geometrischen
Eigenschaften von $M$ werden \"uber diese
Karten definiert, und letztendlich werden wir es
immer nur mit offenen Umgebungen im $\mathbb{R}^n$
zu tun haben, aus denen wir die geometrischen
Eigenschaften von $M$ ablesen k\"onnen.

Etwas genauer definieren wir eine Karte
$(U,\phi)$ als\index{Karte} 
eine offene Teilmenge $U$
der Mannigfaltigkeit $M$
und eine stetige, injektive Abbildung (also
einen lokalen Isomorphismus)
\begin{equation}
         \phi : U\subset M \rightarrow \mathbb{R}^n  \, .
\end{equation}
Da $\phi$ injektiv (lokal bijektiv) sein soll, gibt es auch
eine Umkehrabbildung $\phi^{-1}$ 
von dem Bild von $\phi$
im $\mathbb{R}^n$ zur\"uck in die Mannigfaltigkeit.
Wir k\"onnen also bei einer Karte nahezu beliebig 
zwischen offenen Umgebungen in $M$ und den
zugeh\"origen offenen Umgebungen im $\mathbb{R}^n$ 
-- den eigentlichen Karten -- wechseln. 

Eine Menge von Karten $\{U_i,\phi_i\}$ bezeichnen
wir als\index{Atlas (Differentialgeometrie)} 
{\em Atlas}, sofern $\bigcup U_i = M$ und
f\"ur je zwei Abbildungen 
$\phi_1: U_1 \rightarrow \mathbb{R}^n$ und
$\phi_2: U_2 \rightarrow \mathbb{R}^n$, 
sodass $U_1 \cap U_2$ nicht leer ist,
die Abbildung 
\begin{equation}
  \phi_2 \circ \phi_1 ^{-1}:
     \phi_1(U_1\cap U_2) \rightarrow \mathbb{R}^n 
\end{equation}
eine beliebig oft differenzierbare Abbildung
von der offenen Teilmenge 
$\phi_1(U_1\cap U_2)\subset \mathbb{R}^n$
in den $\mathbb{R}^n$ ist. Diese Eigenschaft
ist sehr intuitiv: In einem Atlas findet man m\"oglicherweise
mehrere Karten, auf denen dasselbe Gebiet (Land
oder Gegend) dargestellt ist. Diese beiden Darstellungen
sollen nat\"urlich \glqq isomorph\grqq\ sein und zwar
beliebig oft ableitbar.% 
%Man bezeichnet solche Abbildungen auch als\index{Diffeomorphismus} {\em Diffeomorphismen}.
\footnote{In der Mathematik
schr\"ankt man den Grad der Ableitbarkeit manchmal
ein und verlangt nur $r$-mal stetig differenzierbar etc.
Umgekehrt kann man noch st\"arker auch Formen von
Analytizit\"at fordern, beispielsweise dass sich die 
Abbildungen beliebig genau durch ihre Taylor-Reihen 
ann\"ahern lassen. Im
Folgenden setzen wir immer alle notwendigen 
Ableitbarkeitseigenschaften voraus.}

\"Uber die Karten sind wir also in der Lage, 
auf der Mannigfaltigkeit so etwas wie
Differential- und Integralrechnung zu betreiben. 
Dies w\"are auf einem einfachen topologischen Raum 
nicht m\"oglich (dort ist nur der Begriff der Stetigkeit
definiert). In der Mathematik spricht man daher
auch schon mal von einer {\em differenzierbaren
Struktur}, die auf $M$ definiert wurde (es
m\"ussen noch bestimmte Konsistenzbedingungen
erf\"ullt sein, auf die ich hier nicht eingehe). 

Haben s\"amtliche Abbildungen 
$\phi_i\circ \phi_j^{-1}$, die sich konstruieren
lassen, die Eigenschaft, dass ihre Jacobi-Determinante
positiv ist 
(wegen der lokalen Bijektivit\"at kann die Jacobi-Determinate
an keinem Punkt verschwinden, sodass sie im gesamten 
Definitionsbereich entweder positiv oder negativ sein muss),
dann spricht man auch von einem\index{orientierbar} 
{\em orientierungserhaltenden Atlas}. Gibt es einen
solchen Atlas, so nennt man die Mannigfaltigkeit
$M$ {orientierbar}. Bekannt ist das M\"obius-Band, das 
in diesem Sinne nicht orientierbar ist.

\subsection{Der Tangentialraum}

An jedem Punkt $p$ einer Mannigfaltigkeit $M$ k\"onnen
wir den Tangentialraum $T_pM$ definieren. Auch 
diese Konstruktion ist eigentlich sehr anschaulich:
Der Tangentialraum\index{Tangentialraum} 
ist der Raum aller Geschwindigkeiten,
die Bahnkurven durch den Punkt $p$ haben k\"onnen.

Ein parametrisierter\index{Weg, parametrisierter} 
Weg $\gamma: I\rightarrow M$ 
ist eine stetige Abbildung von einem Intervall $I\subset \mathbb{R}$ 
in die Mannigfaltigkeit $M$, oft schreiben wir
$t \mapsto \gamma(t)\in M$. 
Im Folgenden soll das Intervall $I$ den Punkt $0$ enthalten
und wir betrachten parametrisierte
Wege $\gamma$, f\"ur die $\gamma(0)=p$, die also bei
$t=0$ durch den Punkt $p$ gehen. 

Wir w\"urden nun gerne die Ableitung des Weges nach
$t$ an der Stelle $t=0$ als eine Tangente an die
Mannigfaltigkeit $M$ im Punkte $p$ definieren, doch
wir k\"onnen auf $M$ nicht ableiten. Daher bedient
man sich eines Tricks: Wir betrachten s\"amtliche 
beliebig oft ableitbaren\footnote{Streng genommen k\"onnen wir von
Funktionen von einem topoligschen Raum $M$ 
nach $\mathbb{R}$ nur  entscheiden, ob sie stetig sind. 
Ableitbarkeit bezieht sich immer auf die differenzierbar
Stuktur auf $M$, die durch die Karten gegeben ist,
sie bezieht sich also auf Funktionen $f\circ \phi^{-1}$,
die Abbildungen von $\mathbb{R}^n$ 
nach $\mathbb{R}$ sind.} Funktionen 
$f:U \rightarrow \mathbb{R}$ (wobei $U$ eine
Umgebung von $p$ sein soll; man beachte, dass auch die
Kartenabbildungen $\phi$, eingeschr\"ankt auf eine ihrer
Komponenten im Bildraum, solche Funktionen 
definieren).  Jeder Weg $\gamma$ und
jede solche Funktion $f$ definiert eine Funktion
$f \circ \gamma: I \rightarrow \mathbb{R}$. Diese
Funktion k\"onnen wir nach dem Argument $t$ ableiten. 
Nun definieren wir auf der Menge aller
Wege (durch $p$) eine \"Aquivalenzrelation:
Zwei Wege $\gamma_1$ und $\gamma_2$ bezeichnen
wir als \"aquivalent, wenn 
f\"ur alle stetigen Funktionen $f:U\rightarrow
\mathbb{R}$ gilt: 
\begin{equation}
     \left.    \frac{{\rm d}}{{\rm d}t} f(\gamma_1(t)) \right|_{t=0}
     =
     \left.    \frac{{\rm d}}{{\rm d}t} f(\gamma_2(t)) \right|_{t=0} \, .
\end{equation}
Eine solche \"Aquivalenzklasse definiert einen
Tangentialvektor\index{Tangentialvektor} 
in $p$ und die Menge aller solcher 
\"Aquivalenzklassen ist der Tangentialraum $T_pM$
an $M$ im Punkte $p$.

Das Verfahren erscheint nur auf den ersten Blick
unn\"otig kompliziert. Der Vorteil der skizzierten
Vorgehensweise liegt darin, dass die Mannigfaltigkeit
$M$ nicht in einen $\mathbb{R}^m$ (mit $m\geq n$)
eingebettet sein muss; diesen Fall betrachten wir
im n\"achsten Abschnitt als Beispiel. Die obigen Definitionen
(wie auch die weiteren geometrischen Konzepte,
insbesondere die Metrik) sind rein intrinsisch.
Mathematisch kann man daher von
\glqq gekr\"ummten R\"aumen\grqq\ sprechen,
ohne dass es einen (euklidischen) Raum geben
muss, {\em in} dem die Mannigfaltigkeit gekr\"ummt
ist. 

Noch ein Wort zur Notation: Das Bild eines Punktes 
$p\in M$ in einer bestimmten Karte ist $\phi(p)$,
die dadurch induzierten {\em Koordinaten} von $p$
bezeichnen wir aber meist mit $x^\mu$ bzw.\
$x^\mu(p)$. Ein Weg in $M$ wird durch
$\gamma(t)$ gekennzeichnet, sein Bild in einer
Karte meist kurz durch $x(t)$. Die Ableitung
eines solchen Weges, $\frac{{\rm d}x(t)}{{\rm d}t}$
ist damit ein Element des Tangentialraums.
Allgemeine Elemente des Tangentialraums 
bezeichne ich
oftmals mit Gro\ss buchstaben, also $X$ mit
Komponenten $X^\mu$ oder $X(p)$ mit 
Komponenten $X(p)^\mu$, da die Kleinbuchstaben
schon f\"ur die Koordinaten von Punkten
verwendet werden. 

\subsection{Eingebettete Mannigfaltigkeiten}
\label{sec_Einbett}

Wir betrachten nun den Spezialfall, dass die
Mannigfaltigkeit\index{Mannigfaltigkeit!eingebettete} 
$M$ eine Untermenge
eines euklidischen Raumes ist, beispielsweise
eine 2-dimensionale Kugeloberfl\"ache, die
als Teilmenge des $\mathbb{R}^3$ aufgefasst
wird und durch die Bedingung
\begin{equation}
          x^2 + y^2 + z^2 = R^2
\end{equation}
definiert ist.\index{Kugeloberfl\"ache!Koordinaten} 
Die \"ubliche Karte f\"ur\index{Kugeloberfl\"ache}
die Kugeloberfl\"ache f\"uhrt zur
Parametrisierung Winkel ein:
\begin{equation}
     x= R \cos \varphi \cos \theta  ~~~ 
     y= R  \sin \varphi \cos \theta ~~~ 
     z= R \sin \theta  \, .
\end{equation}
Ein Vergleich mit einer Weltkarte zeigt, dass
$\varphi\in [0,2\pi)$ dem L\"angengrad und 
$\theta \in (-\pi/2,+\pi/2)$ dem Breitengrad 
entspricht.\footnote{Gew\"ohnlich definiert man 
$\theta\in (0,\pi)$,
wobei $\theta=0$ dem Nord- und $\theta=\pi$
dem S\"udpol entspricht; daher unterscheiden
sich die Formeln hier von denen in den
\"ublichen Formelsammlung dadurch, dass
$\cos \theta$ durch $\sin \theta$ etc.\ zu
ersetzen ist.}
Die Karte verliert ihre G\"ultigkeit am Nord- und 
S\"udpol (dort br\"auchte man andere Karten),
da f\"ur $\theta = \pm \pi /2$  
die Punkte unabh\"angig vom Winkel $\varphi$ 
bereits festliegen. Anders ausgedr\"uckt, alle
Punkte $(\varphi,\pm \pi /2)$ haben als Bild
den Nord- bzw.\ S\"udpol, damit w\"are hier die Beziehung 
zwischen den Punkten auf der Kugel und den
Punkten auf der Karte (w\"urde man die Werte
$\theta=\pm \pi /2$ zulassen)
nicht mehr bijektiv.

Unsere Kartenabbildung lautet somit
\begin{equation}
   \phi~:~ (x,y,z) \rightarrow (\varphi,\theta) 
\end{equation} 
mit
\begin{equation}
   \varphi = \arctan \frac{y}{x}  \hspace{1cm}
    \theta = \arctan \frac{z}{\sqrt{x^2+y^2}}  \, .  
\end{equation}
(Die genauen Definitionsbereiche
und \glqq \"Aste\grqq, die man beim inversen
Tangens zu nehmen hat, sind
technische Details.)

Da die Kugeloberfl\"ache in den $\mathbb{R}^3$ 
eingebettet ist, liegen die Ableitungsvektoren,
die man von Bahnkurven auf der Kugeloberfl\"ache
erh\"alt, ebenfalls im $\mathbb{R}^3$. Man kann
auf diese Weise den Tangentialraum an einen
Punkt der Kugeloberfl\"ache als affinen
Teilraum des $\mathbb{R}^3$ konstruieren.
Das ist allerdings etwas irref\"uhrend, denn
der Tangentenraum an einem Punkt ist
ein Vektorraum (und nicht ein affiner
Vektorraum), d.h., der Nullpunkt des Tangentenraums
ist im Allgemeinen nicht der Nullpunkt des
einbettenden $\mathbb{R}^3$. Streng genommen
handelt es sich um einen anderen Raum:
Die Kugeloberfl\"ache mit all ihren 
Tangentenr\"aumen ist ein 4-dimensionaler
Raum und nicht der einbettende 3-dimensionale
Raum. 

Statt nun s\"amtliche Bahnkurven durch einen
Punkt zu betrachten, kann man einfach
die beiden Kurven durch einen Punkt $p$
(parametrisiert durch $(\varphi_p,\theta_p)$) betrachten, 
bei einen eine der Koordinaten festgehalten wird,
d.h.\ man betrachtet die beiden Kurven, die
den Koordinatenlinien durch $p$ entsprechen.
Die Ableitungsvektoren sind dann
(bei der Kugeloberfl\"ache):
\begin{eqnarray}
\label{eq_Basphi}
  \vec{e}_\varphi &=& \left.  \frac{\partial}{\partial \varphi} \vec{x}(\varphi,\theta_p)
   \right|_{\varphi=\varphi_p}  =
   R(-\sin \varphi_p \cos \theta_p, \cos \varphi_p \cos \theta_p, 0) \\
\label{eq_Bastheta}   
  \vec{e}_\theta &=&  \left.  \frac{\partial}{\partial \theta} \vec{x}(\varphi_p,\theta)
   \right|_{\theta=\theta_p}  =
   R(-\cos \varphi_p \sin \theta_p, -\sin \varphi_p \sin \theta_p,
   \cos \theta )  \,  .
\end{eqnarray}
Diese beiden speziellen Tangentialvektoren spannen
den Tangentialraum am Punkte $p$ auf. Die
Ableitung von einem beliebigen Weg durch $p$
am diesem Punkt (also ein beliebiger Tangentialvektor)
l\"asst sich immer als Linearkombination dieser beiden
Vektoren schreiben.

Wir k\"onnen das oben Gesagte nun leicht
verallgemeinern. Sei der einbettende Raum
$\mathbb{R}^m$ und die Mannigfaltigkeit
$M$ durch eine Abbildung
\begin{equation}
     \phi^{-1} : (u_1, ..., u_n) \mapsto \vec{x}(u_1,...,u_n)
\end{equation}
gegeben. Die Koordinaten $\{(u_1,...,u_n)\}$ parametrisieren
also die Mannigfaltigkeit $M$ als Unterraum des 
$\mathbb{R}^m$. An jedem Punkt definieren die
Tangentialvektoren
\begin{equation}
\label{eq_TangBas}
      \vec{e}_i = \frac{\partial \vec{x}(u_1,...,u_n)}{\partial u_i}
      \hspace{2cm} (i=1,...,n)
\end{equation}
eine Basis, welche den Tangentialraum aufspannen.

Man beachte, dass diese Basis im Allgemeinen
keine normierte oder orthogonale Basis sein muss
(siehe Abschnitt \ref{sec_InduzMet}).

\section{Die Metrik}

Die fundamentale Struktur der Allgemeinen Relativit\"atstheorie ist
\index{Metrik}%
das metrische Feld $g_{\mu \nu}(x)$, wobei $x$ eine beliebige
Parametrisierung der Raumzeit darstellt. 
Ein solches Feld ordnet somit
jedem Punkte $p$ der Raumzeit (also jedem Ereignis) -- 
dargestellt durch seine Koordinaten
$x=\phi(p)$ -- einen Tensor $g_{ \mu \nu}$ zu, 
mit dem wir \glqq Abst\"ande\grqq\ bestimmen k\"onnen. 
Technisch gesprochen ist $g_{\mu \nu}(x)$ eine 
(nicht-entartete) symmetrische Bilinearform 
auf dem Tangentenraum (der nat\"urlich ein Vektorraum ist)
am Punkte $p$. Das bedeutet, $g_{\mu \nu}$ ist auf zwei
Vektoren (Tangentialvektoren, also $\frac{{\rm d}x}{{\rm d}t}$
bez\"uglich eines Weges $x(t)$) anzuwenden, m\"ochte
man einen
invarianten, also vom Koordinatensystem unabh\"angigen
Ausdruck erhalten. 

Es gibt aber auch eine sehr 
anschauliche Beziehung: Seien $p$ und $p'$
zwei \glqq infinitesimal\grqq\ benachbarte Punkte der
Mannigfaltigkeit mit den Koordinaten $x^\mu$ und 
$x^{\prime \,\mu}$. Da die
Punkte sehr eng beieinander liegen, unterscheiden sich
auch die Koordinaten nur um infinitesimale 
Ausdr\"ucke ${\rm d}x^\mu$. Der Abstand ${\rm d}s$ zwischen
$p$ und $p'$ -- hier handelt es sich um einen 
Abstand auf der Mannigfaltigkeit --
ist dann gegeben durch
\begin{equation}
\label{eq_ds2}
     {\rm d}s^2 = g_{\mu \nu} \,{\rm d}x^\mu \,{\rm d}x^\nu
\end{equation} 
bzw.
\begin{equation}
     {\rm d}s = \sqrt{g_{\mu \nu} {\rm d}x^\mu {\rm d}x^\nu} \, .
\end{equation} 
In diesen Ausdr\"ucken wurde die Einstein'sche
Summenkonvention\index{Summenkonvention, Einstein'sche} 
angewand, d.h.\ \"uber doppelt
auftretende Indizes (einer oben und einer unten) ist
zu summieren. Wir werden diese Beziehung in Zukunft
h\"aufiger verwenden und auch noch genauer
untersuchen. Falls es irritiert, dass die Metrix $g_{\mu \nu}$,
die eigentlich auf Vektoren des Tangentialraums
angewandt wird, hier auf infinitesimale Koordinatendifferenzen
angewandt wird, kann man diese Gleichung f\"ur
eine beliebige Kurve $x(t)$ durch den Punkt $p$
(bei $t=0$) und den Punkt $p'$ (bei $t={\rm d}t$) auch in der Form
\begin{equation}
         {\rm d}s = \sqrt{g_{\mu \nu}\frac{{\rm d}x^\mu}{{\rm d}t}
         \frac{{\rm d}x^\nu}{{\rm d}t} } ~{\rm d}t =
         \sqrt{g_{\mu \nu} \dot{x}^\mu \dot{x}^\nu}~ {\rm d}t
\end{equation}
lesen. 

Die Raumzeit hat 
lokal die Struktur eines Minkowski-Raums, das bedeutet, auch
die Metrik hat die
Signatur der Minkowski-Metrik: sgn$(g)=(1,-1,-1,-1)$. Man spricht
in der Relativit\"atstheorie daher
auch manchmal von einer 
Pseudometrik\index{Pseudometrik}. 
Bei der angegebenen Wahl der Vorzeichen hat ein
raumartiger Vektor (der also au\ss erhalb des zuk\"unftigen
bzw.\ auf die Vergangenheit bezogenen Lichtkegels liegt)
eine \glqq negative Norm\grqq, d.h.\ der Ausdruck in Gl.\ \ref{eq_ds2}
wird negativ. Wollen wir einen sinnvollen 
r\"aumlichen Abstand
erhalten, m\"ussen wir daher bei raumartigen Ereignissen
das Negative dieses Ausdrucks w\"ahlen:
\begin{equation}
     {\rm d}l = \sqrt{-g_{\mu \nu} {\rm d}x^\mu {\rm d}x^\nu} \, .
\end{equation} 
Diese Problematik (dass es sich bei $g_{\mu \nu}$ nicht
wirklich um eine Metrik im mathematischen Sinne handelt)
spielt zwar in der Relativit\"atstheorie eine wichtige Rolle --
sie definiert insbesondere die kausale Lichtkegelstruktur --
ist aber vom differentialgeometrischen Standpunkt zum
Gl\"uck nicht so einschneidend, und die meisten Konzepte der
gew\"ohnlichen Differentialgeometrie lassen sich auf die
\glqq Pseudo-Riemann'sche Geometrie\grqq\ \"ubertragen.

Die Metrik erlaubt es, die (Pseudo)-L\"ange von Wegen anzugeben. F\"ur
einen zeitartigen Weg $x^\mu(t)$ 
ergibt sich beispielsweise als Eigenzeit\index{Eigenzeit}
\[  T ~=~  \int {\rm d}\tau = \int 
    \sqrt{g_{\mu \nu} \dot{x}^\mu \dot{x}^\nu}\,  {\rm d} t \;.\]   
Geod\"aten sind station\"are Punkte dieses Funktionals. Ein
Einfluss der Minkwoski-Struktur der Raumzeit ist,
dass es sich bei Geod\"aten
um Wege mit einer {\em maximalen} Eigenzeit (im Gegensatz
zu den Geod\"aten der ge\-w\"ohn\-lichen Geometrie, die
eine {\em minimale} L\"ange haben) handelt. 

In der Relativit\"atstheorie legt die Metrik auch die kausale
Struktur fest\index{kausale Struktur} 
(im Augenblick lassen wir globale topologische
Eigenschaften mal beiseite): Zu jedem Ereignis $p$ gibt es
den Zukunfts- und den\index{Lichtkegel} 
Vergangenheitslichtkegel.
Diese trennen Ereignisse, die mit $p$ durch
physikalische Weltlinien (mit zeitartigen Tangenten, also
$g_{\mu \nu}\dot{x}^\mu \dot{x}^\nu > 0$) verbunden 
werden k\"onnen, von
solchen Ereignissen, f\"ur die es solche Wege nicht gibt.
Zukunfts- und Vergangenheitslichtkegel unterscheiden
sich nochmals durch die Vorzeichen der 0-Komponente
solcher Wege. Innerhalb des Zukunftslichtkegels
befinden sich die Ereignisse, die von $p$ kausal beeinflusst
werden k\"onnen (zu diesen Ereignissen gibt es von $p$
aus Weltlinien mit nicht raumartigen Tangentenvektoren).
Umgekehrt liegen innerhalb des Vergangenheitslichtkegels
die Ereignisse, die $p$ kausal beeinflussen k\"onnen. 
Zu den Ereignissen au\ss erhalb des Lichtkegels gibt
es keine kausale Beziehung. Die kausale Struktur zwischen 
den Ereignissen einer Mannigfaltigkeit legt die Metrik schon
weitgehend fest: Die Freiheit besteht lediglich in einem
Skalarfeld, das die Metrik lokal mit einem Faktor multipliziert.
Allgemein wird in einer Karte der
Lichtkegel nun nicht mehr unbedingt durch gerade Linien
dargestellt. 

\subsection{Induzierte Metrik bei Einbettungen}
\label{sec_InduzMet}

Handelt es sich bei der Mannigfaltigkeit $M$
um eine in den $\mathbb{R}^m$ eingebettete
Untermannigfaltigkeit (vgl.\ Abschnitt \ref{sec_Einbett}),
die durch eine Parameterdarstellung definiert ist,
so bilden die Vektoren $\vec{e}_i$
(Gl.\ \ref{eq_TangBas}) eine Basis des Tangentialraums. 
Wie schon erw\"ahnt, handelt es sich im Allgemeinen
nicht um eine Orthonormalbasis. 
Daher definiert man eine symmetrische
Bilinearform
\begin{equation}
            g_{ij} = \vec{e}_i \cdot \vec{e}_j =
          \frac{\partial \vec{x}(u_1,...,u_n)}{\partial u_i} \cdot 
          \frac{\partial \vec{x}(u_1,...,u_n)}{\partial u_j} \, .
\end{equation}
Diese Gr\"o\ss e ist die durch die Einbettung in den
$\mathbb{R}^m$\index{Metrik!induzierte} 
{\em induzierte} Metrik im Tangentialraum
an jedem Punkt. 

Betrachten wir als Beispiel nochmals die
Kugel\index{Kugeloberfl\"ache!induzierte Metriken} 
und ihre Parametrisierung durch
L\"angengrad- und Breitengradwinkel
$(\varphi, \theta)$. Wir hatten die zugeh\"origen
Basisvektoren im Tangentialraum (an einem
Punkt, parametrisiert durch $(\varphi,\theta)$)
bereits berechnet (Gl.en \ref{eq_Basphi} und
\ref{eq_Bastheta}) und erhalten daraus f\"ur
die Metrik:
\begin{eqnarray}
   g_{\varphi \varphi} = \vec{e}_\varphi \cdot \vec{e}_\varphi
  = R^2 \cos^2 \theta   & &
  \hspace{2cm}  g_{\varphi \theta}=  \vec{e}_\varphi \cdot \vec{e}_\theta = 0 \\ 
    g_{\theta \varphi}=  \vec{e}_\theta \cdot \vec{e}_\varphi = 0  
    \hspace{1.3cm} \mbox{~} & & \hspace{2cm}
   g_{\theta \theta} = \vec{e}_\theta \cdot \vec{e}_\theta
  = R^2      
  \end{eqnarray}
oder
\begin{equation}
     g = R^2 \left( \begin{array}{cc}  \cos^2 \theta & 0 \\
     0 &  1 \end{array} \right)  \, .
\end{equation}
Statt dessen k\"onnen wir auch schreiben:
\begin{equation}
\label{eq_Kugelmetrik}
      {\rm d}s^2 =  R^2 \cos^2 \theta \, ({\rm d}\varphi)^2 +
         R^2 \, ({\rm d}\theta)^2  \, .
\end{equation}


\subsection{Projektionen und die Metrik von Landkarten}

Die Bedeutung der Metrik kann man sich sehr gut anhand einer
Landkarte verdeutlichen. Jede\index{Metrik!Ma\ss stabfaktor} 
Landkarte hat einen
Ma\ss stab, der angibt, wie gro\ss\ der Abstand
zwischen zwei Punkten in Wirklichkeit -- d.h.\ auf der
Erdoberfl\"ache -- ist, wenn der Abstand der zugeh\"origen
Punkte auf der Landkarte bekannt ist. Eine Angabe von
1:30\,000 besagt, dass 1 Zentimeter auf der Landkarte
in Wirklichkeit 30\,000 Zentimetern oder 300 Metern
entspricht. In gewisser Hinsicht entspricht die Metrik
einem solchen Ma\ss stabsfaktor.

Als Koordinaten w\"ahlt man auf der Erdoberfl\"ache
meist die L\"angen- und Breitengrade. In einer
Landkarte entsprechen die L\"angen- und Breitengrade
Linien, die oft horizontal und vertikal verlaufen, insbesondere
also senkrecht aufeinander stehen. Seien $\Delta x$ 
und $\Delta y$ die \glqq Koordinatenabst\"ande\grqq\
zwischen zwei nicht zu weit voneinander entfernten
Punkten auf der Landkarte, also die Abst\"ande 
entlang der Breiten- und L\"angengrade, dann ist
der physikalische (wirkliche) Abstand zwischen den beiden
Punkten gleich
\begin{equation}
         \Delta s^2 = m^2 \Big(  (\Delta x)^2 + (\Delta y)^2 \Big)
\end{equation}
wobei $m$ der oben genannte Ma\ss stabsfaktor ist.
In diesem Fall w\"are $g_{\mu \nu}= m^2 \delta_{\mu \nu}$,
das hei\ss t, die Metrik ist eine Diagonalmatrix und
die Diagonalelemente sind alle gleich und konstant
(gleich dem Quadrat des Ma\ss stabfaktors).
 
Bei Landkarten, die ein gr\"o\ss eres Gebiet
darstellen, reicht die Angabe eines festen Ma\ss stabs
nicht mehr aus. Die Erdoberfl\"ache
ist n\"aherungsweise eine Kugel, die nicht 
verzerrungsfrei (und, wie wir noch diskutieren 
werden, auch nicht \glqq singularit\"atenfrei\grqq) in 
einer Ebene dargestellt werden
kann. Das bedeutet, an unterschiedlichen Punkten
der Karte muss ein unterschiedlicher Ma\ss stab
angesetzt werden. Bei Karten, bei denen es auf
eine genaue Ortsangabe ankommt, beispielsweise
bei See- oder Flugkarten, sind daher manchmal 
Korrekturterme angegeben, denen man entnehmen
kann, wie sich der Ma\ss stab ver\"andert, wenn
man sich beispielsweise vom Zentrum der Karte
entfernt. Au\ss erdem kann der Ma\ss stab
noch von der Richtung abh\"angen. Bei vielen
Weltkarten ist der Ma\ss stab in Nord-S\"udrichtung
nahezu konstant, wohingegen der Ma\ss stab in
Ost-West-Richtung gerade in der N\"ahe der Pole
sehr variieren kann. Und w\"ahlt man schlie\ss lich
noch eine Kartendarstellung, bei der die
L\"angen- und Breitengrade nicht senkrecht
aufeinander stehen, muss man den allgemeinen
Kosinus-Satz anwenden, um aus den 
Koordinatendifferenzen den tats\"achlichen
Abstand berechnen zu k\"onnen, und damit
erh\"alt man in der Formel f\"ur $\Delta s^2$
auch Produktterme  der Form $\Delta x \Delta y$.

\subsection{Koordinatensingularit\"aten}

Koordinatensingularit\"aten 
sind\index{Koordinatensingularit\"aten}
singul\"are Bereiche einer Karte, die aber
keine Singularit\"at der Mannigfaltikeit
sind. Ein singul\"arer Kartenbereich
tritt auf, wenn die Metrik entweder einen
Nullmod oder aber einen singul\"aren
Mod (also eine Unendlichkeit) hat.

Ein Beispiel haben wir schon
bei der L\"angen- und Breitengradbeschreibung
einer Kugel kennengelernt: Die Metrik
wird am Nord- und S\"udpol singul\"ar,
da dort einer der beiden Eigenwerte
verschwindet. Anschaulich bedeutet dies,
dass am Nord- und S\"udpol ein endlicher
Kartenabstand (also endliche
${\rm d}x^\mu$) zu einem verschwindenen
tats\"achlichen Abstand ${\rm d}s$ 
geh\"ort. Bei den Punkten $\theta=\pm \pi/2$ 
verschwindet offensichtlich ein
Eigenwert der Metrik
\begin{equation}
    {\rm d}s^2 = \cos \theta \, {\rm d}\varphi^2
      + {\rm d}\theta^2  \, .
\end{equation}
Selbstverst\"andlich sind diese
singul\"aren Stellen der Metrik in Wirklichkeit
keine Singularit\"aten: Der Nord- und
S\"udpol sind ebenso regul\"ar auf
der Kugel wie jeder andere Punkt.

Es gibt auch noch \glqq schlimmere\grqq\
Koordinatensingularit\"aten. Betrachten
wir dazu als Beispiel eine horizontale
Zylinderprojektion der Kugel bzw.\ die
Zylinderkoordinaten einer Kugel. Das
bedeutet, jeder L\"angengrad wird wieder
auf eine Gerade zu einem Winkel
$\varphi$ abgebildet, aber statt des
Breitengrads verwenden wir nun einfach
die $z$-Koordinate. Wir k\"onnen also
jeden Punkt auf der Kugel durch die
Koordinaten $\varphi$ und $z$
ausdr\"ucken:\index{Kugeloberfl\"ache!Zylinderprojektion} 
\begin{equation}
    \vec{x} = 
    (\sqrt{R^2-z^2} \cos \varphi, \sqrt{R^2-z^2} \sin \varphi, z) \, .
\end{equation}
Die Tangentialvektoren zu den Koordinatenlinien
sind nun
\begin{eqnarray}
  \vec{e}_\varphi &=& \sqrt{R^2-z^2} (-\sin \varphi, \cos \varphi, 0) \\
  \vec{e}_z &=& \left( -\frac{z}{\sqrt{R^2-z^2}} \cos \varphi, 
     - \frac{z}{\sqrt{R^2-z^2}} \sin \varphi , 1 \right) \, ,
\end{eqnarray}
womit wir f\"ur die Metrik erhalten:
\begin{equation}
\label{eq_Kugelsingularitaet}
   {\rm d}s^2 = (R^2- z^2) {\rm d}\varphi^2 +
       \frac{R^2}{R^2-z^2} {\rm d}z^2 \, .
\end{equation}
In diesem Fall verschwindet ein
Eigenwert der Metrik bei $z=\pm R$ (also am
Nord- und S\"udpol) und der andere
Eigenwert wird unendlich. Das Produkt
der beiden Eigenwerte bleibt in diesem
Fall allerdings konstant. Eine ganz
\"ahnliche Situation werden wir sp\"ater
am Horizont eines Schwarzen Loches
wiederfinden. 

\begin{figure}[htb]
\begin{picture}(120,150)(0,0)
\qbezier(0,70)(0,90.7)(14.65,105.36)
\qbezier(14.65,105.36)(29.3,120)(50,120)
\qbezier(50,120)(70.7,120)(85.36,105.36)
\qbezier(85.36,105.36)(100,90.7)(100,70)
\qbezier(0,70)(0,49.3)(14.65,34.65)
\qbezier(14.65,34.65)(29.3,20)(50,20)
\qbezier(50,20)(70.7,20)(85.36,34.65)
\qbezier(85.36,34.65)(100,49.3)(100,70)
%
\qbezier(0,130)(0,135)(50,135)
\qbezier(50,135)(100,135)(100,130)
\qbezier(0,130)(0,125)(50,125)
\qbezier(50,125)(100,125)(100,130)
\qbezier(0,10)(0,15)(50,15)
\qbezier(50,15)(100,15)(100,10)
\qbezier(0,10)(0,5)(50,5)
\qbezier(50,5)(100,5)(100,10)
%
\qbezier(0,70)(0,75)(50,75)
\qbezier(50,75)(100,75)(100,70)
\qbezier(0,70)(0,65)(50,65)
\qbezier(50,65)(100,65)(100,70)
%
\multiput(50,0)(0,6){22}{\line(0,1){4}}
\put(50,132){\vector(0,1){15}}
\put(0,10){\line(0,1){120}}
\put(100,10){\line(0,1){120}}
\put(50,70){\line(1,0){50}}
\put(50,105.36){\line(1,0){50}}
\put(75,80){\makebox(0,0){${\scriptstyle R}$}}
\put(83,99){\makebox(0,0){${\scriptstyle P}$}}
\put(106,105){\makebox(0,0){${\scriptstyle P'}$}}
\put(85.36,105.36){\makebox(0,0){${\scriptstyle \bullet}$}}
\put(100,105.36){\makebox(0,0){${\scriptstyle \bullet}$}}
\end{picture}
\begin{picture}(280,150)(0,0)
\thicklines
\put(0,20){\line(1,0){240}}
\put(255,65){\makebox(0,0){${\scriptstyle \varphi}$}}
\put(0,120){\line(1,0){240}}
\put(0,20){\line(0,1){100}}
\put(240,20){\line(0,1){100}}
\thinlines
\put(0,70){\vector(1,0){260}}
\multiput(13.33,20)(13.33,0){17}{\line(0,1){100}}
%
\put(0,78.7){\line(1,0){240}}
\put(0,87.1){\line(1,0){240}}
\put(0,95){\line(1,0){240}}
\put(0,102.1){\line(1,0){240}}
\put(0,108.3){\line(1,0){240}}
\put(0,113.3){\line(1,0){240}}
\put(0,117){\line(1,0){240}}
\put(0,119.0){\line(1,0){240}}
%
\end{picture}
\caption{\label{fig_karte}%
Spezielle Projektion einer Kugelfl\"ache auf einen
Zylinder. (links) Jeder Punkt wird von der $z$-Achse aus
senkrecht in radialer Richtung auf die Zylinderfl\"ache
projiziert, z.B.\ der Punkt $P$ auf den Punkt $P'$. 
(rechts) L\"angen- und Breitengrade in dieser
Karte.} 
\end{figure}

Abbildung \ref{fig_karte} zeigt die angegebene Projektion
der Kugelfl\"ache auf eine Ebene sowie die resultierende
Karte mit einigen L\"angen- und Breitengraden. Man erkennt,
dass die L\"angengraden in der Karte parallel sind (also
einen konstanten Abstand voneinander haben), wohingegen
die Abst\"ande zwischen den \glqq tats\"achlichen\grqq\ 
L\"angengraden auf der Kugelfl\"ache Richtung Nord- oder
S\"udpol immer kleiner werden. Der obere und untere Rand der
Karte sind singul\"ar, insofern die gesamte Linie nur einem
Punkt entspricht, das horizontale Ma\ss\ dort somit null wird.
Die Breitengrade hingegen haben auf der Kugeloberfl\"ache
immer denselben Abstand voneinander, werden in der
Karte aber zum Nord- bzw.\ S\"udpol hin immer dichter.
Das vertikale Ma\ss\ wird dort beliebig gro\ss. 

Selbstverst\"andlich gilt 
nach wie vor, dass die Kugel am Nord- und
S\"udpol nicht singul\"ar ist, sondern
lediglich eine schlechte Karte gew\"ahlt
wurde. Es gibt nat\"urlich auch 
Karten, die am Nord- und S\"udpol
singularit\"atenfrei sind (siehe jeden
Weltatlas). 

Es gibt nat\"urlich auch tats\"achliche
Singularit\"aten bei Mannigfaltigkeiten,
beispielsweise im Zentrum eines
Schwarzen Loches. Bei solchen
Singularit\"aten wird auch die
Metrik bzw.\ die Karte singul\"ar,
allerdings l\"asst sich diese
Singularit\"at nicht durch die Wahl
einer anderen Karte beheben.
Den Komponenten der Metrik sieht
man nicht sofort an, ob es sich um
eine Koordinatensingularit\"at oder
eine echte Singularit\"at handelt.
Dazu muss man geometrische
Gr\"o\ss en betrachten, die nicht von
der Wahl des Koordinatensystems
abh\"angen, beispielsweise die
skalare Kr\"ummung, die wir in
Abschnitt \ref{sec_curvature}
kennen lernen werden.

\section{Der Levi-Civita-Zusammenhang}

Der n\"achste Schritt beschreibt, 
wie man einen Vektor (also
ein Element eines Tangentialraums)
entlang eines Weges auf der Mannigfaltigkeit
parallel verschiebt.\index{Parallelverschiebung}
Eine solche Parallelverschiebung ist
aus zwei Gr\"unden wichtig: (1) Wir
werden im n\"achsten Abschnitt die 
Kr\"ummung \"uber 
infinitesimale Parallelverschiebungen
entlang geschlossener Wege definieren,
und (2) m\"ochten wir die kr\"aftefreie 
Bewegung definieren als eine 
Bahnkurve, deren Tangentialvektor
(also die Geschwindigkeit) 
im Sinne einer Parallelverschiebung
konstant bleibt.  

\subsection{Die Christoffel-Symbole}

In einem flachen euklidischen Raum mit 
kartesischen Koordinaten k\"onnen wir 
einfach sagen, zwei Vektoren sind
parallel, wenn sie dieselben Komponenten
haben. Bei einer parallelen Verschiebung
in eine beliebige Richtung ${\rm d}x^\mu$ 
\"andert der Vektor $X^\nu$ also seine 
Komponenten nicht, oder
\begin{equation}
      \frac{\partial}{\partial x^\nu} X^\mu = 0 \, .
\end{equation} 
Anders ausgedr\"uckt \"andern sich die
Komponenten eines Vektors nicht, wenn
man ihn in eine Richtung ${\rm d}x^\nu$
verschiebt, d.h.:
\begin{equation}
    \delta X^\mu(p) = X^\mu(p+{\rm d}x^\nu)-
            X^\mu(p) = 0 \, . 
\end{equation}
Die etwas unmathematische Notation
$p+{\rm d}x^\nu$ soll Folgendes andeuten:
Der Punkt $p$ habe die Koordinaten $x^\mu(p)$
und der infinitesimal benachbarte Punkt
$q$ die Koordinaten $x^\mu(q)$. Die 
Koordinaten dieser beiden Punkte unterscheiden
sich um ${\rm d}x^\nu$.

Auf einer allgemeinen Mannigfaltigkeit
werden sich die Komponenten von einem
Vektor $X^\mu$ ver\"andern, wenn er
parallel zu einem Nachbarpunkt verschoben
wird. Seien nun $p$ und $q$ zwei infinitesimal
benachbarte Punkte, sodass f\"ur die Koordinaten
gilt: $x^\mu(p)-x^\mu(q)={\rm d}x^\mu$. Die
Komponenten des 
Vektorfelds am Punkte $p$ seien $X^\mu(p)$
und die am Punkte $q$ entsprechend
$X^\mu(q)$. Wir wollen nun definieren, 
unter welcher Bedingung die beiden Vektoren,
ausgedr\"uckt durch ihre Komponenten, 
parallel sind. Diese Vorschrift l\"asst sich 
durch eine lineare Abbildung angeben, die
nicht nur vom Punkte $p$ sondern auch noch
von der Richtung abh\"angt, in
welche die parallele Verschiebung erfolgt.
Diese Gr\"o\ss e hat somit drei Indizes
und man bezeichnet sie mit $\Gamma^\mu_{\nu \lambda}(p)$.

Die folgende Gleichung definiert nun eine
Parallelverschiebung: 
\begin{equation}
\label{eq_parallel}
       \delta X^\mu = \Gamma^\mu_{\nu \lambda}
                  X^\nu {\rm d}x^\lambda \, .
\end{equation}
Sie bedeutet Folgendes: $\delta X^\mu$ ist die
Differenz zwischen der $\mu$-Koordinate des
Vektors am Punkte $p$ und der $\mu$-Koordinate
des zweiten Vektors am Punkte $q=p+{\rm d}x^\lambda$, 
also $\delta X^\mu=X^\mu(p+{\rm d}x^\lambda) -
X^\mu(p)$. Diese Differenz soll eine lineare
Funktion der Komponenten von $X^\mu$ sein
und ebenfalls linear von der Verschiebung
${\rm d}x^\lambda$ abh\"angen. Wenn die
obige Bedingung erf\"ullt ist, bezeichnet man die
beiden Vektoren, deren Komponenten sich
um $\delta X^\mu$ unterscheiden, als parallel.

Wir m\"ussen uns nun \"uberlegen, welche
Bedingungen wir an diese Abbildung, die man
auch als\index{Zusammenhang} 
{\em Zusammenhang} bezeichnet, stellen.
Bei allgemeinen Vektorr\"aumen ist man hier
vollkommen frei, d.h., es kann sich um eine
vollkommen beliebige (allerdings umkehrbare)
lineare Abbildung handeln. Sind jedoch auf
dem Vektorraum zus\"atzliche Strukturen 
definiert, sollte die Parallelverschiebung diese
Strukturen nach M\"oglichkeit erhalten.
In unserem Fall ist auf der Mannigfaltigkeit
eine Metrik definiert und man wird von einer
Parallelverschiebung erwarten, dass sie das
Skalarprodukt von Vektoren nicht ver\"andert.
Einen solchen Zusammenhang bezeichnet
man auch als\index{Zusammenhang!metrischer} 
{\em metrischen Zusammenhang}.
Wenn also {\em zwei} Vektoren $X$ und $Y$
definiert sind, dann soll ihr Skalarprodukt
am Punkte $p$ dasselbe sein, wie das
Skalarprodukt ihrer Parallelverschiebungen
am Punkte $q$. Das f\"uhrt auf folgende
Bedingung:
\begin{equation}
  g_{\mu \nu}(p) X^\mu(p) Y^\nu(p) =
  g_{\mu \nu}(q) )X^\mu(q)Y^\mu(q)  \, .
  \end{equation} 
Diese Gleichung ist zun\"achst nur
f\"ur infinitesimal benachbarte Punkte
(mit der Verschiebungsrichtung
${\rm d}x^\mu$) sinnvoll, da im allgemeinen
eine Parallelverschiebung von dem
Weg abh\"angt, entlang dem die
Vektoren verschoben wurden (siehe auch 
den Anhang zu Faserb\"undeln in
dem Skript zur QFT).   
  
Wir schreiben nun f\"ur die Koordinaten von 
$q$: $x^\lambda(q)=x^\lambda(p)+{\rm d}x^\lambda$
und entwickeln den Ausdruck auf der rechten Seite
bis zu linearen Termen in ${\rm d}x^\lambda$. Die
f\"uhrende Ordnung hebt sich nat\"urlich auf der
linken und rechten Seite weg.
\begin{equation}
 0 = \frac{\partial g_{\mu \nu}}{\partial x^\lambda}
     {\rm d}x^\lambda X^\mu Y^\nu +
   g_{\mu \nu} \Gamma^{\mu}_{\alpha \lambda}X^\alpha
      {\rm d}x^\lambda Y^\nu + g_{\mu \nu}
     X^\mu \Gamma^{\nu}_{\alpha \lambda} Y^\alpha
      {\rm d}x^\lambda \, .    
\end{equation}
S\"amtliche Terme in diesem Ausdruck sind nun
als Ausdr\"ucke am Punkt $p$ zu verstehen.
In der folgenden Gleichung sind lediglich die
Summationsindizes umbenannt und die gleichen
Terme ausgeklammert worden:
\begin{equation}
 0 = \left( \frac{\partial g_{\mu \nu}}{\partial x^\lambda} +
   g_{\alpha \nu} \Gamma^{\alpha}_{\mu \lambda}
       + g_{\mu \alpha}
      \Gamma^{\alpha}_{\nu \lambda}
       \right) {\rm d}x^\lambda X^\mu Y^\nu\, .    
\end{equation}
Da diese Bedingungsgleichung f\"ur eine
Parallelverschiebung f\"ur beliebige Vektorfelder
$X$ und $Y$ sowie beliebige Verschiebungsrichtungen
${\rm d}x^\lambda$ gelten soll, muss der Ausdruck in
der Klammer verschwinden:
\begin{equation}
\label{eq_kovg}
 0 =  \frac{\partial g_{\mu \nu}}{\partial x^\lambda} +
   g_{\alpha \nu} \Gamma^{\alpha}_{\mu \lambda}
     + g_{\mu \alpha} \Gamma^{\alpha}_{\nu \lambda}
        \, .    
\end{equation}
Diese Gleichung ist noch nicht eindeutig nach den
$\Gamma$-Symbolen, die man auch als
{\em Christoffel-Symbole}\index{Christoffel-Symbole} 
bezeichnet, aufl\"osbar.
Man fordert f\"ur den so genannten\index{Levi-Civita-Zusammenhang} 
Levi-Civita-Zusammenhang
noch die\index{Torsionsfreiheit} 
{\em Torsionsfreiheit}, d.h., die $\Gamma$-Symbole
sollen in den beiden unteren Indizes symmetrisch sein.
Anschaulich bedeutet dies, dass die Ableitung der
$\mu$-Komponente eins Vektorfeldes in $\nu$-Richtung
gleich der Ableitung der $\nu$-Komponente in
$\mu$-Richtung sein soll. Diese beiden Forderungen --
es soll sich um einen metrischen und torsionsfreien
Zusammenhang handeln -- legen den
Levi-Civita-Zusammenhang und damit die
Christoffel-Symbole als Funktion der Metrik
fest:
\[  \Gamma^\lambda_{\mu \nu} ~=~ \frac{1}{2} g^{\lambda \kappa} \left[ 
    \frac{\partial g_{\kappa \nu}}{\partial x^\mu} +
    \frac{\partial g_{\kappa \mu}}{\partial x^\nu} -
    \frac{\partial g_{\mu \nu}}{\partial x^\kappa} \right]  \;.  \]

\subsection{Ein paar mathematische Zwischenbemerkungen}

Der Mathematiker versucht, geometrische Konstruktionen
m\"oglichst unabh\"angig von einer Wahl der Koordinaten
zu definieren. Ausgangspunkt ist meist ein Vektorfeld $X$,
das entlang bestimmter Richtungen abgeleitet werden soll.
Diese Richtungen werden wiederum koordinatenunabh\"angig
durch ein zweites Vektorfeld $Y$ definiert. Dazu muss 
zun\"achst gezeigt werden, dass jedes Vektorfeld $Y$
(zumindest in einer offenen Umgebung, also einer Karte)
Integralkurven besitzt, also eine L\"osungsschar von Wegen
$\gamma(t)$, deren Tangentialvektoren an jedem Punkt
$p$ gleich $Y(p)$ sind. Dann kann man ein
Vektorfeld $X$ nach dem Vektorfeld $Y$ ableiten.

Eine spezielle Ableitung, die so genannte\index{kovariante Ableitung} 
{\em kovariante
Ableitung}, gibt nun an, unter welchen Bedingungen ein
Vektorfeld $X$, abgeleitet nach einem Vektorfeld $Y$
\glqq konstant\grqq\ ist, also in Richtung der Integralkurven
von $Y$ eine Parallelverschiebung von Vektoren darstellt. 
Diese kovariante Ableitung bezeichnet man oft als
$\nabla_Y$. Die Bedingung einer Parallelverschiebung
lautet dann
\begin{equation}
             \nabla_Y X = 0 \, .
\end{equation}  

Zum Vergleich mit unserer Notation betrachten wir
einen speziellen Weg $\gamma(t)$ durch den
Punkt $p$, also $\gamma(0)=p$. In einer Karte
werde dieser Weg durch $y^\mu(t)$ beschrieben. 
Der Tangentialvektor in dieser Karte am Punkte
$p$ ist somit 
\begin{equation}
        Y^\mu(p) = \left. \frac{{\rm d}y^\mu(t)}{{\rm d}t} \right|_{t=0}\,  .
\end{equation}
Nun betrachten wir das Vektorfeld $X$, das entlang des
Weges $\gamma(t)$ die Koordinaten $X^\mu(t)$ haben soll.
Wir sagen, dieses Vektorfeld ist entlang des Weges
$\gamma(t)$ (am Punkte $p$) eine Parallelverschiebung,
wenn
\begin{equation}
          \frac{{\rm d}X^\mu(t)}{{\rm d}t} =
          \frac{\partial X^\mu}{\partial y^\lambda}
          \frac{{\rm d}y^\lambda(t)}{{\rm d}t} 
             = \Gamma^{\mu}_{\nu \lambda} X^\nu(t) 
             \frac{{\rm d}y^\lambda(t)}{{\rm d}t} 
 \end{equation}            
ist, bzw.
\begin{equation}
    \frac{\partial X^\mu}{\partial y^\lambda} =
          \Gamma^{\mu}_{\nu \lambda} X^\nu \, .
\end{equation}
Die kovariante Ableitung in Richtung der
Koordinate $\lambda$ ist somit:
\begin{equation}
     \nabla_\lambda X^\mu :=  
      \partial_\lambda X^\mu - \Gamma^\mu_{\nu \lambda} X^\nu \, ,
\end{equation}
und das Verschwinden der kovarianten Ableitung
ist die Bedingung, dass es sich bei dem Vektorfeld
$X^\mu$ um eine Parallelverschiebung in 
Koordinatenrichtung $\lambda$ handelt.

Abschlie\ss end m\"ochte ich noch anmerken,
dass man auch f\"ur Tensoren eine kovariante
Ableitung definieren kann. Gleichung \ref{eq_kovg}
bedeutet dann, dass die kovariante Ableitung
der Metrik verschwindet. Dies ist eine alternative
Definition f\"ur einen metrischen Zusammenhang.

\subsection{Ein Beispiel aus der klassischen Mechanik -- konstant 
rotierende Bezugssysteme}

In diesem Zwischenabschnitt m\"ochte ich darauf
hinweisen, dass das Konzept der kovarianten
Ableitung schon aus der Mechanik bekannt ist,
allerdings meist nicht unter diesem Namen.
Bei der Herleitung der Coriolis-Kraft betrachtet
man ein ruhendes Inertialsystem und ein
konstant rotierendes zweites Bezugssystem,
dessen Ursprung mit dem ruhenden System
zusammenfallen soll.

Ein Bahnkurve $x^i(t)$ (hier verwende ich lateinische
Indizes, da es sich um rein r\"aumliche Indizes
handelt) im ruhenden System ist eine
Gerade, wenn die zeitliche Ableitung 
$v^i(t)=\frac{{\rm d}x^i(t)}{{\rm d}t}$ konstant 
ist. Im rotierenden\index{kovariante Ableitung!rotierendes Bezugssystem} 
System hingegen
\"andern sich die Komponenten des 
Geschwindigkeitsvektors st\"andig. Seien
\begin{equation}
  x'^i(t) = \sum_j R(t)^i_j x^j(t)            
\end{equation}
 die Koordinaten der Bahnkurve
im rotierenden System (wobei $R(t)$ eine
$t$-abh\"angige Rotationsmatrix darstellt) und entsprechend
\begin{equation}
     v'^{\,i}(t)=\frac{{\rm d}}{{\rm d}t} \sum_j R(t)^i_j x^j(t) 
         = \sum_j \left( \frac{\rm d}{{\rm d}t} R(t)^i_j \right) x^j(t)
         + \sum_j R(t)^i_j  \frac{\rm d}{{\rm d}t} x^j(t)
\end{equation}
die Komponenten der Geschwindigkeit
im rotierenden System, dann gilt bekanntlich
\begin{equation}
       \vec{v}^{\, \prime}(t) = R(t) \left( \vec{v}(t) + 
             \vec{\omega} \times \vec{x}(t) \right)  \, ,
\end{equation}
wobei $\vec{\omega}$ die Drehung charakterisiert,
also die Richtung von $\vec{\omega}$ die
Drehachse und der Betrag die Winkelgeschwindigkeit.
Nun definiert man oft die kovariante
Zeitableitung
\begin{equation}  
    \frac{{\rm D}}{{\rm D}t} \vec{x}(t) = \frac{\rm d}{{\rm d}t} \vec{x}(t) +
         \vec{\omega} \times \vec{x}(t)  
\end{equation}
bzw.\ in Koordinaten
\begin{equation}  
    \frac{{\rm D}}{{\rm D}t} x^i(t) = \frac{\rm d}{{\rm d}t} x^i (t) +
         \sum_{jk} \epsilon^i_{jk} \omega^j  x^k(t)  \, .
\end{equation}
Die Gr\"o\ss e $\Gamma^i_k=\sum_j \epsilon^i_{jk}\omega^j$
definiert den Zusammenhang. Da die \glqq Mannigfaltigkeit\grqq\
nur in der Zeitachse besteht, gibt es keinen gesonderten
Index f\"ur die Richtung der Parallelverschiebung. Die
kovariante Ableitung gibt an, wie sich die Komponenten
eines Vektors im rotierenden System zeitlich
ver\"andern m\"ussen, wenn dieser Vektor im ruhenden
System konstant ist. Und die kovariante Ableitung
verschwindet, wenn sich die Komponenten genau so
ver\"andern, dass es sich bei dem Vektor im Inertialsystem
um einen konstanten Vektor handelt.

\subsection{Die Geod\"atengleichung}

Wir formulieren nun eine\index{Geod\"atengleichung} 
Geod\"atengleichung, d.h.\
eine Gleichung f\"ur die k\"urzeste (in der Relativit\"atstheorie
wegen der Pseudo-Riemann'schen Geometrie
die l\"angste) Verbindungskurve zwischen zwei
Punkten. Wir leiten diese Gleichung aus physikalischen 
\"Uberlegungen zur Konstanz der Geschwindigkeit ab.

Gegeben sei eine Bahnkurve $\gamma(\tau)$ mit ihrer
Koordinatendarstellung $x^\mu(\tau)$, parametrisiert
durch die Eigenzeit $\tau$. An jedem Punkt sei 
\begin{equation}
          u^\mu(\tau) = \frac{{\rm d}x^\mu(\tau)}{{\rm d}\tau}
\end{equation}
der Geschwindigkeitsvektor an die Kurve. Bei dem
Geschwindigkeitsvektor handelt es sich um einen
Tangentialvektor.

Gew\"ohnlich sagen wir, eine Bahnkurve ist eine
Gerade (oder Geod\"ate), wenn die Komponenten der
Geschwindigkeit konstant sind, sich also entlang der 
Bahnkurve nicht \"andern. Auf einer Mannigfaltigkeit
mit einer Metrik bedeutet dies, dass sich die \"Anderung 
der Komponenten von $u^\mu$ {\em in Richtung}
der Geschwindigkeit nicht \"andern, bzw.
\begin{equation}
       {\rm d} u^\mu = -  \Gamma^\mu_{\nu \lambda} u^\nu
          {\rm d}x^\lambda    \, ,
\end{equation}
wobei 
\begin{equation}
     {\rm d}x^\lambda = u^\lambda {\rm d}\tau
\end{equation}
in Richtung des Tangentialvektors zeigt. Damit erhalten
wir als Geod\"atengleichung:
\begin{equation}
\label{geodaete}
     \frac{{\rm d}u^\mu (\tau)}{{\rm d} \tau} ~=~
       - \Gamma^\mu_{\nu \lambda} u^\nu u^\lambda  \;.   
\end{equation}       
Diese Gleichung ist gleichzeitig die Bewegungsgleichung
f\"ur ein \glqq freies\grqq\ Teilchen auf einer
Mannigfaltigkeit mit der Metrik $g_{\mu \nu}$. 
Die Geschwindigkeit muss dabei f\"ur ein massives
Teilchen noch der Bedingung
\begin{equation}
\label{eq_eigenm}
      c^2 = g_{\mu \nu} \frac{{\rm d}x^\mu}{{\rm d}\tau}
      \frac{{\rm d}x^\nu}{{\rm d}\tau}  \hspace{2cm}
      m\neq 0 
\end{equation}
gen\"ugen, die im Wesentlichen angibt, dass es sich
bei $\tau$ um die Eigenzeit handelt (in der gew\"ohnlichen
Geometrie w\"urde sich $\tau$ auf die Bogenl\"ange
beziehen). F\"ur ein masseloses Teilchen ist
${\rm d}\tau=0$ und damit folgt f\"ur eine beliebige
Parametrisierung die Geod\"atengleichung
\begin{equation}
      \frac{{\rm d}^2 x^\mu (\lambda)}{{\rm d} \lambda^2} ~=~
       - \Gamma^\mu_{\nu \lambda} \frac{{\rm d}x^\nu}{{\rm d}\lambda}
       \frac{{\rm d}x^\lambda}{{\rm d}\lambda}           
\end{equation}
mit der Einschr\"ankung
\begin{equation}
\label{eq_eigen0}
      0 = g_{\mu \nu} \frac{{\rm d}x^\mu}{{\rm d}\lambda}
      \frac{{\rm d}x^\nu}{{\rm d}\lambda}  \hspace{2cm}
      m = 0\, . 
\end{equation}

Dieselbe Gleichung erh\"alt man auch als
Euler-Lagrange-Gleichung aus dem L\"angenfunktional
\begin{equation}
\label{eq_freieTeilchen}
      S =  \int  L(x(t),\dot{x}(t)) {\rm d} t  \hspace{0.7cm} {\rm mit}
      \hspace{0.7cm}
    L =  \sqrt{g_{\mu \nu}(x(t)) \dot{x}^\mu \dot{x}^\nu} \, .
\end{equation}
Die Ableitungen sind
\begin{equation}
   \frac{\partial L}{\partial x^\lambda(t)} =
       \frac{1}{2 \sqrt{g_{\mu \nu}(x(t)) \dot{x}^\mu \dot{x}^\nu}}
        \left( \frac{\partial g_{\mu \nu}}{\partial x^\lambda}
          \dot{x}^\mu \dot{x}^\nu \right) 
\end{equation}
und
\begin{equation}
   \frac{{\rm d}}{{\rm d}t}  \frac{\partial L}{\partial \dot{x}^\lambda(t)} =
   \frac{{\rm d}}{{\rm d}t}  \left(  \frac{1}{\sqrt{g_{\mu \nu}(x(t)) \dot{x}^\mu \dot{x}^\nu}}
          g_{\mu \lambda}
            \dot{x}^\mu \right) 
\end{equation}
und die Extrema findet man, indem man die beiden
Ausdr\"ucke gleich setzt. Diese Gleichung wird sehr kompliziert,
daher geht man auch hier meist zur Eigenzeitparametrisierung
(Gl.\ \ref{eq_eigenm} bzw.\ f\"ur masselose Teilchen
Gl.\ \ref{eq_eigen0}) 
\"uber, wodurch man nahezu direkt die Geod\"atengleichung
erh\"alt. 

\section{Kr\"ummungstensoren}
\label{sec_curvature}

Der Levi-Civita-Zusammenhang bzw.\ die Christoffel-Symbole
erlauben eine Parallelverschiebung eines Vektors in
eine bestimmte Richtung. Sie erm\"oglichen damit den Vergleich
zweier Vektoren an verschiedenen Punkten, sofern der eine
Vektor entlang eines vorgegebenen Weges zu dem anderen
Vektor parallel verschoben wird. Nun wollen
wir einen Vektor entlang einer {\em geschlossenen} Bahnkurve
parallel transportieren. Im Allgemeinen wird sich dabei
der Vektor ver\"andern. Diese Ver\"anderung entlang
geschlossener Bahnkurven ist ein Ma\ss\ f\"ur die
geometrische Kr\"ummung der\index{Kugeloberfl\"ache!Parallelverschiebung} 
Mannigfaltigkeit.\index{Parallelverschiebung!auf Kugeloberfl\"ache} 

\begin{figure}[htb]
\begin{picture}(170,120)(-20,0)
\put(10,30){\line(5,1){100}}
\put(10,30){\line(1,3){30}}
\put(110,50){\line(-1,1){70}}
\thicklines
\put(10,30){\vector(-1,-1){12}}
\put(60,40){\vector(-1,-1){12}}
\put(110,50){\vector(-1,-1){12}}
\put(75,85){\vector(-1,-1){12}}
\put(40,120){\vector(-1,-1){12}}
\put(25,75){\vector(-1,-1){12}}
\put(60,5){\makebox(0,0){(a)}}
\end{picture}
%
\begin{picture}(120,120)(0,0)
\qbezier(8,75)(22,110)(60,111)
\qbezier(60,111)(98,110)(112,75)
\qbezier(8,35)(0,55)(8,75)
\qbezier(112,35)(120,55)(112,75)
\qbezier(4,55)(60,25)(116,55)
\put(60,105){\circle*{5}}
%
\qbezier(60,105)(30,90)(23,47)
\qbezier(60,105)(95,100)(110,52)
%
\put(58,104){\vector(4,-1){12}}
\put(23,47){\vector(4,-1){12}}
\put(35,82){\vector(4,-1){12}}
\put(65,40){\vector(1,0){12}}
\put(110,52){\vector(2,1){10}}
\put(96,83){\vector(2,1){10}}
\put(60,105){\vector(2,1){10}}

\thicklines
\put(58,104){\vector(-2,-1){18}}
\put(23,47){\vector(0,-1){18}}
\put(65,40){\vector(0,-1){18}}
\put(110,52){\vector(0,-1){18}}
\put(35,82){\vector(-2,-3){11}}
\put(96,83){\vector(2,-3){10}}
\put(60,105){\vector(4,-1){18}}
\put(60,5){\makebox(0,0){(b)}}
%
\put(55,108){\makebox(0,0){${\scriptstyle 1}$}}
\put(31,82){\makebox(0,0){${\scriptstyle 2}$}}
\put(20,44){\makebox(0,0){${\scriptstyle 3}$}}
\put(61,37){\makebox(0,0){${\scriptstyle 4}$}}
\put(106,47){\makebox(0,0){${\scriptstyle 5}$}}
\put(92,82){\makebox(0,0){${\scriptstyle 6}$}}
\put(61,99){\makebox(0,0){${\scriptstyle 7}$}}
\end{picture}
\caption{\label{fig_Par}%
(a) Wird ein Vektor in einer Ebene entlang der Kanten eines Dreiecks
parallel verschoben, \"andert sich seine Richtung nicht. (b) F\"ur
eine entsprechende Parallelverschiebung entlang von
Gro\ss kreisen (jeweils um eine Viertel Vollkreisl\"ange) 
entlang der Positionen 1 bis 7 (Position 7 = Position 1) 
haben sich die Vektoren um $90^\circ$ gedreht.}
\end{figure} 

Wir betrachten zun\"achst ein vertrautes Beispiel auf
der Kugeloberfl\"ache: Wird ein Vektor beginnend am
Nordpol entlang eines L\"angengrads parallel zum
\"Aquator transportiert, anschlie\ss end entlang
des \"Aquators um einen Viertelkreis parallel verschoben
und dann wieder entlang eines L\"angengrads zum
Nordpol,
so hat sich dieser Vektor im Vergleich zu Anfangszustand
um $90^\circ$ gedreht (siehe Abb.\ \ref{fig_Par}). 
Verkleinert man die Fl\"ache, um die herum ein
Vektor parallel verschoben wird, so verkleinert sich
auch der Winkel, um den dieser Vektor gedreht wird.
Es zeigt sich jedoch, dass der Quotient aus dem
Winkel, um den ein Vektor bei einer Parallelverschiebung
um eine geschlossene Fl\"ache gedreht wird, und
dem Inhalt dieser Fl\"ache gegen eine Konstante
geht. Diese Konstante ist ein Ma\ss\ f\"ur die
Kr\"ummung an dem Punkt, an dem dieser Weg
beginnt und endet.

In Analogie zu Gleichung \ref{eq_parallel} definieren
wir den Kr\"ummungstensor $R^\mu_{\nu \rho \sigma}$
\"uber eine infinitesimale Parallelverschiebung, diesmal
allerdings entlang eines geschlossenen Weges:
\begin{equation}
  \delta X^\mu ~=~ R^\mu_{\nu \rho \sigma} X^\nu {\rm d} 
                \sigma^{\rho \sigma} \, .   
\end{equation}
Diese Gleichung bedeutet Folgendes: Wenn ein Vektor
$X$ mit Komponenten $X^\nu$ um den Rand des 
infinitesimalen Fl\"achenelements ${\rm d}\sigma^{\rho \sigma}$ 
parallel verschoben wird, dann \"andern sich seine
Komponenten um $\delta X^\mu$. Das infinitesimale
Fl\"achenelement ${\rm d}\sigma^{\rho \sigma}$ liegt
in der durch die Koordinaten $\rho$ und $\sigma$
aufgespannten Ebene und hat den Fl\"acheninhalt
$|{\rm d}\sigma^{\rho \sigma}|$. 

Damit l\"asst sich aus den Christoffel-Symbolen der 
Riemann-Christoffel-Kr\"um\-mungstensor\index{Kr\"ummungstensor}
\index{Riemann-Christoffel-Kr\"ummungstensor} berechnen:
\[ R^{\lambda}_{\mu \nu \kappa} ~=~
   \frac{\partial \Gamma^{\lambda}_{\mu \nu}}{\partial x^\kappa} -
   \frac{\partial \Gamma^{\lambda}_{\mu \kappa}}{\partial x^\nu} +
   \Gamma^{\eta}_{\mu \nu}\Gamma^{\lambda}_{\kappa \eta} - 
   \Gamma^{\eta}_{\mu \kappa}\Gamma^{\lambda}_{\nu \eta} \;.  \]
Bei diesem Ausdruck handelt es sich im Wesentlichen
um den Kommutator von zwei kovarianten Ableitungen
\begin{equation}
  R^\lambda_{\mu \nu \kappa}=
      [ (\partial_\mu \delta^\lambda_\alpha - \Gamma^\lambda_{\mu \alpha}),
       (\partial_\nu \delta^\alpha_\kappa - \Gamma^\alpha_{\nu \kappa})] \, ,
\end{equation}
was nochmals zum Ausdruck bringt, dass eine kovariante
Ableitung ein Generator einer Parallelverschiebung in eine
bestimmte Richtung ist, und der Kr\"ummungstensor als
Kommutator zweier solcher Generatoren als Differenz
zwischen \glqq erst Richtung $\kappa$, dann Richtung $\nu$\grqq\
und \glqq erst Richtung $\nu$, dann Richtung $\kappa$\grqq\
aufzufassen ist.

Aus dem Riemann-Christoffel-Kr\"ummungstensor erh\"alt man durch
Kontraktion den Ricci-Tensor\index{Ricci-Tensor}:
\[         R_{\mu \nu } ~=~ R^\lambda_{\mu \lambda \nu}   \;.  \]
Eine weitere Kontraktion f\"uhrt auf den Kr\"ummungsskalar:
\index{Kr\"ummungsskalar}
\[         R ~=~ g^{\mu \nu} R_{\mu \nu}  \;.      \]
In der Riemannschen Geometrie -- mit positiv definiter Metrik -- hat der 
Kr\"ummungs\-skalar eine sehr anschauliche Bedeutung. Es ist der f\"uhrende 
Korrekturfaktor f\"ur das Volumen $V_d(r)$ einer Kugel vom Radius $r$ 
im Vergleich zum 
Volumen der Kugel in einem euklidischen Raum in $d$-Dimensionen
(siehe Pauli \cite{Pauli}, S.~48): 
\[   V_d(r) ~=~ 
   C_d r^d \left\{ 1 + \frac{R}{6} \frac{r^2}{d+2} + \ldots \right\}\;.\]
Die Ableitung nach $r$ liefert eine entsprechende Formel f\"{u}r die 
Oberfl\"{a}che der Kugel:
\[  S_d(r) ~=~ d
    C_d r^{d-1} \left\{ 1 + \frac{R}{6} \frac{r^2}{d}
         + \ldots \right\}\;.  \]
         
\section{Die Einstein'schen Feldgleichungen}
\label{sec_EinsteinGl}
         
Damit haben wir die rein geometrischen Bausteine der Allgemeinen
Relativit\"atstheorie. Es fehlt noch der Anteil der Materie, ausgedr\"uckt
durch den\index{Energie-Impuls-Tensor} 
Energie-Impuls-Tensor $T_{\mu \nu}$. L\"asst sich die Materie 
in einer durch den metrischen Tensor $g_{\mu \nu}$ beschriebenen Raum-Zeit
durch eine Wirkung $S[g;...]$ beschreiben, so gilt formal:
\[     T_{\mu \nu} ~=~ \frac{\delta S}{\delta g^{\mu \nu}} \;.  \]
Diese Formel ist besonders n\"utzlich, wenn die Materie durch Felder
repr\"asentiert wird, beispielsweise im Fall der Maxwell-Theorie (in
einer gekr\"ummten Raumzeit), der Dirac- oder der Klein-Gordon-Theorie.

Die Einstein'schen Feldgleichungen\index{Einstein'sche Gleichungen} 
lauten nun:
\begin{equation}
\label{EinsteinGl}
   R_{\mu \nu} - \frac{1}{2} g_{\mu \nu} R 
     ~=~ - \frac{8 \pi G}{c^4} T_{\mu \nu}   \;. 
\end{equation}     
($G$ ist Newtons Gravitationskonstante.) Die linke Seite dieser Gleichung
enth\"alt rein geometrische Gr\"o\ss en der Raumzeit; die rechte Seite
den Materieanteil in der Raumzeit. Der Energie-Impuls-Tensor h\"angt 
im Allgemeinen ebenfalls von der Metrik ab. Diese Gleichung ist allerdings
nur eine H\"alfte des vollst\"andigen Gleichungssystems der Allgemeinen
Relativit\"atstheorie. Es fehlt noch die Bewegungsgleichung der Materie.
L\"asst sich die Materie durch eine Wirkung beschreiben, so erh\"alt
man diese Bewegungsgleichungen durch Variation der Wirkung nach den
entsprechenden Freiheitsgraden. F\"ur Punktteilchen ist die 
Geod\"atengleichung \ref{geodaete} die Bewegungsgleichung eines Teilchens
in einer gekr\"ummten Raum-Zeit.

Die Einstein'schen Feldgleichungen lassen sich auch
als Euler-Lagrange-Glei\-chungen aus einer Wirkung ableiten.
Dazu betrachten wir folgenden Ausdruck, die so genannte
{\em Einstein-Hilbert-Wirkung}:\index{Einstein-Hilbert-Wirkung}
\begin{equation}
\label{eq_EinsteinHilbert}
           S = \kappa \int {\rm d}^4x \sqrt{-g} R + S[g_{\mu \nu};\varphi, \psi, ...]
\end{equation}
Der erste Term beschreibt ein 4-dimensionales Volumenintegral
\"uber die skalare Kr\"um\-mung. Der Faktor $\sqrt{-g}$ ist die
Quadratwurzel aus der Determinante des metrischen Tensors.
Das Minuszeichen macht diesen Ausdruck positiv,
da in der \"ublichen Signatur die Determinante der
Pseudo-Riemann'schen Metrik negativ ist. Die Kombination
${\rm d}^4 x\sqrt{-g}$ ist das invariante Volumenma\ss. Die
Konstante $\kappa$ h\"angt \"uber
\begin{equation}
          \kappa = \frac{c^3}{16 \pi G}
\end{equation}
mit der Newton'schen Gravitationskonstante $G$ zusammen.

Der zweite Term ist die Wirkung der Materie. Handelt es sich
um Felder (Skalarfelder $\varphi$ oder Dirac-Felder $\psi$) ist
dies die \"ubliche Klein-Gordon-Wirkung bzw.\ Dirac-Wirkung,
allerdings koppeln diese Felder nun (minimal) an die
Metrik. Wie schon erw\"ahnt, liefert die Funktionalableitung dieser
Wirkung nach der Metrik den Energie-Impuls-Tensor der
Materie und damit den rechten Teil der Einstein'schen Gleichungen.
Die Funktionalableitung des Kr\"ummungsskalars nach der
Metrik liefert die linke Seite der Einstein'schen Gleicnungen.

Manchmal addiert man zu der angegebenen Wirkung noch
einen Term
\begin{equation}
          S_\Lambda = - 2\kappa \Lambda \int {\rm d}^4x \, \sqrt{-g} \, ,
\end{equation}
wobei $\Lambda$ die so genannte {\em kosmologische Konstante}
ist.\index{Kosmologische Konstante} 
Dieser Teil der Wirkung besteht also einfach aus einem 
Volumenintegral. Die Einstein-Gleichungen werden damit zu
\begin{equation}
\label{EinsteinLambda}
   R_{\mu \nu} - \frac{1}{2} g_{\mu \nu} R + \Lambda g_{\mu \nu} 
     ~=~  \frac{8 \pi G}{c^4} T_{\mu \nu}   \;. 
\end{equation}     

Den noch fehlenden zweiten Anteil der Bewegungsgleichungen
-- die Bewegungsgleichungen f\"ur die Materie -- erh\"alt
man aus der Wirkung (Gl.\ \ref{eq_EinsteinHilbert}) durch
die Variation nach den Feldern. Betrachtet man freie
Punktteilchen, so ist ihre Wirkung durch das L\"angenfunktional
(Gl.\ \ref{eq_freieTeilchen}) gegeben und die zweite
Bewegungsgleichung ist die Geod\"atengleichung.

Abschlie\ss end wollen wir die Einstein'schen Feldgleichungen
noch etwas umformen, was die L\"osung insbesondere f\"ur
den Fall $T_{\mu \nu}=0$ einfacher macht. Bilden wir auf beiden
Seiten der Gleichung \ref{EinsteinGl} die Spur (genauer, ziehen
einen der Indizes mit der Metrik hoch und bilden dann die Spur)
so erhalten wir
\begin{equation}
       R = \frac{8 \pi G}{c^4} T  \, .
\end{equation}
Wir k\"onnen daher den Term mit der skalaren Kr\"ummung auf die
andere Seite der Gleichung bringen:
\begin{equation}
   R_{\mu \nu} 
     ~=~ - \frac{8 \pi G}{c^4} 
\left( T_{\mu \nu} - \frac{1}{2} g_{\mu \nu} T \right)   \;. 
\end{equation}
Man erkennt an dieser Form sofort, dass die L\"osungen
der Einstein-Gleichungen\index{Einstein'sche Gleichungen!im Vakuum} 
im Vakuum (also f\"ur $T_{\mu \nu}=0$)
durch
\begin{equation}
\label{eq_RiccigleichNull}
        R_{\mu \nu}=0
\end{equation}
gegeben sind. Der Ricci-Tensor (und damit nat\"urlich auch die
skalare Kr\"ummung) verschwinden im materiefreien Raum.

\begin{thebibliography}{99}
%\addcontentsline{toc}{chapter}{Literaturangaben}
%\bibitem{Aichelburg} Peter C.\ Aichelburg (Hrsg.); {\it Zeit im 
%       Wandel der Zeit}; Verlag Vieweg, Braunschweig, Wiesbaden, 1988.
%\bibitem{Barbour3} {\it Mach's Principle -- From Newton's Bucket to
%        Quantum Gravity}; Julian Barbour \& Herbert Pfister (Hrsg.);
%        Birkh\"auser, Boston, Basel, Berlin, 1995.       
%\bibitem{Bekenstein} Jacob D.\ Bekenstein, \textit{Black holes
%          and entropy}, Phys.\ Rev.\ D\,7 (1973) 2333--2346.        
%\bibitem{Bell} John Bell;  {\em Speakable and Unspeakable in 
%        Quantum Physics}, 2.\ edition, Cambridge University Press (2004).       
%\bibitem{Born} Max Born; {\it Optik}; Springer-Verlag, Berlin, Heidelberg,
%        1972.
%\bibitem{Britannica} Encyclopaedia Britannica; 15.th edition, 1988.
%\bibitem{Descartes} Ren\'e Descartes; {\it Die Prinzipien der
%        Philosophie}; Felix Meiner Verlag, Hamburg, 1992; \"ubersetzt
%        von Artur Buchenau.
%\bibitem{EDM} Encyclopaedic Dictionary of Mathematics; Second Edition,
%        MIT Press, 1987.
%\bibitem{Einstein1} Albert Einstein; {\it Zur Elektrodynamik bewegter 
%        K\"orper}; Annalen der Physik, Leipzig, 17 (1905) 891. 
%\bibitem{Einstein2} Albert Einstein; {\it Ist die Tr\"agheit eines
%        K\"orpers von seinem Energieinhalt abh\"angig?} (Ann.\ Phys., 
%        Leipzig, 18 (1905) 639.
%\bibitem{Einstein3} Albert Einstein; {\it Aus meinen sp\"aten Jahren};
%         Ullstein Sachbuch, Verlag Ullstein, Frankfurt, Berlin, 1993.                 
%\bibitem{Einstein4} Albert Einstein; {\it Prinzipielles zur allgemeinen
%        Relativit\"atstheorie}; Annalen der Physik 55 (1918) 241.
%\bibitem{Einstein5} Albert Einstein; {\it \"Uber den Einflu\ss\ der
%        Schwerkraft auf die Ausbreitung des Lichtes}; Annalen der
%        Physik 35 (1911) 898.                 
%\bibitem{Feynman} Richard Feynman; {\it The Character of Physical Law};
%        The MIT Press, 1987.        
%\bibitem{Fierz} Markus Fierz; {\it \"Uber den Ursprung und die Bedeutung
%        der Lehre Isaac Newtons vom absoluten Raum}; Gesnerus, 
%        11.\ Jahrgang (1954), S.\,62--120.
%\bibitem{Fliessbach} Torsten Flie\ss bach; {\it Allgemeine 
%        Relativit\"atstheorie}; BI-Wissenschaftsverlag, Mannheim, Wien
%        Z\"urich, 1990. 
%\bibitem{Galilei} Galilei; {\it Dialog \"uber die beiden haupts\"achlichen
%        Weltsysteme, das ptolem\"aische und das kopernikanische}; 
%        Teubner Stuttgart, 1982; aus dem Italienischen \"ubersetzt von
%        Emil Strauss.   
%  \bibitem{Hawking} Stephen W.\ Hawking, \textit{Particle Creation by
%            black holes}, Comm.\ Math.\ Phys.\ 43 (1976) 199--220.      
%\bibitem{Helmholtz2} Hermann von Helmholtz; {\em \"Uber Wirbelbewegungen,
%        \"Uber Fl\"ussigkeitsbewegungen}, 1858; in Ostwalds Klassiker der 
%       exakten Wissenschaften Bd.\ 1; Verlag Harri Deutsch, Frankfurt, 
%       1996.                   
%\bibitem{Lamb} G.L.\ Lamb, Jr.; {\it Elements of Soliton Theory}; 
%         Pure \& Applied Mathematics, John Wiley \& Sons, 1980. 
%\bibitem{Laue} Max von Laue; {\it Geschichte der Physik}; 
%         Universit\"ats-Verlag Bonn, 1947.
%\bibitem{Lorentz} Hendrik Antoon Lorentz; {\it Electromagnetic phenomena 
 %        in a system moving with any velocity smaller than that of light}; 
%         Proc.\ Acad.\ Sci., Amsterdam, 6 [1904], S.\ 809.
%\bibitem{Mach} Ernst Mach; {\it Die Mechanik in ihrer Entwicklung
%      historisch kritisch dargestellt}; Akademie Verlag, Berlin, 1988.       
%\bibitem{Mainzer} Klaus Mainzer; {\it Philosophie und Geschichte von
%         Raum und Zeit}; in {\it Philosophie und Physik der Raum-Zeit};
%         J\"urgen Audretsch und Klaus Mainzer (Hrsg.); 
%         BI-Wissenschaftsverlag, 1994. 
%\bibitem{Microscope} Touboul, Pierre, et al. (MICROSCOPE Collaboration); \textit{MICROSCOPE Mission:
%         Final Results of the Test of the Equivalence Principle}; Phys.\ Rev.\ Lett.\ \textbf{129} (2022) 121102.
%\bibitem{Misner} C.W.\ Misner, K.S.\ Thorne, J.A.\ Wheeler; 
%        {\it Gravitation}; W.H.\ Freeman and Company, San Francisco,  1973.
%\bibitem{Mittelstaedt} Peter Mittelstaedt; {\it Der Zeitbegriff in der
%        Physik}; BI-Wissenschaftsverlag, 1989.        
%\bibitem{Mittelstaedt2} Peter Mittelstaedt; {\it Philosophische Probleme
%        der modernen Physik}; BI-Wissenschaftsverlag, 1989.        
%\bibitem{Newton}
%   Isaac Newton; {\it Mathematische Grundlagen der Naturphilosophie}; 
%   \"ubersetzt von Ed Dellian; Felix Meiner Verlag, 1988. 
%\bibitem{Newton2} Isaac Newton; {\it \"Uber die Gravitation...};
%       Klostermann Texte Philosophie; Vittorio Klostermann, Frankfurt,
%      1988; \"ubersetzt von Gernot B\"ohme.
%\bibitem{Newton3} Isaac Newton; {\it Optik oder Abhandlung \"uber
%      Spiegelungen, Brechungen, Beugungen und Farben des Lichts};
%      I., II.\ und III.\ Buch (1704); aus dem Englischen \"ubersetzt
%      von W.\ Abendroth; Ostwalds Klassiker der exakten Wissenschaften,
%      Verlag Harri Deutsch 1998.   
%\bibitem{Neumann} Carl Neumann; {\it \"Uber die Principien der
%         Galilei-Newtonschen Theorie}; Akademische Antrittsvorlesung,
%         gehalten in der Aula der Universit\"at Leipzig am 3.\ Nov.\
%         1869; Teubner (Leipzig) 1870.         
\bibitem{Pauli} Wolfgang Pauli; {\it Theory of Relativity}; Dover
      Publications, New York, 1981.      
%\bibitem{Poincare} Jules Henri Poincar\'e; {\it Sur la dynamique de 
%     l'\'electron}, C.R.\ Acad.\ Sci., Paris, 140 (1905) S.~1504; und 
%      Rendiconti del Circolo Matematico di Palermo, Bd.~21 (1906) S.~129.
%\bibitem{Reichenbach1} Hans Reichenbach; {\em Philosophie der 
%       Raum-Zeit-Lehre}; Hans Reichenbach - Gesammelte Werke Bd.\ 2;
%       Vieweg-Verlag, Braunschweig; 1977.
%\bibitem{Reichenbach2} Hans Reichenbach; {\em Axiomatik der
%       relativistischen Raum-Zeit-Lehre}; in {\em Die philosophische
%       Bedeutung der Relativit\"atstheorie}; Hans Reichenbach - Gesammelte
%       Werke Bd.\ 3; Vieweg-Verlag, Braunschweig, 1977. 
%\bibitem{Rovelli} Carlo Rovelli, \textit{Quantum Gravity}; Cambridge
%      University Press, 2007.       
%\bibitem{Schlamminger} Schlamminger, Choi, Wagner, Gundlach,
%         Adelberger; {\em Test of the Equivalence Principle using a
%         rotating torsion balance}; Phys.\ Rev.\ Lett.\ {\bf 100} (2008)
%         041101.     
%\bibitem{Sexl} Roman U.\ Sexl, Helmuth K.\ Urbantke; {\it Relativit\"at,
%      Gruppen, Teilchen}; Springer-Verlag, Wien, New York, 1992.
%\bibitem{Simonyi}
%       K\'aroly Simonyi; {\it Kulturgeschichte der Physik}; Verlag
%       Harri Deutsch, Thun, Frankfurt am Main, 1990.
%\bibitem{Weisberg} Weisberg, J.M., Taylor, J.H.; {\em Relativistic Binary Pulsar
%          B1913+16: Thirty Years of Observations and Analysis}; 
%          \verb+arXiv:astro-ph/0407149v1+; 2004. 
%\bibitem{Thomson} James Thomson; {\it On the Law of Inertia; the
%       Principle of Chronometry; and the Principle of Absolute Clinural
%       Rest, and of Absolute Rotation}; Proc.\ Roy.\ Soc.\ (Edinburgh),
%       Session 1883-84, Vol.\ XII, 568--578.       
%\bibitem{Weizsaecker} Carl Friedrich von Weizs\"acker; {\em Der zweite
%      Hauptsatz und der Unterschied von Vergangenheit und Zukunft};
%      Annalen der Physik 36 (1939) 275--283.       
%\bibitem{Zeh} Zeh, H.D.; {\em The Physical Basis of the Direction of Time},
%      Springer-Verlag, Berlin, 1989.       

%\bibitem{Einstein} Einstein, Albert; {\em ??}, .                   
\end{thebibliography}




\end{document}
