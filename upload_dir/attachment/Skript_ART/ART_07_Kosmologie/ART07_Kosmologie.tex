\documentclass[german,10pt]{book}         
\usepackage{makeidx}
\usepackage{babel}            % Sprachunterstuetzung
\usepackage{amsmath}          % AMS "Grundpaket"
\usepackage{amssymb,amsfonts,amsthm,amscd} 
\usepackage{mathrsfs}
\usepackage{rotating}
\usepackage{sidecap}
\usepackage{graphicx}
\usepackage{color}
\usepackage{fancybox}
\usepackage{tikz}
\usetikzlibrary{arrows,snakes,backgrounds}
\usepackage{hyperref}
\hypersetup{colorlinks=true,
                    linkcolor=blue,
                    filecolor=magenta,
                    urlcolor=cyan,
                    pdftitle={Overleaf Example},
                    pdfpagemode=FullScreen,}
%\newcommand{\hyperref}[1]{\ref{#1}}
%
\definecolor{Gray}{gray}{0.80}
\DeclareMathSymbol{,}{\mathord}{letters}{"3B}
%
\newcounter{num}
\renewcommand{\thenum}{\arabic{num}}
\newenvironment{anmerkungen}
   {\begin{list}{(\thenum)}{%
   \usecounter{num}%
   \leftmargin0pt
   \itemindent5pt
   \topsep0pt
   \labelwidth0pt}%
   }{\end{list}}
%
\renewcommand{\arraystretch}{1.15}                % in Formeln und Tabellen   
\renewcommand{\baselinestretch}{1.15}                 % 1.15 facher
                                                      % Zeilenabst.
\newcommand{\Anmerkung}[1]{{\begin{footnotesize}#1 \end{footnotesize}}\\[0.2cm]}
\newcommand{\comment}[1]{}
\setlength{\parindent}{0em}           % Nicht einruecken am Anfang der Zeile 

\setlength{\textwidth}{15.4cm}
\setlength{\textheight}{23.0cm}
\setlength{\oddsidemargin}{1.0mm} 
\setlength{\evensidemargin}{-6.5mm}
\setlength{\topmargin}{-10mm} 
\setlength{\headheight}{0mm}
\newcommand{\identity}{{\bf 1}}
%
\newcommand{\vs}{\vspace{0.3cm}}
\newcommand{\noi}{\noindent}
\newcommand{\leer}{}

\newcommand{\engl}[1]{[\textit{#1}]}
\parindent 1.2cm
\sloppy

         \begin{document}  \setcounter{chapter}{6}

\chapter{ART-Kosmologie}

Au\ss erdem gibt es L\"osungen der
Einstein'schen Feldgleichungen, die 
kosmologische Modelle beschreiben -- die
bekannteste L\"osung ist in diesem Fall
die Friedman-L\"osung bzw.\ Robertson-Walker-Metrik. 

\section{Kosmologische Modelle}

Eine L\"osung der Einstein'schen Feldgleichungen entspricht einer
vollst\"andigen Raum-Zeit, d.h.\ einem Modell eines Kosmos. Zum ersten
Mal hat die Physik mit der Allgemeinen Relativit\"atstheorie somit
ein Modell an der Hand, mit dem sich kosmologische Fragen, insbesondere
auch zur Entstehungsgeschichte des Universums, 
wissenschaftlich angehen lassen. 

Einstein ging zun\"achst davon aus, dass unser Universum auf gro\ss en
Skalen im Wesentlichen statisch sei. Er musste jedoch rasch feststellen,
dass seine Feldgleichungen ein solch statisches Universum nur in
sehr unphysikalischen Situationen ($T_{\mu \nu}=0$) zulassen.
Um auch f\"ur realistischere Materieverteilungen L\"osungen zu einem
statischen Universum zu erhalten, erweiterte Einstein seine
Feldgleichungen um einen sogenannten kosmologischen 
Term mit einer kosmologischen Konstanten 
$\Lambda$\index{Kosmologische Konstante}, sodass Gl.~\ref{EinsteinGl}
zu folgender Gleichung wird:\index{Einstein'sche Gleichungen}
\begin{equation}
   R_{\mu \nu} - \frac{1}{2} g_{\mu \nu} R + \Lambda g_{\mu \nu}
     ~=~ - \frac{8 \pi G}{c^4} T_{\mu \nu}   \;. 
\end{equation}
Geometrisch k\"onnte man $\Lambda$ als eine 
\glqq negative Raumkr\"ummung\grqq\ des Vakuums interpretieren, die durch
die vorhandene Materie nahezu ausgeb\"ugelt wird. Schl\"agt man den
kosmologischen Term der rechten Seite der Gleichung zu, so kann man ihn
als eine Art Energiedichte des Vakuums interpretieren, die zu einer
negativen Raumkr\"ummung f\"uhrt. 

Durch die kosmologische Konstante hoffte Einstein, statische L\"osungen
der Feldgleichungen mit Materiefeldern zu erhalten. Er wurde aber rasch
entt\"auscht. Angeblich (die einzige Quelle f\"ur dieses Zitat scheint der
Physiker George Gamow zu sein) 
hat Einstein bei sp\"aterer Gelegenheit die Einf\"uhrung
dieses Terms als seine \glqq gr\"o\ss te Eselei\grqq\ bezeichnet. 

\subsection{Das Olbers'sche Paradoxon}
\index{Olbers, Heinrich Wilhelm Matthias}

Im Rahmen der klassischen Kosmologie war schon bekannt, dass die
Annahme eines homogenen, seit unendlichen Zeiten in gleicher Form
bestehenden Kosmos zu einem Widerspruch f\"uhrt. Heute bezeichnet man
dieses Paradoxon meist nach dem Astronom und Arzt Heinrich Wilhelm 
Matthias Olbers ({\em geb.\ 11.10.1758 in Arbergen bei Bremen; gest.\
2.3.1840 in Bremen}), obwohl entsprechende \"Uberlegungen bereits von 
Edmund Halley (1656--1742)\index{Halley, Edmund} angestellt wurden.

Olbers argumentierte, dass der Himmel in alle Richtungen dieselbe
Helligkeit wie die Sonne haben m\"usste. Insbesondere m\"usste es
auch Nachts \glqq taghell\grqq\ sein. Das Argument basiert auf der
Annahme eines unendlich ausgedehnten, homogenen Universums 
(d.h.\ die Sternendichte ist \"uberall nahezu konstant), das in dieser
Form auch seit unendlicher Zeit existiert hat. In diesem Fall m\"usste
n\"amlich aus jeder Raumrichtung das Licht eines Sterns auf die Erde
treffen. 

Olbers selber glaubte das Paradoxon dadurch umgehen zu k\"onnen, dass 
er Wolken im Kosmos annahm, die das Licht von sehr weit entfernten
Sternen verdecken. Man wei\ss\ heute jedoch, dass sich diese Wolken
durch die einfallende Strahlung h\"atten erw\"armen m\"ussen, bis sie
schlie\ss lich ins thermische Gleichgewicht mit dieser Strahlung gekommen
w\"aren, d.h.\ ebenfalls die Strahlung emittieren w\"urden. Auch die
Annahme einer endlichen Lebensdauer der Sterne umgeht das Paradoxon
nicht, wenn man zus\"atzlich fordert, dass die mittlere Sterndichte
konstant bleibt, also auch st\"andig neue Sterne entstehen.

Aus heutiger Sicht gibt es zwei L\"osungen dieses Olbers'schen
Paradoxons (vgl.\ \cite{Fliessbach}, S.~347):
\begin{enumerate}
\item
F\"ur ein endliches Weltalter gibt es den\index{Ereignishorizont} 
Ereignishorizont, jenseits
dessen wir nichts sehen. Auch in einem unendlich ausgedehnten Universum
erreicht uns nur Licht aus einem Bereich, der in unserer kausalen 
Vergangenheit liegt.
\item
Bei einer Expansion des Universums nimmt die Fluchtgeschwindigkeit mit
dem Abstand zu. Geht diese Fluchtgeschwindigkeit gegen die
Lichtgeschwindigkeit, so muss die Rotverschiebung des wahrgenommenen
Lichtes gegen Unendlich gehen. Auch hierdurch wird die empfangene
Helligkeit begrenzt. Diese Schranke der Wahrnehmbarkeit ist f\"ur
Robertson-Walker-Universen (s.u.) mit dem Ereignishorizont identisch.
\end{enumerate}

\subsection{Expandierende Universen}

Im Jahre 1924 zeigte Edwin Powell Hubble ({\em geb.\ 20.11.1889 in
Marshfield (Missouri), gest.\ 28.9.1953 in San Marino (Kalifornien)})
\index{Hubble, Edwin Powell}
die Existenz von Galaxien au\ss erhalb unseres Sternensystems. F\"unf
Jahre sp\"ater entdeckte er die Expansion des Weltalls \"uber die
Rotverschiebung entfernter Galaxien. Zu dem Zeitpunkt, als Einstein
die Allgemeine Relativit\"atstheorie entwickelt hatte, waren also weder
au\ss ergalaktische Objekte noch die Expansion des Universums bekannt.

Doch schon im Jahre 1917 fand der sowjetische Kosmologe Aleksandr
Alexandrovich Friedmann 
(1888--1925)\index{Friedmann, Aleksandr Alexandrovich}
L\"osungen der Einstein-Gleichungen, die ein expandierendes Universum
beschreiben. Er legte so die Grundlagen f\"ur unsere heutige
Big-Bang- bzw.\ Urknall-Theorie\index{Big-Bang}\index{Urknall}.

Die wesentliche Annahme, die f\"ur kosmologische L\"osungen der 
Einstein-Glei\-chungen
meist gemacht wird, ist die Homogenit\"at und Isotropie unseres
Universums. Darunter versteht man, dass auf sehr gro\ss en Skalen
kein Ort und keine Richtung im Universum ausgezeichnet sind. Diese
Annahme bezeichnet man auch als 
kosmologisches Prinzip\index{Kosmologisches Prinzip}. F\"ur
den geometrischen Anteil der Einstein-Gleichungen bedeutet dies, dass 
die dreidimensionale Kr\"ummung r\"aumlich konstant sein mu\ss. Lediglich
eine Zeitabh\"angigkeit dieser Kr\"ummung ist noch erlaubt. Es zeigt sich,
dass unter diesen Bedingungen nur noch eine Metrik der Form
\begin{equation}
    {\rm d}s^2 ~=~ c^2 {\rm d}t^2 - R(t)^2 
    \left( \frac{{\rm d}r^2}{1-kr^2} + r^2( {\rm d}\vartheta^2 +
    \sin^2 \vartheta \; {\rm d}\phi^2) \right)    
\end{equation}
m\"oglich ist. Diese Metrik bezeichnet man als 
Robertson-Walker-Metrik\index{Robertson-Walker-Metrik}. 

Zwei freie Parameter kennzeichnen diese Metrik: Der Parameter $k$, der
durch geeignete Skalierung von $r$ auf die Werte $k=0,+1,-1$ beschr\"ankt
werden kann, und der Wert $R(t)$, der \"uber die Gleichung
\begin{equation}
        K ~=~ \frac{k}{R(t)^2}    
\end{equation}
mit dem dreidimensionalen Kr\"ummungsskalar $K$ in Beziehung steht.
Der Wert von $k$ unterscheidet somit, ob die dreidimensionale
skalare Kr\"ummung positiv, null oder negativ ist. Dem entsprechen
drei unterschiedlichen Formen von Universen.
Insbesondere ist f\"ur $k=1$ der dreidimensionale Raum endlich, aber ohne
Grenze (Kugel). 

Das kosmologische Prinzip wird gelegentlich angezweifelt, und man
kann zurecht fragen, ob wir wirklich eine Homogenit\"at und Isotropie
des Raumes beobachten. Der sichtbare Teil des Universums hat einen
Radius von ungef\"ahr $10^{10}$\,Lichtjahren. Unsere Galaxie andererseits
hat einen Radius von $10^5$\,Lichtjahren. Die meisten Galaxien sind in
Clustern oder Haufen mit einem Durchmesser von rund $10^7$\,Lichtjahren 
konzentriert. Bis zu dieser Skala beobachten wir somit durchaus
reichhaltige Strukturen auch in der Form der Materieverteilung. Es handelt 
sich also um maximal zwei bis drei Gr\"o\ss enordnungen, f\"ur die das
kosmologische Prinzip g\"ultig sein k\"onnte. Ob das der Fall ist, oder
ob es weitere charakteristische Strukturen jenseits der Galaxiencluster
gibt, m\"ussen zuk\"unftige Messungen entscheiden.

Noch wurde nichts \"uber die zeitliche Entwicklung des Universums
ausgesagt. Diese steckt in der Abh\"angigkeit des \glqq Radius\grqq\ $R(t)$
-- genauer sollte man von einer Skala sprechen -- des Universums von
der Zeit und sollte aus der Einstein-Gleichung bestimmt werden.
Dazu macht man \"ublicherweise Annahmen \"uber den 
Energie-Impuls-Tensor\index{Energie-Impuls-Tensor}
der Materie, der nach dem kosmologischen Prinzip ebenfalls r\"aumlich
konstant und isotrop sein sollte. Die wesentliche Freiheit besteht
in der Relation zwischen der Materiedichte $\rho$ und dem 
\glqq Radius\grqq\
$R(t)$. F\"ur \glqq normale\grqq\ Materie gilt 
\begin{equation}
       \rho_m R(t)^3 ~=~ {\rm const}   \hspace{2cm}
      \mbox{(materiedominiert)}  \;,   
\end{equation}
also die bekannte Relation, dass die Dichte umgekehrt proportional zum
Volumen ist. F\"ur Strahlung beispielsweise gilt
\begin{equation}
       \rho_s R(t)^4 ~=~ {\rm const}   \hspace{2cm}
      \mbox{(strahlungsdominiert)}  \;.   
\end{equation}
Mit diesen Relationen erh\"alt man aus den Einstein-Gleichungen eine 
einfache Differentialgleichung f\"ur $R(t)$,
\begin{equation}
\label{Friedmann}
     \dot{R}^2 + V(R) ~=~ - k    \;,   
\end{equation}
mit
\begin{equation}
  V(R) ~=~ - \frac{a}{R^2} - \frac{b}{R} - \frac{1}{3} \Lambda R^2 \;.
\end{equation}  
$a$ und $b$ sind Konstanten, die den Anteil an Strahlung bzw.\
normaler Materie im Universum angeben, und $\Lambda$ ist die kosmologische
Konstante. Modelle, bei denen $R(t)$ der Gl.~\ref{Friedmann} gen\"ugt, 
bezeichnet man als Friedmann-Modelle\index{Friedmann-Modelle}.

Qualitativ lassen sich die L\"osungen von Gl.~\ref{Friedmann} leicht
durch die physikalische Analogie mit der Energie eines eindimensionalen
Teilchens in einem effektivem Potential $V(R)$ diskutieren. F\"ur
$\Lambda=0$ beispielsweise kann es Universen
geben, deren Radius (Skala) nach oben beschr\"ankt ist
 -- in diesem Fall kommt es wieder zu einem Kollaps. Oder
aber das Universum expandiert f\"ur alle Zeiten. 

\begin{SCfigure}[50][htb]
\begin{picture}(210,200)(-10,0)
\put(20,0){\vector(0,1){180}}
\put(16,100){\vector(1,0){174}}
\put(10,190){\makebox(0,0){$V(R)$}}
\put(190,90){\makebox(0,0){$R$}}
\put(175,145){\makebox(0,0){${\scriptstyle \Lambda < 0}$}}
\put(160,90){\makebox(0,0){${\scriptstyle \Lambda = 0}$}}
\put(161,35){\makebox(0,0){${\scriptstyle \Lambda > 0}$}}
\put(5,100){\makebox(0,0){${\scriptstyle k=0}$}}
\put(5,120){\makebox(0,0){${\scriptstyle k=1}$}}
\put(5,80){\makebox(0,0){${\scriptstyle k=-1}$}}
\put(16,80){\line(1,0){8}}
\put(16,120){\line(1,0){8}}
\qbezier(90,90)(150,100)(180,180)
\qbezier(90,90)(140,95)(180,20)
\qbezier(90,90)(140,97)(180,97)
\thicklines
\qbezier(25,0)(30,80)(90,90)
\end{picture}
\caption{\label{fig_cosmos}%
Das \glqq effektive Potenzial\grqq\ f\"ur die Dynamik des
Skalenfaktors $R(t)$ in einer Friedman-L\"osung f\"ur
verschiedene Werte der kosmologischen Konstanten $\Lambda$.
In Abh\"angigkeit von $k$ verschiebt sich die $V(R)=0$-Achse
nach oben oder untern.}
\end{SCfigure}

Der wesentliche
Parameter f\"ur diese Unterscheidung ist die Materiedichte im Universum.
Aus der sichtbaren Materie in unserem Universum w\"urde man auf einen
Wert von $\rho$ schlie\ss en, der zu einem ewig expandierenden
Universum f\"uhrt. Allerdings deuten genaue Untersuchungen der
Bewegungen von Galaxien darauf hin, dass der gr\"o\ss te Teil der
Materie in unserem Universum unsichtbar ist. 

Nachdem vor einigen Jahren aufgrund genauer Beobachtungen und Messugen
an Supernovaexplosionen die Entfernungsskalen f\"ur Objekte, deren
Entfernung nicht mehr durch Paralaxenmessugn m\"oglich ist, revidiert
werden mussen, ergibt sich heute (Stand Januar 2015) das Bild, dass sich
unser Universum in einem Stadium befindet, in dem die Geschwindigkeit der
Expansion wieder zunimmt, nachdem es vor rund 8 Milliarden Jahren
eine \glqq minimale Expansionsrate\grqq\ durchlaufen hat. 
Die Beobachtungen lassen sich zwar durch eine positive kosmologische
Konstante beschreiben, doch die Natur dieser Konstanten bleibt
ungewiss. Manche Modelle postulieren eine ganz neue 
Energie- oder gar Materieform (man spricht auch manchmal von
{\em Dunkler Energie}), die\index{Dunkle Energie} 
sich durch einen negativen Druck 
auszeichnet und das Universum zur Expansion \glqq dr\"angt\grqq. 
In kaum einem Gebiet der Physik \"andern sich derzeit die 
grundlegenden Vorstellungen innerhalb weniger Jahre so oft und so 
einschneidend wie in der Kosmologie.

\subsection{Das de Sitter-Universum}

Sollten sich die augenblicklichen Vorstellungen als richtig erweisen
und die Skala des Universums beschleunigt zunehmen, wird der
Materiegehalt immer d\"unner und spielt eine zunehmend
geringere Rolle. In diesem Fall ist das Verhalten des Universums
durch die Kosmologische\index{Kosmologische Konstante} 
Konstante $\Lambda$ dominiert, die
bei einem beschleunigt expandierenden Universum positiv sein muss.

Ein Universum ohne Materiegehalt aber mit einer positiven
Kosmologischen Konstanten bezeichnet man als 
\textit{de Sitter-Universum}.\index{de Sitter-Universum}
Die \glqq Bewegungsgleichung\grqq\ f\"ur $R(t)$ (Gl.\ \ref{Friedmann}) 
wird in diesem Fall zu
\begin{equation}
           \dot{R}(t) = \frac{1}{\sqrt{3}} \sqrt{\Lambda} R(t) = H R(t) 
\end{equation}  
mit Expansionsrate $H$. Die L\"osung lautet:
\begin{equation}
              R(t) = R_0 {\rm e}^{H t}  \, .
\end{equation}
Dies ist die \textit{de Sitter-L\"osung} der Einstein'schen Gleichungen
f\"ur einen materiefreien Raum mit einer kosmologischen Konstanten.
Sie beschreibt nicht nur m\"oglicherweise das asymptotische
Verhalten unseres Universums in der fernen Zukunft, sondern sie
hat auch viele interessante mathematische Eigenschaften.

\begin{thebibliography}{99}
%\addcontentsline{toc}{chapter}{Literaturangaben}
%\bibitem{Aichelburg} Peter C.\ Aichelburg (Hrsg.); {\it Zeit im 
%       Wandel der Zeit}; Verlag Vieweg, Braunschweig, Wiesbaden, 1988.
%\bibitem{Barbour3} {\it Mach's Principle -- From Newton's Bucket to
%        Quantum Gravity}; Julian Barbour \& Herbert Pfister (Hrsg.);
%        Birkh\"auser, Boston, Basel, Berlin, 1995.       
%\bibitem{Bekenstein} Jacob D.\ Bekenstein, \textit{Black holes
%          and entropy}, Phys.\ Rev.\ D\,7 (1973) 2333--2346.        
%\bibitem{Bell} John Bell;  {\em Speakable and Unspeakable in 
%        Quantum Physics}, 2.\ edition, Cambridge University Press (2004).       
%\bibitem{Born} Max Born; {\it Optik}; Springer-Verlag, Berlin, Heidelberg,
%        1972.
%\bibitem{Britannica} Encyclopaedia Britannica; 15.th edition, 1988.
%\bibitem{Descartes} Ren\'e Descartes; {\it Die Prinzipien der
%        Philosophie}; Felix Meiner Verlag, Hamburg, 1992; \"ubersetzt
%        von Artur Buchenau.
%\bibitem{EDM} Encyclopaedic Dictionary of Mathematics; Second Edition,
%        MIT Press, 1987.
%\bibitem{Einstein1} Albert Einstein; {\it Zur Elektrodynamik bewegter 
%        K\"orper}; Annalen der Physik, Leipzig, 17 (1905) 891. 
%\bibitem{Einstein2} Albert Einstein; {\it Ist die Tr\"agheit eines
%        K\"orpers von seinem Energieinhalt abh\"angig?} (Ann.\ Phys., 
%        Leipzig, 18 (1905) 639.
%\bibitem{Einstein3} Albert Einstein; {\it Aus meinen sp\"aten Jahren};
%         Ullstein Sachbuch, Verlag Ullstein, Frankfurt, Berlin, 1993.                 
%\bibitem{Einstein4} Albert Einstein; {\it Prinzipielles zur allgemeinen
%        Relativit\"atstheorie}; Annalen der Physik 55 (1918) 241.
%\bibitem{Einstein5} Albert Einstein; {\it \"Uber den Einflu\ss\ der
%        Schwerkraft auf die Ausbreitung des Lichtes}; Annalen der
%        Physik 35 (1911) 898.                 
%\bibitem{Feynman} Richard Feynman; {\it The Character of Physical Law};
%        The MIT Press, 1987.        
%\bibitem{Fierz} Markus Fierz; {\it \"Uber den Ursprung und die Bedeutung
%        der Lehre Isaac Newtons vom absoluten Raum}; Gesnerus, 
%        11.\ Jahrgang (1954), S.\,62--120.
%\bibitem{Fliessbach} Torsten Flie\ss bach; {\it Allgemeine 
%        Relativit\"atstheorie}; BI-Wissenschaftsverlag, Mannheim, Wien
%        Z\"urich, 1990. 
%\bibitem{Galilei} Galilei; {\it Dialog \"uber die beiden haupts\"achlichen
%        Weltsysteme, das ptolem\"aische und das kopernikanische}; 
%        Teubner Stuttgart, 1982; aus dem Italienischen \"ubersetzt von
%        Emil Strauss.   
%  \bibitem{Hawking} Stephen W.\ Hawking, \textit{Particle Creation by
%            black holes}, Comm.\ Math.\ Phys.\ 43 (1976) 199--220.      
%\bibitem{Helmholtz2} Hermann von Helmholtz; {\em \"Uber Wirbelbewegungen,
%        \"Uber Fl\"ussigkeitsbewegungen}, 1858; in Ostwalds Klassiker der 
%       exakten Wissenschaften Bd.\ 1; Verlag Harri Deutsch, Frankfurt, 
%       1996.                   
%\bibitem{Lamb} G.L.\ Lamb, Jr.; {\it Elements of Soliton Theory}; 
%         Pure \& Applied Mathematics, John Wiley \& Sons, 1980. 
%\bibitem{Laue} Max von Laue; {\it Geschichte der Physik}; 
%         Universit\"ats-Verlag Bonn, 1947.
%\bibitem{Lorentz} Hendrik Antoon Lorentz; {\it Electromagnetic phenomena 
 %        in a system moving with any velocity smaller than that of light}; 
%         Proc.\ Acad.\ Sci., Amsterdam, 6 [1904], S.\ 809.
%\bibitem{Mach} Ernst Mach; {\it Die Mechanik in ihrer Entwicklung
%      historisch kritisch dargestellt}; Akademie Verlag, Berlin, 1988.       
%\bibitem{Mainzer} Klaus Mainzer; {\it Philosophie und Geschichte von
%         Raum und Zeit}; in {\it Philosophie und Physik der Raum-Zeit};
%         J\"urgen Audretsch und Klaus Mainzer (Hrsg.); 
%         BI-Wissenschaftsverlag, 1994. 
%\bibitem{Microscope} Touboul, Pierre, et al. (MICROSCOPE Collaboration); \textit{MICROSCOPE Mission:
%         Final Results of the Test of the Equivalence Principle}; Phys.\ Rev.\ Lett.\ \textbf{129} (2022) 121102.
%\bibitem{Misner} C.W.\ Misner, K.S.\ Thorne, J.A.\ Wheeler; 
%        {\it Gravitation}; W.H.\ Freeman and Company, San Francisco,  1973.
%\bibitem{Mittelstaedt} Peter Mittelstaedt; {\it Der Zeitbegriff in der
%        Physik}; BI-Wissenschaftsverlag, 1989.        
%\bibitem{Mittelstaedt2} Peter Mittelstaedt; {\it Philosophische Probleme
%        der modernen Physik}; BI-Wissenschaftsverlag, 1989.        
%\bibitem{Newton}
%   Isaac Newton; {\it Mathematische Grundlagen der Naturphilosophie}; 
%   \"ubersetzt von Ed Dellian; Felix Meiner Verlag, 1988. 
%\bibitem{Newton2} Isaac Newton; {\it \"Uber die Gravitation...};
%       Klostermann Texte Philosophie; Vittorio Klostermann, Frankfurt,
%      1988; \"ubersetzt von Gernot B\"ohme.
%\bibitem{Newton3} Isaac Newton; {\it Optik oder Abhandlung \"uber
%      Spiegelungen, Brechungen, Beugungen und Farben des Lichts};
%      I., II.\ und III.\ Buch (1704); aus dem Englischen \"ubersetzt
%      von W.\ Abendroth; Ostwalds Klassiker der exakten Wissenschaften,
%      Verlag Harri Deutsch 1998.   
%\bibitem{Neumann} Carl Neumann; {\it \"Uber die Principien der
%         Galilei-Newtonschen Theorie}; Akademische Antrittsvorlesung,
%         gehalten in der Aula der Universit\"at Leipzig am 3.\ Nov.\
%         1869; Teubner (Leipzig) 1870.         
%\bibitem{Pauli} Wolfgang Pauli; {\it Theory of Relativity}; Dover
%      Publications, New York, 1981.      
%\bibitem{Poincare} Jules Henri Poincar\'e; {\it Sur la dynamique de 
%     l'\'electron}, C.R.\ Acad.\ Sci., Paris, 140 (1905) S.~1504; und 
%      Rendiconti del Circolo Matematico di Palermo, Bd.~21 (1906) S.~129.
%\bibitem{Reichenbach1} Hans Reichenbach; {\em Philosophie der 
%       Raum-Zeit-Lehre}; Hans Reichenbach - Gesammelte Werke Bd.\ 2;
%       Vieweg-Verlag, Braunschweig; 1977.
%\bibitem{Reichenbach2} Hans Reichenbach; {\em Axiomatik der
%       relativistischen Raum-Zeit-Lehre}; in {\em Die philosophische
%       Bedeutung der Relativit\"atstheorie}; Hans Reichenbach - Gesammelte
%       Werke Bd.\ 3; Vieweg-Verlag, Braunschweig, 1977. 
%\bibitem{Rovelli} Carlo Rovelli, \textit{Quantum Gravity}; Cambridge
%      University Press, 2007.       
%\bibitem{Schlamminger} Schlamminger, Choi, Wagner, Gundlach,
%         Adelberger; {\em Test of the Equivalence Principle using a
%         rotating torsion balance}; Phys.\ Rev.\ Lett.\ {\bf 100} (2008)
%         041101.     
%\bibitem{Sexl} Roman U.\ Sexl, Helmuth K.\ Urbantke; {\it Relativit\"at,
%      Gruppen, Teilchen}; Springer-Verlag, Wien, New York, 1992.
%\bibitem{Simonyi}
%       K\'aroly Simonyi; {\it Kulturgeschichte der Physik}; Verlag
%       Harri Deutsch, Thun, Frankfurt am Main, 1990.
%\bibitem{Weisberg} Weisberg, J.M., Taylor, J.H.; {\em Relativistic Binary Pulsar
%          B1913+16: Thirty Years of Observations and Analysis}; 
%          \verb+arXiv:astro-ph/0407149v1+; 2004. 
%\bibitem{Thomson} James Thomson; {\it On the Law of Inertia; the
%       Principle of Chronometry; and the Principle of Absolute Clinural
%       Rest, and of Absolute Rotation}; Proc.\ Roy.\ Soc.\ (Edinburgh),
%       Session 1883-84, Vol.\ XII, 568--578.       
%\bibitem{Weizsaecker} Carl Friedrich von Weizs\"acker; {\em Der zweite
%      Hauptsatz und der Unterschied von Vergangenheit und Zukunft};
%      Annalen der Physik 36 (1939) 275--283.       
%\bibitem{Zeh} Zeh, H.D.; {\em The Physical Basis of the Direction of Time},
%      Springer-Verlag, Berlin, 1989.       

%\bibitem{Einstein} Einstein, Albert; {\em ??}, .                   
\end{thebibliography}

\end{document}
