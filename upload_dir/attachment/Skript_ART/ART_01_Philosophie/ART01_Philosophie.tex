\documentclass[german,10pt]{book}    
\usepackage{makeidx}
\usepackage{babel}            % Sprachunterstuetzung
\usepackage{amsmath}          % AMS "Grundpaket"
\usepackage{amssymb,amsfonts,amsthm,amscd} 
\usepackage{mathrsfs}
\usepackage{rotating}
\usepackage{sidecap}
\usepackage{graphicx}
\usepackage{color}
\usepackage{fancybox}
\usepackage{tikz}
\usetikzlibrary{arrows,snakes,backgrounds}
\usepackage{hyperref}
\hypersetup{colorlinks=true,
                    linkcolor=blue,
                    filecolor=magenta,
                    urlcolor=cyan,
                    pdftitle={Overleaf Example},
                    pdfpagemode=FullScreen,}
%\newcommand{\hyperref}[1]{\ref{#1}}
%
\definecolor{Gray}{gray}{0.80}
\DeclareMathSymbol{,}{\mathord}{letters}{"3B}
%
\newcounter{num}
\renewcommand{\thenum}{\arabic{num}}
\newenvironment{anmerkungen}
   {\begin{list}{(\thenum)}{%
   \usecounter{num}%
   \leftmargin0pt
   \itemindent5pt
   \topsep0pt
   \labelwidth0pt}%
   }{\end{list}}
%
\renewcommand{\arraystretch}{1.15}                % in Formeln und Tabellen   
\renewcommand{\baselinestretch}{1.15}                 % 1.15 facher
                                                      % Zeilenabst.
\newcommand{\Anmerkung}[1]{{\begin{footnotesize}#1 \end{footnotesize}}\\[0.2cm]}
\newcommand{\comment}[1]{}
\setlength{\parindent}{0em}           % Nicht einruecken am Anfang der Zeile 

\setlength{\textwidth}{15.4cm}
\setlength{\textheight}{23.0cm}
\setlength{\oddsidemargin}{1.0mm} 
\setlength{\evensidemargin}{-6.5mm}
\setlength{\topmargin}{-10mm} 
\setlength{\headheight}{0mm}
\newcommand{\identity}{{\bf 1}}
%
\newcommand{\vs}{\vspace{0.3cm}}
\newcommand{\noi}{\noindent}
\newcommand{\leer}{}

\newcommand{\engl}[1]{[\textit{#1}]}
\parindent 1.2cm
\sloppy

      \begin{document}
 
\chapter{Die ART -- Allgemeine Vorbemerkungen}

Bei der speziellen Relativit\"atstheorie kann man noch dar\"uber
streiten, welchen Beitrag Albert Einstein zu ihrer Entwicklung wirklich
geleistet hat und ob sie nicht vielleicht auch ohne ihn kurz vor ihrer
Entdeckung stand. Die wichtigen Formeln waren alle bekannt und
insbesondere Henri Poincar\'e hatte schon wesentliche Fortschritte
hinsichtlich der Interpretation dieser Formeln erzielt. Selbst die 
transformierte Zeit war als \glqq lokale Hilfsgr\"o\ss e\grqq\ schon 
aufgetaucht und in Gebrauch. 

Bei der allgemeinen Relativit\"atstheorie ist aber unbestritten, dass 
ihre grundlegenden Ideen das Werk von Einstein alleine sind. Es gab zwar
hinsichtlich der Formulierung der korrekten Feldgleichungen der
allgemeinen Relativit\"atstheorie f\"ur lange Zeit einen Priorit\"atsstreit
mit David Hilbert (vgl.~\cite{Simonyi}, S.~418/19 und 424), der allerdings
in weiten Z\"ugen mittlerweile zugunsten von Einstein entschieden ist.

\section{Die Motivationen f\"ur die allgemeine Relativi\-t\"atstheorie}
\index{Einstein, Albert}
\vspace{0.4cm}

Einstein ({\em geb.~14.3.1879 in Ulm; gest.~18.4.1955 in Princeton 
(New Jersey)}) hatte sicherlich eine besondere F\"ahigkeit zu sp\"uren, wann
er etwas nicht genau verstanden hatte, wo die Ursachen daf\"ur lagen, und
wie er das Problem mit einfachen \"Uberlegungen und Gedankenexperimenten 
angehen konnte. Gerade im Zusammenhang mit der Entwicklung der ART hat
er das immer wieder unter Beweis gestellt.
Welche \"Uberlegungen genau Einstein zur allgemeinen Relativit\"atstheorie
gef\"uhrt haben, ist schwer nachzuvollziehen. Die folgenden,
eng zusammenh\"angenden Argumente geh\"orten sicherlich dazu.

\subsection{Das \"Aquivalenzprinzip}
\label{Equivalenz}\index{Aequivalenzprinzip@\"Aquivalenzprinzip}

Eine erste, vereinfachte Form des \"Aquivalenzprinzips lautet
\begin{itemize}
\item
Schwere und tr\"age Masse sind gleich.
\end{itemize}
Die tr\"age Masse $m_{\rm t}$ ist hierbei \"uber
das zweite Newton'sche Gesetz bei {\em bekannter} Kraft $F$ definiert: 
\index{Masse!traege@tr\"age}
\begin{equation}
               m_{\rm t} ~=~ \frac{F}{a}   \;.   
\end{equation}
               
Die schwere Masse steht im Gravitationsgesetz. Zwischen zwei K\"orpern
mit den schweren Massen $m_{\rm g}$ und $m_{\rm g}'$ im Abstand $r$ wirkt 
die Kraft:\index{Masse!schwere}
\begin{equation}        
           F ~=~ G \frac{m_{\rm g} m_{\rm g}'}{r^2}  \;.   
\end{equation}
Hierbei ist $G$ die Newton'sche Konstante. Insbesondere erf\"ahrt 
ein K\"orper (nahe der Erdoberfl\"ache) der Masse $m_{\rm g}$ im Gravitationsfeld der Erde
(schwere Masse $M$, Radius $R$) die Schwerkraft 
\begin{equation}
        F ~=~ m_{\rm g} g   \hspace{1cm} {\rm mit} \hspace{1cm}  g = G \frac{M}{R^2} \, ,   
\end{equation}
wobei $g$ die Erdbeschleunigung (meist als $9,81\,{\rm ms^{-2}}$ angenommen) bezeichnet. 

Experimentell \"uberpr\"ufbar ist nur, dass tr\"age und
schwere Masse zueinander proportional sind. Ihre Gleichheit ist
unter diesen Bedingungen eine Konvention. \"Ublicher\-weise w\"ahlt man die 
Newton'sche
Konstante bzw.\ die Erdbeschleunigung so, dass der 
Proportionalit\"atsfaktor zwischen tr\"ager und schwerer Masse eins wird.
Diese Proportionalit\"at wurde mittlerweile mit einer Genauigkeit von rund $10^{-15}$ 
\"uberpr\"uft (vgl.\ z.B.\ \cite{Schlamminger,Microscope}). Derzeit ist ein weiteres Experiment geplant
 - STEP, Satellite Test of the Equivalence Principle -, mit dem eine Genauigkeit von $10^{-18}$ 
 angestrebt wird.

Die obige Formulierung des \"Aquivalenzprinzips ist in mehrfacher Hinsicht
vereinfacht. In erster Linie liegt das daran, dass wir die tr\"age wie 
auch die schwere Masse durch nicht-relativistische Gleichungen definiert 
haben. In einer veralteten Sprechweise (wie sie Einstein
noch verwandte) k\"onnte man sagen, dass wir das
\"Aquivalenzprinzip f\"ur \glqq ponderable Materie\grqq\ definiert haben.
Es ist damit noch nicht eindeutig gekl\"art, inwiefern auch f\"ur
andere Energieformen das \"Aquivalenzprinzip gilt.

Experimentell zeigt sich, dass eine allgemeinere Formulierung des 
\"Aquivalenz\-prinzips gilt, die sich weniger auf die Massen als auf die 
Kr\"afte bezieht:
\begin{itemize}
\item
Gravitationskr\"afte sind \"aquivalent zu Beschleunigungskr\"aften.
\end{itemize}
In der Formulierung von Einstein (\cite{Einstein4}) hei\ss t es sogar:
\glqq Tr\"agheit und Schwere sind wesensgleich.\grqq\ Doch was bedeutet hier
\glqq \"aquivalent\grqq\ oder \glqq wesensgleich\grqq?

\"Ublicherweise dr\"uckt man diese \"Aquivalenz folgenderma\ss en aus:
Die Wirkung eines konstanten Gravitationsfeldes 
l\"asst sich durch den \"Ubergang
zu einem beschleunigten Bezugssystem eliminieren. 
F\"ur ein nicht homogenes Gravitationsfeld
muss man sich auf ein {\em lokales Bezugssystem}
beschr\"anken. Die Abmessungen eines solchen 
lokalen Bezugssystems (das schlie\ss t auch die
Zeitdauer f\"ur Experimente ein) m\"ussen klein genug sein, 
dass bei vorgegebener Messgenauigkeit keine Effekte der
Inhomogenit\"at des Feldes nachgewiesen werden k\"onnen. Sie h\"angen
somit sowohl vom Inhomogenit\"atsgrad des Feldes als auch von dem
zul\"assigen Messfehler ab. Ein lokales Bezugssystems ist daher 
nur approximativ realisierbar, \"ahnlich wie die 
\glqq nicht-relativistische Mechanik\grqq\ oder die \glqq klassische
Mechanik\grqq\ (im Sinne von Nicht-Quantenmechanik).

Die \glqq \"Aquivalenz\grqq\ zwischen einem Gravitationsfeld
und einer Beschleunigung l\"asst sich folgenderma\ss en
pr\"azisieren: 
\begin{itemize}
\item
In einem lokalen Bezugssystem
l\"asst sich der Einfluss einer Gravitationskraft nicht von
der Wirkung einer Beschleunigung unterscheiden. 
\end{itemize}
Der Begriff des lokalen
Bezugssystems ist somit auch auf Beschleunigungskr\"afte zu \"ubertragen.
Eine Rotation des Bezugssystems um eine Achse durch das System ist
beispielsweise nicht erlaubt, ebensowenig kann die Wirkung des
Gravitationsfeldes von einem Gegenstand {\em innerhalb} eines
Bezugssystems durch Beschleunigungskr\"afte beschrieben werden.

F\"ur einfache F\"alle der nicht-relativistischen Mechanik ist diese
zweite Form des \"Aquivalenzprinzips leicht aus der ersten Formulierung 
(der Gleichheit von tr\"ager und schwerer Masse) herzuleiten. F\"ur 
kompliziertere F\"alle, beispielsweise der relativistischen Mechanik oder 
der Elektrodynamik, muss die Wirkung der Gravitation bzw.\ der
Beschleunigung zun\"achst genau bekannt sein, um das \"Aquivalenzprinzip
\"uberpr\"ufen zu k\"onnen. Insofern ist die eigentliche Aussage des
\"Aquivalenzprinzips, wie sie der allgemeinen Relativit\"atstheorie 
zugrunde liegt, eher ein Postulat:
\begin{itemize}
\item
In einem lokalen Inertialsystem gelten die Gesetze der speziellen
Relativit\"ats\-theorie.
\end{itemize}
Durch kein Experiment soll somit die Wirkung der Gravitation in einem
lokalen Bezugssystem von der Wirkung einer Beschleunigung unterschieden
werden k\"onnen.

Trotz dieser scheinbar klaren Formulierung des \"Aquivalenzprinzips
gibt es noch weitreichende Feinheiten und Unterteilungen. Insbesondere
auf die Unterscheidung zwischen dem sogenannten \glqq starken\grqq\ und
\glqq schwachen\grqq\ \"Aquivalenzprinzip soll hier nicht eingegangen
werden (vgl.~beispielsweise \cite{Misner}).

\subsection{Das Mach'sche Prinzip}
\label{MachsPrinzip}

Im Rahmen der Newton'schen Mechanik wie auch der speziellen 
Relativit\"atstheorie gibt es keine Erkl\"arung f\"ur das
\"Aquivalenzprinzip. Es erhebt sich hier allerdings die Frage, was
man als Erkl\"arung bezeichnen kann. 

Ernst Mach hat in seiner {\em Mechanik} das Newton'sche 
Konzept des absoluten Raums angegriffen, weil es sich
seiner Meinung nach dabei um eine unbeobachtbare Entit\"at
handelt. Auch Newtons Argumente, dass sich die
Wirkung des absoluten Raums durch die Tr\"agheitskr\"afte
bemerkbar macht, wenn K\"orper relativ zum absoluten
Raum beschleunigt werden, hat er nicht gelten lassen. 
Zu dem ber\"uhmten \glqq Eimerexperiment\grqq\ von Newton
-- die Wasseroberfl\"ache in einem Eimer ist gew\"olbt,
wenn sich das Wasser relativ zum absoluten Raum
bewegt, wobei die Bewegung relativ zu den
Eimerw\"anden keine Bedeutung hat -- bemerkt er, 
dass man schlie\ss lich nicht
w\"usste, ob sich das Wasser wirklich relativ zum absoluten
Raum bewegt oder nur relativ zum Fixsternhimmel.
Das entscheidende Experiment -- den Eimer in Ruhe
lassen und den Fixsternhimmel relativ dazu in
Rotation zu versetzen; wobei nach Newton das Wasser
nicht gew\"olbt sein solle -- k\"onne man schlie\ss lich
nicht durchf\"uhren.

Mach war der Ansicht, dass das Gravitationsgesetz
in seiner bekannten Form nicht vollst\"andig sei, 
sondern dass ein weiterer Beitrag zu
ber\"ucksichtigen sei, der sich allerdings nur
bei einer beschleunigten Bewegung relativ zu
einer gro\ss en Massenansammlung (z.B.\ der Menge
aller stellaren Objekte) bemerkbar machen w\"urde.
In diesem Fall w\"are Tr\"agheit eine besondere
Form der Gravitation, das hei\ss t,
die Tr\"agheit eines K\"orpers w\"urde
durch dieselbe Wechselwirkung verursacht wie sein Gewicht. 
Ein solches Gesetz w\"are eine Erkl\"arung f\"ur das 
\"Aquivalenzprinzip.

Einstein hat
lange Zeit geglaubt, dass die allgemeine Relativit\"atstheorie in
diesem Sinne eine Erkl\"arung f\"ur das \"Aquivalenzprinzip liefere.
So schreibt er in einem Brief an Mach vom 25.~Juni 1913:
\vspace{0.3cm}

\small
Hochgeehrter Herr Kollege!

Dieser Tage haben Sie wohl meine neue Arbeit \"uber Relativit\"at und
Gravitation erhalten, die nach unendlicher M\"uhe und qu\"alendem 
Zweifel nun endlich fertig geworden ist. N\"achstes Jahr bei der
Sonnenfinsternis soll sich zeigen, ob die Lichtstrahlen an der Sonne
gekr\"ummt werden, ob mit anderen Worten die zugrunde gelegte fundamentale 
Annahme von der Aequivalenz von Beschleunigung des Bezugssystems einerseits
und Schwerefeld andererseits wirklich zutrifft.

Wenn ja, so erfahren ihre genialen Untersuchungen \"uber die Grundlagen
der Mechanik -- Planck's ungerechtfertigter Kritik zum Trotz -- eine
gl\"anzende Best\"atigung. Denn es ergibt sich mit Notwendigkeit, dass 
die {\it Tr\"agheit} in einer Art {\it Wechselwirkung} der K\"orper
ihren Ursprung hat, ganz im Sinne ihrer \"Uberlegungen zum Newtonschen
Eimer-Versuch.

Eine erste Konsequenz in diesem Sinne finden Sie oben auf Seite 6 der
Arbeit. Es hat sich ferner folgendes ergeben:\\[0.2cm]
\begin{minipage}[b]{12cm}
1) Beschleunigt man eine tr\"age Kugelschale $S$, so erf\"ahrt nach der
Theorie ein von ihr eingeschlossener K\"orper eine beschleunigende
Kraft.\\
2) Rotiert die Schale $S$ um eine durch ihren Mittelpunkt gehende
Achse (relativ zum System der Fixsterne (\glqq Restsystem\grqq)), so
entsteht im Innern der Schale ein Coriolis-Feld, d.h.\ die Ebene des
Foucault-Pendels wird (mit einer allerdings praktisch unmessbar kleinen
Geschwindigkeit) mitgenommen.
\end{minipage}
\hfill
\begin{picture}(50,100)(0,0)
\put(10,10){\line(0,1){10}}
\put(10,25){\line(0,1){10}}
\put(10,50){\circle{30}}
\put(10,50){\makebox(0,0){$\bullet$}}
\put(18,50){\makebox(0,0){$K$}}
\put(33,50){\makebox(0,0){$S$}}
\put(10,65){\line(0,1){10}}
\put(10,80){\line(0,1){10}}
\end{picture}


Es ist mir eine gro\ss e Freude, Ihnen dies mitteilen zu k\"onnen, zumal
jene Kritik Plancks mir schon immer h\"ochst ungerechtfertigt erschienen
war.

\mbox{~} \hfill Mit gr\"o\ss ter Hochachtung gr\"u\ss t Sie herzlich

\mbox{~} \hfill Ihr ergebener A.\ Einstein
\vspace{0.3cm}

\normalsize
Das in diesem Brief angesprochene Experiment zur Bestimmung der
\label{p_solareclipse}%
Licht\-ab\-lenkung an der Sonne wurde 1914 wegen des 
Ausbruchs des ersten Weltkrieges
nicht durchgef\"uhrt -- zum Gl\"uck, wie manche meinen. Die allgemeine 
Relativit\"atstheorie lag n\"amlich in ihrer endg\"ultigen Form
noch nicht vor, und Einstein hat zur Berechnung des Ablenkungswinkels 
nur das \"Aquivalenzprinzip benutzt, ohne die inhomogene 
Kr\"ummung des Raumes
in der N\"ahe der Sonne korrekt zu ber\"ucksichtigen. 
Der Wert war in diesem Fall um einen Faktor 2 zu klein (\cite{Einstein5}).
Einstein korrigierte diesen Fehler bevor das Experiment dann 1919
wirklich durchgef\"uhrt wurde und seine Vorhersagen best\"atigte. 

Es zeigte sich sp\"ater, dass die angesprochenen Effekte der
allgemeinen Relativit\"atstheorie die Tr\"agheitskr\"afte nicht
erkl\"aren k\"onnen. Zwei umeinander rotierende Kugeln, die durch
einen Faden zusammengehalten werden, werden auch in der ART durch
Kr\"afte nach au\ss en getrieben, die sich nicht als Wechselwirkung
verstehen lassen. Es erhebt sich somit die Frage, inwieweit die
allgemeine Relativit\"atstheorie wirklich eine Erkl\"arung f\"ur das
\"Aquivalenzprinzip liefert, wie es oft behauptet wird. 

Dass diese Frage alles andere als trivial zu beantworten ist,
hat beispielsweise eine Konferenz zu dem Thema \glqq Mach's Principle --
From Newton's Bucket to Quantum Gravity\grqq\ gezeigt, die vom 26--30 Juli 1993
in T\"ubingen stattfand (die Proceedings sind als Buch erschienen 
\cite{Barbour3}). Mehrere Sessions und Diskussionsrunden waren auf dieser
Konferenz dem Thema gewidmet, inwieweit die allgemeine Relativit\"atstheorie
das Mach'sche Prinzip realisiert. 

Abschlie\ss end m\"ochte ich noch kurz darauf eingehen, woher der
Begriff \glqq Mach'sches Prinzip\grqq\ eigentlich stammt und was genau er
bedeuten soll. Gepr\"agt wurde dieser Begriff 1918 von
Einstein \cite{Einstein4}. In diesem kurzen
Artikel mit dem Titel \glqq Prinzipielles zur allgemeinen 
Relativit\"atstheorie\grqq\ will Einstein auf einige Kritikpunkte antworten,
die gegen die allgemeine Relativit\"atstheorie vorgebracht wurden.
Er schreibt:
\vspace{0.3cm}

\small
Die Theorie, wie sie mir heute vorschwebt, beruht auf drei 
Hauptgesichtspunkten, die allerdings keineswegs voneinander unabh\"angig
sind. Sie seien im folgenden kurz angef\"uhrt und charakterisiert und
hierauf im nachfolgenden von einigen Seiten beleuchtet:

a) {\em Relativit\"atsprinzip:} Die Naturgesetze sind nur Aussagen
\label{Relprinz}
\"uber zeitr\"aumliche Koinzidenzen; sie finden deshalb ihren einzig
nat\"urlichen Ausdruck in allgemein kovarianten Gleichungen.

b) {\em \"Aquivalenzprinzip:} Tr\"agheit und Schwere sind wesensgleich.
Hieraus und aus den Ergebnissen der speziellen Relativit\"atstheorie
folgt notwendig, dass der symmetrische \glqq Fundamentaltensor\grqq
($g_{\mu \nu}$) die metrischen Eigenschaften des Raumes, das
Tr\"agheitsverhalten der K\"orper in ihm, sowie die Gravitationswirkungen
bestimmt. Den durch den Fundamentaltensor beschriebenen Raumzustand
wollen wir als \glqq $G$-Feld\grqq\ bezeichnen.

c) {\em Machsches Prinzip:} Das $G$-Feld ist {\em restlos} durch die
Massen der K\"orper bestimmt. Da Masse und Energie nach den Ergebnissen
der speziellen Relativit\"atstheorie das Gleiche sind und die Energie
formal durch den symmetrischen Energietensor ($T_{\mu \nu}$) beschrieben
wird, so besagt dies, dass das $G$-Feld durch den Energietensor der
Materie bedingt und bestimmt sei.
\vspace{0.3cm}

\normalsize
An den Begriff \glqq Machsches Prinzip\grqq\ hat Einstein eine Fu\ss note
angeschlossen:
\vspace{0.3cm}

\small
Bisher habe ich die Prinzipe a) und c) nicht auseinandergehalten, was aber
verwirrend wirkte. Den Namen \glqq Machsches Prinzip\grqq\ habe ich deshalb
gew\"ahlt, weil dieses Prinzip eine Verallgemeinerung der Machschen
Forderung bedeutet, dass die Tr\"agheit auf eine Wechselwirkung der
K\"orper zur\"uckgef\"uhrt werden m\"usse.
\vspace{0.3cm}

\normalsize
Die Einstein'sche Formulierung aller drei Prinzipien bedarf sicherlich
einiger Kommentare. Auf Punkt a) gehen wir im n\"achsten Abschnitt noch
ein. Punkt b) wurde im vorherigen Abschnitt erl\"autert und wird 
vielleicht klarer, wenn man die vierte (letzte) dort angegebene Formulierung
des \"Aquivalenzprinzips betrachtet. Auf Punkt c) m\"ochte ich hier
kurz eingehen.

Heute verstehen wir unter dem Mach'schen Prinzip meist die Formulierung,
die Einstein in der Fu\ss note gew\"ahlt hat: \glqq Tr\"agheit beruht
auf einer Wechselwirkung zwischen K\"orpern.\grqq\ 
Vielleicht hat Einstein zur Zeit des Zitats (1918) noch geglaubt, dass 
die allgemeine Relativit\"atstheorie das Mach'sche Prinzip in dieser
Form tats\"achlich erf\"ullt. Schon die Bemerkung unter Punkt b)
(der Fundamentaltensor legt das Tr\"agheitsverhalten der K\"orper
fest) gilt nicht mehr, wenn man neben der Gravitation noch
andere Wechselwirkungen ber\"ucksichtigt. Ein Beispiel ist das von
Newton beschriebene System zweier umeinander rotierender Kugeln, die 
durch einen Faden miteinander verbunden sind. Die Kr\"afte im Faden, 
die die beiden Kugeln auf konstantem Abstand halten, sind keine
Gravitationskr\"afte. Es treten selbst im flachen Minkowski-Raum die 
bekannten Fliehkr\"afte als Tr\"agheitskr\"afte auf.

Das Mach'sche Prinzip in der Einstein'schen Formulierung gilt aber noch aus
einem anderen Grund nicht, den Einstein selber in der erw\"ahnten
Arbeit diskutiert: Es w\"urde verlangen, dass es in einem Universum
ohne Materie nur die Minkowski-Metrik $g_{\mu \nu}={\rm diag}(-1,1,1,1)$
als L\"osung gibt. Einstein selber zeigt aber, dass $g_{\mu \nu} =
{\rm konst}$ (f\"ur alle Komponenten verschiedene Konstanten) ebenfalls
eine L\"osung der Gleichungen darstellt. Daher gibt es zum selben
Materiezustand des Universums verschiedene L\"osungen f\"ur das
$G$-Feld, was nach der Einstein'sche Version des
Mach'schen Prinzips verboten w\"are. 
%Der Ausweg
%Einsteins, die Einf\"uhrung der kosmologischen Konstanten, l\"ost das
%Problem allerdings nicht. 

\subsection{Das Relativit\"atsprinzip}

Wir haben gesehen, dass sowohl in der Newton'schen Mechanik als auch
in der speziellen Relativit\"atstheorie eine bestimmte Klasse von
Bezugssystemen dadurch ausgezeichnet ist, dass in ihnen die
Newton'schen Gesetze (bzw.\ die relativistischen Verallgemeinerungen)
gelten, insbesondere das Tr\"agheitsgesetz. Diese Bezugssysteme 
bezeichnet man als
Inertialsysteme. Sie sind durch die Poincar\'e-Gruppe untereinander
verbunden, d.h.\ je zwei Inertialsysteme lassen sich durch ein
Element dieser Gruppe ineinander \"uberf\"uhren.

Diese Auszeichnung einer bestimmten Klasse von Systemen als
Inertialsysteme empfand Einstein als willk\"urlich. Die Willk\"ur
besteht dabei in zwei Aspekten, die von der allgemeinen 
Relativit\"atstheorie unterschiedlich gel\"ost werden.

Eine Willk\"ur liegt in der Auszeichnung der \glqq Gleichf\"ormigkeit\grqq,
d.h.\ der Wahl bestimmter Skalen sowohl hinsichtlich
der Zeit als auch hinsichtlich des Raumes. Physikalisch ist keine
Zeit- oder Raumskala ausgezeichnet. Wir w\"ahlen die Skalen nach dem
Gesichtspunkt der Einfachheit bzw.\ der Bequemlichkeit, n\"amlich so,
dass die Newton'schen Gesetze f\"ur die einfachsten Bewegungsformen
-- die geradlinig-gleichf\"ormige Bewegung des kr\"aftefreien K\"orpers
oder auch die gleichf\"ormig periodische Bewegung eines Pendels oder
Planeten -- die einfachste Gestalt annehmen. Physikalisch darf jedoch
eine beliebige (nicht notwendigweise gleichf\"ormige) Reparametrisierung
von Raum und Zeit die Gesetze nicht ab\"andern. 

Auf diese Eigenschaft bezieht sich Einstein in der obigen (Seite
\pageref{Relprinz}) Erl\"auterung des Relativit\"atsprinzips:
\glqq Die Naturgesetze sind nur Aussagen
\"uber zeitr\"aumliche Koinzidenzen; sie finden deshalb ihren einzig
nat\"urlichen Ausdruck in allgemein kovarianten Gleichungen.\grqq\
Statt \glqq zeitr\"aumliche Koinzidenzen\grqq\ w\"urden wir vielleicht heute
\glqq (lokale) Ereignisse\grqq\ sagen. Wie wir diese Ereignisse 
parametrisieren darf die G\"ultigkeit der Naturgesetze nat\"urlich
nicht beeinflussen. Daher m\"ussen die Gleichungen allgemein kovariant,
d.h.\ reparametrisierungsinvariant, sein. Dieses Prinzip ist in der
allgemeinen Relativit\"atstheorie erf\"ullt.

Die zweite Willk\"ur liegt in der Auszeichnung der 
\glqq Geradlinigkeit\grqq. Geradlinigkeit ist nur f\"ur eine flache
Raumzeit definiert, im Sinne der allgemeinen Relativit\"atstheorie
also f\"ur eine Raumzeit, in der es keine Gravitationsfelder gibt.
Die allgemeine Relativit\"atstheorie erweitert den Begriff des
Inertialsystems auch auf nicht-flache Raumzeiten: 
In einem Gravitationsfeld frei fallende Bezugs\-sys\-teme sind ebenfalls
Inertialsysteme. Geometrisch folgen sie ebenso Geod\"aten wie
die herk\"ommlichen Inertialsysteme in einer flachen, feldfreien
Raumzeit. Hierzu schreibt Einstein (\cite{Einstein5}, S.~899):%
\vspace{0.3cm}

\small
Man kann bei dieser Auffassung ebensowenig von der {\it absoluten
Beschleunigung} des Bezugssystems sprechen, wie man nach der
gew\"ohnlichen Relativit\"atstheorie von der {\it absoluten
Geschwindigkeit} eines Systems reden kann. 
\vspace{0.3cm}

\normalsize
Und als Fu\ss note f\"ugt er noch an: 
\vspace{0.3cm}

\small
Nat\"urlich kann man ein {\it beliebiges} Schwerefeld
nicht durch einen Bewegungszustand des Systems ohne Gravitationsfeld
ersetzen, ebensowenig, als man durch eine Relativit\"ats\-transformation
alle Punkte eines beliebig bewegten Mediums auf Ruhe transformieren
kann.%
\vspace{0.3cm}

\normalsize
Diese Aussage, dass in einem Gravitationsfeld frei fallende Systeme 
ebenfalls Inertialsysteme sind, ist von nicht zu untersch\"atzender 
praktischer Bedeutung: Die Newton'schen Inertialsysteme bzw.\ die 
Inertialsysteme der speziellen Relativit\"atstheorie sind kaum zu
realisierende Idealf\"alle, da es in unserem Universum 
kaum Orte ohne Gravitationskr\"afte gibt. Frei fallende
Systeme hingegen lassen sich vergleichsweise leicht verwirklichen.

In einer Hinsicht bleibt jedoch die allgemeine Relativit\"atstheorie
unbefriedigend, und das h\"angt nat\"urlich wieder mit dem Mach'schen
Prinzip zusammen. In einer flachen Raumzeit ohne Gravitationsfelder
folgt ein (auch von anderen Kr\"aften unbeeinflusster) K\"orper
einer geraden Linie. Doch was bedeutet \glqq gerade Linie\grqq, wenn
es keine Materie im Raum gibt. Die Minkowski-Metrik erlaubt die
Definition einer geraden Linie ohne Bezug auf irgendeinen anderen
K\"orper im Universum. Die Kritik Machs an diesem Konzept wird auch
von der allgemeinen Relativit\"atstheorie nicht beantwortet.

\subsection{Raum und Zeit nehmen nicht an der Dynamik teil}

Die Aymmetrie zwischen einerseits der Wirkung von Raum bzw.\ Raumzeit auf
die Materie, wie sie sich in den Tr\"agheitskr\"aften (z.B.\ der
Fliehkraft) offenbart, und andererseits der fehlenden Wirkung von Materie 
auf den Raum bzw.\ die Zeit empfand Einstein immer als unbefriedigend.

Im Vorwort zur Neuauflage von Emil Strauss' deutscher \"Ubersetzung 
von Galilei's {\em Dialog} vergleicht Einstein das
Konzept eines absouten Raumes mit der aristotelischen Vorstellung 
eines Weltmittelpunktes (\cite{Galilei}, Vorwort, S.~XI). 
Er schreibt zun\"achst zum Anliegen des Dialogs: 
\glqq Galileo wendet sich gegen die Einf\"uhrung
dieses \glq Nichts\grq\ (Weltmittelpunkt), das doch auf die materiellen
Dinge einwirken soll; dies findet er ganz unbefriedigend.\grqq\ 
Dann geht er ganz explizit auf die Analogie zur Relativit\"atstheorie ein:
\vspace{0.3cm}

\small
Ich m\"ochte hier -- in Form einer Einschaltung -- darauf aufmerksam
machen, dass eine weitgehende Analogie besteht zwischen Galileos
Ablehnung der Setzung eines Weltmittelpunktes zur Erkl\"arung des
Fallens der K\"orper und der Ablehnung der Setzung
des Inertialsystems zur Erkl\"arung des Tr\"agheitsverhaltens
der K\"orper (welche Ablehnung der allgemeinen Relativit\"atstheorie
zugrunde liegt). Beiden Setzungen gemeinsam ist n\"amlich die 
Einf\"uhrung eines begrifflichen Dinges mit folgenden Eigenschaften:
\begin{enumerate}
\item
Es ist nicht als etwas Reales gedacht, von der Art der ponderablen
Materie (bzw.\ des \glqq Feldes\grqq).
\item
Es ist ma\ss gebend f\"ur das Verhalten der realen Dinge, ist aber
umgekehrt keiner Einwirkung durch die realen Dinge unterworfen.
\end{enumerate}
Die Einf\"uhrung derartiger begrifflichen Elemente ist zwar vom rein
logischen Gesichtspunkte nicht schlechthin unzul\"assig, widerstrebt
aber dem wissenschaftlichen Instinkt.
\vspace{0.3cm}

\normalsize
Diese Kritik an der absoluten, an der Dynamik nicht beteiligten Form
von Raum und Zeit wird durch die allgemeine Relativit\"atstheorie
\"uberwunden. Die Einstein'schen Feldgleichungen beschreiben den 
Einfluss der Materie auf das metrische Feld der Raumzeit. 
Dar\"uberhinaus gibt es sogar nicht-triviale L\"osungen der 
Einstein'schen Gleichungen, selbst wenn der Energie-Impuls-Tensor der
Materie verschwindet, z.B.\ die so genannten Gravitationswellen.
Hierbei handelt es sich um wellenf\"ormige Schwankungen des
metrischen Feldes um die L\"osung der flachen Minkowski-Raumzeit. 

\section{Geometrisierung des Raumes}

Die allgemeine Relativit\"atstheorie verbindet man oft mit der
Vorstellung einer \glqq gekr\"ummten\grqq\ Raumzeit. Die Geometrie der
Raumzeit denkt man sich dabei meist durch den metrischen Tensor
\index{Metrischer Tensor}
$g_{\mu \nu}$ gegeben, beispielsweise als L\"osung der Einstein'schen
Feldgleichungen. Hier gibt es jedoch physikalisch noch einige
Interpretationsprobleme. Zun\"achst ist der metrische Tensor gar nicht
experimentell bestimmbar. Beliebige Diffeomorphismen der 
Raumzeit-Mannigfaltigkeit lassen die Geometrie unver\"andert, \"an\-dern
aber die Komponenten des metrischen Tensors. Damit erhebt
sich die Frage, ob bzw.\ wie man experimentell zu geometrischen
Gr\"o\ss en gelangen kann.

Um Geometrie betreiben zu k\"onnen, m\"ussen wir Abst\"ande
ausmessen k\"onnen. Selbst wenn wir uns die Punkte als identifiziert
denken, haben wir noch keine eindeutige Vorschrift, zwei nahe beieinander
liegenden Punkten einen Abstand zuzuordnen. Ein Gro\ss teil der folgenden
Darstellung entstammt der \glqq Philosophie der Raum-Zeit-Lehre\grqq\
von Hans Reichenbach \cite{Reichenbach1}. Wir konzentrieren uns zun\"achst
auf die Ausmessung r\"aumlicher Abst\"ande.

Angenommen, gegeben seien zwei Paare von Raumpunkten -- $(A,B)$ und $(a,b)$ -- 
und wir wollen den Abstand von $(A,B)$ mit dem Abstand von $(a,b)$ 
vergleichen. Was machen wir? Wir tragen den Abstand $(A,B)$ auf einem
\glqq Lineal\grqq\ ab, transportieren das Lineal zu den Punkten $(a,b)$ und
vergleichen die auf dem Lineal abgetragene Strecke mit dem Punktepaar
$(a,b)$. Doch was garantiert uns, dass sich die L\"ange des
Lineals bei dem Transport vom Punktepaar $(A,B)$ zum Punktepaar $(a,b)$ 
nicht ver\"andert? Daf\"ur gibt es keine Garantie! Wir k\"onnen
nur unter geeigneten Voraussetzungen {\em definieren}, dass sich dieser
Abstand nicht \"andert. 

Was sind das f\"ur Voraussetzungen? Aus der Erfahrung sind wir bereit
zu glauben, dass sich das Lineal beim Transport nicht verformt. Wir
betrachten das Lineal im Rahmen der \"ublichen Genauigkeiten als einen
\index{Starrer K\"orper}
\glqq starren K\"orper\grqq. Abgesehen von der Unhandlichkeit des Verfahrens
w\"urden wir sicherlich nicht zwei auf einer Wasser\-oberfl\"ache schwimmende
Korken zur Abstandsmessung heranziehen. Wir m\"ussen also zun\"achst
definieren, was wir unter einem starren K\"orper verstehen wollen.
Wir wissen, dass sich ein K\"orper bei Erhitzung im Allgemeinen
ausdehnt. Zug- oder Druckkr\"afte k\"onnen ebenfalls die Abmessungen
eines K\"orpers ver\"andern. Auch wenn wir ein Material w\"ahlen, bei
dem die inneren Kr\"afte die Form gegen\"uber solchen \"au\ss eren 
Zug- und Druckkr\"aften nahezu unver\"andert lassen, wissen wir doch
aus der speziellen Relativit\"atstheorie, dass es keinen idealen
starren K\"orper gibt. Die Wirkung eines pl\"otzlichen Sto\ss es kann
sich nur mit maximal Lichtgeschwindigkeit im K\"orper ausbreiten, d.h.\
eine gewisse Verformung l\"asst sich bei keinem K\"orper vermeiden.
Wir k\"onnen aber definieren, dass wir einen K\"orper als {\em starr}
bezeichnen, wenn sich seine Form ohne Einwirkung erkennbarer \"au\ss erer 
Kr\"afte nicht ver\"andert. \"Ahnlich wie schon beim Tr\"agheitsprinzip
sind wir also darauf angewiesen, \"uber das Vorhandensein \"au\ss erer 
Kr\"afte Aussagen machen zu k\"onnen. 

Dies ist sicherlich nicht immer der Fall. Angenommen, eine Kraft wirkt
auf alle K\"orper gleicherma\ss en und zwar so, dass s\"amtliche
Abstandsverh\"altnisse zwischen den K\"orpern und an den K\"orpern
im Vergleich mit dem kr\"aftefreien Fall
unver\"andert bleiben, dann kann die Wirkung einer solchen Kraft nicht
nachgewiesen werden. Reichenbach spricht in diesem Zusammenhang von einer
\glqq kongruenzerhaltenden, universellen Kraft\grqq.
\index{Kraft!universelle}\index{Kraft!kongruenzerhaltende}%
\index{Kongruenzerhaltende Kraft}\index{Universelle Kraft}%
Dass sich solche Kr\"afte physikalisch nicht nachweisen lassen,
hat Newton schon in seiner {\em Principia} (Korrolar 6) beschrieben. 

Wir m\"ussen also verschiedene Arten von Kr\"aften unterscheiden.
Reichenbach f\"uhrt zun\"achst die Unterscheidung zwischen einer
{\em universellen} Kraft und einer {\em differentiellen} Kraft ein.
\index{Kraft!differentielle}\index{Differentielle Kraft}%
Eine universelle Kraft wird durch folgende zwei Eigenschaften definiert:
\begin{enumerate}
\item
Sie wirkt auf jede Form von Materie gleicherma\ss en.
\item
Es gibt keine Abschirmung gegen sie.
\end{enumerate}
Die bekannten Kr\"afte -- Zug- und Druckkr\"afte, elektromagnetische
Kr\"afte, W\"arme -- sind differentielle Kr\"afte: Sie wirken auf
verschiedene Materialien unterschiedlich. Au\ss erdem gibt es in den
meisten F\"allen eine Abschirmung gegen diese Kr\"afte. Anders ist es
mit der Gravitation. Sie wirkt material\-un\-abh\"angig auf alle K\"orper 
gleicherma\ss en -- das ist eine Form des \"Aquivalenzprinzips --
und es gibt keine Abschirmung gegen die Gravitation.
Gravitation z\"ahlt nach Reichenbach also zu den universellen Kr\"aften.
\index{Gravitation!als universelle Kraft}

Unter den universellen Kr\"aften bilden die kongruenzerhaltenden Kr\"afte
noch einen Spezialfall. Hierzu z\"ahlt beispielsweise ein \"uberall
homogenes Gravitationsfeld. Solche Kr\"afte lassen sich durch keinen
beobachtbaren Effekt nachweisen. Im Folgenden sei \glqq keine Kraft\grqq\
immer gleichbedeutend mit \glqq keine Kraft oder universelle, 
kongruenzerhaltende Kraft\grqq. Ein {\em in}homogenes Gravitationsfeld 
entspricht einer universellen, aber {\em nicht} kongruenzerhaltenden Kraft.
Betrachten wir beispielsweise vier Massepunkte. Zwei dieser Massepunkte
seien durch eine Feder miteinander verbunden, die anderen beiden
Massepunkte seien frei. Im kr\"aftefreien Fall k\"onnen wir die 
Anfangsbedingungen so einrichten, dass sich die Abst\"ande zwischen
den vier Massepunkten nicht ver\"andern. Das Gleiche gilt auch in einem
kongruenzerhaltenden, universellen Kraftfeld. In einem inhomogenen
Gravitationsfeld, beispielsweise dem Zentralfeld eines Planteten, werden 
die Massen relativ zu einander jedoch bewegt: Die beiden freien Massen
werden ihren Abstand rascher verringern als die beiden durch eine Feder
auf einen bestimmten Abstand gehaltenen Massen. Dieser Effekt l\"asst
sich nachweisen. 

Wir wollen nun den Abstand zwischen Raumpunkten durch starre K\"orper
definieren. Dass dies \"uberhaupt sinnvoll ist, h\"angt von einer
wesentlichen Tatsache der Erfahrung ab: Ohne differentielle Kr\"afte
h\"angt die L\"ange eines starren K\"orpers weder vom Transportweg 
(im Zustandsraum des K\"orpers, d.h.\ Drehungen sind mit eingeschlossen)
noch von seinem Material ab. Wenn \"uberhaupt, so ist seine L\"ange
nur eine Funktion des Ortes und der Orientierung des starren K\"orpers.
Hier hilft uns aber keine experimentell \"uberpr\"ufbare Beobachtung
weiter -- die L\"ange eines starren K\"orpers an einem Ort mit einer 
bestimmten Orientierung ist eine sogenannte \glqq Zuordnungsdefinition\grqq\ 
(\cite{Reichenbach1}), die wir frei w\"ahlen k\"onnen. 

Eine m\"ogliche Definition ist, die L\"ange eines starren K\"orpers ohne 
differentielle Kr\"afte als konstant anzunehmen. In diesem Fall kann es
sich aber herausstellen, dass wir eine Geometrie finden, die nicht
euklidisch ist. Wir k\"onnen aber auch umgekehrt f\"ur den Raum eine
euklidische Geometrie {\em definieren}. Dann werden wir 
Abweichungen von einer euklidischen Geometrie, wie
sie mit einem scheinbar starren K\"orper festgestellt werden,
auf universelle Kr\"afte zur\"uckf\"uhren m\"ussen, die
den K\"orper deformieren.

Das ist auch der Grund, warum wir zwischen differentiellen und 
universellen Kr\"aften unterscheiden. Der Unterschied 
zwischen diesen beiden Kraftarten liegt n\"amlich darin, dass wir die
universellen Kr\"afte durch eine Umdefinition der geometrischen Abst\"ande
eliminieren k\"onnen. Die Wirkung einer universellen Kraft k\"onnen wir
als Eigenschaft des Raumes auffassen, wohingegen die Wirkung
differentieller Kr\"afte ohne Willk\"ur nicht als Eigenschaft des
Raumes formuliert werden kann.

\begin{SCfigure}[50][htb]
\begin{picture}(250,110)(0,5)
\put(20,20){\line(1,0){220}}
\put(20,50){\line(1,0){85}}
\put(105,70){\oval(40,40)[br]}
\put(145,70){\oval(40,40)[t]}
\put(185,70){\oval(40,40)[bl]}
\put(185,50){\line(1,0){55}}
\multiput(50,21)(0,6){5}{\line(0,1){3}}
\multiput(70,21)(0,6){5}{\line(0,1){3}}
\multiput(145,21)(0,6){12}{\line(0,1){3}}
\multiput(158,21)(0,6){11}{\line(0,1){3}}
\multiput(172,21)(0,6){6}{\line(0,1){3}}
\put(50,13){\makebox(0,0){$P$}}
\put(70,13){\makebox(0,0){$Q$}}
\put(145,13){\makebox(0,0){$A$}}
\put(158,13){\makebox(0,0){$B$}}
\put(172,13){\makebox(0,0){$C$}}
\put(50,57){\makebox(0,0){$P'$}}
\put(70,57){\makebox(0,0){$Q'$}}
\put(145,98){\makebox(0,0){$A'$}}
\put(160,94){\makebox(0,0){$B'$}}
\put(175,62){\makebox(0,0){$C'$}}
%
\put(10,20){\makebox(0,0){$E$}}
\put(10,50){\makebox(0,0){$G$}}
\end{picture}
\caption{\label{figreichen}%
Projektion einer nichteuklidischen Geometrie $G$ auf eine Ebene $E$.}
\end{SCfigure}

Das folgende Beispiel (aus Reichenbach \cite{Reichenbach1}, \S2 -- \S8) 
soll das Gesagte verdeutlichen (vgl.\ Abb.~\ref{figreichen}). Gegeben seien
zwei Fl\"achen, eine euklidische Ebene $E$ und eine Fl\"ache $G$
mit einer \glqq Beule\grqq. Auf diesen Fl\"achen leben zweidimensionale
Wesen, die ihre Welt intrinsisch ausmessen wollen. Wir stellen uns nun
vor, dass jeder Punkt der Fl\"ache $G$ mit der nicht-euklidischen 
Geometrie auf die Fl\"ache $E$ senkrecht projiziert wird. Au\ss erdem
nehmen wir an, dass eine universelle Kraft (eine Art 
\glqq universelle\grqq\ Erw\"armung der Fl\"ache) die 
L\"angenma\ss st\"abe der Wesen auf der euklidischen Fl\"ache $E$ so   
verformt, dass die Abst\"ande von Punkten auf $E$ genau den Abst\"anden
entsprechen, die die Wesen auf $G$ mit ihren unverformten Ma\ss st\"aben
den entsprechenden Punkten zuordnen. Die Wesen auf $E$ ordnen also
den Punkten $A$, $B$ und $C$ untereinander dieselben Abst\"ande zu, die
auch die Wesen auf $G$ den Punkten $A'$, $B'$ und $C'$ zuordnen.
W\"urden nicht die Wesen auf der Fl\"ache $E$ dieselbe Geometrie
rekonstruieren, die auch die Wesen auf $G$ finden? Gibt es \"uberhaupt
einen Unterschied zwischen den beiden Geometrien? Liegt der Unterschied
nicht nur in der unterschiedlichen Form der Einbettung der beiden
Fl\"achen in einen dreidimensionalen Raum? Diese Einbettung sollte aber
mit der intrinsischen Geometrie nichts zu tun haben. 

\begin{figure}
\begin{picture}(380,160)(0,0)
\put(30,40){\line(1,0){115}}
\put(30,40){\line(1,2){50}}
\put(40,46){\line(1,0){105}}
\put(145,40){\line(0,1){6}}
\put(40,46){\line(1,2){40}}
\put(80,126){\line(1,-2){40}}
\put(80,140){\line(1,-2){47}}
\put(123,40){\line(1,-2){6}}
\put(130,40){\line(1,-2){6}}
\put(129,28){\line(1,0){7}}
\multiput(120,40)(2,0){10}{\line(0,1){2}}
\put(135,53){\makebox(0,0){$S$}}
%
\qbezier(257,36)(206,36)(206,85)
\qbezier(206,85)(206,134)(257,134)
\qbezier(260,30)(200,30)(200,85)
\qbezier(200,85)(200,140)(260,140)
\put(257,36){\line(0,1){110}}
\put(263,36){\line(0,1){110}}
\put(257,146){\line(1,0){6}}
\qbezier(263,36)(314,36)(314,85)
\qbezier(314,85)(314,134)(263,134)
\qbezier(260,30)(320,30)(320,85)
\qbezier(320,85)(320,140)(260,140)
\multiput(257,125)(0,2){10}{\line(1,0){2}}
\put(250,147){\makebox(0,0){$S$}}
\multiput(257,125)(0,2){10}{\line(1,0){2}}
\end{picture}
\caption{\label{figgeom}%
Zwei Geometrie-Messer. Auf der linken Seite bildet ein starrer K\"orper ein
Dreieck, das am Punkt $S$ offen ist. Auf einer Skala l\"asst sich dort
ablesen, wie sehr das Dreieck von einem euklidischen Dreieck abweicht.
Auf der rechten Seite bildet der starre K\"orper einen Kreis, der noch 
zus\"atzlich eine Diagonalverbindung hat.
Am Punkt $S$ l\"asst sich wieder ablesen, wie sehr Umfang und
Durchmesser von ihrem euklidischen Verh\"altnis abweichen.}
\end{figure}

Wir als \glqq au\ss enstehende\grqq\ Wesen haben den Eindruck gewonnen, dass 
es einen Unterschied zwischen der Fl\"ache $G$ und der Fl\"ache $E$ gibt:
$G$ entspricht wirklich einer nicht-euklidischen Geometrie, w\"ahrend auf
$E$ eben eine Kraft wirkt. Doch was bedeutet hier \glqq wirklich\grqq?
Wir haben in der Physik keine M\"oglichkeit, zwischen einer universellen
Kraft und einer nicht-euklidischen Geometrie des Raumes zu unterscheiden.
Es bleibt uns \"uberlassen, welche Anschauung wir {\em f\"ur denselben
physikalischen Sachverhalt} benutzen wollen. Reichenbach dr\"uckt das
dadurch aus, dass er f\"ur die Verformungen $U$ eines L\"angenma\ss stabes
-- idealisierte Beispiele f\"ur solche \glqq Geometrie-Messer\grqq\ sind in
Abb.~\ref{figgeom} skizziert -- an jedem Punkt drei Ursachen verantwortlich 
macht:
\[      U ~=~ G + K_{\rm u} + K_{\rm d}  \;.   \]
Hierbei ist $G$ die Geometrie des Raumes, $K_{\rm u}$ eine universelle 
Kraft und $K_{\rm d}$ eine differentielle Kraft. Die differentielle Kraft
k\"onnen wir durch Vergleich verschiedener Materialien bzw.\ durch
Abschirmung des Ma\ss stabes von dieser Kraft erkennen und entsprechend
ber\"ucksichtigen. Doch dar\"uberhinaus bemerken wir nur die Wirkung der 
Summe $G+K_{\rm u}$. Es bleibt uns \"uberlassen, welchen Teil wir der
Geometrie $G$ des Raumes zuschreiben und welchen Teil wir als universelle
Kraft $K_{\rm u}$ interpretieren. Die Einstein'sche Konvention setzt
$K_{\rm u}=0$; es bleibt also ausschlie\ss lich die Geometrie. Jemand
anders wird vielleicht die Geometrie $G$ gleich der euklidischen Geometrie
$E$ w\"ahlen und s\"amtliche beobachteten Abweichungen davon einer
universellen Kraft zuschreiben. 

Der Vorteil der Einstein'schen Konvention ist ihre Einfachheit. Warum
sollen wir irgendeine Geometrie besonders auszeichnen, die von den
am starren K\"orper gemessenen Ma\ss stabsangaben abweicht? Die
euklidische Geometrie ist nur in unserer Anschauung ausgezeichnet, weil
wir durch unsere Alltagsvorstellung an sie gew\"ohnt sind. Mathematisch
gibt es keine ausgezeichnete Geometrie. 

Auch andere Versuche zur Auszeichnung einer euklidischen Geometrie
-- beispielsweise die Idee von Dingler und Lorenzen, euklidische Ebenen 
durch \glqq Aneinanderreiben und Abschabung\grqq\ starrer K\"orper zu erzeugen 
(siehe beispielsweise \cite{Mittelstaedt2}) -- kann man heute als 
fehlgeschlagen betrachten.

Abschlie\ss end sollte ich noch anmerken, dass die hier
angestellten \"Uberlegungen nicht davon abh\"angen, dass
wir starre K\"orper zur Ausmessung von Abst\"anden verwendet
haben. Auch die Zeitmessungen von Lichtstrahlen, die zwischen
Ereignissen hin- und herlaufen, erfordern zum Vergleich, dass
die Messung der Zeit an verschiedenen Raumpunkten 
unabh\"angig von diesen Raumpunkten ist. Au\ss erdem wird
vorausgesetzt, dass Licht zwischen zwei Punkten immer
den \glqq geod\"atisch k\"urzesten\grqq\ Weg nimmt, aber damit
hat man schon eine Konvention getroffen. Wer garantiert, dass
bei universellen Kr\"aften das Licht nicht von einer Geod\"aten
abgelenkt wird?

\section{Ein Beispiel - das Fermat'sche Prinzip}

In der geometrischen Optik kennen wir
das Fermat'sche Prinzip: Der von einem Lichtstrahl
durchlaufene Weg hat die k\"urzeste 
(streng genommen handelt es sich allgemein um
eine lokal station\"are) optische Wegl\"ange.
Die optische Wegl\"ange $l_{\rm opt}$ ergibt sich 
dabei aus der geometrischen Wegl\"ange $l$
multipliziert mit dem Brechungsindex $n$ des Mediums.
Bei einem ortsabh\"angigen Brechungsindex 
folgt
\begin{equation}
         l_{\rm opt} = \int_\gamma  n(x(l)) \,{\rm d}l
\end{equation} 
Hierbei ist $x(l)$ ein durch die geometrische
Wegl\"ange $l$ parametrisierter Weg $\gamma$.

Angenommen, es g\"abe nur Photonen. W\"are
es dann sinnvoll, die optische Dichte $n(x)$ am
Ort $x$ als eine geometrische Eigenschaft
zu definieren? Da Elektronen und andere Teilchen
von einem Glaskristall, Wasser, etc.\ vollkommen
anders abgelenkt werden als Licht, machen wir hier
die Einschr\"ankung auf Photonen.

\begin{figure}
\begin{picture}(220,80)(-20,0)
\qbezier(50,5)(60,20)(80,20)
\qbezier(80,20)(100,20)(110,5)
\qbezier(0,20)(80,40)(160,20)
\put(80,10){\makebox(0,0){$M$}}
%
\end{picture}
\begin{picture}(160,80)(0,0)
\put(60,5){\line(1,0){60}}
\put(60,5){\line(1,2){30}}
\put(120,5){\line(-1,2){30}}
\put(10,20){\line(4,1){66}}
\put(75.5,36.2){\line(1,0){29}}
\put(104.5,36.2){\line(4,-1){66}}
%
\end{picture}
\caption{\label{fig_Prisma}%
(Links) Ein Lichtstrahl wird an einer gro\ss en Masse --
z.B.\ der Sonne -- abgelenkt. (Rechts) Ein Lichtstrahl
wird in einem Prisma abgelenkt.}
\end{figure}

Abbildung \ref{fig_Prisma} zeigt zwei \"ahnliche
Situationen: einmal die Ablenkung von Licht an
einer Gravitationsquelle wie der Sonne und einmal die Ablenkung
von Licht in einem Prisma. In der ART wird die
Lichtablenkung an der Sonne als Beispiel f\"ur
die nicht-euklidische Geometrie gewertet, d.h.\ der
Lichtstrahl breitet sich entlang einer 
L\"osung der Geod\"atengleichung aus. 

Die Antwort, ob auch im Fall unterschiedlicher
optischer Dichten die Geometrie des Raums
f\"ur die Ablenkung verantwortlich gemacht
werden kann, h\"angt von einer wichtigen 
Eigenschaft ab: Ob die Ursachen der Ablenkung
universell sind. Im Allgemeinen zeigen
lichtdurchl\"assige K\"orper wie Glas oder Wasser
eine Dispersion, d.h., der Brechungsindex h\"angt von der Wellenl\"ange
ab und somit werden unterschiedliche Wellenl\"angen auch unterschiedlich
gebrochen. Licht unterschiedlicher Wellenl\"ange nimmt also bei gleichem
Anfangs- und Endpunkt unterschiedliche Wege.
Insofern wirkt ein Prisma \glqq differentiell\grqq.
Die Gravitation hingegen lenkt das Licht
unabh\"angig von seiner Wellenl\"ange immer
gleich ab.\footnote{Dies gilt, obwohl Licht mit k\"urzeren
Wellenl\"angen energiereicher ist als Licht mit
langen Wellenl\"angen. Nach dem \"Aquivalenzprinzip
ist diese Tatsache klar: Die Ablenkung von Licht
in einem beschleunigten System hat nichts mit
seiner Wellenl\"ange zu tun, sondern lediglich mit
dem Grad der Beschleunigung und der 
Lichtgeschwindigkeit.} 

\begin{thebibliography}{99}
%\addcontentsline{toc}{chapter}{Literaturangaben}
%\bibitem{Aichelburg} Peter C.\ Aichelburg (Hrsg.); {\it Zeit im 
%       Wandel der Zeit}; Verlag Vieweg, Braunschweig, Wiesbaden, 1988.
\bibitem{Barbour3} {\it Mach's Principle -- From Newton's Bucket to
        Quantum Gravity}; Julian Barbour \& Herbert Pfister (Hrsg.);
        Birkh\"auser, Boston, Basel, Berlin, 1995.       
%\bibitem{Bekenstein} Jacob D.\ Bekenstein, \textit{Black holes
%          and entropy}, Phys.\ Rev.\ D\,7 (1973) 2333--2346.        
%\bibitem{Bell} John Bell;  {\em Speakable and Unspeakable in 
%        Quantum Physics}, 2.\ edition, Cambridge University Press (2004).       
%\bibitem{Born} Max Born; {\it Optik}; Springer-Verlag, Berlin, Heidelberg,
%        1972.
%\bibitem{Britannica} Encyclopaedia Britannica; 15.th edition, 1988.
%\bibitem{Descartes} Ren\'e Descartes; {\it Die Prinzipien der
%        Philosophie}; Felix Meiner Verlag, Hamburg, 1992; \"ubersetzt
%        von Artur Buchenau.
%\bibitem{EDM} Encyclopaedic Dictionary of Mathematics; Second Edition,
%        MIT Press, 1987.
%\bibitem{Einstein1} Albert Einstein; {\it Zur Elektrodynamik bewegter 
%        K\"orper}; Annalen der Physik, Leipzig, 17 (1905) 891. 
%\bibitem{Einstein2} Albert Einstein; {\it Ist die Tr\"agheit eines
%        K\"orpers von seinem Energieinhalt abh\"angig?} (Ann.\ Phys., 
%        Leipzig, 18 (1905) 639.
%\bibitem{Einstein3} Albert Einstein; {\it Aus meinen sp\"aten Jahren};
%         Ullstein Sachbuch, Verlag Ullstein, Frankfurt, Berlin, 1993.                 
\bibitem{Einstein4} Albert Einstein; {\it Prinzipielles zur allgemeinen
        Relativit\"atstheorie}; Annalen der Physik 55 (1918) 241.
\bibitem{Einstein5} Albert Einstein; {\it \"Uber den Einflu\ss\ der
        Schwerkraft auf die Ausbreitung des Lichtes}; Annalen der
        Physik 35 (1911) 898.                 
%\bibitem{Feynman} Richard Feynman; {\it The Character of Physical Law};
%        The MIT Press, 1987.        
%\bibitem{Fierz} Markus Fierz; {\it \"Uber den Ursprung und die Bedeutung
%        der Lehre Isaac Newtons vom absoluten Raum}; Gesnerus, 
%        11.\ Jahrgang (1954), S.\,62--120.
%\bibitem{Fliessbach} Torsten Flie\ss bach; {\it Allgemeine 
%        Relativit\"atstheorie}; BI-Wissenschaftsverlag, Mannheim, Wien
%        Z\"urich, 1990. 
\bibitem{Galilei} Galilei; {\it Dialog \"uber die beiden haupts\"achlichen
        Weltsysteme, das ptolem\"aische und das kopernikanische}; 
        Teubner Stuttgart, 1982; aus dem Italienischen \"ubersetzt von
        Emil Strauss.   
%  \bibitem{Hawking} Stephen W.\ Hawking, \textit{Particle Creation by
%            black holes}, Comm.\ Math.\ Phys.\ 43 (1976) 199--220.      
%\bibitem{Helmholtz2} Hermann von Helmholtz; {\em \"Uber Wirbelbewegungen,
%        \"Uber Fl\"ussigkeitsbewegungen}, 1858; in Ostwalds Klassiker der 
%       exakten Wissenschaften Bd.\ 1; Verlag Harri Deutsch, Frankfurt, 
%       1996.                   
%\bibitem{Lamb} G.L.\ Lamb, Jr.; {\it Elements of Soliton Theory}; 
%         Pure \& Applied Mathematics, John Wiley \& Sons, 1980. 
%\bibitem{Laue} Max von Laue; {\it Geschichte der Physik}; 
%         Universit\"ats-Verlag Bonn, 1947.
%\bibitem{Lorentz} Hendrik Antoon Lorentz; {\it Electromagnetic phenomena 
 %        in a system moving with any velocity smaller than that of light}; 
%         Proc.\ Acad.\ Sci., Amsterdam, 6 [1904], S.\ 809.
%\bibitem{Mach} Ernst Mach; {\it Die Mechanik in ihrer Entwicklung
%      historisch kritisch dargestellt}; Akademie Verlag, Berlin, 1988.       
%\bibitem{Mainzer} Klaus Mainzer; {\it Philosophie und Geschichte von
%         Raum und Zeit}; in {\it Philosophie und Physik der Raum-Zeit};
%         J\"urgen Audretsch und Klaus Mainzer (Hrsg.); 
%         BI-Wissenschaftsverlag, 1994. 
\bibitem{Microscope} Touboul, Pierre, et al. (MICROSCOPE Collaboration); \textit{MICROSCOPE Mission:
         Final Results of the Test of the Equivalence Principle}; Phys.\ Rev.\ Lett.\ \textbf{129} (2022) 121102.
\bibitem{Misner} C.W.\ Misner, K.S.\ Thorne, J.A.\ Wheeler; 
        {\it Gravitation}; W.H.\ Freeman and Company, San Francisco,  1973.
%\bibitem{Mittelstaedt} Peter Mittelstaedt; {\it Der Zeitbegriff in der
%        Physik}; BI-Wissenschaftsverlag, 1989.        
\bibitem{Mittelstaedt2} Peter Mittelstaedt; {\it Philosophische Probleme
        der modernen Physik}; BI-Wissenschaftsverlag, 1989.        
%\bibitem{Newton}
%   Isaac Newton; {\it Mathematische Grundlagen der Naturphilosophie}; 
%   \"ubersetzt von Ed Dellian; Felix Meiner Verlag, 1988. 
%\bibitem{Newton2} Isaac Newton; {\it \"Uber die Gravitation...};
%       Klostermann Texte Philosophie; Vittorio Klostermann, Frankfurt,
%      1988; \"ubersetzt von Gernot B\"ohme.
%\bibitem{Newton3} Isaac Newton; {\it Optik oder Abhandlung \"uber
%      Spiegelungen, Brechungen, Beugungen und Farben des Lichts};
%      I., II.\ und III.\ Buch (1704); aus dem Englischen \"ubersetzt
%      von W.\ Abendroth; Ostwalds Klassiker der exakten Wissenschaften,
%      Verlag Harri Deutsch 1998.   
%\bibitem{Neumann} Carl Neumann; {\it \"Uber die Principien der
%         Galilei-Newtonschen Theorie}; Akademische Antrittsvorlesung,
%         gehalten in der Aula der Universit\"at Leipzig am 3.\ Nov.\
%         1869; Teubner (Leipzig) 1870.         
%\bibitem{Pauli} Wolfgang Pauli; {\it Theory of Relativity}; Dover
%      Publications, New York, 1981.      
%\bibitem{Poincare} Jules Henri Poincar\'e; {\it Sur la dynamique de 
%     l'\'electron}, C.R.\ Acad.\ Sci., Paris, 140 (1905) S.~1504; und 
%      Rendiconti del Circolo Matematico di Palermo, Bd.~21 (1906) S.~129.
\bibitem{Reichenbach1} Hans Reichenbach; {\em Philosophie der 
       Raum-Zeit-Lehre}; Hans Reichenbach - Gesammelte Werke Bd.\ 2;
       Vieweg-Verlag, Braunschweig; 1977.
%\bibitem{Reichenbach2} Hans Reichenbach; {\em Axiomatik der
%       relativistischen Raum-Zeit-Lehre}; in {\em Die philosophische
%       Bedeutung der Relativit\"atstheorie}; Hans Reichenbach - Gesammelte
%       Werke Bd.\ 3; Vieweg-Verlag, Braunschweig, 1977. 
%\bibitem{Rovelli} Carlo Rovelli, \textit{Quantum Gravity}; Cambridge
%      University Press, 2007.       
\bibitem{Schlamminger} Schlamminger, Choi, Wagner, Gundlach,
         Adelberger; {\em Test of the Equivalence Principle using a
         rotating torsion balance}; Phys.\ Rev.\ Lett.\ {\bf 100} (2008)
         041101.     
%\bibitem{Sexl} Roman U.\ Sexl, Helmuth K.\ Urbantke; {\it Relativit\"at,
%      Gruppen, Teilchen}; Springer-Verlag, Wien, New York, 1992.
\bibitem{Simonyi}
       K\'aroly Simonyi; {\it Kulturgeschichte der Physik}; Verlag
       Harri Deutsch, Thun, Frankfurt am Main, 1990.
%\bibitem{Weisberg} Weisberg, J.M., Taylor, J.H.; {\em Relativistic Binary Pulsar
%          B1913+16: Thirty Years of Observations and Analysis}; 
%          \verb+arXiv:astro-ph/0407149v1+; 2004. 
%\bibitem{Thomson} James Thomson; {\it On the Law of Inertia; the
%       Principle of Chronometry; and the Principle of Absolute Clinural
%       Rest, and of Absolute Rotation}; Proc.\ Roy.\ Soc.\ (Edinburgh),
%       Session 1883-84, Vol.\ XII, 568--578.       
%\bibitem{Weizsaecker} Carl Friedrich von Weizs\"acker; {\em Der zweite
%      Hauptsatz und der Unterschied von Vergangenheit und Zukunft};
%      Annalen der Physik 36 (1939) 275--283.       
%\bibitem{Zeh} Zeh, H.D.; {\em The Physical Basis of the Direction of Time},
%      Springer-Verlag, Berlin, 1989.       

%\bibitem{Einstein} Einstein, Albert; {\em ??}, .                   
\end{thebibliography}

\end{document}
