\documentclass[german,10pt]{book}         
\usepackage{makeidx}
\usepackage{babel}            % Sprachunterstuetzung
\usepackage{amsmath}          % AMS "Grundpaket"
\usepackage{amssymb,amsfonts,amsthm,amscd} 
\usepackage{mathrsfs}
\usepackage{rotating}
\usepackage{sidecap}
\usepackage{graphicx}
\usepackage{color}
\usepackage{fancybox}
\usepackage{tikz}
\usetikzlibrary{arrows,snakes,backgrounds}
\usepackage{hyperref}
\hypersetup{colorlinks=true,
                    linkcolor=blue,
                    filecolor=magenta,
                    urlcolor=cyan,
                    pdftitle={Overleaf Example},
                    pdfpagemode=FullScreen,}
%\newcommand{\hyperref}[1]{\ref{#1}}
%
\definecolor{Gray}{gray}{0.80}
\DeclareMathSymbol{,}{\mathord}{letters}{"3B}
%
\newcounter{num}
\renewcommand{\thenum}{\arabic{num}}
\newenvironment{anmerkungen}
   {\begin{list}{(\thenum)}{%
   \usecounter{num}%
   \leftmargin0pt
   \itemindent5pt
   \topsep0pt
   \labelwidth0pt}%
   }{\end{list}}
%
\renewcommand{\arraystretch}{1.15}                % in Formeln und Tabellen   
\renewcommand{\baselinestretch}{1.15}                 % 1.15 facher
                                                      % Zeilenabst.
\newcommand{\Anmerkung}[1]{{\begin{footnotesize}#1 \end{footnotesize}}\\[0.2cm]}
\newcommand{\comment}[1]{}
\setlength{\parindent}{0em}           % Nicht einruecken am Anfang der Zeile 

\setlength{\textwidth}{15.4cm}
\setlength{\textheight}{23.0cm}
\setlength{\oddsidemargin}{1.0mm} 
\setlength{\evensidemargin}{-6.5mm}
\setlength{\topmargin}{-10mm} 
\setlength{\headheight}{0mm}
\newcommand{\identity}{{\bf 1}}
%
\newcommand{\vs}{\vspace{0.3cm}}
\newcommand{\noi}{\noindent}
\newcommand{\leer}{}

\newcommand{\engl}[1]{[\textit{#1}]}
\parindent 1.2cm
\sloppy

         \begin{document}  \setcounter{chapter}{4}

\chapter{Schwarze L\"ocher}

Die wohl spektakul\"arsten Vorhersagen der
Einstein'schen Feldgleichungen beziehen sich
auf Schwarze L\"ocher. Einige L\"osungen kennt
man in analytischer Form, dazu geh\"oren die
bekannte Schwarzschild-L\"osungen eines
ruhenden ungeladenen Schwarzen Lochs, die
Kerr-L\"osung f\"ur ungeladene rotierende
Schwarze L\"ocher, die Reissner-Nordstr\"om-L\"osung
f\"ur ruhende geladene Schwarze L\"ocher und
schlie\ss lich die Kerr-Newman-L\"osung f\"ur
rotierende geladene Schwarze L\"ocher.
Au\ss erdem gibt es L\"osungen der
Einstein'schen Feldgleichungen, die 
kosmologische Modelle beschreiben -- die
bekannteste L\"osung ist in diesem Fall
die Friedman-L\"osung bzw.\ Robertson-Walker-Metrik. 

\section{Isotrope, statische L\"osungen}
\label{sec_IsoStat}

Bevor wir uns konkret mit der Schwarzschild-Metrik 
besch\"aftigen, betrachten wir den allgemeineren Fall
einer isotropen, statischen Geometrie. Die genauen
L\"osungen h\"angen vom Energie-Impuls-Tensor ab.
Hier geht es mir jedoch nur um vorbereitende Bemerkungen
zur allgemeinen Form solcher Geometrien.
Die folgenden \"Uberlegungen sind gr\"o\ss tenteils dem Buch von
Flie\ss bach entnommen \cite{Fliessbach}.

\subsection{Standardform einer asymptotisch flachen isotropen
und statischen Metrik}

Neben \glqq isotrop und statisch\grqq\ verlangen wir noch, 
dass die L\"osung
asymptotisch f\"ur sehr gro\ss e Werte der Koordinate
$r$ (die eine r\"aumlich radiale Achse vom Zentrum der L\"osung
parametrisiert) gegen die Minkowski-Metrik gehen soll, also
in Kugelkoordinaten:
\begin{equation}
    {\rm d}s^2 = c^2 {\rm d}t^2 - {\rm d}r^2 - r^2 ({\rm d}\theta^2 + \sin^2 \theta {\rm d}\varphi^2) \, . 
\end{equation}
\glqq Isotrop\grqq\ bedeutet, dass\index{isotrope Raumzeit} 
die Abh\"angigkeit der Metrik von
den Winkeln $\theta$ und $\varphi$ unver\"andert bleibt.
\glqq Statisch\grqq\ soll sich in diesem\index{statische Raumzeit} 
Fall auf den Zeitparameter $t$ 
beziehen, d.h.\ auf die physikalische Zeit eines \glqq unendlich\grqq\ 
weit entfernten Beobachters. Die Metrik soll nicht von diesem
Zeitparameter abh\"angen.

Ganz allgemein setzen wir f\"ur eine solche Metrik
die folgende Form an:
\begin{equation}
    {\rm d}s^2 = B(r) c^2 {\rm d}t^2 - A(r) {\rm d}r^2 - C(r)  
    r^2 ({\rm d}\theta^2 + \sin^2 \theta {\rm d}\varphi^2) \, . 
\end{equation}
Wegen der Isotropie kann es keine gemischten Terme
in den Winkeln geben, und einen gemischten Term der
Form ${\rm d}t\,{\rm d}r$ kann man immer durch eine
geeignete Umdefinition $t'=t+\psi(r)$ loswerden. F\"ur
$r\rightarrow \infty$ muss $\psi(r)$ verschwinden und
damit hat auch $t'$ asymtotisch immer noch die Bedeutung der 
physikalischen Zeit eines weit entfernten Beobachters.
Ebenso hat $r$ in diesem Grenzfall die Bedeutung einer
gew\"ohnlichen Radialkoordinate. Die Winkel $\theta$ und
$\varphi$ sind ohnehin die gew\"ohnlichen Winkel in
Kugelkoordinaten.

Zus\"atzlich zu diesen Einschr\"ankungen k\"onnen wir immer noch 
eine Umparametrisierung von $r$ vornehmen, sodass wir den
Term $C(r)$ loswerden. Auch dies \"andert
asymptotisch an der Interpretation von $r$ nichts.
Als {\em Standardform} einer\index{Standardform einer isotropen, statischen Metrik} 
isotropen, statischen
Metrik bezeichnet man daher folgende Parametrisierung:
\begin{equation}
\label{eq_isostat1}
    {\rm d}s^2 = B(r) c^2 {\rm d}t^2 - A(r) {\rm d}r^2 - 
    r^2 ({\rm d}\theta^2 + \sin^2 \theta {\rm d}\varphi^2) \, . 
\end{equation}
In dieser Parametrisierung hat eine Kugelschale
f\"ur einen Abstand, der durch den Parameter $r$ 
charakterisiert ist, eine Fl\"ache von $4 \pi r^2$. 
Allerdings bezeichnet $r$ nun nicht mehr den
physikalischen Abstand vom Ursprung; dieser ist durch
\begin{equation}
          R(r) = \int_0^r \sqrt{A(r')} {\rm d}r'
\end{equation}
gegeben. Insofern verh\"alt sich die Fl\"ache einer
Kugelschale als Funktion des Abstands $R(r)$ vom
Ursprung nicht wie im Euklidischen. Dadurch kann
der dreidimensionale Raum eine Kr\"ummung haben. 
Betrachten wir eine zweidimensionale Fl\"ache f\"ur
feste Werte von $t$ und $r$, also
parametrisiert durch $\theta$ und $\varphi$, so
handelt es sich um eine gew\"ohnliche Kugeloberfl\"ache.

\subsection{Robertson-Entwicklung}

Bei der experimentellen \"Uberpr\"ufung der
ART ist man oft an dem Verhalten der L\"osungen
f\"ur gro\ss e Wert von $r$ interessiert, also an den
f\"uhrenden Korrekturen zur Minkowski-Raumzeit.
Dazu betrachtet man eine Entwicklung der Funktionen
$A(r)$ und $B(r)$ f\"ur gro\ss e Werte von $r$. Da
$A(r)$ und $B(r)$ dimensionslos sind, k\"onnen sie
nur Funktionen einer dimensionslosen Gr\"o\ss e
sein, doch aus $G$, $c$ und $r$ (das sind die
Parameter, die uns zun\"achst zur Verf\"ugung stehen)
erhalten wir keine dimensionslose
Gr\"o\ss e. Die L\"osung muss durch
mindestens einen weiteren Parameter, den wir
mit $M$ bezeichnen und der sich im Prinzip aus
dem Energie-Impuls-Tensor ergeben sollte, charakterisiert 
sein. $M$ ist ein Massenparameter,
der durch die f\"uhrende Korrektur, die der Newton'schen
Gravitation entspricht, definiert ist. (Im n\"achsten
Abschnitt werden wir sehen, dass sich die
Schwarzschild-L\"osung als L\"osung der Vakuumgleichungen
-- $T_{\mu \nu}=0$ -- ergibt. Trotzdem enth\"alt die
L\"osung einen Parameter $M$. Dies ist \"ahnlich wie
in der Newton'schen Mechanik, wo das Gravitationspotenzial
zu einer Punktmasse bei $r=0$ nicht definiert ist, aber au\ss erhalb
von $r=0$ eine L\"osung der Vakuumgleichungen ist.
Die \glqq St\"arke der Singularit\"at\grqq\ bei $r=0$ definiert
den Massenparameter.) 

Wir nehmen nun an, dass f\"ur gro\ss e Abst\"ande
vom Zentrum der Verteilung nur dieser eine
Parameter $M$ relevant ist. Dann ist
$GM/c^2 r$ dimensionslos und wir k\"onnen
schreiben:
\begin{eqnarray}
    A(r)  &=&  1 + 2 \beta \frac{GM}{c^2 r} + \ldots  \\
    B(r) &=&  1 - 2 \alpha \frac{GM}{c^2 r} + 2(\gamma - \beta)
     \left( \frac{GM}{c^2 r} \right)^2 + \ldots \, .
\end{eqnarray}
(Die seltsame Kombination der Parameter $\alpha, \beta,
\gamma$ hat historische Gr\"unde.) 
Diese Entwicklung bezeichnet man
als {\em Robertson-Entwicklung}.\index{Robertson-Entwicklung}.  
Wie wir schon in Abschnitt \ref{sec_NewtonGrenzfall}
gesehen haben, gilt f\"ur den Newton'schen
Grenzfall:
\begin{equation}
\label{eq_betagammaNewton}
  \mbox{Newton:  } ~ \alpha = 1~, ~~ \beta= \gamma = 0 \, .
\end{equation}
Im n\"achsten Abschnitt werden wir sehen, dass 
die Schwarzschild-L\"osung der ART folgende
Werte liefert:
\begin{equation}
\label{eq_betagammaART}
  \mbox{ART:  } ~ \alpha = \beta= \gamma = 1 \, .
\end{equation}
Diese Parameter haben wir auch schon im 
Zusammenhang mit der Lichtablenkung (Abschnitt
\ref{sec_Lichtablenkung}) und der Periheldrehung
(Abschnitt \ref{sec_Perihel}) betrachtet.

\subsection{Der Ricci-Tensor als Funktion von $A(r)$ und $B(r)$}

Wir k\"onnen nun aus der Metrik (\ref{eq_isostat1}) bzw.
\begin{equation}
      g_{\mu \nu} = {\rm diag}~ ( B(r) , - A(r) , -r^2 , - r^2 \sin^2 \theta )
\end{equation}
die Christoffel-Symbole, den Riemann'schen Kr\"ummungstensor
und den Ricci-Tensor berechnen. Man ben\"otigt dazu noch,
dass die Metrik mit oben stehenden Indizes das Inverse
der\index{Metrik!inverse} 
normalen Metrik ist, also
\begin{equation}
      g_{\mu \nu} = {\rm diag} ~
      \left( \frac{1}{B(r)} , - \frac{1}{A(r)} , -\frac{1}{r^2} , -\frac{1}{r^2 \sin^2 \theta} \right) \, .
\end{equation}
Die Rechnung ist in \cite{Fliessbach} durchgef\"uhrt. Hier beschr\"anke
ich mich auf das Ergebnis\index{Ricci-Tensor} 
f\"ur den Ricci-Tensor:
\begin{eqnarray}
\label{eq_RicciAB1}
      R_{00} &=& R_{tt}~ =~  - \frac{B''}{2A} + \frac{B'}{4A} \left( \frac{A'}{A} +
         \frac{B'}{B} \right) - \frac{B'}{rA}  \\ 
      R_{11} &=& R_{rr} \; =~   \frac{B''}{2B} - \frac{B'}{4B} \left( \frac{A'}{A} +
         \frac{B'}{B} \right) - \frac{A'}{rA}  \\
      R_{22} &=& R_{\theta \theta} \; = ~  
      - 1 -  \frac{r}{2A} \left( \frac{A'}{A} -
         \frac{B'}{B} \right) + \frac{1}{A}  \\
\label{eq_RicciAB2}
      R_{33} &=&  R_{\varphi \varphi} = ~ 
      R_{22} \, \sin^2 \theta \, .             
\end{eqnarray}
Alle gemischen Terme verschwinden: $R_{\mu \nu}=0$ f\"ur
$\mu \neq \nu$. 

\section{Schwarze L\"ocher -- Schwarzschild-Metrik}

Eine der ersten nicht-trivialen L\"osung der materiefreien Feldgleichungen 
der Allgemeinen Relativit\"atstheorie fand 1916 der deutsche Astronom
Karl\index{Schwarzschild, Karl} 
Schwarzschild ({\em geb.\ 9.10.1873 in Frankfurt am Main; 
gest.\ 11.5.1916 in Pottsdam}). Diese Schwarzschild-Metrik
\index{Schwarzschild-Metrik}
beschreibt nicht nur schwarze L\"ocher\index{Schwarzes Loch}, sondern
auch das Gravitationsfeld in der Umgebung von Sternen oder Planeten.
Daher kann man diese L\"osung zur Berechnung der Periheldrehung
des Merkur oder der Lichtablenkung im Gravitationsfeld der Sonne
heranziehen. 

\subsection{Klassische Berechnung des kritischen Radius}

Schon im Jahre 1783 entwickelte der Geologe 
John Michell\index{Michell, John} 
die\index{Dunkler Stern@\glqq Dunkler Stern\grqq} 
Vorstellung \glqq Dunkler Sterne\grqq,
von denen wegen ihrer Masse kein Licht entweichen
kann. Seine \"Uberlegungen basierten auf der
Newton'schen Mechanik und setzten voraus,
dass sich Licht wie jede andere Materie verh\"alt
(im Sinne der Korpuskeltheorie des Lichts von
Newton). Ausgangspunkt seiner \"Uberlegungen
war die klassische Fluchtgeschwindigkeit.

Ein Probek\"orper der Masse $m$, der im 
Unendlichen in Ruhe ist
($v=0$) und frei fallend auf einen 
schweren K\"orper der Masse $M$ zuf\"allt, hat im 
Abstand $r$ die kinetische Energie:
\begin{equation}
     E_{\rm kin} = \frac{1}{2} m v^2 = G \frac{mM}{r}
\end{equation} 
und damit die Geschwindigkeit
\begin{equation}
        v = \sqrt{2G \frac{M}{r}} \, .         
\end{equation}
Dies ist umgekehrt die\index{Fluchtgeschwindigkeit} 
Fluchtgeschwindigkeit
beim Radius $r$ aus dem Feld eines
K\"orpers der Masse $M$. 
Nach klassischen \"Uberlegungen hat
er somit die Lichtgeschwindigkeit $c$
bei einem Abstand:
\begin{equation}
\label{eq_Schwarzschildradius}
       R_{\rm S} = \frac{2GM}{c^2}  \, .
\end{equation}
Wenn der Radius des schweren K\"orpers
der Masse $M$ kleiner als $R_{\rm S}$ ist, kann
somit selbst Licht diesem K\"orper nicht
entweichen, da die Fluchtgeschwindigkeit
gr\"o\ss er als $c$ w\"are. Wie wir im n\"achsten
Abschnitt sehen werden, ist diese
Gleichung f\"ur einen kritischen Radius
(dem Radius des Horizonts eines schwarzen
Lochs -- dem Schwarzschild-Radius) auch 
in der Relativit\"atstheorie exakt.
\vspace{0.3cm}

{\small 
Ein solcher K\"orper ben\"otigt nicht
unbedingt eine sehr hohe Dichte, sofern
er nur gro\ss\ genug ist.  
Ausgedr\"uckt durch die Dichte ist die
Masse eines K\"orpers vom
Radius $R$ 
\begin{equation}
           M = \rho V = \rho \frac{4\pi}{3}R^3 \, .
\end{equation}
Setzen wir dies in Gleichung \ref{eq_Schwarzschildradius}
f\"ur den kritischen Radius ein, erhalten wir 
die Bedingung:
\begin{equation}
      R = \frac{2G}{c^2} \rho \frac{4\pi}{3} R^3
  \hspace{1cm} {\rm oder} \hspace{1cm}
      R = c  \sqrt{\frac{3}{8\pi \rho G} }  \, . 
\end{equation}
Mit 
\begin{equation}
       G = 6,6738 \cdot 10^{-11} \frac{{\rm m}^3}{{\rm kg}\cdot {\rm s}^2}
\end{equation}
und der Dichte von
Wasser, also $\rho = 1000\,{\rm kg}/{\rm m}^3$,
kommen wir zu einem Radius von rund 
$R=4\cdot 10^{8}$\,km oder dem rund
2,5-fachen des Abstands Erde--Sonne. 
Ein K\"orper dieser Gr\"o\ss e mit der Dichte
von Wasser h\"atte einen Radius, der
gleich seinem Schwarzschild-Radius ist.}

\subsection{Herleitung der Schwarzschild-L\"osung}

Wie wir im letzten Kapitel gesehen haben (vgl.\
Gl.\ \ref{eq_RiccigleichNull}), lassen sich
die Einstein'schen Feldgleichungen ohne Materie
(also f\"ur $T_{\mu \nu} = 0$) 
in der Form\index{Einstein'sche Gleichungen!ohne Materie}
\begin{equation}
       R_{\mu \nu} = 0  
\end{equation} 
schreiben. Wir verwenden nun die Komponenten
des Ricci-Tensors f\"ur die Standardform der isotropen,
statischen L\"osung (Gl.\ \ref{eq_RicciAB1}--\ref{eq_RicciAB2}), 
um eine L\"osung zu finden. Auch hierbei halte ich mich eng
an \cite{Fliessbach}. Es muss gelten:
$R_{00}=R_{11}=R_{22}=0$. (Wegen Gl.\ \ref{eq_RicciAB2}
folgt aus $R_{22}=0$ auch $R_{33}=0$.) 
Wir bilden die Summe 
\begin{equation}
\label{eq_SSL0}
   \frac{R_{00}}{B} + \frac{R_{11}}{A} = - \frac{1}{rA}
      \left( \frac{B'}{B} + \frac{A'}{A} \right) = 0  \, ,
\end{equation}
was auf
\begin{equation}
       \left( \frac{B'}{B} + \frac{A'}{A} \right) =  \frac{\rm d}{{\rm d}r}
       \ln (AB) = 0  
\end{equation}  
f\"uhrt. Daraus folgt
\begin{equation}
      A(r) B(r) = {\rm const} \, ,
\end{equation}
und wegen des asymptotischen Verhaltens von $A(r)$ und
$B(r)$ f\"ur gro\ss e Werte von $r$ ist die Konstante 1, also
\begin{equation}
\label{eq_SSL1}
                A(r) = \frac{1}{B(r)}  \, . 
\end{equation} 
Wir nutzen diese Beziehung in $R_{11}$ und gelangen zu der
Gleichung
\begin{equation}
\label{eq_SSL2}
           R_{22} = -1 + r B' + B = 0 \, .
\end{equation}
Da $R_{22}$ identisch verschwinden soll, muss auch
die Ableitung verschwinden und damit folgt:
\begin{equation}
\label{eq_SSL3}
           \frac{{\rm d}R_{22}}{{\rm d}r} = B' + r B'' + B' = rB'' + 2B' = 0 \, .
\end{equation}
Setzen wir Gl.\ \ref{eq_SSL1} in $R_{11}$ ein, erhalten wir
\begin{equation}
     R_{11} = \frac{B''}{2B} + \frac{B'}{rB} = \frac{rB'' + 2B}{2rB} \, .
\end{equation}
Wegen Gl.\ \ref{eq_SSL3} verschwindet dieser Ausdruck aber,
also ist $R_{11}=0$ (und damit wegen Gl.\ \ref{eq_SSL0} auch 
$R_{00}=0$), sofern $R_{22}$ verschwindet. Wir haben also
nur noch Gleichung \ref{eq_SSL2} zu l\"osen. Da $(rB)' = rB' + B$
f\"uhrt dies auf:
\begin{equation}
             \frac{{\rm d}(rB)}{{\rm d}r} = 1 \, .
\end{equation} 
Aus dieser Gleichung folgt $rB=r -2a$, wobei wir die
freie Integrationskonstante mit $-2a$ bezeichnet haben. 
Damit erhalten wir als L\"osungen:
\begin{equation}
     B(r) = 1 - \frac{2a}{r} \hspace{1cm} {\rm und}
     A(r) = \frac{1}{1 - 2a/r}  \, .
\end{equation}
Die Minkowski-Metrik $A(r)=B(r)=1$ ergibt sich aus dem
Spezialfall $a=0$. 

Die physikalische Bedeutung der Integrationskonstanten
$a$ ergibt sich aus einem Vergleich mit der Newton'schen
N\"aherung (siehe Abschnitt \ref{sec_NewtonGrenzfall}, 
Gl.\ \ref{eq_gnnNewton}). F\"ur gro\ss e Werte von $r$ sollte
die L\"osung dem Newton'schen Grenzfall entsprechen, und 
damit erhalten wir 
\begin{equation} 
            a=GM/c^2\, . 
\end{equation}  

\subsection{Die Schwarzschild-L\"osung}

Wie wir im letzten Abschnitt gesehen haben, l\"asst sich die 
Schwarzschild-Metrik als statische und 
rotationssymmetrische L\"osung schreiben, d.h.\ sie h\"angt in diesem
Fall nur von einer Koordinate ab -- dem Radius $r$ vom Zentrum der 
L\"osung:
\begin{equation}
\label{schwarz}
  {\rm d}s^2 ~=~ \left( 1 - \frac{2GM}{c^2 r} \right)c^2 {\rm d}t^2 -
  \frac{ {\rm d}r^2}{ 1 - 2GM/(c^2 r)}  - r^2 \left( {\rm d} \vartheta^2 +
  \sin^2 \vartheta \; {\rm d} \phi^2  \right)   
\end{equation}     
Au\ss erdem enth\"alt die Schwarzschild-Metrik noch einen Parameter $M$, 
der die St\"arke des Gravitationsfeldes in einem bestimmten Abstand angibt.
$M$ wird meist so gew\"ahlt, dass die Schwarzschild-L\"osung f\"ur
$r\rightarrow \infty$ einem Newton'schen Gravitationspotential 
einer Masse $M$ im Abstand $r$ entspricht. 

Der winkelabh\"angige (letzte) Term der Metrik entspricht
gew\"ohnlichen Kugelkoordinaten (vgl.\ Gl.\ \ref{eq_Kugelmetrik}). 
Die ersten beiden Terme werden bei 
\begin{equation}
            R_{\rm S} = \frac{2 G M}{c^2}
\end{equation}
singul\"ar. $R_{\rm S}$ bezeichnet man auch als
Schwarzschild-Radius\index{Schwarzschild-Radius}. 
Vergleichen wir die Art der Singularit\"at dieser
beiden Elemente der Metrik mit der speziellen
Kugelprojektion in Gl.\ \ref{eq_Kugelsingularitaet},
so entdeckt man (bis auf das relative Vorzeichen,
das durch die Minkowski-Struktur hereinkommt)
eine deutliche \"Ahnlichkeit, was die Vermutung
nahelegt, dass es sich hierbei nur um eine
Koordinatensingularit\"at handelt. 

Obwohl Eddington
\index{Eddington, Arthur Stanley}
1924 schon gezeigt hatte, dass es sich bei dieser Singularit\"at
nur um eine Koordinatensingularit\"at handelt, blieb ihre Natur doch
unklar. Bis in die sechziger Jahre schien man sich auch wenig daf\"ur
zu interessieren, da man sich kaum vorstellen konnte, dass schwarze
L\"ocher wirklich existierten. Erst 1958 untersuchte David Finkelstein  
\index{Finkelstein, David}
die Singularit\"at in der Metrik genauer und \glqq entdeckte\grqq\ das
alte Ergebnis von Eddington wieder. Bei $r=0$ gibt es allerdings wirklich
eine geometrische Singularit\"at, denn dort wird die Kr\"ummung unendlich.

\subsubsection{Veranschaulichungen der Schwarzschild-Metrik}

Als ersten Schritt zum Versuch einer Veranschaulichung
betrachten wir einen r\"aum\-lichen Schnitt durch die
\"Aquatorebene der Schwarzschild-Metrik. 
Das bedeutet,\index{Schwarzschild-Metrik!Aequa@\"Aquatorialebene}
wir setzen $\theta=0$ (dieser Winkel wird nicht variiert) und
$t={\rm const}$ (diese Koordinate wird ebenfalls nicht
variiert). Die verbleibende Metrik in der \"Aquatorebene
ist somit:
\begin{equation}
    {\rm d}s^2 = 
      \frac{ {\rm d}r^2}{ 1 - R_{\rm S}/r}  + r^2  {\rm d} \varphi^2 
\end{equation}
Wir suchen nun eine \glqq H\"ohenfunktion\grqq $f(r)$ \"uber der 
von $r$ und $\varphi$ parametrisierten Ebene, sodass die
Projektion der Fl\"ache $(r,\varphi,f(r))$ im $\mathbb{R}^3$
auf die Ebene $(r,\varphi)$ die oben angegebene Metrik
liefert. Nach den \"Uberlegungen aus Abschnitt \ref{sec_InduzMet} 
wird die $rr$-Komponente der Metrik (alle
anderen Komponenten werden nicht beeinflusst, da $f$ nicht
vom Winkel abh\"angen soll) zu
\begin{equation}
     g_{rr} = \frac{\partial (r,\varphi,f(r))}{\partial r} \cdot
     \frac{\partial (r,\varphi,f(r))}{\partial r} = (1,0,f'(r)) \cdot (1,0,f'(r)) 
     = 1 + (f'(r))^2 \, .
\end{equation}
Wir m\"ussen also die Differentialgleichung
\begin{equation}
       \frac{1}{1-R_{\rm S}/r} = 1 + f'^2
\end{equation}
bzw.
\begin{equation}
      f'(r) = \sqrt{\frac{R_{\rm S}}{r - R_{\rm S}}}    
\end{equation}
l\"osen (zun\"achst nur f\"ur $r > R_{\rm S}$). Das Ergebnis
ist
\begin{equation}
     f(r) = 2\sqrt{R_{\rm S}} \sqrt{r - R_{\rm S}} \, .
\end{equation} 
Wir erhalten somit einen Trichter von der Form einer
Wurzelfunktion (siehe Abb.\ \ref{fig_trichter}). Diesen
Trichter bezeichnet man auch als Flamm'schen 
Paraboloid.\index{Flamm'scher Paraboloid} 

\begin{figure}[ht]
\begin{picture}(210,120)(-10,0)
\put(0,0){\line(1,0){200}}
\put(0,0){\line(1,2){20}}
\put(200,0){\line(-1,2){20}}
\put(20,40){\line(1,0){55}}
\put(125,40){\line(1,0){55}}
%
\qbezier(110,20)(110,65)(150,110)
\qbezier(90,20)(90,65)(50,110)
\put(100,20){\line(0,1){90}}
%
\qbezier(80,20)(80,25)(100,25)
\qbezier(100,25)(120,25)(120,20)
\qbezier(0,110)(0,115)(100,115)
\qbezier(100,115)(200,115)(200,110)
\qbezier(50,75)(50,80)(100,80)
\qbezier(100,80)(150,80)(150,75)
\qbezier(70,52)(70,57)(100,57)
\qbezier(100,57)(130,57)(130,52)
\thicklines
\qbezier(120,20)(120,65)(200,110)
\qbezier(80,20)(80,65)(0,110)
\end{picture}
%
\hfill
\begin{picture}(160,150)(-50,-70)
\put(20,0){\line(1,0){50}}
\put(0,20){\line(0,1){50}}
\put(-20,0){\line(-1,0){50}}
\put(0,-20){\line(0,-1){50}}
\put(14.14,14.14){\line(1,1){35.4}}
\put(-14.14,14.14){\line(-1,1){35.4}}
\put(-14.14,-14.14){\line(-1,-1){35.4}}
\put(14.14,-14.14){\line(1,-1){35.4}}
%
\qbezier(-50,0)(-50,20.7)(-35.36,35.36)
\qbezier(-35.36,35.36)(-20.7,50)(0,50)
\qbezier(0,50)(20.7,50)(35.36,35.36)
\qbezier(35.36,35.36)(50,20.7)(50,0)
\qbezier(-50,0)(-50,-20.7)(-35.36,-35.36)
\qbezier(-35.36,-35.36)(-20.7,-50)(0,-50)
\qbezier(0,-50)(20.7,-50)(35.36,-35.36)
\qbezier(35.36,-35.35)(50,-20.7)(50,0)
%
\qbezier(-70,0)(-70,29)(-49.5,49.5)
\qbezier(-49.5,49.5)(-29,70)(0,70)
\qbezier(0,70)(29,70)(49.5,49.5)
\qbezier(49.5,49.5)(70,29)(70,0)
\qbezier(-70,0)(-70,-29)(-49.5,-49.5)
\qbezier(-49.5,-49.5)(-29,-70)(0,-70)
\qbezier(0,-70)(29,-70)(49.5,-49.5)
\qbezier(49.5,-49.5)(70,-29)(70,0)
%
\qbezier(-32,0)(-32,13.25)(-22.63,22.63)
\qbezier(-22.63,22.63)(-13.25,32)(0,32)
\qbezier(0,32)(13.25,32)(22.63,22.63)
\qbezier(22.63,22.63)(32,13.25)(32,0)
\qbezier(-32,0)(-32,-13.25)(-22.63,-22.63)
\qbezier(-22.63,-22.63)(-13.25,-32)(0,-32)
\qbezier(0,-32)(13.25,-32)(22.63,-22.63)
\qbezier(22.63,-22.63)(32,-13.25)(32,0)
%
\qbezier(-25,0)(-25,10.35)(-17.68,17.68)
\qbezier(-17.68,17.68)(-10.35,25)(0,25)
\qbezier(0,25)(10.35,25)(17.68,17.68)
\qbezier(17.68,17.68)(25,10.35)(25,0)
\qbezier(-25,0)(-25,-10.35)(-17.68,-17.68)
\qbezier(-17.68,-17,68)(-10.35,-25)(0,-25)
\qbezier(0,-25)(10.35,-25)(17.68,-17.68)
\qbezier(17.68,-17.68)(25,-10.35)(25,0)
%
\qbezier(-22,0)(-22,9.1)(-15.56,15.56)
\qbezier(-15.56,15.56)(-9.1,22)(0,22)
\qbezier(0,22)(9.1,22)(15.56,15.56)
\qbezier(15.56,15.56)(22,9.1)(22,0)
\qbezier(-22,0)(-22,-9.1)(-15.56,-15.56)
\qbezier(-15.56,-15.56)(-9.1,-22)(0,-22)
\qbezier(0,-22)(9.1,-22)(15.56,-15.56)
\qbezier(15.56,-15.56)(22,-9.1)(22,0)
%
\thicklines
\qbezier(-20,0)(-20,8.3)(-14.14,14.14)
\qbezier(-14.14,14.14)(-8.3,20)(0,20)
\qbezier(0,20)(8.3,20)(14.14,14.14)
\qbezier(14.14,14.14)(20,8.3)(20,0)
\qbezier(-20,0)(-20,-8.3)(-14.14,-14.14)
\qbezier(-14.14,-14.14)(-8.3,-20)(0,-20)
\qbezier(0,-20)(8.3,-20)(14.14,-14.14)
\qbezier(14.14,-14.14)(20,-8.3)(20,0)
\end{picture}
\caption{\label{fig_trichter}%
(links) Projiziert man den Trichter auf die $(r,\varphi)$-Ebene so induziert
dieser eine Metrik, die gleich der Metrik der Schwarzschild-L\"osung
in der \"Aquatorialebene ist. (rechts) Die Dichte der Kreise, die
auf dem Trichter einen konstanten Abstand voneinander haben, wird in 
dieser Karte immer h\"oher, je n\"aher man sich dem Schwarzschild-Radius
n\"ahert.}
\end{figure}

Obwohl $f(r)$ f\"ur sehr gro\ss e Werte von $r$ nicht
verschwindet, ist die durch $f(r)$ induzierte Metrik
f\"ur sehr gro\ss e Werte von $r$ gleich der Minkowski-Metrik,
da die Ableitung von $f(r)$ f\"ur $r\rightarrow \infty$ gegen null
geht. 

Was passiert f\"ur $r< R_{\rm S}$? In diesem Fall dreht sich
das Vorzeichen in der $rr$-Komponente der Metrik um. Wir
erhalten also f\"ur den Bereich innerhalb des Schwarzschild-Radius
eine kausale Struktur in $r$-Richtung. Dies werden wir sp\"ater
noch diskutieren. 

\begin{SCfigure}[50][ht]
\begin{picture}(260,230)(0,0)
\thinlines
\put(120,20){\line(0,1){180}}
\thicklines
\put(40,20){\vector(0,1){180}}
\put(43,20){\line(0,1){180}}
%\put(220,20){\line(0,1){200}}
\thinlines
\put(40,20){\vector(1,0){210}}
\qbezier(130,50)(139,33)(250,32)
\qbezier(130,50)(121,60)(121,200)
\put(40,10){\makebox(0,0){\footnotesize Singularit\"at}}
\put(120,10){\makebox(0,0){$R_{\rm S}$}}
\put(250,13){\makebox(0,0){$r$}}
\put(35,200){\makebox(0,0){$t$}}
%
\put(140,20){\line(0,1){23}}
\put(147,30){\makebox(0,0){${\scriptstyle \Delta \tau}$}}
%
\put(240,17){\line(0,1){6}}
\put(220,17){\line(0,1){6}}
\put(200,17){\line(0,1){6}}
\put(180.5,17){\line(0,1){6}}
\put(162,17){\line(0,1){6}}
\put(146,17){\line(0,1){6}}
\put(135,17){\line(0,1){6}}
\put(127,17){\line(0,1){6}}
\put(124,17){\line(0,1){6}}
\put(122,17){\line(0,1){6}}
\put(121,17){\line(0,1){6}}
\end{picture}
\caption{\label{figSMrt}%
Schwarzschild-Metrik au\ss erhalb des Schwarz\-schild-Radius. 
Physikalisch gleiche r\"aumliche
Abst\"ande werden in der Karte durch immer k\"urzere
Abst\"ande dargestellt, wenn $r$ sich dem Schwarzschild-Radius
n\"ahert. Andererseits entsprechen physikalisch gleiche 
Eigenzeitintervalle in $t$-Richtung
immer gr\"o\ss ere Intervallen in der Karte.}
\end{SCfigure}

Nun betrachten wir die Schwarzschild-Metrik in
$(t,r)$-Koordinaten\index{Schwarzschild-Metrik} 
(vgl.\ Abbildung \ref{figSMrt}).
Wir haben in der Abbildung gleiche physikalische
r\"aumliche Distanzen auf der $r$-Achse markiert.
In der Karte entsprechen diesen Distanzen immer
k\"urzere Abst\"ande. Man vergleiche dieses Verhalten
mit den Abst\"anden der Breitengraden auf einer Kugeloberfl\"ache
in einer senkrechten Zylinderprojektion auf die $z$-Achse
(Abb.\ \ref{fig_karte}). Umgekehrt werden gleiche Eigenzeiten in $t$-Richtung
in der Karte durch immer gr\"o\ss ere Abst\"ande wiedergegeben
(die Punkte auf der durchgezogenen Kurve haben von der
$r$-Achse immer denselben senkrechten Abstand $\Delta \tau$).    
Dies entspricht den Verh\"altnissen f\"ur horizontale Abst\"ande
(entlang von Breitengraden) in Abb.\ \ref{fig_karte}. Dieselbe physikalische
Distanz auf der Kugeloberfl\"ache wird in der Karte durch
immer gr\"o\ss ere Abst\"ande in horizontaler Richtung 
wiedergegeben, wenn man sich den Polen n\"ahert. 

\subsubsection{Weltlinien von Beobachtern}


\begin{SCfigure}[50][ht]
\begin{picture}(260,230)(0,0)
\thinlines
\put(120,20){\line(0,1){200}}
\thicklines
\put(40,20){\vector(0,1){200}}
\put(43,20){\line(0,1){200}}
\put(220,20){\line(0,1){200}}
\thinlines
\put(40,20){\vector(1,0){210}}
\qbezier(220,20)(220,50)(170,100)
\qbezier(170,100)(120,150)(122,220)
\put(40,10){\makebox(0,0){Singularit\"at}}
\put(120,10){\makebox(0,0){$R_{\rm S}$}}
\put(227,70){\makebox(0,0){$A$}}
\put(187,70){\makebox(0,0){$B$}}
\put(82,105){\makebox(0,0){$C$}}
\put(180,160){\makebox(0,0){I}}
\put(80,160){\makebox(0,0){II}}
\put(250,13){\makebox(0,0){$r$}}
\put(35,220){\makebox(0,0){$t$}}
\multiput(218,20)(0,6){34}{\line(1,0){4}}
\put(217,27){\line(1,0){4}}
\put(215,34){\line(1,0){4}}
\put(213,42){\line(1,0){4}}
\put(208,52){\line(1,0){4}}
\put(200,64){\line(1,0){4}}
\put(187,80){\line(1,0){4}}
\put(154,116){\line(1,0){4}}
\put(120,214){\line(1,0){4}}
\qbezier(117,200)(100,70)(43,70)
\put(114,193){\line(1,0){4}}
\put(96.5,120){\line(1,0){4}}
\put(79.5,90){\line(1,0){4}}
\put(67,78){\line(1,0){4}}
\put(61.5,75){\line(1,0){4}}
\put(57,73){\line(1,0){4}}
\put(51,71){\line(1,0){4}}
\put(43,70){\line(1,0){4}}
\end{picture}
\caption{\label{figschwarz}%
Schwarzschild-Metrik. Beobachter $A$ h\"alt einen konstanten
Abstand von der Singularit\"at; Beobachter $B$ trennt sich von
Beobachter $A$ und fliegt frei auf das schwarze Loch zu. Die
Markierungen auf den jeweiligen Weltlinien entsprechen 
(qualitativ) gleichen
Taktzeiten. Bereich I entspricht $r>R_{\rm S}$, Bereich II ist der
Bereich innerhalb des Horizonts. Eine Weltlinie hinter dem
Horizont ($C$) \glqq kommt von oben\grqq\ und 
endet unweigerlich in der Singularit\"at.}
\end{SCfigure}

Abb.~\ref{figschwarz} skizziert die Verh\"altnisse zweier 
Beobachter\index{Schwarzschild-Metrik!Weltlinien} 
im Feld der Schwarzschild-Metrik, ausgedr\"uckt in den
Koordinaten $(t,r)$, entsprechend der obigen Form (Gl.~\ref{schwarz}).
Die Winkelvariable interessieren uns im 
Folgenden nicht weiter, da die L\"osung rotationssymmetrisch ist.
Die Koordinate $t$ entspricht der Eigenzeit eines Beobachters
in konstantem \glqq unendlichen\grqq\ Abstand vom Schwarzen
Loch und ist proportional zur Eigenzeit
eines Beobachters $A$, der einen konstanten Abstand 
au\ss erhalb vom Schwarzschild-Radius h\"alt. 
N\"ahert sich dieser Abstand dem
Schwarzschild-Radius, so wird dieser 
Proportionalit\"atsfaktor kleiner, d.h.\ umso 
weiter liegen in diesen Koordinaten Takte konstanter Eigenzeit auf der 
Weltlinie auseinander. Am Schwarzschild-Radius wird dieser
Faktor null. Die senkrechte Linie an diesem Punkt entspricht
einer \glqq Null-Linie\grqq, d.h.\ zwei Punkte auf dieser Linie haben 
den Minkowski-Abstand Null. 

\begin{SCfigure}[50][ht]
\begin{picture}(260,230)(0,0)
\thinlines
\put(120,20){\line(0,1){200}}
\thicklines
\put(40,20){\vector(0,1){200}}
\put(43,20){\line(0,1){200}}
\put(220,20){\line(0,1){200}}
\thinlines
\put(40,20){\vector(1,0){210}}
\qbezier(220,20)(220,50)(170,100)
\qbezier(170,100)(120,150)(122,220)
\put(40,10){\makebox(0,0){Singularit\"at}}
\put(120,10){\makebox(0,0){$R_{\rm S}$}}
\put(227,70){\makebox(0,0){$A$}}
\put(187,70){\makebox(0,0){$B$}}
\put(180,160){\makebox(0,0){I}}
\put(80,160){\makebox(0,0){II}}
\put(250,13){\makebox(0,0){$r$}}
\put(35,220){\makebox(0,0){$t$}}
%
\qbezier(220,130)(250,131)(220,132)
\qbezier(220,130)(190,131)(220,132)
\qbezier(220,115)(250,116)(220,117)
\qbezier(220,115)(190,116)(220,117)
\put(204,116){\line(2,1){30}}
\put(236,116){\line(-2,1){30}}
%
\put(180,123){\makebox(0,0){
\qbezier(0,7)(20,8)(0,9)
\qbezier(0,7)(-20,8)(0,9)
\qbezier(0,-8)(20,-7)(0,-6)
\qbezier(0,-8)(-20,-7)(0,-6)
\put(-10.5,-7){\line(4,3){20}}
\put(10.5,-7){\line(-4,3){20}}
}}
%
\put(151,123){\makebox(0,0){
\qbezier(0,7)(15,8)(0,9)
\qbezier(0,7)(-15,8)(0,9)
\qbezier(0,-8)(15,-7)(0,-6)
\qbezier(0,-8)(-15,-7)(0,-6)
\put(-7,-7){\line(1,1){15}}
\put(7,-7){\line(-1,1){15}}
}}
%
\put(132,123){\makebox(0,0){
\qbezier(0,7)(8,8)(0,9)
\qbezier(0,7)(-8,8)(0,9)
\qbezier(0,-8)(8,-7)(0,-6)
\qbezier(0,-8)(-8,-7)(0,-6)
\put(-3,-7){\line(2,5){6}}
\put(3,-7){\line(-2,5){6}}
}}
%
\thicklines
\put(120.5,114){\line(0,1){18}}
\put(119.5,114){\line(0,1){18}}
%
\thinlines
\put(110,123){\makebox(0,0){
\qbezier(-6.5,0)(-5,15)(-3.5,0)
\qbezier(-6.5,0)(-5,-15)(-3.5,0)
\qbezier(6.5,0)(5,15)(3.5,0)
\qbezier(6.5,0)(5,-15)(3.5,0)
\put(-4,-8){\line(1,2){8}}
\put(4,-8){\line(-1,2){8}}
}}
%
\put(70,123){\makebox(0,0){
\qbezier(-12,0)(-10,10)(-8,0)
\qbezier(-12,0)(-10,-10)(-8,0)
\qbezier(8,0)(10,10)(12,0)
\qbezier(8,0)(10,-10)(12,0)
\put(-10,-5){\line(2,1){20}}
\put(10,-5){\line(-2,1){20}}
}}
\end{picture}
\caption{\label{figschwarz2}%
Lichtkegelstruktur der Schwarz\-schild-Metrik. 
Je kleiner $r$, umso enger wird der Lichtkegel in
den Schwarzschild-Koordinaten.
Beim Schwarzschild-Radius entartet der Kegel
zu einer Linie (\"Offnungswinkel null).
Innerhalb des Horizonts
haben $r$ und $t$ ihre Rollen als \glqq r\"aumliche\grqq\ 
und \glqq zeitliche\grqq\ Koordinaten vertauscht.}
\end{SCfigure}


F\"ur einen Beobachter $B$, der sich von
$A$ trennt und \glqq frei\grqq\ auf das schwarze Loch zuf\"allt,
werden die Intervalle gleicher Eigenzeit durch immer gr\"o\ss ere
Abst\"ande wiedergegeben. 
Dies wird durch die Markierungen in Abb.~\ref{figschwarz}
symbolisiert. Integriert man die L\"ange der Weltlinie von $B$ bis zum
Wert $t=\infty$ (d.h.\ unendlicher Eigenzeit des Beobachters $A$)
auf, so findet man einen endlichen Wert. Beobachter $B$ erreicht somit
den Schwarzschild-Radius nach einer f\"ur ihn endlichen Eigenzeit. Sein
weiteres Schicksal wird aus dieser Darstellung zun\"achst
nicht deutlich. Man kann jedoch seine Geod\"ate vervollst\"andigen
und erh\"alt dabei eine Trajektorie $C$ innerhalb des Schwarzschild-Radius,
die in den angegebenen Kooridanten \glqq von oben\grqq\ kommt
und in der Singularit\"at bei $r=0$ endet. 

F\"ur $r<R_{\rm S}$ haben die Koeffizienten von ${\rm d}t$ und ${\rm d}r$
ihre Vorzeichen gewechselt, d.h.\ innerhalb des Schwarzschild-Radius
sind die Verh\"altniss von 
\glqq raumartig\grqq\ und \glqq zeitartig\grqq\ umgekehrt: waagerechte
Linien sind zeitartig und senkrechte Linien sind raumartig. Daher
liegt die Singularit\"at eigentlich in der zeitlichen Zukunft und ist
raumartig. 

Das \glqq Universum\grqq\ von Beobachter $A$ ist der Bereich I. Er kann 
Bereich II\index{Schwarzschild-Metrik!kausale Beziehungen} 
innerhalb des Schwarzschild-Radius nicht einsehen. Der
Schwarzschild-Radius bildet somit f\"ur $A$ einen
Ereignishorizont.\index{Ereignishorizont}
Er sieht einen Beobachter $B$ auf diesen Ereignishorizont zufliegen,
allerdings bleibt $B$ f\"ur alle Zeiten von $A$ au\ss erhalb dieses
Bereichs. Er verschwindet f\"ur $A$ im infraroten Bereich des
Spektrums. 

\begin{figure}[ht]
\begin{picture}(360,250)(0,0)
\put(160,70){\vector(0,1){100}}
\put(60,120){\vector(1,0){240}}
\thicklines
%\put(195,63){\line(0,1){114}}
\put(60,20){\line(1,1){200}}
\put(60,220){\line(1,-1){200}}
\qbezier(232,70)(200,120)(195,177)
\qbezier(270,20)(170,120)(270,220)
\qbezier(80,210)(160,130)(240,210)
\qbezier(80,213)(160,133)(240,213)
\qbezier(80,30)(160,110)(240,30)
\qbezier(80,27)(160,107)(240,27)
\thinlines
\put(155,160){\makebox(0,0){$t'$}}
\put(300,115){\makebox(0,0){$x$}}
\put(160,200){\makebox(0,0){Zukunftssingularit\"at}}
\put(160,40){\makebox(0,0){Vergangenheitssingularit\"at}}
\put(250,130){\makebox(0,0){{\Large I}}}
\put(150,80){\makebox(0,0){{\Large IV}}}
\put(60,130){\makebox(0,0){{\Large III}}}
\put(175,155){\makebox(0,0){{\Large II}}}
\put(160,128){\makebox(0,0){O}}
\put(230,150){\makebox(0,0){A}}
\put(205,110){\makebox(0,0){B}}
\end{picture}
\caption{\label{figkruskal}%
Kruskal-Szekeres-Darstellung der geod\"atisch vervollst\"andigten 
Schwarzschild-L\"osung. Beobachter $A$ sieht das schwarze Loch nur
von au\ss en. Seine Welt ist der Bereich I. Aus dem Bereich der
Vergangenheitssingularit\"at IV kann Strahlung in Bereich I dringen.
Beobachter $A$ sieht diesen Bereich als \glqq wei\ss es Loch\grqq.
Andererseits kann ein Beobachter $B$ auch von Bereich I hinter
den Horizont in Bereich II dringen. Er trifft dann unweigerlich auf
die Zukunftssingularit\"at. Bereich II ist f\"ur $A$ ein schwarzes Loch.}
\end{figure}

Eine interessante\index{Schwarzes Loch!Kruskal-Szekeres-Karte} 
Darstellung der Verh\"altnisse an einem schwarzen
Loch, die insbesondere das Schicksal von Beobachter $B$ deutlich macht,
fanden 1960 Martin Kruskal\index{Kruskal, Martin David} und unabh\"angig 
von ihm George Szekeres\index{Szekeres, George} (vgl.\ Abb.~\ref{figkruskal}).
Der Darstellung \ref{figschwarz} entsprechen dabei zun\"achst nur die
Bereiche I und II. Die Bereiche III und IV sind eine Erweiterung der
Schwarzschild-Metrik zu einem geod\"atisch vollst\"andigen 
Universum:\index{Geod\"atisch Vollst\"andig}
Geod\"aten enden entweder an einer Singulari\"at oder im Unendlichen.

Der wesentliche qualitative Unterschied zwischen Abb.~\ref{figkruskal} 
und der Darstellung des Rindler-Universums,
Abb.~\ref{figrindler}, liegt in der Existenz einer Zukunfts- und
Vergangenheitssingularit\"at. Dort wird die Kr\"ummung singul\"ar,
d.h.\ nach unserem klassischen Verst\"andnis enden dort Raum und Zeit.
Ein weiterer Unterschied zum Rindler-Universum ist die
Rotationsinvarianz. Erweitert man das Rindler-Universum um
zwei r\"aumliche Dimensionen, wird die Struktur der
Horizonte dort zu einem Keil, wohingegen es beim Schwarzen
Loch ein Kegel ist. 

Bereich I entspricht einem \"au\ss eren Beobachter $A$ des schwarzen Loches.
Er muss eine Kraft aufwenden, um sich dem Einfluss des schwarzen
Loches entziehen zu k\"onnen, und sp\"urt somit das Gravitationsfeld. 
Aus Bereich IV kann Strahlung in seine Welt dringen, nichts aus seiner
Welt kann aber in diesen Bereich hinein. Man bezeichnet diesen Bereich
manchmal auch als \glqq wei\ss es Loch\grqq\index{Wei\ss es Loch}. 
Die Trennungsfl\"ache zwischen Bereich I und II entspricht dem 
Ereignishorizont.
\index{Schwarzes Loch}\index{Ereignishorizont}
Nichts, was einmal aus Bereich I in diesen Bereich gelangt ist, kann
jemals wieder in den Bereich I zur\"uck. Der \"au\ss ere Beobachter
$A$ sieht den Beobachter $B$ an der Oberfl\"ache des Horizonts 
verschwinden, ganz \"ahnlich wie im Fall des Rindler-Universums.

Bereich III ist wiederum von Bereich I kausal getrennt. F\"ur eine
geod\"atisch vollst\"andige L\"osung -- d.h.\ eine L\"osung, f\"ur die
jede Geod\"ate entweder fortgesetzt werden kann oder an einer
Singularit\"at endet -- ist dieser Bereich jedoch notwendig. Er ist
wie ein zweites Universum - ebenfalls au\ss erhalb des schwarzen Loches -
aber trotzdem mit I durch keine zeitartige Linie verbunden.

Eine ebenfalls elegante Darstellung von L\"osungen 
der Einstein-Gleichungen sind so genannte 
Penrose-Diagramme.\index{Penrose-Diagramm}
Hierbei handelt es sich um eine konforme Koordinatentransformation,
d.h.\ Winkel werden getreu dargestellt, allerdings sind L\"angen
verzerrt. Durch eine konforme Transformation l\"asst sich die
Kruskal-Szekeres-Raumzeit auf ein Diagramm wie in
Abb.\ \ref{fig_Penrose} abbilden.

\begin{SCfigure}[50][htb]
\begin{picture}(210,130)(0,0)
\put(60,20){\line(1,0){100}}
\put(60,120){\line(1,0){100}}
\put(60,20){\line(-1,1){50}}
\put(60,20){\line(1,1){100}}
\put(160,20){\line(1,1){50}}
\put(160,20){\line(-1,1){100}}
\put(10,70){\line(1,1){50}}
\put(210,70){\line(-1,1){50}}
\multiput(62,20)(4,0){25}{\makebox(0,0){${\scriptstyle \vee}$}}
\multiput(62,120)(4,0){25}{\makebox(0,0){${\scriptstyle \vee}$}}
\put(110,10){\makebox(0,0){\footnotesize Vergangenheitssingularit\"at}}
\put(110,130){\makebox(0,0){\footnotesize Zukunftssingularit\"at}}
\put(175,74){\makebox(0,0){\footnotesize unser}}
\put(175,66){\makebox(0,0){\footnotesize Universum}}
\put(45,74){\makebox(0,0){\footnotesize anderes}}
\put(45,66){\makebox(0,0){\footnotesize Universum}}
\put(110,110){\makebox(0,0){\footnotesize Schwarzes}}
\put(110,101){\makebox(0,0){\footnotesize Loch}}
\put(110,40){\makebox(0,0){\footnotesize Wei\ss es}}
\put(110,31){\makebox(0,0){\footnotesize Loch}}
%
\put(139,90){\begin{rotate}{45}
\makebox(0,0){\bf\tiny Ereignishorizont}
\end{rotate}}
%
\put(196,100){\makebox(0,0){${\scriptstyle t=+\infty}$}}
\put(196,40){\makebox(0,0){${\scriptstyle t=-\infty}$}}
\end{picture}
\caption{\label{fig_Penrose}%
Penrose-Diagramm einer vervollst\"andigten 
Schwarzschild-L\"osung. Die Zukunftssingularit\"at
und die Vergangenheitssingularit\"at entsprechen
den horizontanten Linien am oberen und unteren
Rand. Topologisch sind die Bereiche \"ahnlich
dargestellt wie bei den Kruskal-Szekeres-Koordinaten.
Lokal sind Lichtkegel immer unter $\pm 45^\circ$ geneigt,
also wie bei einem gew\"ohnlichen Minkowski-Diagramm.}
\end{SCfigure}

Durch die konforme Transformation wird der Bereich
\glqq unseres\grqq\ Universums endlich. Er wird oben
rechts von der Zukunftslinie $t=+\infty$ berandet und
unten rechts von der Vergangenheitslinie $t=-\infty$.
S\"amtliche Weltlinien in unserem Universum (die also nie
einen Ereignishorizont \"uberschreiten) beginnen
auf der Linie unten rechts und enden auf der Linie
oben rechts. Weltlinien aus unserem Universum, die
den Ereignishorizont zum Schwarzen Loch
\"uberschreiten, enden an der Zukunftssingularit\"at.

\section{Schwarze L\"ocher mit Ladung und Drehimpuls}

Geschlossene L\"osungen der Einstein-Gleichung sind
noch f\"ur Schwarze L\"ocher mit einer elektrischen Ladung $Q$
und einem Drehimpuls $J$ bekannt. Die Kerr-Newman-Metrik 
beschreibt eine\index{Kerr-Newman-Metrik} 
L\"osung mit allen drei Qualit\"atsmerkmalen.
Sie lautet:
\begin{equation}
{\rm d}s^2 =  \frac{\Delta}{\rho^2} ({\rm d}t - a \sin^2 \theta\, {\rm d}\varphi)^2
    - \frac{\sin^2 \theta}{\rho^2} \Big( (r^2+ a^2){\rm d}\varphi - a\, {\rm d}t \Big)^2
    - \frac{\rho^2}{\Delta} {\rm d}r^2 - \rho^2\, {\rm d}\theta^2
\end{equation}
mit folgenden Abk\"urzungen:
\begin{equation}
   \Delta = r^2 - 2Mr + a^2 + Q^2  \hspace{1cm}
   \rho^2 = r^2 + a^2 \cos^2 \theta \hspace{1cm}
   a = \frac{J}{M}   \, .
\end{equation}
F\"ur den Spezialfall $Q=0$ und $J=0$ wird
diese Metrik zur Schwarzschild-Metrik (Gl.\ \ref{schwarz}). 
Den Spezialfall $Q=0$ und $J\neq 0$ bezeichnet man
als Kerr-Metrik, und den Spezialfall $Q\neq 0$ und $J=0$
als Reissner-Nordstr\"om-Metrik. 

Theoretisch kann ein Schwarzes Loch noch 
andere Quantenzahlen haben -- Baryonenzahl, Leptonenzahl, etc. --
aber es gibt zu diesen Quantenzahlen keine weiteren
Parameter. Insbesondere gibt es zur Massenverteilung oder
Ladungsverteilung keine h\"oheren Momente, oder zum Drehimpuls
keinen Tr\"agheitstensor mit unterschiedlichen Tr\"agheitsmomenten.
Diese Tatsache bezeichnet man manchmal 
auch als\index{No-Hair-Theorem}
\glqq No-Hair-Theorem\grqq, womit angedeutet werden soll, dass
ein Schwarzes Loch neben diesen angedeuteten Quantenzahlen
keine weiteren Freiheitsgrade besitzen kann. 

Besonders interessant sind einige Eigenschaften
rotierender\index{Schwarzes Loch!rotierendes} 
Schwarzer L\"ocher, die ich
qualitativ kurz ansprechen m\"ochte, da sie
einerseits bei Sch\"ulern auf
gro\ss es Interesse sto\ss en d\"urften und
andererseits mit gro\ss er Wahrscheinlichkeit die
riesigen Schwarzen L\"ocher in den Zentren
von Galaxien sehr rasch rotieren werden, diese
L\"osungen also wichtig sind.

\begin{SCfigure}[50][htb]
\begin{picture}(185,120)(0,20)
\qbezier(50,70)(50,90.7)(64.65,105.36)
\qbezier(64.65,105.36)(79.3,120)(100,120)
\qbezier(100,120)(120.7,120)(135.36,105.36)
\qbezier(135.36,105.36)(150,90.7)(150,70)
\qbezier(50,70)(50,49.3)(64.65,34.65)
\qbezier(64.65,34.65)(79.3,20)(100,20)
\qbezier(100,20)(120.7,20)(135.36,34.65)
\qbezier(135.36,34.65)(150,49.3)(150,70)
%
\qbezier(20,70)(20,90.7)(43.42,105.36)
\qbezier(43.42,105.36)(66.9,120)(100,120)
\qbezier(100,120)(133.1,120)(156.6,105.36)
\qbezier(156.6,105.36)(180,90.7)(180,70)
\qbezier(20,70)(20,49.3)(43.42,34.65)
\qbezier(43.42,34.65)(66.9,20)(100,20)
\qbezier(100,20)(133.1,20)(156.6,34.65)
\qbezier(156.6,34.65)(180,49.3)(180,70)
\put(163,73){\makebox(0,0){\tiny Ergo-}}
\put(163,67){\makebox(0,0){\tiny sph\"are}}
\put(100,5){\vector(0,1){130}}
\end{picture}
\caption{\label{fig_rotblackhole}%
Ein rotierendes Schwarzes Loch besitzt
einen inneren und einen \"au\ss eren
Ereignishorizont. Den Bereich zwischen den
beiden Horizonten bezeichnet man als
Ergosph\"are.}
\end{SCfigure}

Ein rotierendes Schwarzes Loch besitzt zwei
Ereignishorizonte (siehe Abb.\ \ref{fig_rotblackhole}).
Der innere Ereignishorizont hat Kugelform, und
ein Gegenstand, der diesen Horizont \"uberquert hat,
kann nicht mehr in den \"au\ss eren Bereich, aus dem
er gekommen ist, zur\"uck. Zwischen den beiden
Horizonten befindet sich die so genannte
Ergosph\"are.\index{Ergosph\"are} 
In diesem Bereich kann sich ein
Beobachter nur halten, wenn er sich um das
Schwarze Loch herumbewegt. Der Raum 
in diesem Bereich wird durch das rotierende
Schwarze Loch derart \glqq mitgedreht\grqq,
dass der Lichtkegel so stark in Rotationsrichtung
gekippt ist, dass eine Weltlinie innerhalb des
Lichtkegels nicht an ihrem Ort bleiben kann, sondern
sich ebenfalls um das Schwarze Loch winden muss.
Trotzdem kann ein Beobachter diesem Bereich
auch wieder entkommen. 

Roger Penrose hat auf einen interessanten Prozess
hingewiesen\index{Penrose-Prozess} 
(den so genannten \textit{Penrose-Prozess}), bei
dem man an einem rotierenden Schwarzen Loch durch
die Entsorgung von M\"ull Energie gewinnen kann:
Eine Rakete mit M\"ull fliegt in den Bereich der
Ergosph\"are und trennt sich dort von dem
M\"ull. Dies kann in einer Weise geschehen, dass
der M\"ull in das Schwarze Loch hineinfliegt und
gleichzeitig der Rakete ein gr\"o\ss erer Impuls
\"ubertragen wird, mit dem sie das Schwarze
Loch wieder verlassen kann. Insgesamt hat die
Rakete nachher eine h\"ohere Energie als vorher,
und zwar einschlie\ss lich der urspr\"unglichen
Energie des M\"ulls,
und der M\"ull ist im Schwarzen Loch 
verschwunden. Bei diesem Prozess verliert das
rotierende Schwarze Loch allerdings einen
Teil seines Drehimpulses (und damit einen Teil seiner
Rotationsenergie, die nun auf die Rakete \"ubertragen
wurde), sodass er nicht beliebig
oft wiederholt werden kann.

Allgemein sind die Eigenschaften der L\"osungen zu rotierenden 
bzw.\ geladenen Schwarzen L\"o\-chern wesentlich 
komplizierter als bei der Schwarzschild-L\"osung. 
Abbildung \ref{fig_Penrose2}) zeigt das 
Penrose-Diagramm\index{Penrose-Diagramm}
zur geod\"atisch vervollst\"andigten L\"osung. Man erkennt
die beiden Horizonte, den inneren und \"au\ss eren Horizont.
Au\ss erdem sind die Singularit\"aten nun zeitartig und
k\"onnen von einer Weltlinie umgangen werden, so dass
es bei diesen Schwarzen L\"ochern tats\"ahlich m\"oglich
ist, dem Inneren wieder zu entkommen, allerdings in
einen anderen Teil des Universums.

\begin{SCfigure}[50][htb]
\begin{picture}(220,350)(0,0)
\put(160,120){\line(0,1){100}}
\put(60,120){\line(0,1){100}}
\put(70,10){\line(-1,1){60}}
\put(50,10){\line(1,1){160}}
\put(150,10){\line(1,1){60}}
\put(170,10){\line(-1,1){160}}
\put(10,70){\line(1,1){200}}
\put(210,70){\line(-1,1){200}}
\put(10,170){\line(1,1){160}}
\put(210,170){\line(-1,1){160}}
\put(10,270){\line(1,1){60}}
\put(210,270){\line(-1,1){60}}
\put(60,320){\line(0,1){10}}
\put(160,320){\line(0,1){10}}
\put(60,10){\line(0,1){10}}
\put(160,10){\line(0,1){10}}
\multiput(60,122)(0,4){25}{\makebox(0,0){${\scriptstyle >}$}}
\multiput(160,122)(0,4){25}{\makebox(0,0){${\scriptstyle >}$}}
\multiput(60,322)(0,4){3}{\makebox(0,0){${\scriptstyle >}$}}
\multiput(160,322)(0,4){3}{\makebox(0,0){${\scriptstyle >}$}}
\multiput(60,10)(0,4){3}{\makebox(0,0){${\scriptstyle >}$}}
\multiput(160,10)(0,4){3}{\makebox(0,0){${\scriptstyle >}$}}
%
\put(40,179){\makebox(0,0){\tiny zeitartige}}
\put(37,170){\makebox(0,0){\tiny Singularit\"at}}
\put(180,179){\makebox(0,0){\tiny zeitartige}}
\put(183,170){\makebox(0,0){\tiny Singularit\"at}}
\put(175,74){\makebox(0,0){\footnotesize unser}}
\put(175,66){\makebox(0,0){\footnotesize Universum}}
\put(45,74){\makebox(0,0){\footnotesize anderes}}
\put(45,66){\makebox(0,0){\footnotesize Universum}}

\put(45,274){\makebox(0,0){\footnotesize anderes}}
\put(45,266){\makebox(0,0){\footnotesize Universum}}
\put(170,274){\makebox(0,0){\footnotesize anderes}}
\put(170,266){\makebox(0,0){\footnotesize Universum}}

%
\put(138,91){\begin{rotate}{45}
\makebox(0,0){\bf\tiny Ereignishorizont}
\end{rotate}}
\put(131,97){\begin{rotate}{45}
\makebox(0,0){\bf\tiny \"au\ss erer}
\end{rotate}}
\put(132,143){\begin{rotate}{-45}
\makebox(0,0){\bf\tiny Ereignishorizont}
\end{rotate}}
\put(135,153){\begin{rotate}{-45}
\makebox(0,0){\bf\tiny innerer}
\end{rotate}}
\put(137,190){\begin{rotate}{45}
\makebox(0,0){\bf\tiny Ereignishorizont}
\end{rotate}}
\put(131,197){\begin{rotate}{45}
\makebox(0,0){\bf\tiny innerer}
\end{rotate}}
\put(132,243){\begin{rotate}{-45}
\makebox(0,0){\bf\tiny Ereignishorizont}
\end{rotate}}
\put(135,253){\begin{rotate}{-45}
\makebox(0,0){\bf\tiny \"au\ss erer}
\end{rotate}}
\qbezier(135,60)(70,170)(135,280)
%
\put(196,100){\makebox(0,0){${\scriptstyle t=+\infty}$}}
\put(196,40){\makebox(0,0){${\scriptstyle t=-\infty}$}}
\put(95,230){\makebox(0,0){\bf\tiny erlaubte}}
\put(92,223){\makebox(0,0){\bf\tiny Weltlinie}}

\end{picture}
\caption{\label{fig_Penrose2}%
Penrose-Diagramm einer vervollst\"andigten 
Kerr-L\"osung. Die Singularit\"aten sind nun
zeitartig und k\"onnen daher von Weltlinien
vermieden werden. Das Diagramm wiederholt sind
in Zeitrichtung beliebig oft. In dieser Raumzeit
gibt es zeitartige Weltlinien, die von unserem
Universum in andere Universen f\"uhren, ohne
auf eine Singularit\"at zu treffen.}
\end{SCfigure}


\begin{thebibliography}{99}
%\addcontentsline{toc}{chapter}{Literaturangaben}
%\bibitem{Aichelburg} Peter C.\ Aichelburg (Hrsg.); {\it Zeit im 
%       Wandel der Zeit}; Verlag Vieweg, Braunschweig, Wiesbaden, 1988.
%\bibitem{Barbour3} {\it Mach's Principle -- From Newton's Bucket to
%        Quantum Gravity}; Julian Barbour \& Herbert Pfister (Hrsg.);
%        Birkh\"auser, Boston, Basel, Berlin, 1995.       
%\bibitem{Bekenstein} Jacob D.\ Bekenstein, \textit{Black holes
%          and entropy}, Phys.\ Rev.\ D\,7 (1973) 2333--2346.        
%\bibitem{Bell} John Bell;  {\em Speakable and Unspeakable in 
%        Quantum Physics}, 2.\ edition, Cambridge University Press (2004).       
%\bibitem{Born} Max Born; {\it Optik}; Springer-Verlag, Berlin, Heidelberg,
%        1972.
%\bibitem{Britannica} Encyclopaedia Britannica; 15.th edition, 1988.
%\bibitem{Descartes} Ren\'e Descartes; {\it Die Prinzipien der
%        Philosophie}; Felix Meiner Verlag, Hamburg, 1992; \"ubersetzt
%        von Artur Buchenau.
%\bibitem{EDM} Encyclopaedic Dictionary of Mathematics; Second Edition,
%        MIT Press, 1987.
%\bibitem{Einstein1} Albert Einstein; {\it Zur Elektrodynamik bewegter 
%        K\"orper}; Annalen der Physik, Leipzig, 17 (1905) 891. 
%\bibitem{Einstein2} Albert Einstein; {\it Ist die Tr\"agheit eines
%        K\"orpers von seinem Energieinhalt abh\"angig?} (Ann.\ Phys., 
%        Leipzig, 18 (1905) 639.
%\bibitem{Einstein3} Albert Einstein; {\it Aus meinen sp\"aten Jahren};
%         Ullstein Sachbuch, Verlag Ullstein, Frankfurt, Berlin, 1993.                 
%\bibitem{Einstein4} Albert Einstein; {\it Prinzipielles zur allgemeinen
%        Relativit\"atstheorie}; Annalen der Physik 55 (1918) 241.
%\bibitem{Einstein5} Albert Einstein; {\it \"Uber den Einflu\ss\ der
%        Schwerkraft auf die Ausbreitung des Lichtes}; Annalen der
%        Physik 35 (1911) 898.                 
%\bibitem{Feynman} Richard Feynman; {\it The Character of Physical Law};
%        The MIT Press, 1987.        
%\bibitem{Fierz} Markus Fierz; {\it \"Uber den Ursprung und die Bedeutung
%        der Lehre Isaac Newtons vom absoluten Raum}; Gesnerus, 
%        11.\ Jahrgang (1954), S.\,62--120.
%\bibitem{Fliessbach} Torsten Flie\ss bach; {\it Allgemeine 
%        Relativit\"atstheorie}; BI-Wissenschaftsverlag, Mannheim, Wien
%        Z\"urich, 1990. 
%\bibitem{Galilei} Galilei; {\it Dialog \"uber die beiden haupts\"achlichen
%        Weltsysteme, das ptolem\"aische und das kopernikanische}; 
%        Teubner Stuttgart, 1982; aus dem Italienischen \"ubersetzt von
%        Emil Strauss.   
%  \bibitem{Hawking} Stephen W.\ Hawking, \textit{Particle Creation by
%            black holes}, Comm.\ Math.\ Phys.\ 43 (1976) 199--220.      
%\bibitem{Helmholtz2} Hermann von Helmholtz; {\em \"Uber Wirbelbewegungen,
%        \"Uber Fl\"ussigkeitsbewegungen}, 1858; in Ostwalds Klassiker der 
%       exakten Wissenschaften Bd.\ 1; Verlag Harri Deutsch, Frankfurt, 
%       1996.                   
%\bibitem{Lamb} G.L.\ Lamb, Jr.; {\it Elements of Soliton Theory}; 
%         Pure \& Applied Mathematics, John Wiley \& Sons, 1980. 
%\bibitem{Laue} Max von Laue; {\it Geschichte der Physik}; 
%         Universit\"ats-Verlag Bonn, 1947.
%\bibitem{Lorentz} Hendrik Antoon Lorentz; {\it Electromagnetic phenomena 
 %        in a system moving with any velocity smaller than that of light}; 
%         Proc.\ Acad.\ Sci., Amsterdam, 6 [1904], S.\ 809.
%\bibitem{Mach} Ernst Mach; {\it Die Mechanik in ihrer Entwicklung
%      historisch kritisch dargestellt}; Akademie Verlag, Berlin, 1988.       
%\bibitem{Mainzer} Klaus Mainzer; {\it Philosophie und Geschichte von
%         Raum und Zeit}; in {\it Philosophie und Physik der Raum-Zeit};
%         J\"urgen Audretsch und Klaus Mainzer (Hrsg.); 
%         BI-Wissenschaftsverlag, 1994. 
%\bibitem{Microscope} Touboul, Pierre, et al. (MICROSCOPE Collaboration); \textit{MICROSCOPE Mission:
%         Final Results of the Test of the Equivalence Principle}; Phys.\ Rev.\ Lett.\ \textbf{129} (2022) 121102.
%\bibitem{Misner} C.W.\ Misner, K.S.\ Thorne, J.A.\ Wheeler; 
%        {\it Gravitation}; W.H.\ Freeman and Company, San Francisco,  1973.
%\bibitem{Mittelstaedt} Peter Mittelstaedt; {\it Der Zeitbegriff in der
%        Physik}; BI-Wissenschaftsverlag, 1989.        
%\bibitem{Mittelstaedt2} Peter Mittelstaedt; {\it Philosophische Probleme
%        der modernen Physik}; BI-Wissenschaftsverlag, 1989.        
%\bibitem{Newton}
%   Isaac Newton; {\it Mathematische Grundlagen der Naturphilosophie}; 
%   \"ubersetzt von Ed Dellian; Felix Meiner Verlag, 1988. 
%\bibitem{Newton2} Isaac Newton; {\it \"Uber die Gravitation...};
%       Klostermann Texte Philosophie; Vittorio Klostermann, Frankfurt,
%      1988; \"ubersetzt von Gernot B\"ohme.
%\bibitem{Newton3} Isaac Newton; {\it Optik oder Abhandlung \"uber
%      Spiegelungen, Brechungen, Beugungen und Farben des Lichts};
%      I., II.\ und III.\ Buch (1704); aus dem Englischen \"ubersetzt
%      von W.\ Abendroth; Ostwalds Klassiker der exakten Wissenschaften,
%      Verlag Harri Deutsch 1998.   
%\bibitem{Neumann} Carl Neumann; {\it \"Uber die Principien der
%         Galilei-Newtonschen Theorie}; Akademische Antrittsvorlesung,
%         gehalten in der Aula der Universit\"at Leipzig am 3.\ Nov.\
%         1869; Teubner (Leipzig) 1870.         
%\bibitem{Pauli} Wolfgang Pauli; {\it Theory of Relativity}; Dover
%      Publications, New York, 1981.      
%\bibitem{Poincare} Jules Henri Poincar\'e; {\it Sur la dynamique de 
%     l'\'electron}, C.R.\ Acad.\ Sci., Paris, 140 (1905) S.~1504; und 
%      Rendiconti del Circolo Matematico di Palermo, Bd.~21 (1906) S.~129.
%\bibitem{Reichenbach1} Hans Reichenbach; {\em Philosophie der 
%       Raum-Zeit-Lehre}; Hans Reichenbach - Gesammelte Werke Bd.\ 2;
%       Vieweg-Verlag, Braunschweig; 1977.
%\bibitem{Reichenbach2} Hans Reichenbach; {\em Axiomatik der
%       relativistischen Raum-Zeit-Lehre}; in {\em Die philosophische
%       Bedeutung der Relativit\"atstheorie}; Hans Reichenbach - Gesammelte
%       Werke Bd.\ 3; Vieweg-Verlag, Braunschweig, 1977. 
%\bibitem{Rovelli} Carlo Rovelli, \textit{Quantum Gravity}; Cambridge
%      University Press, 2007.       
%\bibitem{Schlamminger} Schlamminger, Choi, Wagner, Gundlach,
%         Adelberger; {\em Test of the Equivalence Principle using a
%         rotating torsion balance}; Phys.\ Rev.\ Lett.\ {\bf 100} (2008)
%         041101.     
%\bibitem{Sexl} Roman U.\ Sexl, Helmuth K.\ Urbantke; {\it Relativit\"at,
%      Gruppen, Teilchen}; Springer-Verlag, Wien, New York, 1992.
%\bibitem{Simonyi}
%       K\'aroly Simonyi; {\it Kulturgeschichte der Physik}; Verlag
%       Harri Deutsch, Thun, Frankfurt am Main, 1990.
%\bibitem{Weisberg} Weisberg, J.M., Taylor, J.H.; {\em Relativistic Binary Pulsar
%          B1913+16: Thirty Years of Observations and Analysis}; 
%          \verb+arXiv:astro-ph/0407149v1+; 2004. 
%\bibitem{Thomson} James Thomson; {\it On the Law of Inertia; the
%       Principle of Chronometry; and the Principle of Absolute Clinural
%       Rest, and of Absolute Rotation}; Proc.\ Roy.\ Soc.\ (Edinburgh),
%       Session 1883-84, Vol.\ XII, 568--578.       
%\bibitem{Weizsaecker} Carl Friedrich von Weizs\"acker; {\em Der zweite
%      Hauptsatz und der Unterschied von Vergangenheit und Zukunft};
%      Annalen der Physik 36 (1939) 275--283.       
%\bibitem{Zeh} Zeh, H.D.; {\em The Physical Basis of the Direction of Time},
%      Springer-Verlag, Berlin, 1989.       

%\bibitem{Einstein} Einstein, Albert; {\em ??}, .                   
\end{thebibliography}

\end{document}
