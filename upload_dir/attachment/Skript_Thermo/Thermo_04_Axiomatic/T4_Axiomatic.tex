%\documentclass[german,10pt]{book}      
\usepackage{makeidx}
\usepackage{babel}            % Sprachunterstuetzung
\usepackage{amsmath}          % AMS "Grundpaket"
\usepackage{amssymb,amsfonts,amsthm,amscd} 
\usepackage{mathrsfs}
\usepackage{rotating}
\usepackage{sidecap}
\usepackage{graphicx}
\usepackage{color}
\usepackage{fancybox}
\usepackage{tikz}
\usetikzlibrary{arrows,snakes,backgrounds}
\usepackage{hyperref}
\hypersetup{colorlinks=true,
                    linkcolor=blue,
                    filecolor=magenta,
                    urlcolor=cyan,
                    pdftitle={Overleaf Example},
                    pdfpagemode=FullScreen,}
%\newcommand{\hyperref}[1]{\ref{#1}}
%
\definecolor{Gray}{gray}{0.80}
\DeclareMathSymbol{,}{\mathord}{letters}{"3B}
%
\newcounter{num}
\renewcommand{\thenum}{\arabic{num}}
\newenvironment{anmerkungen}
   {\begin{list}{(\thenum)}{%
   \usecounter{num}%
   \leftmargin0pt
   \itemindent5pt
   \topsep0pt
   \labelwidth0pt}%
   }{\end{list}}
%
\renewcommand{\arraystretch}{1.15}                % in Formeln und Tabellen   
\renewcommand{\baselinestretch}{1.15}                 % 1.15 facher
                                                      % Zeilenabst.
\newcommand{\Anmerkung}[1]{{\begin{footnotesize}#1 \end{footnotesize}}\\[0.2cm]}
\newcommand{\comment}[1]{}
\setlength{\parindent}{0em}           % Nicht einruecken am Anfang der Zeile 

\setlength{\textwidth}{15.4cm}
\setlength{\textheight}{23.0cm}
\setlength{\oddsidemargin}{1.0mm} 
\setlength{\evensidemargin}{-6.5mm}
\setlength{\topmargin}{-10mm} 
\setlength{\headheight}{0mm}
\newcommand{\identity}{{\bf 1}}
%
\newcommand{\vs}{\vspace{0.3cm}}
\newcommand{\noi}{\noindent}
\newcommand{\leer}{}

\newcommand{\engl}[1]{[\textit{#1}]}
\parindent 1.2cm
\sloppy

         \begin{document}  \setcounter{chapter}{2}

\chapter{Axiomatische Zug\"ange zur Thermodynamik}
% Kap 2
\label{chap_Entropie_Axiom}

\info{Thomas Filk}{30.03.2024}%
Dieser Kurztext beschreibt skizzenhaft die axiomatischen Ans\"atze
von Constantin Carath\'{e}odory (1873--1950) \cite{Caratheodory},\index{Carath\'{e}odory, Constantin} 
Josef-Maria Jauch (1914--1974) \cite{Jauch}\index{Jauch,Josef-Maria} 
sowie Elliott Lieb und Jakob Yngvason \cite{Lieb1,Lieb2})
zur Thermodynamik und speziell zur Definition der Entropie. Der Inhalt dieses Kurztextes geht
weit \"uber das hinaus, was man als Lehrkraft in der Schule zur Thermodynamik wissen sollte, und
selbst in einem normalen Physikstudium wird man diesen axiomatischen Zug\"angen selten
begegnen.

F\"ur mathematisch interessierte Physikerinnen und Physiker ist jedoch gerade die Thermodynamik
oftmals etwas unbefriedigend. Einerseits wird die Thermodynamik gerne als sehr allgemeine
ph\"anomenologische Theorie bezeichnet, die in allen physikalischen Modellen erf\"ullt sein sollte.
Andererseits wird selten deutlich gemacht, worin die entscheidenden mathematischen Annahmen
hinter dieser Theorie bestehen. Daher gibt dieser Kurztext einen groben Einblick in axiomatische
Zug\"ange zur Thermodynamik. Auch wenn die wichtigsten Begriffe nochmals kurz erl\"autert
werden, setzt dieser Kurztext einerseits die Kenntnis des 2.\ Hauptsatzes sowie des
Carnot-Prozesses voraus (siehe \glqq \hyperref[chap_ZweiterHS]{Der Zweite Hauptsatz der Thermodynamik})\grqq, 
andererseits sollten auch die Konzepte der Zustandsgr\"o\ss en und
Prozessgr\"o\ss en sowie der Unterschied zwischen einer exakten 1-Form und einer allgemeinen 1-Form
(siehe \glqq \hyperref[chap_Zustand]{Zustands- und Prozessgr\"o\ss en}\grqq) bekannt sein. 
 
\section{Axiomatische Definitionen der Entropie}
\label{sec_AxiomeEntropie}

Der Carnot-Prozess zeigt, dass es zu der Prozessgr\"o\ss e (1-Form) W\"arme $\delta Q$ 
einen integrierenden Faktor $T$ gibt, sodass $\delta Q=T{\rm d}S$. Die meisten axiomatischen Zug\"ange
(z.B.\ Carath\'{e}odory und Jauch \cite{Caratheodory,Jauch}) leiten diese Tatsache aus rein mathematischen
\"Uberlegungen her, wobei eine genaue Analyse zeigt, dass sie das Konzept des Carnot-Prozesses
mathematisieren und dann beweisen, dass die 1-Form \glqq W\"arme\grqq\ einen integrierenden
Faktor besitzt und dass dieser integrierende Faktor nur von einer empirischen Temperatur
$\theta$ abh\"angt. Sie definieren dann die absolute Temperatur $T$ als diesen integrierenden
Faktor. 

\subsection{Die geometrische Bedeutung eines integrierenden Faktors}

In diesem Abschnitt geht es um einige Beziehungen\index{Integrierender Faktor} 
zwischen 1-Formen und 0-Formen. N\"aheres
zu diesen Formen sowie zu der hier verwendeten Notation 
findet man in dem Kurztext \glqq \hyperref[chap_Zustand]{Zustands- und Prozessgr\"o\ss en}\grqq. Hier
werden nur die wichtigsten Definitionen angegeben. 

Die Menge der Gleichgewichtszust\"ande\index{Gleichgewichtszustand}
in der Thermodynamik l\"asst sich als Mannigfaltigkeit $M$ darstellen, d.h., sie l\"asst sich lokal durch
Koordinaten $\pmb{x}=(x_0,x_1,...,x_n)$ beschreiben. Bei diesen Koordinaten handelt es sich um einen Satz von 
unabh\"angigen Zustandsgr\"o\ss en, wobei eine Koordinate ($x_0$) die innere Energie sein soll und die
anderen Koordinaten ein Satz von Gr\"o\ss en, durch welche die Arbeit des Systems an der Umgebung
kontrolliert werden kann (z.B.\ das Volumen, die Stoffmengen, Ladungsmengen, etc.), daher bezeichnet man
diese Koordinaten auch als Arbeitskoordinaten. Eine\index{Zustandsgr\"o\ss e}\index{0-Form} 
Zustandsgr\"o\ss e oder auch 0-Form ist eine Observable, die an einem Gleichgewichtszustand gemessen 
werden kann, d.h., es handelt sich um eine Funktion $f:M\rightarrow \mathbb{R}$ (bzw.\ in Koordinaten
$f:U\subset \mathbb{R}^n \rightarrow \mathbb{R}$; ich verwende hier dasselbe Symbol). 
Unter einem\index{Prozess}
Prozess bzw.\ einer Zustands\"anderung verstehen wir einen Weg $\gamma: [0,1] \rightarrow M$ (in Koordinaten
$t \mapsto \pmb{x}(t)=(x_0(t),...,x_n(t))$) auf der Mannigfaltigkeit der Gleichgewichtszust\"ande. Damit
dieser Weg tats\"achlich auf dieser Mannigfaltigkeit verl\"auft, muss die Zustands\"anderung reversibel 
erfolgen, d.h., sie muss quasistation\"ar sein (zu jedem Zeitpunkt muss sich das System in einem Gleichgewichtszustand
befinden, d.h., der Prozess verl\"auft beliebig langsam) und durch minimale \"Anderungen in der Umgebung
muss der Prozess umkehrbar sein. Man \"uberlege sich, dass jeder solche Weg auf der Menge der
Gleichgewichtszust\"ande tats\"achlich durch einen reversiblen Prozess physikalische realisiert
werden kann.

Das totale Differential\index{totales Differential}\index{1-Form}\index{Prozessgr\"o\ss e}
${\rm d}f$ (oder auch der Gradient $\pmb{\nabla}f$) ist eine 1-Form und damit eine sogenannte Prozessgr\"o\ss e.
Das bedeutet, es handelt sich um ein Feld, das an jedem Punkt $\pmb{x}$ auf einen Tangentenvektor 
(die \glqq Geschwindigkeit\grqq\ $\dot{\pmb{x}}$ eines Wegs $\gamma:[0,1] \rightarrow M; t \mapsto \pmb{x}(t)$) an 
diesem Punkt angewandt werden muss. Die so erhaltene Funktion l\"asst sich entlang des Weges  integrieren:
\begin{equation}
         f(\pmb{x}) - f(\pmb{x}_0) =  \int_\gamma {\rm d}f = \int_0^1 {\rm d}f_{(\pmb{x}(t))}( \dot{\pmb{x}})\,{\rm d}t \, .
\end{equation}
Das Differential ${\rm d}f_{(\pmb{x})}$ definiert an jedem Punkt $\pmb{x}\in M$ eine Hyperfl\"ache (einen Vektorraum,
der eine Dimension weniger als $M$ hat, also die Kodimension 1):
Alle Tangentialvektoren $\dot{\pmb{x}}$ in $T_{\pmb{x}}M$ (dem Tangentialraum an $M$ im Punkte $\pmb{x}$, der
durch die m\"oglichen Geschwindigkeiten von Wegen durch diesen Punkt gegeben ist), 
f\"ur die ${\rm d}f_{(\pmb{x}(t))}( \dot{\pmb{x}})=0$ gilt, liegen in dieser Hyperebene, d.h.\ es handelt sich bei dieser 
Hyperebene um den Tangentialraum an 
eine \"Aquipotentialfl\"ache von $f$.\footnote{Oft stellt man sich ${\rm d}f_{(\pmb{x})}$ als ein Vektorfeld
vor, das an jedem Punkt dem Gradienten von $f$ entspricht. Hierbei sollte man jedoch beachten, dass es
sich dabei um eine lineare Abbildung (also ein Element des Dualraums) auf dem Tangentenraum an $M$ -- also
dem Raum aller \glqq Geschwindigkeiten von Wegen\grqq\ in einem Punkt $\pmb{x}\in M$ -- handelt.} 
Ausgedr\"uckt in Koordinaten hat ${\rm d}f$ die Form:
\begin{equation}
           {\rm d}f_{(\pmb{x})} = \sum_{i=1}^n \frac{\partial f(\pmb{x})}{\partial x_i}\, {\rm d} x_i \, ,
\end{equation}
wobei ${\rm d}x_i$ die Differentiale der Koordinatenfunktion $x_i:M\rightarrow \mathbb{R}$ sind. Diese
sollten nicht als \glqq kleine Inkremente\grqq\ interpretiert werden, sondern als die Abbildung, die einem
Tangentialvektor (an einem Punkt $\pmb{x}$) seine $i$-te Komponente zuordnet. Es ist also
\begin{equation}
                  {\rm d}x_i (\dot{\pmb{x}}) = \dot{x}_i \, . 
\end{equation}
Eine allgemeine 1-Form auf $M$ hat (ausgedr\"uckt in lokalen Koordinaten) die Form
\begin{equation}
           \omega_{(\pmb{x})} = \sum_{i=1}^n \omega_i \, {\rm d} x_i \, .
\end{equation}
Auch eine solche 1-Form -- anschlaulich ein Vektorfeld mit Komponenten $\omega_i$ --
definiert an jedem Punkt $\pmb{x}$ von $M$ eine Hyperebene durch die Bedingung
$\omega_{(\pmb{x})} (\dot{\pmb{x}}) = 0$. Dies f\"uhrt auf die Frage: Unter welchen Bedingungen
lassen sich diese Hyperebenen in einer Umgebung eines Punktes $\pmb{x}$ als Tangentialr\"aume
an die \"Aquipotentialfl\"achen einer Funktion $f$ deuten? Ist $\omega$ das
totale Differential zu einer Funktion $f$, wenn also $\omega = {\rm d}f$ gilt -- in diesem Fall bezeichnet
man $\omega$ als \textit{exakt} --, ist das der Fall. Eine solche Funktion $f$ gibt es, wenn $\omega$
rotationsfrei ist, d.h. wenn gilt:
\begin{equation}
                                  \frac{\partial \omega_i}{\partial x_j} =  \frac{\partial \omega_j}{\partial x_i} \, .
\end{equation}
(In einem Differentialformenkalk\"ul kann man daf\"ur auch ${\rm d}\omega =0$ schreiben.) 

Die Bedingung $\omega_{(\pmb{x})} (\dot{\pmb{x}}) = 0$, durch welche bei gegebenem $\omega$ die
Hyperebene definiert ist, erlaubt aber eine gewisse Freiheit: Wenn $\omega$ an jedem Punkt mit einem
beliebigen (von null verschiedenen) Faktor multipliziert wird, \"andert sich diese Bedingung f\"ur die
Hyperebenen nicht. Es wird lediglich die \glqq L\"ange\grqq\ der Normalenvektoren an die Ebene
ge\"andert, das \"andert aber die Ebene selbst nicht. Also gilt auch: Wenn es eine Funktion
$\alpha(\pmb{x})$ gibt, sodass $\alpha(\pmb{x}) \omega_{(\pmb{x})}$ rotationsfrei ist, dann gibt es
(in einem zusammenh\"angenden Gebiet) eine Funktion $f$, sodass $\alpha \omega= {\rm d}f$. Die
Funktion $\alpha(\pmb{x})$ bezeichnet man als \glqq integrierenden Faktor\grqq. Die Gleichung
$\omega(\dot{\pmb{x}})=0$ hat dieselben L\"osungen wie $\alpha(\pmb{x}) \omega(\dot{\pmb{x}})=0$ 
bzw.\ ${\rm d}f(\dot{\pmb{x}})=0$ (sofern es diese Funktion $f$ gibt), 
und dies sind die Vektoren der Tangentialr\"aume an die
\"Aquipotentialfl\"achen von $f$. 

Gibt es eine Funktion $f$, sodass $\alpha \omega ={\rm d}f$, ist nat\"urlich auch jede Funktion 
$F(f)$ eine L\"osung des Problems, da die \"Aquipotentialfl\"achen dieselben sind. 
Da ${\rm d}F(f)= F'(f) {\rm d}f$, \"andert sich entsprechend der integrierende Faktor um
$ \alpha \rightarrow \alpha F'(f)$. Die L\"osung des Problems ist somit nicht eindeutig. 

Die Beziehung zur Thermodynamik ist nun folgende: Die W\"arme\index{Waerme@W\"arme} 
$\delta Q$ (in der Mathematik w\"urde
man eher $\omega_Q$ schreiben) ist eine 1-Form ebenso wie das Differential ${\rm d}U$ 
der inneren Energie U. Allerdings ist ${\rm d}U$ eine exakte 1-Form -- es gibt die Zustandsfunktion 
(0-Form) der inneren Energie $U$ --, wohingegen $\delta Q$ keine exakte 1-Form ist. 
Allerdings kann man beweisen -- und dies tun der Carnot-Prozess wie auch die axiomatischen 
Ans\"atze von Carath\'{e}odory und Jauch --, dass es einen integrierenden Faktor zu $\delta Q$ gibt, 
n\"amlich $1/T$, wobei $T$ als absolute Temperatur definiert wird, sodass es eine 
Zustandsgr\"o\ss e -- die Entropie $S$ -- gibt, sodass $\delta Q/T = {\rm d}S$. 
  
\subsection{Die Arbeit von Constantin Carath\'{e}odory}

Die Arbeit von Constantin Carath\'{e}odory von 1913 \cite{Caratheodory} wird heute oftmals als der
erste axiomatische Zugang zur Thermodynamik angesehen, bei dem der zweite Hauptsatz der
Thermodynamik -- die Existenz einer Zustandsgr\"o\ss e Entropie sowie einer absoluten Temperatur --
bewiesen wird, ohne dass das sehr unscharf definierte Konzept der W\"arme explizit verwendet wird.  
Der erste Hauptsatz der Thermodynamik -- bei Carath\'{e}odory ist dies das Axiom I -- wird
so umgeformt, dass die W\"arme $\delta Q$ durch die besser definierten Gr\"o\ss en innere Energie $U$ 
bzw.\ das totale Differential ${\rm d}U$ und die Arbeit $\delta W$ ausgedr\"uckt wird. Lokal 
wird dadurch postuliert, dass\index{Erster Hauptsatz der Thermodynamik}\index{Energieerhaltung}
\begin{equation}
                          \delta Q = {\rm d}U + \delta W \, .
\end{equation}
Prozesse bzw.\ Zustands\"anderungen sind auch bei Carath\'{e}odory reversible Wege $\pmb{x}(t)$ 
auf der Mannigfaltigkeit $M$ der Gleichgewichtszust\"ande. Zun\"achst sehr unanschaulich 
ist das Axiom II bei Carath\'{e}odory:\\[0.3cm]
\textit{In jeder beliebigen Umgebung eines willk\"urlich vorgeschriebenen Anfangszustands
gibt es Zust\"ande, die durch adiabatische Zustands\"anderungen nicht beliebig approximiert
werden k\"onnen.}
\vspace{0.3cm}

Eine adiabatische Zustands\"anderung -- beschrieben durch einen Tangentialvektor $\dot{\pmb{x}}$ --
erf\"ullt dabei die Bedingung $\delta Q(\dot{\pmb{x}})=0$.\index{adiabatisch} 
Bei einer solchen Zustands\"anderung flie\ss t
also keine W\"arme bzw.\ findet kein W\"armeaustausch mit der Umgebung statt. Die Forderung, dass es 
solche Zustands\"anderungen \"uberhaupt gibt, 
setzt als empirische Tatsache die Existenz von adiabatisch abgeschlossenen W\"anden voraus, sodass man
experimentell eine reversible Zustands\"anderung realisieren kann, bei der keine W\"arme mit der Umgebung
ausgetauscht wird. 

Carath\'{e}odory
betrachtet nun adiabatische Wege (Zustands\"anderungen) $\pmb{x}(t)$, ausgehend von einem Anfangszustand
$\pmb{x}_0$ zu einem beliebigen Zustand $\pmb{x}$, f\"ur die gilt
\begin{equation}
                          \delta Q_{(\pmb{x}(t))}(\dot{\pmb{x}}(t))  = 0 \, . 
\end{equation}
Wenn es zu $\delta Q$ einen integrierenden Faktor $\alpha(\pmb{x})$ und eine Funktion $S(\pmb{x})$
gibt, so dass $\alpha(\pmb{x}) \delta Q = {\rm d}S$, bedeutet dies, dass ein solcher Weg
(eine solche Zustands\"anderung) nur innerhalb einer \"Aquipotentialfl\"ache zu dieser Funktion $S$ verlaufen
kann. Dies bedeutet wiederum, dass bestimmte Punkte in einer beliebig kleinen Umgebung eines
Gleichgewichtszustands $\pmb{x}$ -- n\"amlich solche mit einem anderen Funktionswert f\"ur $S$ als der Punkt $\pmb{x}$ --
durch adiabatische Wege nicht erreicht werden k\"onnen. Nach einem Theorem von \'{E}lie Cartan
gilt aber auch das Umgekehrte: Wenn es in der Umgebung eines Punktes $\pmb{x}$ andere
Punkte gibt, die durch Wege, entlang derer $\omega(\dot{\pmb{x}})=0$ ist, nicht erreicht werden
k\"onnen, dann gibt es eine Funktion $S(\pmb{x})$, sodass diese Wege auf einer Hyperfl\"ache 
$S(\pmb{x})={\rm const}$ verlaufen. Mit anderen Worten: Wenn $\omega$ keinen integrierenden Faktor
besitzt, dann kann man mit Wegen, f\"ur die $\omega(\dot{\pmb{x}})=0$ gilt, jeden Punkt in einer
Umgebung eines Punkte $\pmb{x}$ beliebig approximieren. 

Damit hat Carath\'{e}odory gezeigt, dass es zu der 1-Form W\"arme einen integrierenden Faktor gibt,
so dass diese 1-Form zu einem totalen Differential einer Zustandsgr\"o\ss e $S$ wird, die wir dann
Entropie nennen. Es verbleibt zu zeigen, dass dieser integrierende Faktor lediglich eine Funktion der
sogenannten \textit{empirischen Temperatur} $\theta$ sein kann. Au\ss erdem hatten wir gesagt, dass
die Funktion $S$ nicht eindeutig ist, sofern sie existiert. Allerdings kann man zeigen, dass unter den
Funktionen $F(S)$, die m\"oglich w\"aren, nur eine (bis auf einen konstanten Faktor) die Eigenschaft
hat, additiv zu sein. D.h., wenn man zwei Systeme zu einem Gesamtsystem vereint, soll die
Entropie $S$ additiv (in der Thermodynamik spricht man meist von einer mengenartigen Gr\"o\ss e
oder auch einer extensiven Grr\"o\ss e)\index{extensiv} 
sein: $S_{\rm ges} = S_1 + S_2$. Dies legt sowohl die Funktion $S$ als auch
den integrierenden Faktor bis auf einen konstanten Faktor fest. Von diesem kann man nun zeigen,
dass er nur von der Temperatur abh\"angt, denn wenn der integrierende Faktor eine intensive
Zustandsgr\"o\ss e sein soll, kann er nicht mehr von einer extensiven Gr\"o\ss e (der Entropie) 
abh\"angen. 

\subsection{Der Zugang von Joseph-Maria Jauch}

In seinem Artikel von 1970 \cite{Jauch} geht Joseph-Maria Jauch ganz \"ahnlich vor, wie 
Carathh\'{e}odory. Durch ein erstes Axiom wird die Energieerhaltung postuliert, sodass es
eine 1-Form (genannt W\"arme) gibt, die sich als
\begin{equation}
                 \delta Q ={\rm d}U + \delta W 
\end{equation}
schreiben l\"asst, wobei vorausgesetzt wird, dass $U$ -- die innere Energie -- eine an einem
Gleichgewichtszustand messbare Gr\"o\ss e -- also eine Zustandsgr\"o\ss e -- darstellt, 
die bei der Verwendung von Arbeitskoordinaten mit $x_0$ bezeichnet wird,
und dass $\delta W$ die von dem System an der Umgebung geleistete Arbeit bezeichnet, die
durch die Arbeitskoordinaten $(x_1,...,x_n)$ kontrolliert werden kann. 

Jauch betrachtet nun reversible adiabatische Wege, also wiederum Wege, f\"ur die
$\delta Q(\dot{\pmb{x}}(t))=0$ ist. Er beweist das folgende Theorem: Ist ein adiabatischer Weg
in den Arbeitskoordinaten $(x_1,...x_n)$ geschlossen, dann ist er auch in $M$ geschlossen, d.h.\
in den Koordinaten $(x_0,x_1,...,x_n)$. Dies ist nicht trivial: Es k\"onnte adiabatische \glqq Spiralen\grqq\ 
geben, sodass in $(x_1,...,x_n)$ ein Weg geschlossen ist, dieser aber in der verbliebenen Koordinate
$x_0$ einen anderen Wert annimmt. 

Wir betrachten also nun einen Weg $\gamma$, der bez\"uglich der Arbeitskoordinaten $(x_1,...,x_n)$
geschlossen ist, und f\"ur den $\delta Q(\dot{\pmb{x}})=0$. Insbesondere gilt f\"ur diesen Weg
\begin{equation}
      \int_{\gamma} \delta Q = 0 \, .
\end{equation}      
Damit m\"usste f\"ur diesen Weg und die Koordinate $x_0$ (die der inneren Energie $U$ entspricht) 
gelten:      
\begin{equation}      
               x_0(1) - x_0(0) =  \Delta U  =  \int_{\gamma} \delta W \, .
\end{equation}
Da die Arbeitskoordinaten $(x_1,...,x_n)$ aber am Ende des Weges wieder dieselben Werte wie
zu Beginn annehmen, kann die Energie $\Delta U$ dem System nur zugef\"ugt worden sein, indem
bei dem Prozess diese Energie aus einem W\"armereservoir entnommen wurde. Doch dies widerspricht
dem zweiten Hauptsatz in der Formulierung von Planck: Es gibt keine periodisch arbeitende Maschine
(die Arbeitskoordinaten sind nach einem Zyklus dieselben wie vorher), deren einziger Effekt darin besteht, 
dass ein Gewicht angehoben wird und sich ein W\"armereservoir abgek\"uhlt hat.

Das bedeutet aber, dass ein adiabatischer Weg, f\"ur den somit $\int_{\gamma} \delta Q = 0$, und
der in den Arbeitskoordinaten geschlossen ist, automatisch insgesamt geschlossen ist. 


\subsection{Lieb und Yngvason: Entropie als Ma\ss\ f\"ur Irreversibilit\"at}

In einigen j\"ungeren Arbeiten (z.B.\ \cite{Lieb1,Lieb2}) haben Lieb und Yngvason versucht, den
Begriff der Entropie direkt als Ma\ss\ f\"ur Irreversibilit\"at herzuleiten.\index{Entropie!Ma\ss\ f\"ur Irreversibilit\"at} 
Dazu betrachten sie wieder den\index{Gleichgewichtszustand}
Raum aller Gleichgewichtszust\"ande und definieren darauf eine Halbordnung: Wenn ein
Prozess f\"ur zwei Gleichgewichtszust\"ande $A$ und $B$ spontan abl\"auft (nicht notwendigerweise 
reversibel oder \"uber eine Folge vonGleichgewichtszust\"anden), definieren sie $A\rightarrow B$. 
Wenn es Prozesse gibt, die in beide Richtungen ablaufen k\"onnen, gelte $A \leftrightarrow B$.
Die Relation $A\rightarrow B$ ist transitiv, d.h., wenn gilt $A\rightarrow B$ und $B \rightarrow C$, gilt
auch $A \rightarrow C$. Auch die Relation $A \leftrightarrow B$ ist transitiv (und symmetrisch). 
Das bedeutet: Es gibt auf der Menge der Gleichgewichtszust\"ande eine \"Aquivalenzrelation
($\leftrightarrow$) und auf der Menge der \"Aquivalenzklassen eine Halbordnung. 

Lieb und Yngvason definieren nun auf der Menge der Menge der Gleichgewichtszust\"ande eine
Funktion $\sigma$, die innerhalb einer \"Aquivalenzklasse konstant ist und die die Halbordnung respektiert, 
d.h., falls $A \rightarrow B$ gilt, soll auch $\sigma(A) \leq \sigma(B)$ gelten, und falls $A \leftrightarrow B$
gilt, soll $\sigma(A) = \sigma(B)$ gelten. Diese Funktion $\sigma$ k\"onnte man als empirische
Entropie interpretieren. 

Auch diese Funktion $\sigma$ ist nicht eindeutig, da jede monotone Funktion von $\sigma$ ebenfalls
die geforderten Eigenschaften hat. \"Ahnlich wie Carath\'{e}odory und Jauch nutzen die Autoren nun
bestimmte Forderungen an die Funktion $\sigma$, die im Wesentlichen
aus der Additivit\"at bzw.\ der Skalierungseigenschaft (also der Forderung, dass die Entropie eine
extensive Zustandsgr\"o\ss e sein soll)
folgen, um $\sigma$ bis auf einen konstanten
Faktor festzulegen. Die so erhaltene Funktion $S$ definieren sie als die Entropie eines Systems. 

\begin{thebibliography}{99}
\bibitem{Caratheodory} Carath\'{e}odory, C.; \textit{Untersuchungen \"uber die Grundlagen der
          Thermodynamik}; Mathematische Annalen, Bd.\ 67 (1909) 355--386.
\bibitem{Carnot} Carnot, S.; \textit{R\'{e}flexions sur la puissance motrice du feu et sur les machines propres \`{a} d\'{e}velopper cette
               puissance}; Annales scientifiques de l'\'{E}cole Normale Sup\'{e}ieure S\'{e}r.\ 2 (1872), 393--457.\\Deutsch:
               \textit{Betrachtungen \"uber die bewegende Kraft des Feuers und die zur Entwickelung dieser Kraft geeigneten Maschinen} (1824);
               \"ubersetzt und herausgegeben von W.\ Ostwald, in Ostwald's Klassiker der exakten Wissenschaften, Nr.\ 37.        
\bibitem{Jauch} Jauch, J.-M.; \textit{Analytical Thermodynamics. Part I. Thermostatistics---General Theory};
               Found.\ Phys.\ 5, No.\ 1 (1975), 111--132; Verlag von Wilhelm Engelmann, Leibzig, 1892.
\bibitem{Lieb1} Lieb, E.\,H., Yngvason, J.; \textit{The Physics and Mathematics of the Second Law of
            Thermodynamics}; Phys.\ Rep.\ 310 (1999) 1--96.
\bibitem{Lieb2} Lieb, E.\,H., Yngvason, J.; \textit{The Mathematical Structure of the Second Law of
            Thermodynamics}; Current Developments in Mathematics, 2001; International Press, Cambridge (2002)
            89--130.
\bibitem{Sommerfeld} Sommerfeld, A.; \textit{Vorlesungen \"uber theoretische Physik; Band 5: Thermodynamik und
             Statistik}; Verlag Harri Deutsch, Frankfurt (1992).                                                                           
\end{thebibliography}


%\end{document}

