\documentclass[german,10pt]{book}      
\usepackage{makeidx}
\usepackage{babel}            % Sprachunterstuetzung
\usepackage{amsmath}          % AMS "Grundpaket"
\usepackage{amssymb,amsfonts,amsthm,amscd} 
\usepackage{mathrsfs}
\usepackage{rotating}
\usepackage{sidecap}
\usepackage{graphicx}
\usepackage{color}
\usepackage{fancybox}
\usepackage{tikz}
\usetikzlibrary{arrows,snakes,backgrounds}
\usepackage{hyperref}
\hypersetup{colorlinks=true,
                    linkcolor=blue,
                    filecolor=magenta,
                    urlcolor=cyan,
                    pdftitle={Overleaf Example},
                    pdfpagemode=FullScreen,}
%\newcommand{\hyperref}[1]{\ref{#1}}
%
\definecolor{Gray}{gray}{0.80}
\DeclareMathSymbol{,}{\mathord}{letters}{"3B}
%
\newcounter{num}
\renewcommand{\thenum}{\arabic{num}}
\newenvironment{anmerkungen}
   {\begin{list}{(\thenum)}{%
   \usecounter{num}%
   \leftmargin0pt
   \itemindent5pt
   \topsep0pt
   \labelwidth0pt}%
   }{\end{list}}
%
\renewcommand{\arraystretch}{1.15}                % in Formeln und Tabellen   
\renewcommand{\baselinestretch}{1.15}                 % 1.15 facher
                                                      % Zeilenabst.
\newcommand{\Anmerkung}[1]{{\begin{footnotesize}#1 \end{footnotesize}}\\[0.2cm]}
\newcommand{\comment}[1]{}
\setlength{\parindent}{0em}           % Nicht einruecken am Anfang der Zeile 

\setlength{\textwidth}{15.4cm}
\setlength{\textheight}{23.0cm}
\setlength{\oddsidemargin}{1.0mm} 
\setlength{\evensidemargin}{-6.5mm}
\setlength{\topmargin}{-10mm} 
\setlength{\headheight}{0mm}
\newcommand{\identity}{{\bf 1}}
%
\newcommand{\vs}{\vspace{0.3cm}}
\newcommand{\noi}{\noindent}
\newcommand{\leer}{}

\newcommand{\engl}[1]{[\textit{#1}]}
\parindent 1.2cm
\sloppy

         \begin{document}  \setcounter{chapter}{1}

%\newcommand{\solution}[1]{}

\chapter{Der zweite Hauptsatz der Thermodynamik}
% Kap x
\label{chap_ZweiteHS}

Die \textit{thermodynamische Entropie} geh\"ort erfahrungsgem\"a\ss\ 
zu den schwierigen Konzepten in der Physik, sowohl f\"ur Sch\"ulerinnen und Sch\"uler als
auch f\"ur Studierende. Als
Zustandsgr\"o\ss e bleibt sie vergleichsweise unanschaulich. Sie hat ihren Ursprung
in der wichtigen Arbeit von Nicolas L\'{e}onard Sadi Carnot (1796--1832) von 1824 \cite{Carnot}, 
und auch heute noch dient der Carnot-Prozesses h\"aufig der Herleitung dieser Zustandsgr\"o\ss e
und ihrem Bezug zum zweiten Hauptsatz der Thermodynamik. In der heutigen Form 
erfasst wurde dieser Begriff allerdings erst von Rudolf Clausius (1822--1888), der 
dieses Konzept 1854 zum ersten Mal formulierte und ihm 1865 die heutige Bezeichnung \glqq Entropie\grqq\ sowie
die heutige thermodynamische Formulierung gab. Die statistische Formulierung der Entropie
in Form der \textit{Boltzmann-Entropie} (siehe der Kurztext 
\glqq \hyperref[chap_Entropie]{Entropie -- Statistische Zug\"ange}\grqq\ gegen Ende des 19.\ Jahrhunderts 
gab ihr zwar eine gewisse Anschauung, allerdings ist der Zusammenhang zur thermodynamischen Entropie
gerade in der Schule schwierig zu vermitteln. 

Ausgangspunkt f\"ur den thermodynamischen Entropiebegriff ist der zweite Hauptsatz der
Thermodynamik. Hier gibt es einfache Formulierungen (z.B.\ von Clausius und von Kelvin bzw.\ Planck),
die aber wenig unmittelbare Schlussfolgerungen erlauben, und komplexere Formulierung, die zwar
ergiebig sind aber meist auch sehr unanschaulich (in diesem Kurztext verwende ich die Formulierung
von Sommerfeld \cite{Sommerfeld}). Der Carnot-Prozess zeigt die Gleichwertigkeit dieser
Definitionen. In einem abschlie\ss enden Kapitel beschreibe ich skizzenhaft die axiomatischen Ans\"atze
von Carath\'{e}odory \cite{Caratheodory}, Jauch \cite{Jauch} sowie Lieb und Yngvason \cite{Lieb1,Lieb2}). 

\section{Der zweite Hauptsatz der Thermodynamik}

Der zweite Hauptsatz der Thermodynamik besitzt viele Formulierungen, von denen einige
sehr einfach und unmittelbare Erfahrungstatsachen sind, andere hingegen eher 
komplex aber geeignet, um weitreichende Schlussfolgerungen daraus ziehen zu k\"onnen.
Das Ziel des Carnot-Prozesses oder auch der axiomatischen Zug\"ange von Carath\'{e}odory und
Jauch ist, die komplexere aber reichhaltigere Aussage des zweiten Hauptsatzes auf die
einfacheren Aussagen zur\"uckzuf\"uhren, d.h., deren \"Aquivalenz zu zeigen. Die bekanntesten
einfachen Formulierungen des zweiten Hauptsatzes sind die Formulierung von 
Clausius sowie die Formulierung von Kelvin bzw.\ Planck. 

\subsection{Der zweite Hauptsatz in der Formulierung von Clausius und Kelvin}

Die Formulierungen des zweiten Hauptsatzes der Thermodynamik von Clausius und
Kelvin, wobei die Formulierung von Kelvin erst von Max Planck in die hier verwendete Form
gebracht wurde, geh\"oren zu den unmittelbar einsichtigen Formen des zweiten Hauptsatzes. 
\vspace{0.3cm}

\noindent
Die Formulierung von Clausius lautet:\\
Es gibt keinen Prozess, bei dem W\"arme von einem k\"alteren System in ein w\"armeres
System flie\ss t, ohne dass weitere Ver\"anderungen aufgetreten sind. 
\vspace{0.3cm}

Gemeint ist damit, dass es kein System $M$ gibt, das der Umgebung Energie in Form von W\"arme oder
Arbeit (also W\"armeenergie, elektrische Energie, chemische Energie, ...) entzieht, diese dazu nutzt, aus
einem k\"alteren System mit der Temperatur $T_2$ W\"arme in ein w\"armeres Reservoir mit der Temperatur
$T_1>T_2$ zu pumpen, und anschlie\ss end die aufgenommene Energie in derselben Form und Menge
wieder an die Umgebung abgibt. Letztendlich soll sich nichts ge\"andert haben, au\ss er dass das
k\"altere System noch k\"alter geworden und das w\"armere System noch w\"armer geworden ist (vgl.\
Abb.\ \ref{fig_ZweiterHauptsatz}).

\begin{figure}[htb]
\begin{picture}(170,150)(-20,0)
\put(-10,140){\makebox(0,0){(a)}}
\put(10,10){\line(1,0){120}}
\put(10,10){\line(0,1){20}}
\put(10,30){\line(1,0){120}}
\put(130,10){\line(0,1){20}}
\put(10,120){\line(1,0){120}}
\put(10,120){\line(0,1){20}}
\put(10,140){\line(1,0){120}}
\put(130,120){\line(0,1){20}}
\put(70,55){\line(1,0){40}}
\put(70,55){\line(0,1){40}}
\put(70,95){\line(1,0){40}}
\put(110,55){\line(0,1){40}}
%
\put(80,30){\line(0,1){15}}
\put(100,30){\line(0,1){15}}
\put(75,45){\line(1,0){5}}
\put(100,45){\line(1,0){5}}
\put(75,45){\line(3,2){15}}
\put(105,45){\line(-3,2){15}}
\put(80,95){\line(0,1){15}}
\put(100,95){\line(0,1){15}}
\put(75,110){\line(1,0){5}}
\put(100,110){\line(1,0){5}}
\put(75,110){\line(3,2){15}}
\put(105,110){\line(-3,2){15}}
%
\put(30,60){\line(1,0){30}}
\put(60,55){\line(1,1){10}}
\put(60,75){\line(1,-1){10}}
\put(60,55){\line(0,1){5}}
\put(60,70){\line(0,1){5}}
\put(30,70){\line(1,0){30}}
\put(40,80){\line(1,0){30}}
\put(30,85){\line(1,1){10}}
\put(30,85){\line(1,-1){10}}
\put(40,75){\line(0,1){5}}
\put(40,90){\line(0,1){5}}
\put(40,90){\line(1,0){30}}
%
\put(70,20){\makebox(0,0){$T_2$}}
\put(70,130){\makebox(0,0){$T_1>T_2$}}
\put(90,75){\makebox(0,0){$M$}}
\put(50,65){\makebox(0,0){${\scriptstyle Q_2,W}$}}
\put(50,85){\makebox(0,0){${\scriptstyle Q_2,W}$}}
\put(90,40){\makebox(0,0){$Q_1$}}
\put(90,105){\makebox(0,0){$Q_1$}}
\end{picture}
\hspace{1cm}
%%
\begin{picture}(150,150)(0,0)
\put(0,140){\makebox(0,0){(b)}}
\put(10,10){\line(1,0){120}}
\put(10,10){\line(0,1){20}}
\put(10,30){\line(1,0){120}}
\put(130,10){\line(0,1){20}}
\put(70,95){\line(1,0){40}}
\put(70,95){\line(0,1){40}}
\put(70,135){\line(1,0){40}}
\put(110,95){\line(0,1){40}}
%
\put(80,30){\line(0,1){55}}
\put(100,30){\line(0,1){55}}
\put(75,85){\line(1,0){5}}
\put(100,85){\line(1,0){5}}
\put(75,85){\line(3,2){15}}
\put(105,85){\line(-3,2){15}}
%
\put(30,115){\line(1,0){40}}
\put(30,105){\circle{20}}
\put(20,65){\line(0,1){40}}
\put(10,45){\line(1,0){20}}
\put(10,45){\line(0,1){20}}
\put(30,45){\line(0,1){20}}
\put(10,65){\line(1,0){20}}
\put(37,45){\vector(0,1){20}}
\put(70,20){\makebox(0,0){$T_1$}}
\put(20,55){\makebox(0,0){$m$}}
\put(30,105){\makebox(0,0){$\bullet$}}
\put(90,115){\makebox(0,0){$M$}}
\put(90,60){\makebox(0,0){$Q$}}
%
\put(110,100){\line(1,0){30}}
\put(140,95){\line(1,1){10}}
\put(140,115){\line(1,-1){10}}
\put(140,95){\line(0,1){5}}
\put(140,110){\line(0,1){5}}
\put(110,110){\line(1,0){30}}
\put(120,120){\line(1,0){30}}
\put(110,125){\line(1,1){10}}
\put(110,125){\line(1,-1){10}}
\put(120,115){\line(0,1){5}}
\put(120,130){\line(0,1){5}}
\put(120,130){\line(1,0){30}}
\put(130,105){\makebox(0,0){${\scriptstyle Q_2,W}$}}
\put(130,125){\makebox(0,0){${\scriptstyle Q_2,W}$}}
\end{picture}
\caption{\label{fig_ZweiterHauptsatz}%
Zur Formulierung des zweiten Hauptsatzes durch Clausius und Kelvin bzw.\ Planck. (a) Eine
Maschine $M$ wird dazu genutzt, W\"armeenergie $Q_1$ aus einem k\"alteren in ein w\"armeres System zu pumpen.
Sie nimmt dabei Energie in Form von Arbeit $W$ oder W\"arme $Q_2$ aus der Umgebung auf, gibt diese aber
in derselben Form wieder an die Umgebung zur\"uck. Eine W\"armepumpe erh\"alt man, wenn $Q_2$ aus
dem w\"armeren System entnommen und (zus\"atzlich zu $Q_1$) an dieses wieder zur\"uckgegeben wird.  
(b) Eine zyklisch arbeitende Maschine wird dazu genutzt, W\"armeenergie $Q$ in mechanische Arbeit -- das
Hochheben eines Gewichtes zur Masse $m$ -- umzusetzen. }
\end{figure} 

Manchmal formuliert man den Satz f\"ur eine W\"armepumpe: In diesem Fall entnimmt die Maschine dem 
w\"armeren System zun\"achst eine W\"armemenge $Q_2$ und nutzt diese, um dem k\"alteren System die W\"arme 
$Q_1$ zu entziehen. Schlie\ss lich gibt sie die W\"arme $Q_1+Q_2$ wieder an das w\"armere Sys\-tem ab. Die
W\"arme $Q_2$ wurde also dem w\"armeren System entnommen und an dieses zur\"uckgegeben. Die W\"arme
$Q_1$ wurde vom k\"alteren System in das w\"armere System gepumpt.
\vspace{0.3cm}

\noindent
Die Formulierung von Kelvin bzw.\ Planck lautet:\\  
Es ist unm\"oglich, eine periodisch arbeitende Maschine zu konstruieren, die weiter nichts bewirkt als die
Hebung einer Last und die Abk\"uhlung eines W\"armereservoirs.
\vspace{0.3cm}

Letztendlich bedeutet dies, dass man nicht W\"armeenergie vollst\"andig in mechanische Arbeit umwandeln kann.
Die Maschine darf der Umgebung f\"ur ihren Betrieb zwar Energie in Form von W\"arme oder Arbeit entziehen,
muss diese aber wieder vollst\"andig an die Umgebung abgeben, sodass sich in der Umgebung nach einem
Zyklus nichts ge\"andert hat. 

Die beiden Aussagen sind \"aquivalent. G\"abe es die Maschine von Clausius, k\"onnte man die in das
w\"armere System gepumpte W\"armemenge nutzen, um eine Masse anzuheben, wobei ein Teil dieser
W\"armemenge in das k\"altere System zur\"uckflie\ss en k\"onnte. Insgesamt h\"atte man aber W\"arme aus
dem k\"alteren System genutzt, um eine Masse anzuheben. Umgekehrt, g\"abe es die Maschine von
Kelvin (diese bezeichnet man auch gelegentlich als Perpetuum Mobile 2.\ Art), k\"onnte man die angehobene
Masse wieder absinken lassen und dabei (analog zu dem Experiment von Joule zur Bestimmung des
W\"arme\"aquivalents) ein beliebiges System erw\"armen. 

\subsection{Der zweite Hauptsatz in der Formulierung von Sommerfeld}

Arnold Sommerfeld formuliert den zweiten Hauptsatz der Thermodynamik in seinen
\textit{Vorlesungen zur theoretischen Physik} folgenderma\ss en \cite{Sommerfeld}:\\[0.2cm]
Jedes thermodynamische System besitzt eine Zustandsgr\"o\ss e, Entropie genannt. Man 
berechnet sie, indem man das System aus einem willk\"urlich gew\"ahlten Anfangszustand in den jeweiligen Zustand
des Systems durch eine Folge von Gleichgewichtszust\"anden \"uberf\"uhrt, die hierbei schrittweise
zugef\"uhrte W\"arme $\delta Q$ bestimmt, letztere durch die erst bei dieser Gelegenheit zu definierende
\glqq absolute Temperatur\grqq\ $T$ dividiert und s\"amtliche Quotienten summiert.\\
Bei den wirklichen (nicht ideellen) Vorg\"angen nimmt die Entropie eines nach au\ss en abgeschlossenen Systems
zu. 
\vspace{0.3cm}
 
Hier wird sofort ein anderer Charakter der Aussage deutlich. Diese Aussagen sind alles andere als sofort
einsichtig. Die Tatsache, dass es eine Zustandsgr\"o\ss e Entropie geben soll, ist schon f\"ur sich nicht trivial.
Die Vorgehensweise, wie man diese bestimmt, erscheint ebenfalls sehr eigenartig. Sommerfeld spricht von einer
\glqq Folge von Gleichgewichtszust\"anden\grqq, er meint jedoch einen reversiblen Prozess. Damit ein
Prozess reversibel ist (also durch beliebig kleine \"Anderungen in der Umgebung auch in umgekehrter 
Richtung ausgef\"uhrt werden kann) ist die Folge von Gleichgewichtszust\"anden -- dies bezeichnet man
auch als quasistation\"ar -- nur eine notwendige Bedingung. Es gibt Prozesse, die beliebig langsam ablaufen und
daher eine Folge von Gleichgewichtszust\"anden sind, die trotzdem nicht reversibel sind. Ein Beispiel ist
der Joule-Thomson-Effekt, bei dem eine Druckdifferenz zwischen benachbarten Systemen durch eine
kleine D\"use ausgeglichen wird. Die D\"usen\"offnung l\"asst sich beliebig klein machen, wodurch der
Prozess beliebig langsam verl\"auft. Trotzdem ist dieser Prozess nicht reversibel. Es ist daher
nicht selbstverst\"andlich, dass je zwei beliebige Gleichgewichtszust\"ande durch einen reversiblen Prozess
miteinander verbunden werden k\"onnen. 

Ebenfalls eigenartig ist die Formulierung \glqq die erst bei dieser Gelegenheit zu definierende absolute Temperatur\grqq.
Gemeint ist, dass es zur W\"arme $Q$, die eine Prozessgr\"o\ss e ist und keine Zustandsgr\"o\ss e (diese Begriffe
werden in dem Kurztext \hyperref[chap_Zustand]{Zustands- und Prozessgr\"o\ss en} genauer erl\"autert), 
einen integrierenden Faktor gibt, sodass $\delta Q$ lokal proportional zum totalen Differential (d.h.\ zum Gradienten) 
einer Zustandsgr\"o\ss e ist: $\delta Q = T {\rm d}S$. 
Dieser integrierende Faktor ist bis auf einen Skalenfaktor festgelegt und man kann zeigen, dass dadurch
eine absolute Temperatur definiert werden kann. Dies wird im n\"achsten Abschnitt \ref{sec_Carnot} gezeigt und
dies ist auch das Ziel der axiomatischen Formulierungen (siehe Abschnitt \ref{sec_AxiomeEntropie}). 

Die eigentliche Aussage des zweiten Hauptsatzes besteht in dem letzten Satz: Die Entropie nimmt
in wirklichen abgeschlossenen Systemen immer zu. Insbesondere nimmt sie in abgeschlossenen Systemen
nie ab. Dies erkl\"art, weshalb manche Prozesse in der Natur beobachtet werden, die umgekehrt
ablaufenden Prozesse aber nie. Au\ss erdem zeichnet diese Aussage eine Zeitrichtung -- den sogenannten
thermodynamischen Zeitpfeil -- aus: Die thermodynamische Zeitrichtung ist so, dass die Entropie in
dieser Richtung zunimmt.      

Sommerfeld f\"uhrt seine nicht triviale Formulierung des zweiten Hauptsatzes der Thermodynamik auf
eine triviale Formulierung (von Kelvin bzw.\ Planck) zur\"uck. Dazu verwendet er den Carnot-Prozess. 

\subsection{Carnot-Prozesse}
\label{sec_Carnot}

Unter Berufung auf Carnot gibt Sommerfeld in seiner Vorlesung \cite{Sommerfeld} eine
sehr elegante Begr\"undung, weshalb $\delta Q/T$ integrabel ist, oder anders ausgedr\"uckt,
weshalb f\"ur jeden beliebigen zyklischen reversiblen Prozess (mathematisch ein geschlossener
Weg $\gamma$ im Raum der Gleichgewichtszust\"ande)
\begin{equation}
           \oint_\gamma  \frac{\delta Q}{T} = 0 
\end{equation}
ist. 

Ich m\"ochte an dieser Stelle den in meinen Augen sehr eleganten Beweis nur
andeuten, und das auch nur f\"ur ein homogenes Fluid (Fl\"ussigkeit oder Gas), bei
dem bei vorgegebener Stoffmenge der Druck $p$ und das Volumen $V$ den thermischen Zustand
bereits festlegen. Es gen\"ugt daher, geschlossene Wege im $p$-$V$-Diagramm zu betrachten. 

\begin{SCfigure}[50][htb]
\begin{picture}(220,160)(0,0)
\put(10,10){\vector(1,0){200}}
\put(10,10){\vector(0,1){140}}
\qbezier(40,130)(80,110)(140,100)
\qbezier(70,60)(110,40)(190,30)
\qbezier(40,130)(50,80)(70,60)
\qbezier(140,100)(150,60)(190,30)
\put(210,0){\makebox(0,0){$V$}}
\put(5,150){\makebox(0,0){$p$}}
\put(37,135){\makebox(0,0){$1$}}
\put(145,100){\makebox(0,0){$2$}}
\put(195,30){\makebox(0,0){$3$}}
\put(65,55){\makebox(0,0){$4$}}
\put(100,120){\makebox(0,0){${\scriptstyle T_1={\rm const}}$}}
\put(120,35){\makebox(0,0){${\scriptstyle T_2={\rm const}}$}}
\put(38,90){\makebox(0,0){${\scriptstyle \delta Q=0}$}}
\put(166,70){\makebox(0,0){${\scriptstyle \delta Q=0}$}}
\end{picture}
\caption{\label{fig_Carnot}%
Der Carnot-Prozess in der 
$p$-$V$-Ebene. Der Prozess $1\rightarrow 2$ verl\"auft
entlang einer Isothermen bei der Temperatur
$T_1$; der Prozess $2\rightarrow 3$ ist adiabatisch; 
der Prozess $3\rightarrow 4$ verl\"auft
wieder entlang einer Isothermen, diesmal bei der
Temperatur $T_2<T_1$; und der Prozess
$4\rightarrow 1$ ist wieder adiabatisch.}
\end{SCfigure}

Ein {\em Carnot-Prozess} ist ein reversibler
Kreisprozess, d.h., alle Teilprozesse sind reversibel
und nach einem vollen Durchlauf
befindet sich das System wieder in seinem 
Ausgangszustand. Er setzt sich zusammen aus
zwei isothermen Teilprozessen, bei denen das
System an zwei W\"armeb\"ader bei den jeweiligen 
Temperaturen $T_1 > T_2$ gekoppelt wird, und   
zwei adiabatischen Prozessen, bei denen also kein
W\"armeaustausch mit der Umgebung
stattfindet. Adiabaten sind \glqq steiler\grqq\ als
Isothermen im $p$-$V$-Diagramm, weil dem
System f\"ur eine Volumenausdehnung entlang 
einer Isothermen W\"arme zugef\"uhrt
wird. F\"ur dieselbe Volumenausdehnung
entlang einer Adiabaten nimmt die Temperatur
wegen der fehlenden W\"armezufuhr ab
und damit nimmt auch der Druck rascher
ab als entlang einer Isothermen. 

Da Anfangs- und Endzustand bei einem
Durchlauf gleich sind, folgt
\begin{equation}
   \oint \delta Q = \oint \delta W 
   \hspace{1cm} {\rm bzw.} \hspace{0.7cm}
   W = Q_{1} - Q_{2}  \, ,
\end{equation}
wobei $Q_{1}$ die entlang des Weges
$1\rightarrow 2$ vom System aus dem hei\ss en Reservoir
($T_1$) aufgenommene W\"armemenge
und $Q_{2}$ die von dem System entlang
des Weges $3\rightarrow 4$ an das k\"uhle Reservoir 
($T_2$) abgegebene W\"armemenge sind.
$W$ ist die bei dem Kreisprozess $1\rightarrow 2\rightarrow 3 \rightarrow 4$
geleistete Arbeit. 

Der Wirkungsgrad eines solchen Prozesses
ist definiert als das Verh\"altnis von geleisteter
Arbeit $W$ zu aufgenommener W\"armemenge
$Q_{1}$:
\begin{equation}
       \eta = \frac{W}{Q_{1}} = 1 - \frac{Q_{2}}{Q_{1}} \, .
\end{equation} 

Sommerfeld (in Anlehnung an Carnot) beweist nun, dass alle
reversiblen Carnot-Prozesse, die zwischen
denselben Temperaturen $T_1$ und $T_2$
operieren, denselben Wirkungsgrad haben
m\"ussen, unabh\"angig von der Stoffbeschaffenheit
oder anderer Parameter. Das bedeutet,
das Verh\"altnis $Q_2/Q_1$ kann nur von den
Temperaturen $T_1$ und $T_2$ abh\"angen:
\begin{equation}
           \frac{Q_2}{Q_1} = f(T_1,T_2) \, .
\end{equation}
Zum Beweis argumentiert er, dass die Prozesse
reversibel sein sollen und daher
auch in umgekehrter Reihenfolge
durchlaufen werden k\"onnen, wobei in
diesem Fall aus der Umgebung Arbeit
$W$ aufgenommen wird und dazu dient,
eine bestimmte W\"armemenge $Q_2$
aus dem k\"uhlen Reservoir aufzunehmen
und $Q_1$ in das hei\ss e Reservoir abzugeben.
Die Arbeit wird also genutzt, um W\"arme aus
dem kalten in das warme Reservoir zu
pumpen. H\"atten zwei Carnot-Prozesse zwischen
denselben Temperaturen verschiedene
Wirkungsgrade, k\"onnte man den Prozess mit
dem gr\"o\ss eren Wirkungsgrad zur Erzeugung von 
Arbeit aus W\"arme verwenden, und den anderen
in umgekehrter Richtung zur Verwendung dieser
Arbeit als W\"armepumpe einsetzen und h\"atte
so effektiv (ohne weitere Ver\"anderungen in der
Umgebung) W\"arme vom k\"alteren in das 
w\"armere Bad gepumpt. Dies widerspricht aber
dem 2.\ Hauptsatz der Thermodynamik in der Formulierung von Clausius.  

\begin{SCfigure}[50][htb]
\begin{picture}(220,160)(0,0)
\put(10,10){\vector(1,0){200}}
\put(10,10){\vector(0,1){140}}
\qbezier(40,130)(80,110)(140,100)
\qbezier(70,60)(110,40)(190,30)
\qbezier(50,95)(100,75)(156,65)
\qbezier(40,130)(50,80)(70,60)
\qbezier(140,100)(150,60)(190,30)
\put(210,0){\makebox(0,0){$V$}}
\put(5,150){\makebox(0,0){$p$}}
\put(37,135){\makebox(0,0){$1$}}
\put(145,100){\makebox(0,0){$2$}}
\put(161,65){\makebox(0,0){$3$}}
\put(45,95){\makebox(0,0){$4$}}
\put(195,30){\makebox(0,0){$5$}}
\put(65,55){\makebox(0,0){$6$}}
\put(100,120){\makebox(0,0){${\scriptstyle T_1={\rm const}}$}}
\put(110,85){\makebox(0,0){${\scriptstyle T_2={\rm const}}$}}
\put(120,35){\makebox(0,0){${\scriptstyle T_3={\rm const}}$}}
\put(45,75){\makebox(0,0){${\scriptstyle \delta Q=0}$}}
\put(160,83){\makebox(0,0){${\scriptstyle \delta Q=0}$}}
\end{picture}
\caption{\label{fig_Carnot2}%
Zwei Carnot-Prozesse ($1$-$2$-$3$-$4$-$1$ bei Temperaturen $T_1>T_2$
und $4$-$3$-$5$-$6$-$4$ bei $T_2>T_3$) k\"onnen zu einem Carnot-Prozess
$1$-$2$-$5$-$6$-$1$ bei den Temperaturen $T_1>T_3$ verkn\"upft werden,
in dem die Punkte in der Reihenfolge $1$-$2$-$3$-$4$-$3$-$5$-$6$-$4$-$1$
durchlaufen werden.}
\end{SCfigure}

Hat man sich davon \"uberzeugt, dass f\"ur einen reversiblen Carnot-Prozess das Verh\"altnis von
abgegebener zu aufgenommener W\"armemenge nur von den Temperaturen 
abh\"angen kann, betrachtet man nun zwei Carnot-Prozesse bei den Temperaturen
$T_1>T_2$ und $T_2>T_3$, sodass eine geeignete Hintereinanderschaltung
der Wege einen Carnot-Prozess bei den Temperaturen $T_1>T_3$ beschreibt (vgl.\
Abb.\ \ref{fig_Carnot2}). Wegen
\begin{equation}
     \frac{Q_3}{Q_1} =      \frac{Q_2}{Q_1}  \frac{Q_3}{Q_2} =   f(T_1,T_2) f(T_2,T_3)  =  f(T_1,T_3)
\end{equation}
muss es eine Funktion $f(T)$ geben, sodass
\begin{equation}
     \frac{Q_2}{Q_1} =   \frac{f(T_2)}{f(T_1)}  \, .
\end{equation}
Wir {\em definieren} nun eine Temperaturskala (die f\"ur die bisherige Argumentation noch
nicht vorausgesetzt wurde), indem wir f\"ur einen reversiblen Carnot-Prozess
\begin{equation}
     \frac{Q_2}{Q_1} =    \frac{T_2}{T_1}  \hspace{1cm} ({\rm bzw.})  \hspace{1cm}  \frac{Q_1}{T_1} = \frac{Q_2}{T_2} 
\end{equation}
festlegen. Dies ist eine M\"oglichkeit, eine Temperaturskala zu bestimmen. (Sie legt die
Temperatur bis auf eine multiplikative Konstante, also eine lineare Umdefinition der Skala,
fest.) Dass dieses Verfahren \"ubereinstimmt mit der Definition einer
Temperaturskala \"uber das Verhalten idealer Gase w\"are allerdings noch zu zeigen, was hier
nicht geschehen soll. Bez\"uglich der so gew\"ahlten Temperaturskala ist also bei einem reversiblen
Carnot-Prozess das Verh\"altnis $Q_1/T_1$ entlang der oberen Isothermen gleich dem
Verh\"altnis $Q_2/T_2$ entlang der unteren Isothermen. Ganz nebenbei haben wir auf diese Weise
auch den Wirkungsgrad eines reversiblen Carnot-Prozesses abgeleitet:
\begin{equation}
        \eta = 1 - \frac{T_2}{T_1}  \, .
\end{equation}

\begin{SCfigure}[50][htb]
\begin{picture}(220,160)(-20,0)
\put(0,10){\vector(1,0){180}}
\put(0,10){\vector(0,1){140}}
\qbezier(50,110)(55,107)(60,105)
\qbezier(80,40)(85,37)(90,35)
\qbezier(50,110)(60,60)(80,40)
\qbezier(60,105)(70,55)(90,35)
%
\qbezier(110,103)(113,101)(120,100)
\qbezier(140,45)(143,44)(150,43)
\qbezier(110,103)(120,63)(140,45)
\qbezier(120,100)(130,60)(150,43)
%
\thicklines
\qbezier(30,100)(50,117)(150,93)
\qbezier(150,93)(200,75)(145,43)
\qbezier(145,43)(100,30)(50,45)
\qbezier(50,45)(-10,70)(30,100)

\put(190,0){\makebox(0,0){$V$}}
\put(5,150){\makebox(0,0){$p$}}
\put(115,110){\makebox(0,0){${\scriptstyle \delta Q_1/T_1}$}}
\put(145,35){\makebox(0,0){${\scriptstyle \delta Q_2/T_2}$}}
\end{picture}
\caption{\label{fig_Carnot3}%
Jeder beliebige Kreisprozess l\"asst sich n\"aherungsweise in Carnot-Prozesse unterteilen,
die bei verschiedenen Temperaturen aber infinitesimal benachbarten Adiabaten ablaufen. F\"ur jeweils
zu einem Carnot-Prozess geh\"orende gegen\"uberliegende isotherme
Abschnitte heben sich die Beitr\"age $\delta Q_1/T_1$ und $\delta Q_2/T_2$
bei der Integration \"uber den Prozessweg paarweise weg.}
\end{SCfigure}

Als letzten Schritt zerlegt Sommerfeld einen
beliebigen Kreisprozess in der $p$-$V$-Ebene
in Carnot-Prozesse zu infinitesimal benachbarten
Adiabaten (d.h., die W\"armemengen $Q_1$ und
$Q_2$ werden zu infinitesimalen Elementen
$\delta Q_1$ und $\delta Q_2$), die aber bei
verschiedenen Temperaturen arbeiten
(Abb.\ \ref{fig_Carnot3}). F\"ur
diese Prozesse gilt
\begin{equation}
     \frac{\delta Q_1}{T_1} =
     \frac{\delta Q_2}{T_2} \, .  
\end{equation}
Da sich diese Elemente bei der Summation
\"uber einen geschlossenen Weg immer paarweise 
wegheben, erhalten wir insgesamt die
zu beweisende Identit\"at
\begin{equation}
              \oint \frac{\delta Q}{T} = 0 \, .
\end{equation}
Offensichtlich ist die \"uber einen
Carnot-Prozess definierte Temperaturskala
(bis auf einen multiplikativen Faktor) identisch
zu der Temperaturskala, die man aus der
Forderung erh\"alt, dass $T$ der integrierende
Faktor zwischen der 1-Form $\delta Q$ und
der zugeh\"origen exakten 1-Form ${\rm d}S$
sein soll.

\section{Axiomatische Definitionen der Entropie}
\label{sec_AxiomeEntropie}

Der Carnot-Prozess hat gezeigt, dass es zu der Prozessgr\"o\ss e (1-Form) W\"arme $\delta Q$ 
einen integrierenden Faktor $T$ gibt, sodass $\delta Q=T{\rm d}S$. Die meisten axiomatischen Zug\"ange
(z.B.\ Carath\'{e}odory und Jauch \cite{Caratheodory,Jauch}) leiten diese Tatsache aus rein mathematischen
\"Uberlegungen her, wobei ein genaue Analyse zeigt, dass sie das Konzept des Carnot-Prozesses
mathematisieren und dann beweisen, dass die 1-Form \glqq W\"arme\grqq\ einen integrierenden
Faktor besitzt und dass dieser integrierende Faktor nur von einer empirischen Temperatur
$\theta$ abh\"angt. Sie definieren dann die absolute Temperatur $T$ als diesen integrierenden
Faktor. Ich skizziere hier einige dieser axiomatischen Zug\"ange, obwohl dies weit \"uber die
Anforderungen in der Schule hinausgeht. Aber f\"ur den interessierten Leser stecken in diesen Zug\"angen
doch sehr viele zus\"atzliche Einsichten.

\subsection{Die geometrische Bedeutung eines integrierenden Faktors}

In diesem Abschnitt geht es um einige Beziehungen zwischen 1-Formen und 0-Formen. N\"aheres
zu diesen Formen sowie zu der hier verwendeten Notation 
findet man in dem Kurztext \hyperref[chap_Zustand]{Zustands- und Prozessgr\"o\ss en}. Hier
werden nur die wichtigsten Definitionen angegeben. 

Die Menge der Gleichgewichtszust\"ande
in der Thermodynamik l\"asst sich als Mannigfaltigkeit $M$ darstellen, d.h., sie l\"asst sich lokal durch
Koordinaten $\pmb{x}=(x_0,x_1,...,x_n)$ beschreiben. Bei diesen Koordinaten handelt es sich um einen Satz von 
unabh\"angigen Zustandsgr\"o\ss en, wobei eine Koordinate ($x_0$) die innere Energie sein soll und die
anderen Koordinaten ein Satz von Gr\"o\ss en, durch welche die Arbeit des Systems an der Umgebung
kontrolliert werden kann (z.B.\ das Volumen, die Stoffmengen, Ladungsmengen, etc.), daher bezeichnet man
diese Koordinaten auch als Arbeitskoordinaten. Eine 
Zustandsgr\"o\ss e oder auch 0-Form ist eine Observable, die an einem Gleichgewichtszustand gemessen 
werden kann, d.h., es handelt sich um eine Funktion $f:M\rightarrow \mathbb{R}$ (bzw.\ in Koordinaten
$f:U\subset \mathbb{R}^n \rightarrow \mathbb{R}$; ich verwende hier dasselbe Symbol). Unter einem
Prozess bzw.\ einer Zustands\"anderung verstehen wir einen Weg $\gamma: [0,1] \rightarrow M$ (in Koordinaten
$t \mapsto \pmb{x}(t)=(x_0(t),...,x_n(t))$) auf der Mannigfaltigkeit der Gleichgewichtszust\"ande. Damit
dieser Weg tats\"achlich auf dieser Mannigfaltigkeit verl\"auft, muss die Zustands\"anderung reversibel 
erfolgen, d.h., sie muss quasistation\"ar sein (zu jedem Zeitpunkt muss sich das System in einem Gleichgewichtszustand
befinden, d.h., der Prozess verl\"auft beliebig langsam) und durch minimale \"Anderungen in der Umgebung
muss der Prozess umkehrbar sein. Man \"uberlege sich, dass jeder solche Weg auf der Menge der
Gleichgewichtszust\"ande tats\"achlich durch einen reversiblen Prozess physikalische realisiert
werden kann.

Das totale Differential
${\rm d}f$ (oder auch der Gradient $\pmb{\nabla}f$) ist eine 1-Form und damit eine sogenannte Prozessgr\"o\ss e.
Das bedeutet, es handelt sich um ein Feld, das an jedem Punkt $\pmb{x}$ auf einen Tangentenvektor 
(die \glqq Geschwindigkeit\grqq\ $\dot{\pmb{x}}$ eines Wegs $\gamma:[0,1] \rightarrow M; t \mapsto \pmb{x}(t)$) an 
diesem Punkt angewandt werden muss. Die so erhaltene Funktion l\"asst sich entlang des Weges  integrieren:
\begin{equation}
         f(\pmb{x}) - f(\pmb{x}_0) =  \int_\gamma {\rm d}f = \int_0^1 {\rm d}f_{(\pmb{x}(t))}( \dot{\pmb{x}})\,{\rm d}t \, .
\end{equation}
Das Differential ${\rm d}f_{(\pmb{x})}$ definiert an jedem Punkt $\pmb{x}\in M$ eine Hyperfl\"ache (einen Vektorraum,
der eine Dimension weniger als $M$ hat, also die Kodimension 1):
Alle Tangentialvektoren $\dot{\pmb{x}}$ in $T_{\pmb{x}}M$ (dem Tangentialraum an $M$ im Punkte $\pmb{x}$, der
durch die m\"oglichen Geschwindigkeiten von Wegen durch diesen Punkt gegeben ist), 
f\"ur die ${\rm d}f_{(\pmb{x}(t))}( \dot{\pmb{x}})=0$ gilt, liegen in dieser Hyperebene, d.h.\ es handelt sich bei dieser 
Hyperebene um den Tangentialraum an 
eine \"Aquipotentialfl\"ache von $f$.\footnote{Oft stellt man sich ${\rm d}f_{(\pmb{x})}$ als ein Vektorfeld
vor, das an jedem Punkt dem Gradienten von $f$ entspricht. Hierbei sollte man jedoch beachten, dass es
sich dabei um eine lineare Abbildung (also ein Element des Dualraums) auf dem Tangentenraum an $M$ -- also
dem Raum aller \glqq Geschwindigkeiten von Wegen\grqq\ in einem Punkt $\pmb{x}\in M$ -- handelt.} 
Ausgedr\"uckt in Koordinaten hat ${\rm d}f$ die Form:
\begin{equation}
           {\rm d}f_{(\pmb{x})} = \sum_{i=1}^n \frac{\partial f(\pmb{x})}{\partial x_i}\, {\rm d} x_i \, ,
\end{equation}
wobei ${\rm d}x_i$ die Differentiale der Koordinatenfunktion $x_i:M\rightarrow \mathbb{R}$ sind. Diese
sollten nicht als \glqq kleine Inkremente\grqq\ interpretiert werden, sondern als die Abbildung, die einem
Tangentialvektor (an einem Punkt $\pmb{x}$) seine $i$-te Komponente zuordnet. Es ist also
\begin{equation}
                  {\rm d}x_i (\dot{\pmb{x}}) = \dot{x}_i \, . 
\end{equation}
Eine allgemeine 1-Form auf $M$ hat (ausgedr\"uckt in lokalen Koordinaten) die Form
\begin{equation}
           \omega_{(\pmb{x})} = \sum_{i=1}^n \omega_i \, {\rm d} x_i \, .
\end{equation}
Auch eine solche 1-Form -- anschlaulich ein Vektorfeld mit Komponenten $\omega_i$ --
definiert an jedem Punkt $\pmb{x}$ von $M$ eine Hyperebene durch die Bedingung
$\omega_{(\pmb{x})} (\dot{\pmb{x}}) = 0$. Dies f\"uhrt auf die Frage: Unter welchen Bedingungen
lassen sich diese Hyperebenen in einer Umgebung eines Punktes $\pmb{x}$ als Tangentialr\"aume
an die \"Aquipotentialfl\"achen einer Funktion $f$ deuten? Ist $\omega$ das
totale Differential zu einer Funktion $f$, wenn also $\omega = {\rm d}f$ gilt -- in diesem Fall bezeichnet
man $\omega$ als \textit{exakt} --, ist das der Fall. Eine solche Funktion $f$ gibt es, wenn $\omega$
rotationsfrei ist, d.h. wenn gilt:
\begin{equation}
                                  \frac{\partial \omega_i}{\partial x_j} =  \frac{\partial \omega_j}{\partial x_i} \, .
\end{equation}
(In einem Differentialformenkalk\"ul kann man daf\"ur auch ${\rm d}\omega =0$ schreiben.) 

Die Bedingung $\omega_{(\pmb{x})} (\dot{\pmb{x}}) = 0$, durch welche bei gegebenem $\omega$ die
Hyperebene definiert ist, erlaubt aber eine gewisse Freiheit: Wenn $\omega$ an jedem Punkt mit einem
beliebigen (von null verschiedenen) Faktor multipliziert wird, \"andert sich diese Bedingung f\"ur die
Hyperebenen nicht. Es wird lediglich die \glqq L\"ange\grqq\ der Normalenvektoren an die Ebene
ge\"andert, das \"andert aber die Ebene selbst nicht. Also gilt auch: Wenn es eine Funktion
$\alpha(\pmb{x})$ gibt, sodass $\alpha(\pmb{x}) \omega_{(\pmb{x})}$ rotationsfrei ist, dann gibt es
(in einem zusammenh\"angenden Gebiet) eine Funktion $f$, sodass $\alpha \omega= {\rm d}f$. Die
Funktion $\alpha(\pmb{x})$ bezeichnet man als \glqq integrierenden Faktor\grqq. Die Gleichung
$\omega(\dot{\pmb{x}})=0$ hat dieselben L\"osungen wie $\alpha(\pmb{x}) \omega(\dot{\pmb{x}})=0$ 
bzw.\ ${\rm d}f(\dot{\pmb{x}})=0$ (sofern es diese Funktion $f$ gibt), 
und dies sind die Vektoren der Tangentialr\"aume an die
\"Aquipotentialfl\"achen von $f$. 

Gibt es eine Funktion $f$, sodass $\alpha \omega ={\rm d}f$, ist nat\"urlich auch jede Funktion 
$F(f)$ eine L\"osung des Problems, da die \"Aquipotentialfl\"achen dieselben sind. 
Da ${\rm d}F(f)= F'(f) {\rm d}f$, \"andert sich entsprechend der integrierende Faktor um
$ \alpha \rightarrow \alpha F'(f)$. Die L\"osung des Problems ist somit nicht eindeutig. 

Die Beziehung zur Thermodynamik ist nun folgende: Die W\"arme $\delta Q$ (in der Mathematik w\"urde
man eher $\omega_Q$ schreiben) ist eine 1-Form ebenso wie das Differential ${\rm d}U$ 
der inneren Energie U. Allerdings ist ${\rm d}U$ eine exakte 1-Form -- es gibt die Zustandsfunktion 
(0-Form) der inneren Energie $U$ --, wohingegen $\delta Q$ keine exakte 1-Form ist. 
Allerdings kann man beweisen -- und dies tun der Carnot-Prozess wie auch die axiomatischen 
Ans\"atze von Carath\'{e}odory und Jauch --, dass es einen integrierenden Faktor zu $\delta Q$ gibt, 
n\"amlich $1/T$, wobei $T$ als absolute Temperatur definiert wird, sodass es eine 
Zustandsgr\"o\ss e -- die Entropie $S$ -- gibt, sodass $\delta Q/T = {\rm d}S$. 
  
\subsection{Die Arbeit von Constantin Carath\'{e}odory}

Die Arbeit von Constantin Carath\'{e}odory von 1913 \cite{Caratheodory} wird heute oftmals als der
erste axiomatische Zugang zur Thermodynamik angesehen, bei dem der zweite Hauptsatz der
Thermodynamik -- die Existenz einer Zustandsgr\"o\ss e Entropie sowie einer absoluten Temperatur --
bewiesen wird, ohne dass das sehr unscharf definierte Konzept der W\"arme explizit verwendet wird.  
Der erste Hauptsatz der Thermodynamik -- bei Carath\'{e}odory ist dies das Axiom I -- wird
so umgeformt, dass die W\"arme $\delta Q$ durch die besser definierten Gr\"o\ss en innere Energie $U$ 
bzw.\ das totale Differential ${\rm d}U$ und die Arbeit $\delta W$ ausgedr\"uckt wird. Lokal 
wird dadurch postuliert, dass
\begin{equation}
                          \delta Q = {\rm d}U + \delta W \, .
\end{equation}
Prozesse bzw.\ Zustands\"anderungen sind auch bei Carath\'{e}odory reversible Wege $\pmb{x}(t)$ 
auf der Mannigfaltigkeit $M$ der Gleichgewichtszust\"ande. Zun\"achst sehr unanschaulich 
ist das Axiom II bei Carath\'{e}odory:\\[0.3cm]
\textit{In jeder beliebigen Umgebung eines willk\"urlich vorgeschriebenen Anfangszustands
gibt es Zust\"ande, die durch adiabatische Zustands\"anderungen nicht beliebig approximiert
werden k\"onnen.}
\vspace{0.3cm}

Eine adiabatische Zustands\"anderung -- beschrieben durch einen Tangentialvektor $\dot{\pmb{x}}$ --
erf\"ullt dabei die Bedingung $\delta Q(\dot{\pmb{x}})=0$. Bei einer solchen Zustands\"anderung flie\ss t
also keine W\"arme bzw.\ findet kein W\"armeaustausch mit der Umgebung statt. Die Forderung, dass es 
solche Zustands\"anderungen \"uberhaupt gibt, 
setzt als empirische Tatsache die Existenz von adiabatisch abgeschlossenen W\"anden voraus, sodass man
experimentell eine reversible Zustands\"anderung realisieren kann, bei der keine W\"arme mit der Umgebung
ausgetauscht wird. 

Carath\'{e}odory
betrachtet nun adiabatische Wege (Zustands\"anderungen) $\pmb{x}(t)$, ausgehend von einem Anfangszustand
$\pmb{x}_0$ zu einem beliebigen Zustand $\pmb{x}$, f\"ur die gilt
\begin{equation}
                          \delta Q_{(\pmb{x}(t))}(\dot{\pmb{x}}(t))  = 0 \, . 
\end{equation}
Wenn es zu $\delta Q$ einen integrierenden Faktor $\alpha(\pmb{x})$ und eine Funktion $S(\pmb{x})$
gibt, so dass $\alpha(\pmb{x}) \delta Q = {\rm d}S$, bedeutet dies, dass ein solcher Weg
(eine solche Zustands\"anderung) nur innerhalb einer \"Aquipotentialfl\"ache zu dieser Funktion $S$ verlaufen
kann. Dies bedeutet wiederum, dass bestimmte Punkte in einer beliebig kleinen Umgebung eines
Gleichgewichtszustands $\pmb{x}$ -- n\"amlich solche mit einem anderen Funktionswert f\"ur $S$ als der Punkt $\pmb{x}$ --
durch adiabatische Wege nicht erreicht werden k\"onnen. Nach einem Theorem von \'{E}lie Cartan
gilt aber auch das Umgekehrte: Wenn es in der Umgebung eines Punktes $\pmb{x}$ andere
Punkte gibt, die durch Wege, entlang derer $\omega(\dot{\pmb{x}})=0$ ist, nicht erreicht werden
k\"onnen, dann gibt es eine Funktion $S(\pmb{x})$, sodass diese Wege auf einer Hyperfl\"ache 
$S(\pmb{x})={\rm const}$ verlaufen. Mit anderen Worten: Wenn $\omega$ keinen integrierenden Faktor
besitzt, dann kann man mit Wegen, f\"ur die $\omega(\dot{\pmb{x}})=0$ gilt, jeden Punkt in einer
Umgebung eines Punkte $\pmb{x}$ beliebig approximieren. 

Damit hat Carath\'{e}odory gezeigt, dass es zu der 1-Form W\"arme einen integrierenden Faktor gibt,
so dass diese 1-Form zu einem totalen Differential einer Zustandsgr\"o\ss e $S$ wird, die wir dann
Entropie nennen. Es verbleibt zu zeigen, dass dieser integrierende Faktor lediglich eine Funktion der
sogenannten \textit{empirischen Temperatur} $\theta$ sein kann. Au\ss erdem hatten wir gesagt, dass
die Funktion $S$ nicht eindeutig ist, sofern sie existiert. Allerdings kann man zeigen, dass unter den
Funktionen $F(S)$, die m\"oglich w\"aren, nur eine (bis auf einen konstanten Faktor) die Eigenschaft
hat, additiv zu sein. D.h., wenn man zwei Systeme zu einem Gesamtsystem vereint, soll die
Entropie $S$ additiv sein: $S_{\rm ges} = S_1 + S_2$. Dies legt sowohl die Funktion $S$ als auch
den integrierenden Faktor bis auf einen konstanten Faktor fest. Von diesem kann man nun zeigen,
dass er nur von der Temperatur abh\"angt, denn wenn der integrierende Faktor eine intensive
Zustandsgr\"o\ss e sein soll, kann er nicht mehr von einer extensiven Gr\"o\ss e (der Entropie) 
abh\"angen. 

\subsection{Der Zugang von Joseph-Maria Jauch}

In seinem Artikel von 1970 \cite{Jauch} geht Joseph-Maria Jauch ganz \"ahnlich vor, wie 
Carathh\'{e}odory. Durch ein erstes Axiom wird die Energieerhaltung postuliert, sodass es
eine 1-Form (genannt W\"arme) gibt, die sich als
\begin{equation}
                 \delta Q ={\rm d}U + \delta W 
\end{equation}
schreiben l\"asst, wobei vorausgesetzt wird, dass $U$ -- die innere Energie -- eine an einem
Gleichgewichtszustand messbare Gr\"o\ss e -- also eine Zustandsgr\"o\ss e -- darstellt, 
die bei der Verwendung von Arbeitskoordinaten mit $x_0$ bezeichnet wird,
und dass $\delta W$ die von dem System an der Umgebung geleistete Arbeit bezeichnet, die
durch die Arbeitskoordinaten $(x_1,...,x_n)$ kontrolliert werden kann. 

Jauch betrachtet nun reversible adiabatische Wege, also wiederum Wege, f\"ur die
$\delta Q(\dot{\pmb{x}}(t))=0$ ist. Er beweist das folgende Theorem: Ist ein adiabatischer Weg
in den Arbeitskoordinaten $(x_1,...x_n)$ geschlossen, dann ist er auch in $M$ geschlossen, d.h.\
in den Koordinaten $(x_0,x_1,...,x_n)$. Dies ist nicht trivial: Es k\"onnte adiabatische \glqq Spiralen\grqq\ 
geben, sodass in $(x_1,...,x_n)$ ein Weg geschlossen ist, dieser aber in der verbliebenen Koordinate
$x_0$ einen anderen Wert annimmt. 

Wir betrachten also nun einen Weg $\gamma$, der bez\"uglich der Arbeitskoordinaten $(x_1,...,x_n)$
geschlossen ist, und f\"ur den $\delta Q(\dot{\pmb{x}})=0$. Insbesondere gilt f\"ur diesen Weg
\begin{equation}
      \int_{\gamma} \delta Q = 0 \, .
\end{equation}      
Damit m\"usste f\"ur diesen Weg und die Koordinate $x_0$ (die der inneren Energie $U$ entspricht) 
gelten:      
\begin{equation}      
               x_0(1) - x_0(0) =  \Delta U  =  \int_{\gamma} \delta W \, .
\end{equation}
Da die Arbeitskoordinaten $(x_1,...,x_n)$ aber am Ende des Weges wieder dieselben Werte wie
zu Beginn annehmen, kann die Energie $\Delta U$ dem System nur zugef\"ugt worden sein, indem
bei dem Prozess diese Energie aus einem W\"armereservoir entnommen wurde. Doch dies widerspricht
dem zweiten Hauptsatz in der Formulierung von Planck: Es gibt keine periodisch arbeitende Maschine
(die Arbeitskoordinaten sind nach einem Zyklus dieselben wie vorher), deren einziger Effekt darin besteht, 
dass ein Gewicht angehoben wird und sich ein W\"armereservoir abgek\"uhlt hat.

Das bedeutet aber, dass ein adiabatischer Weg, f\"ur den somit $\int_{\gamma} \delta Q = 0$, und
der in den Arbeitskoordinaten geschlossen ist, automatisch insgesamt geschlossen ist. 


\subsection{Lieb und Yngvason: Entropie als Ma\ss\ f\"ur Reversibilit\"at}

In einigen j\"ungeren Arbeiten (z.B.\ \cite{Lieb1,Lieb2}) haben Lieb und Yngvason versucht, den
Begriff der Entropie direkt als Ma\ss\ f\"ur Irreversibilit\"at herzuleiten. Dazu betrachten sie wieder den
Raum aller Gleichgewichtszust\"ande und definieren darauf eine Halbordnung: Wenn ein
Prozess f\"ur zwei Gleichgewichtszust\"ande $A$ und $B$ spontan abl\"auft (nicht notwendigerweise 
reversibel oder \"uber eine Folge vonGleichgewichtszust\"anden), definieren sie $A\rightarrow B$. 
Wenn es Prozesse gibt, die in beide Richtungen ablaufen k\"onnen, gelte $A \leftrightarrow B$.
Die Relation $A\rightarrow B$ ist transitiv, d.h., wenn gilt $A\rightarrow B$ und $B \rightarrow C$, gilt
auch $A \rightarrow C$. Auch die Relation $A \leftrightarrow B$ ist transitiv (und symmetrisch). 
Das bedeutet: Es gibt auf der Menge der Gleichgewichtszust\"ande eine \"Aquivalenzrelation
($\leftrightarrow$) und auf der Menge der \"Aquivalenzklassen eine Halbordnung. 

Lieb und Yngvason definieren nun auf der Menge der Menge der Gleichgewichtszust\"ande eine
Funktion $\sigma$, die innerhalb einer \"Aquivalenzklasse konstant ist und die die Halbordnung respektiert, 
d.h., falls $A \rightarrow B$ gilt, soll auch $\sigma(A) \leq \sigma(B)$ gelten, und falls $A \leftrightarrow B$
gilt, soll $\sigma(A) = \sigma(B)$ gelten. Diese Funktion $\sigma$ k\"onnte man als empirische
Entropie interpretieren. 

Auch diese Funktion $\sigma$ ist nicht eindeutig, da jede monotone Funktion von $\sigma$ ebenfalls
die geforderten Eigenschaften hat. \"Ahnlich wie Carath\'{e}odory und Jauch nutzen die Autoren nun
bestimmte Forderungen an die Funktion $\sigma$, die im Wesentlichen
aus der Additivit\"at bzw.\ der Skalierungseigenschaft folgen, um $\sigma$ bis auf einen konstanten
Faktor festzulegen. Die so erhaltene Funktion $S$ definieren sie als die Entropie eines Systems. 

\begin{thebibliography}{99}
\bibitem{Caratheodory} Carath\'{e}odory, C.; \textit{Untersuchungen \"uber die Grundlagen der
          Thermodynamik}; Mathematische Annalen, Bd.\ 67 (1909) 355--386.
\bibitem{Carnot} Carnot, S.; \textit{R\'{e}flexions sur la puissance motrice du feu et sur les machines propres \`{a} d\'{e}velopper cette
               puissance}; Annales scientifiques de l'\'{E}cole Normale Sup\'{e}ieure S\'{e}r.\ 2 (1872), 393--457.\\Deutsch:
               \textit{Betrachtungen \"uber die bewegende Kraft des Feuers und die zur Entwickelung dieser Kraft geeigneten Maschinen} (1824);
               \"ubersetzt und herausgegeben von W.\ Ostwald, in Ostwald's Klassiker der exakten Wissenschaften, Nr.\ 37.        
\bibitem{Jauch} Jauch, J.-M.; \textit{Analytical Thermodynamics. Part I. Thermostatistics---General Theory};
               Found.\ Phys.\ 5, No.\ 1 (1975), 111--132; Verlag von Wilhelm Engelmann, Leibzig, 1892.
\bibitem{Lieb1} Lieb, E.\,H., Yngvason, J.; \textit{The Physics and Mathematics of the Second Law of
            Thermodynamics}; Phys.\ Rep.\ 310 (1999) 1--96.
\bibitem{Lieb2} Lieb, E.\,H., Yngvason, J.; \textit{The Mathematical Structure of the Second Law of
            Thermodynamics}; Current Developments in Mathematics, 2001; International Press, Cambridge (2002)
            89--130.
\bibitem{Sommerfeld} Sommerfeld, A.; \textit{Vorlesungen \"uber theoretische Physik; Band 5: Thermodynamik und
             Statistik}; Verlag Harri Deutsch, Frankfurt (1992).                                                                           
\end{thebibliography}


\end{document}

