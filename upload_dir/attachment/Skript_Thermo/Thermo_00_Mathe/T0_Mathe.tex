%\documentclass[german,10pt]{book}     
\usepackage{makeidx}
\usepackage{babel}            % Sprachunterstuetzung
\usepackage{amsmath}          % AMS "Grundpaket"
\usepackage{amssymb,amsfonts,amsthm,amscd} 
\usepackage{mathrsfs}
\usepackage{rotating}
\usepackage{sidecap}
\usepackage{graphicx}
\usepackage{color}
\usepackage{fancybox}
\usepackage{tikz}
\usetikzlibrary{arrows,snakes,backgrounds}
\usepackage{hyperref}
\hypersetup{colorlinks=true,
                    linkcolor=blue,
                    filecolor=magenta,
                    urlcolor=cyan,
                    pdftitle={Overleaf Example},
                    pdfpagemode=FullScreen,}
%\newcommand{\hyperref}[1]{\ref{#1}}
%
\definecolor{Gray}{gray}{0.80}
\DeclareMathSymbol{,}{\mathord}{letters}{"3B}
%
\newcounter{num}
\renewcommand{\thenum}{\arabic{num}}
\newenvironment{anmerkungen}
   {\begin{list}{(\thenum)}{%
   \usecounter{num}%
   \leftmargin0pt
   \itemindent5pt
   \topsep0pt
   \labelwidth0pt}%
   }{\end{list}}
%
\renewcommand{\arraystretch}{1.15}                % in Formeln und Tabellen   
\renewcommand{\baselinestretch}{1.15}                 % 1.15 facher
                                                      % Zeilenabst.
\newcommand{\Anmerkung}[1]{{\begin{footnotesize}#1 \end{footnotesize}}\\[0.2cm]}
\newcommand{\comment}[1]{}
\setlength{\parindent}{0em}           % Nicht einruecken am Anfang der Zeile 

\setlength{\textwidth}{15.4cm}
\setlength{\textheight}{23.0cm}
\setlength{\oddsidemargin}{1.0mm} 
\setlength{\evensidemargin}{-6.5mm}
\setlength{\topmargin}{-10mm} 
\setlength{\headheight}{0mm}
\newcommand{\identity}{{\bf 1}}
%
\newcommand{\vs}{\vspace{0.3cm}}
\newcommand{\noi}{\noindent}
\newcommand{\leer}{}

\newcommand{\engl}[1]{[\textit{#1}]}
\parindent 1.2cm
\sloppy

      \begin{document}   %\setcounter{chapter}{0}

\chapter{Mathematische Grundlagen der Statistischen Mechanik}
\label{chap_MathStat}

\info{Thomas Filk}{30.03.2024}%
Wie schon der Name andeutet, handelt es sich bei
der Statistischen Mechanik um eine \glqq statistische\grqq\
Theorie, d.h., man interessiert sich f\"ur die
gemittelten Eigenschaften von Systemen
mit vielen Freiheitsgraden bzw.\ von einem
Ensemble von vielen gleichartigen Systemen. 
Die Thermodynamik ist eine ph\"anomenologische
Theorie f\"ur den Grenzfall sehr vieler (im Idealfall
unendlich vieler) Freiheitsgrade bzw.\ 
Systemkomponenten. In diesem Grenzfall wird
unter sehr allgemeinen Bedingungen das
Verhalten von Mittelwerten exakt. 

Das zentrale mathematische R\"ustzeug f\"ur
die Statistische Mechanik umfasst somit die
Wahrscheinlichkeitstheorie bzw.\ die Statistik, und der
Schwerpunkt beruht 
speziell auf dem Verhalten von Kenngr\"o\ss en von
Wahrscheinlichkeitsverteilungen im Grenzfall
unendlich gro\ss er Systeme (oder von Systemen
mit unendlich vielen Freiheitsgraden). Dieses
Werkzeug soll in diesem Kurztext kurz angerissen 
werden. 

\section{Wahrscheinlichkeiten}

Das Thema \glqq Wahrscheinlichkeit\grqq\ wurde zu\index{Wahrscheinlichkeit|(}
einem Zweig der Mathematik, als man in der Zeit
zwischen dem 15.\ und 17.\,Jahrhundert 
mehr und mehr erkannte, dass sich die relativen
H\"aufigkeiten, mit der bestimmte Ereignisse bei
Gl\"ucksspielen auftreten, mathematisch bestimmen
lassen. Nach langen Schwierigkeiten bei der Suche
nach einer Definition, was genau Wahrscheinlichkeit
eigentlich sei, konnte Andrei Nikolajewitsch Kolmogorow (1903--1987) 1933\index{Kolmogorow, Andrei}
den Begriff der Wahrscheinlichkeit axiomatisch
fassen. Diese Formulierung umgeht die philosophische
Frage nach dem Wesen der Wahrscheinlichkeit und
ersetzt sie durch einen Satz von Bedingungen,
die von einer mathematische Struktur erf\"ullt sein
m\"ussen, um von Wahrscheinlichkeiten im 
herk\"ommlichen Sinne sprechen zu k\"onnen.

Die Hauptschwierigkeit eines solchen 
Formalismus bezieht sich auf die mathematisch
exakte Beschreibung von Wahrscheinlichkeiten
bei kontinuierlichen Ereignismengen. Als Beispiel
sei die Wahrscheinlichkeit genannt, auf einer (mathematisch
idealisierten) Dart-Scheibe mit einer (mathematisch
idealisierten) Pfeilspitze einen (mathematisch
idealisieren) Punkt zu treffen. Die erste Schwierigkeit
besteht darin, dass wir elementaren Ereignissen
keine von null verschiedenen 
Wahrscheinlichkeiten zuschreiben 
k\"onnen, sondern nur Ereignismengen (beispielsweise
einem Fl\"achenausschnitt der Dart-Scheibe). 
Die zweite,
f\"ur die Physik allerdings meist weniger wichtige
Schwierigkeit liegt darin, dass manchen
Teilmengen des Ereignisraums \"uberhaupt keine 
Wahrscheinlichkeiten (weder null noch von null
verschieden)
zugeschrieben werden k\"onnen. Heute z\"ahlt
die Wahrscheinlichkeitstheorie in der Mathematik
zur Ma\ss theorie. Allerdings spielt gerade bei
endlichen Ereignisr\"aumen oft die Kombinatorik
eine wichtigere Rolle. 

\subsection{Ereignisse als Teilmengen eines
Ereignisraumes}

Wenn wir im Alltag im Zusammenhang mit 
Wahrscheinlichkeiten von einem Ereignis sprechen,\index{Ereignis}\index{Elementarereignis}
meinen wir meist sogenannte {\em Elementarereignisse},
die sich auf eine einzelne konkrete Realisierung beziehen.
Bekannte Beispiele sind das Ereignis, mit einem
W\"urfel bei einem Spiel eine bestimmte Zahl zu
w\"urfeln oder bei einem M\"unzwurf \glqq Kopf\grqq\ zu
erhalten. Haben wir Grund zu der Annahme, dass 
jedes solche Ereignis gleich wahrscheinlich ist,
k\"onnen wir die Wahrscheinlichkeit f\"ur ein
solches Ereignis bestimmen, wenn wir die Gesamtzahl
aller Ereignisse kennen. Beim W\"urfel gibt es 
sechs Zahlen, daher ist die Wahrscheinlichkeit
f\"ur das Auftreten einer bestimmten Zahl (bei einem
\glqq ehrlichen\grqq\ W\"urfel) gleich $1/6$. Bei
einer M\"unze gibt es zwei m\"ogliche Elementarereignisse
und wenn die M\"unze \glqq fair\grqq\ ist, sollte jedes
Ereignis mit der Wahrscheinlichkeit $1/2$ auftreten.

In vielen F\"allen verstehen wir unter einem
Ereignis aber auch eine ganze Klasse von
Elementarereignissen, beispielsweise wenn
wir bei einem W\"urfelspiel mit zwei W\"urfeln 
nach der Wahrscheinlichkeit fragen, ein
\glqq M\"axle\grqq\ (eine Eins und eine Zwei)
zu w\"urfeln oder eine bestimmte Gesamtaugenzahl
zu erreichen (dies ist
z.B.\ bei dem Spiel \glqq Siedler von Catan\grqq\
wichtig). Das Ereignis \glqq Gesamtaugenzahl
Sieben\grqq\ besteht somit aus mehreren
Elementarereignissen: 
$\{(1,6),(2,5),(3,4),(4,3),(5,2),(6,1)\}$, wobei $(p,q)$
das Elementarereignis bezeichnet, mit dem ersten
W\"urfel (der z.B.\ durch seine Farbe markiert sein
kann) die Zahl $p$ und mit dem zweiten W\"urfel
die Zahl $q$ zu w\"urfeln. Sind die Elementarereignisse
gleich wahrscheinlich, erhalten wir die Wahrscheinlichkeit
f\"ur ein solches zusammengesetztes Ereignis nach der 
\textit{Laplace'schen Formel}\index{Laplace'sche Formel f\"ur Wahrscheinlichkeiten}\index{Wahrscheinlichkeit!Laplace'sche Formel}
\begin{equation}
   \mbox{Wahrscheinlichkeit} = 
   \frac{\mbox{Menge der g\"unstigen Ereignisse}}{\mbox{Menge
   der m\"oglichen Ereignisse}} \, .
\end{equation}

Bei kontinuierlichen Ereignismengen (z.B.\ dem
oben angesprochenen mathematisch idealisierten
Dart-Spiel) k\"onnen wir den Einzelereignissen
im Allgemeinen nur noch die Wahrscheinlichkeiten 
null zuschreiben, obwohl es nicht unm\"oglich 
ist, einen bestimmten Punkt zu treffen.
In einem solchen Fall k\"onnen wir nur 
Mengen von Elementarereignissen eine
nicht-verschwindende
Wahrscheinlichkeit zuschreiben. Diese Mengen
bestehen auch nicht aus endlich vielen oder abz\"ahlbar
unendlich vielen einzelnen Punkten, sondern
sind typischerweise zusammenh\"angende Gebiete.

\subsection{Borel-Mengen bzw.\ Borel-Algebren}

W\"ahrend man in der Physik in den meisten F\"allen
damit zufrieden ist, dass solche Gebiete ein
endliches Ma\ss\ haben (je nach Ereignismenge
eine endliche L\"ange bzw.\ Dauer, Fl\"ache, Volumen, 
etc.), m\"ochte man in der Mathematik diese Gebiete
genauer charakterisieren k\"onnen. Insbesondere
zeigt sich, dass es Teilmengen von kontinuierlichen
Mengen gibt, denen man \"uberhaupt kein sinnvolles
Ma\ss\ zuordnen kann. 

Zur Umgehung dieses Problems definiert man\index{Ereignismenge}
zu einer {\em Ereignismenge} $\Omega$ eine bestimmte 
Klasse von Teilmengen, die bez\"uglich der Bildung
von Komplementen sowie abz\"ahlbaren 
Vereinigungen abgeschlossen ist. Eine solche 
Menge von Teilmengen bezeichnet man als\index{Sigma@$\sigma$-Algebra}
$\sigma$-Algebra.\footnote{Der Begriff der Algebra mag
hier \"uberraschen. In der Mathematik bezeichnet man
ganz allgemein eine Menge (oder einen Satz von Mengen),
auf der mehr als eine Verkn\"upfungsrelation definiert ist,
als eine Algebra.} Eine $\sigma$-Algebra bezeichnet
man als Borel-Menge oder Borel-Algebra, wenn sie die\index{Borel-Algebra} 
offenen Teilmengen von $\Omega$ enth\"alt. Insbesondere
enth\"alt $\sigma$ in diesem Fall auch $\Omega$ selbst
und die leere Menge. 

Die mathematischen Details bei allgemeinen
topologischen R\"aumen spielen hier keine
Rolle. Bei den reellen Zahlen bzw.\ dem
$\mathbb{R}^n$ sind die relevanten Teilmengen
alle Mengen, die man aus Intervallen bzw.\
offenen Kugeln durch abz\"ahlbar viele
Vereinigungen und Bildung von Komplementmengen
erh\"alt. Damit sind pathologischen F\"alle, beispielsweise
die \"Aquivalenzklassen von reellen Zahlen, deren Differenz eine
rationale Zahl ist, ausgeschlossen. Im Folgenden spreche ich
manchmal von \glqq sinnvollen\grqq\ Teilmengen,
wenn die Elemente einer Borel-Algebra gemeint sind.

\subsection{Das Axiomensystem von Kolmogorow}

Das Kolmogorow'sche Axiomensystem beschreibt
die Voraussetzungen, um von Wahrscheinlichkeiten
sprechen zu k\"onnen.\index{Kolmogorow-Axiome}\index{Wahrscheinlichkeitsraum}
\vspace{0.2cm}

\noindent
{\em Definition:} Ein Tripel $(\Omega, \sigma, \omega)$
hei\ss t Wahrscheinlichkeitsraum, wenn $\sigma$
eine Borel-Menge von $\Omega$ ist (also eine
Menge von sinnvollen Teilmengen von $\Omega$)
und $\omega: \sigma \longrightarrow \mathbb{R}$ eine
Abbildung, die folgende Bedingungen erf\"ullt:
\begin{eqnarray*}
     \omega(\Omega) &=&  1 \\
     \omega(A) & \geq & 0 ~~~ \mbox{f\"ur alle } A \in \sigma \\
     \omega \Big( \bigcup_i A_i \Big) &=& \sum_i \omega(A_i) ~~~~~  
     {\rm sofern} ~~ A_i \in \sigma ~~{\rm und} ~~ 
     A_i \cap A_j = \emptyset ~(i \neq j)  \, .
\end{eqnarray*}
$\omega$ hei\ss t Wahrscheinlichkeitsfunktion auf $\sigma$.\index{Wahrscheinlichkeitsfunktion}

Die erste Bedingung bedeutet, dass die Wahrscheinlichkeit
f\"ur das Ereignis $\Omega$ (also dass irgendein 
Elementarereignis aus $\Omega$ realisiert wird)
gleich 1 ist. Die zweite Bedingung bedeutet,
dass Wahrscheinlichkeiten positive Zahlen sind.
Die nicht-triviale Bedingung ist die dritte Bedingung: 
Seien $\{A_i\}$ h\"ochstens abz\"ahlbar viele und
paarweise disjunkte Teilmengen von $\sigma$, dann
ist die Wahrscheinlichkeit f\"ur das Eintreten eines 
Elementarereignisses
in irgendeiner dieser Ereignismengen gleich der Summe der
Wahrscheinlichkeiten, dass dieses Ereignis in einer
bestimmten dieser Mengen liegt. Anschaulich bedeutet
es: Die Wahrscheinlichkeit f\"ur das Auftreten eines 
von mehreren unabh\"angigen Ereignissen
ist gleich der Summe der Wahrscheinlichkeiten 
der einzelnen Ereignisse. Aus diesem Gesetz ergibt
sich z.B.\ unmittelbar die oben erw\"ahnte Laplace'sche Formel.


\section{Zufallsvariable und Kenngr\"o\ss en}

Die Natur des Ereignisraumes $\Omega$ kann sehr
unterschiedlich sein.
In den meisten F\"allen m\"ochten wir jedoch den 
Elementarereignissen Zahlen zuordnen und dann die Frage
stellen k\"onnen, mit welcher Wahrscheinlichkeit
bestimmte Zahlen auftreten.

Dazu definieren wir das Konzept einer
{\em Zufallsvariablen}:\index{Zufallsvariable}
\vspace{0.2cm}
 
 \noindent
 {\em Definition:} Eine {\em Zufallsvariable} $X$ zu einem
 Wahrscheinlichkeitsraum $(\Omega, \sigma, \omega)$ ist eine
 Abbildung $X: \Omega \longrightarrow \mathbb{R}$,
 sodass das Urbild jeder messbaren Menge aus $\mathbb{R}$
 in $\sigma$ liegt. 
 \vspace{0.2cm}
 
 Konkret bedeutet dies: Die Menge aller
 Elementarereignisse $a\in \Omega$, f\"ur die $X(a)$
 in einem bestimmten Intervall in $\mathbb{R}$ liegt,
 ist eine Menge, der man eine Wahrscheinlichkeit
 zuschreiben kann. Zu jeder Zufallsvariablen $X$ 
 definieren wir nun eine {\em Verteilungsfunktion} $P_X$
 auf $\mathbb{R}$:
\begin{equation}
    P_X: \mathbb{R} \rightarrow [0,1] ~~~ {\rm mit} ~~ 
     P_X(x) = \omega \Big( \{a \in \Omega| X(a) < x \} \Big) \, .
\end{equation}
$P(x)$ ist somit die Wahrscheinlichkeit, dass bei
einer Realisierung ein Ereignis $a$ auftritt, f\"ur
das die Zufallsvariable $X(a)$ einen Wert 
kleiner als $x$ hat. Formal k\"onnen wir nun einer
Zufallsvariablen eine Wahrscheinlichkeitsdichte
zuordnen, indem wir die Ableitung von $P(x)$
betrachten:\index{Wahrscheinlichkeitsdichte}
\begin{equation}
    w_X(x) = \frac{{\rm d}P_X}{{\rm d}x}  \, .
\end{equation}
Allerdings ist diese Wahrscheinlichkeitsdichte nicht
immer eine gew\"ohnliche Funktion auf den reellen
Zahlen sondern eher eine Distribution, d.h., 
sinnvoll sind nur Integrale \"uber messbare Teilmengen
von $\mathbb{R}$. 

Wie in der Physik oft \"ublich, werde ich nicht immer
explizit von einem Wahrscheinlichkeitsraum $(\Omega, \sigma,\omega)$ 
sowie einer Zufallsvariablen $X$ und ihrer Verteilungsfunktion
$P_X$ sprechen, sondern einfacher von einer (reellen) Variablen $x$
und ihrer Wahrscheinlichkeitsdichte $w(x)$. Ganz grob
m\"ochte ich die Zusammenh\"ange angeben, mit denen wir
es in der Statistischen Mechanik zu tun haben werden:
$\Omega$ wird der Raum aller Mikrozust\"ande sein, das\index{Mikrozustand}
ist klassisch der Phasenraum aller beteiligten Teilchen
und in der Quantenmechanik meist die Menge der Eigenzust\"ande
zum Energieoperator. $\sigma$ besteht aus (sinnvollen)
Teilmengen im Phasenraum und $\omega$ ist eine
Wahrscheinlichkeit (in der QM) oder eine 
Wahrscheinlichkeitsdichte (\"uber dem klassischen
Phasenraum), die angibt, mit welcher
Wahrscheinlichkeit ein bestimmter Mikrozustand
vorliegt. Eine Zufallsvariable ist dann eine Observable --\index{Observable}
beispielsweise die Energie -- auf dem Phasenraum:
Sie ordnet jedem Mikrozustand eine Zahl zu. 
$w$ ist die Wahrscheinlichkeitsdichte f\"ur das
Auftreten eines bestimmten Messwerts, wenn an einem
konkreten System diese Observable gemessen wird. 

Bei einem diskreten System ($\Omega$ endlich oder abz\"ahlbar)
besteht $\sigma$ aus allen Teilmengen von $\Omega$ und
enth\"alt damit auch die Elementarereignisse
$i\in \Omega$. Damit gen\"ugt es bei diskreten Systemen,
f\"ur jedes Elementarereignis $i$ seine Wahrscheinlichkeit
$\omega(i)$ anzugeben. Die Wahrscheinlichkeit f\"ur
ein Ereignis $A\subset \Omega$ ist dann einfach:
\begin{equation}
      \omega (A) = \sum_{i \in A} \omega(i) \, . 
\end{equation}
Da es sich nun um diskrete Ereignisse handelt,
wird die Verteilung $P$ zu einer Stufenfunktion
(die an einer Stelle $x$ um einen diskreten Sprung 
zunimmt, wenn es ein Ereignis $a$
gibt, sodass $X(a)=x$). Die Ableitung $w(x)$
wird damit zu einer Summe \"uber $\delta-$Funktionen.
Das ist auch richtig, wenn man die Summe \"uber
diskrete Ereignisse als Integral schreiben m\"ochte.
Im Allgemeinen ist das aber nicht notwendig: Man
spricht von Wahrscheinlichkeiten und berechnet
diese durch die Bildung von Summen \"uber
diskrete Ereignismengen.

Ist eine Wahrscheinlichkeitsdichte gegeben,
k\"onnen wir verschiedene Kenngr\"o\ss en
bestimmen. Diese charakterisieren eine
Verteilung und reichen oftmals f\"ur einfache
Zusammenh\"ange aus. F\"ur eine allgemeine
(messbare, d.h.\ lokal integrierbare) Funktion $f(x)$ kann man den 
Erwartungswert berechnen:\index{Mittelwert}\index{Erwartungswert}
\begin{equation}
     \langle f(x) \rangle = \int f(x) w(x)\, {\rm d}x \, .
\end{equation}
Besondere Erwartungswerte sind der
{\em Mittelwert}
\begin{equation}
     \bar{x} := \langle x \rangle =  \int x w(x)\, {\rm d}x 
\end{equation}
und die {\em Varianz}\index{Varianz}
\begin{equation}
     \sigma_x^2 := \langle (x - \bar{x})^2  \rangle 
     = \int (x-\bar{x})^2 w(x)\, {\rm d}x \, .
\end{equation}
Die Wurzel aus der Varianz, also $\sigma_x$, 
bezeichnet man
als {\em Standardabweichung}.\index{Standardabweichung}
Allgemein sind die {\em Momente}\index{Moment einer Verteilung}
der Verteilung durch
\begin{equation}
     \langle x^n \rangle = \int x^n w(x)\, {\rm d}x 
\end{equation}
gegeben, f\"ur die man auch die {\em erzeugende
Funktion}\index{erzeugende Funktion}
\begin{equation}
     \tilde{f}(k) = \langle {\rm e}^{{\rm i}kx} \rangle = \int 
     {\rm e}^{{\rm i}kx} w(x)\, {\rm d}x 
\end{equation}
definiert. Die Momente erh\"alt man aus den 
Ableitungen von $\tilde{f}$:
\begin{equation}
     \langle x^n \rangle = (-{\rm i})^n \frac{{\rm d}^n \tilde{f}(k)}{{\rm d}k^n}
     \Big|_{k=0} \, .
\end{equation}
Bis auf Faktoren ist $\tilde{f}(k)$ die Fourier-Transformierte
der Wahrscheinlichkeitsdichte $w(x)$. 

Bei mehreren Zufallsvariablen $X_i$ ($i=1,...,n$) 
definiert man eine gemeinsame Verteilungsfunktion
\begin{equation}
     P(x_1,...,x_n) = \omega \Big( \{ a \in \Omega | \,
       X_1(a)<x_1 \, , \, X_2(a)<x_2 \, , ..., \, X_n(a)<x_n\} \Big)  \, ,
\end{equation}
und durch Ableitungen nach $x_i$ ergibt sich die 
gemeinsame Wahrscheinlichkeitsdichte
\begin{equation}
    w(x_1,...,x_n) = \frac{{\rm d}^n P(x_1,...,x_n)}{{\rm d}x_1 \cdots {\rm d}x_n} \, .
\end{equation}
Daraus erh\"alt man z.B.\ die {\em Kovarianzmatrix}\index{Kovarianzmatrix}
\begin{equation}
     C_{ij} = \int (x_i - \bar{x}_i)(x_j - \bar{x}_j) \, w(x_1,...,x_n)\, {\rm d}^n x \, ,
\end{equation}
die ein guter Indikator f\"ur Korrelationen zwischen den
beiden Variablen $x_i$ und $x_j$ ist. Die Diagonale der
Kovarianzmatrix enth\"alt die Varianzen zu den einzelnen
Gr\"o\ss en.

Gilt f\"ur die Wahrscheinlichkeitsdichte von zwei Gr\"o\ss en 
$x$ und $y$ die Beziehung
\begin{equation}
     w(x,y) = w(x) w(y) \, ,
\end{equation}
so bezeichnet man $x$ und $y$ als {\em statistisch unabh\"angig}.\index{statistisch unabh\"angig}
In diesem Fall faktorisieren alle Erwartungswerte
von Produkten der beiden Gr\"o\ss en zu Produkten
von den entsprechenden Erwartungswerten. Insbesondere
ist die Kovarianzmatrix eine Diagonalmatrix, d.h., die
Nicht-Diagonalelemente sind null. Allerdings gilt im
Allgemeinen die Umkehrung nicht: Wenn die Kovarianzen
verschwinden bedeutet das nicht notwendigerweise, dass
die zugeh\"origen Gr\"o\ss en statistisch unabh\"angig sind.

\subsection{Die Gau\ss-Verteilung}

Unter den kontinuierlichen Wahrscheinlichkeitsverteilungen
spielt die Gau\ss-Verteilung eine besondere Rolle.\index{Gauss@Gau\ss-Verteilung}
Sie ist durch die Angabe ihres Mittelwerts $\bar{x}$ und
ihrer Standardabweichung $\sigma_x$ bzw.\ ihrer
Varianz $\sigma_x^2$ festgelegt:
\begin{equation}
     w(x) = \frac{1}{\sqrt{2\pi \sigma_x^2}} \exp \left( 
     - \frac{(x-\bar{x})^2}{2 \sigma_x^2} \right) \, .
\end{equation}

Allgemeiner kann man f\"ur eine positive
symmetrische Matrix $A_{ij}$ (positiv bedeutet in diesem
Fall, dass s\"amtliche Eigenwerte positiv sind) die
Verteilungsfunktion 
\begin{equation}
     w(x_1,...,x_n) = \frac{\sqrt{{\rm det}\,A}}{(2\pi)^{n/2}} \exp \left( 
     - \frac{1}{2} \sum_{i,j=1}^n (x_i-\bar{x}_i) A_{ij} (x_j -\bar{x}_j) \right) 
\end{equation}
definieren. Die Kovarianzmatrix dieser Verteilung ist
\begin{equation}
          C_{ij} = (A^{-1})_{ij} \, ,
\end{equation}
wobei $A^{-1}$ die 
inverse Matrix zu $A$ ist. 

\section{Grenzwerts\"atze}

In der Wahrscheinlichkeitstheorie unterscheidet man viele 
Arten von Grenzwerts\"atzen, die h\"aufig auch noch 
in starker und schwacher Form vorkommen 
k\"onnen. Ihnen allen gemein ist, dass Aussagen
\"uber die statistischen Eigenschaften von Funktionen von 
sehr vielen Zufallsvariablen (im Grenzfall unendlich
vieler Zufallsvariablen) gemacht werden. An dieser
Stelle m\"ochte ich nur auf zwei Grenzwerts\"atze eingehen,
und auch bei diesen nur eine schwache Version
angeben, die f\"ur physikalische Anwendungen 
ausreicht: das \glqq Gesetz der gro\ss en Zahlen\grqq\
und den \glqq zentralen Grenzwertsatz\grqq.  Beide
Grenzwerts\"atze machen Aussagen zu der
Wahrscheinlichkeitsverteilung der Summe bzw.\ dem
Mittelwert von vielen Zufallszahlen. 

\subsection{Das Gesetz der gro\ss en Zahlen}

Das Gesetz der gro\ss en Zahlen besagt, dass es\index{Gesetz der gro\ss en Zahlen}
f\"ur den Mittelwert von Zufallsvariablen
zunehmend (mit wachsender Anzahl der Zufallsvariablen)
unwahrscheinlicher wird, von dem Durchschnitt der
Mittelwerte dieser Zufallsvariablen wesentlich 
abzuweichen. Konkretisiert werden nun die
Begriffe \glqq zunehmend unwahrscheinlicher\grqq\ 
und \glqq wesentlich abweichen\grqq. 

Generell gilt f\"ur die Summe von zwei
Zufallsvariaben $S=X_1+X_2$, dass ihr
Mittelwert gleich der Summe der Mittelwerte
von $X_1$ und $X_2$ ist. Das folgt unmittelbar
aus der Definition des Mittelwerts und den
Normierungseigenschaften der
Wahrscheinlichkeitsdichten:
\begin{eqnarray}
      \bar{s} &=& \int (x_1+x_2) w(x_1,x_2)\, {\rm d}^2 x \\
       &=& \int x_1 w(x_1,x_2) {\rm d} x_1 {\rm d}x_2 +
       \int x_2 w(x_1,x_2) {\rm d} x_1 {\rm d}x_2 =
       \bar{x}_1 + \bar{x}_2 \, .      
\end{eqnarray}
Im Folgenden betrachten wir den Mittelwert $M$
und die einfache Summe $S$ von $N$ Zufallsvariablen:
\begin{equation}
       M = \frac{1}{N} \sum_i X_i 
       ~~~~ {\rm und} ~~~~
       S = \sum_i X_i \, ,
\end{equation}
die sich nur in dem Normierungsfaktor $1/N$
unterscheiden. Beide Gr\"o\ss en sind selbst wieder
Zufallsvariable. F\"ur ihre Mittelwerte gilt allgemein:
\begin{equation}
       \bar{m} = \frac{1}{N} \sum_i \bar{x}_i 
       ~~~~ {\rm und} ~~~~ 
      \bar{s} = \sum_i \bar{x}_i    \,  .
\end{equation}
Bisher haben wir noch keine Annahme \"uber
die Varianzen oder die statistische
Unabh\"angigkeit der Zufallsvariablen gemacht.
Nun betrachten wir einen Satz $X_1,...,X_N$ von
$N$ Zufallsvariablen, die verschiedene 
Verteilungsfunktionen $P_i(x_i)$ und
damit verbundene Wahrscheinlichkeitsdichten
$w_i(x_i)$ mit 
zugeh\"origen Mittelwerten $\bar{x}_i$ haben
k\"onnen, deren Varianzen $\sigma_{x_i}^2$ 
aber endlich
und durch eine Konstante beschr\"ankt sein
sollen. Au\ss erdem nehmen wir an, dass die
Zufallsvariablen statistisch unabh\"angig sein sollen.
Das bedeutet, die gemeinsame Wahrscheinlichkeitsdichte
faktorisiert:
\begin{equation}
   w(x_1,...,x_N) = w_1(x_1) w_2(x_2) \cdots w_N(x_N) \, .
\end{equation}
Die eigentliche Aussage des Gesetzes der
gro\ss en Zahlen ist, dass die Wahrscheinlichkeit,
dass der Mittelwert von $N$ Zufallszahlen um
mehr als ein vorgegebenes $\epsilon$ von dem
Mittelwert der Mittelwerte abweicht, im Grenzfall
$N\rightarrow \infty$ gegen null geht. Es wird also
f\"ur gen\"ugend gro\ss e Werte von $N$ beliebig
unwahrscheinlich, dass der Mittelwert der Summe 
um mehr als eine beliebig kleine vorgegebene 
Konstante vom Mittelwert der Mittelwerte abweicht.
Dieses Gesetz der gro\ss en Zahlen ist
gleichzeitig ein Beweis f\"ur die h\"aufig
axiomatisch eingef\"uhrte Behauptung, dass die
relative H\"aufigkeit bei gen\"ugend vielen
Realisierungen eines Ereignisses gegen die
Wahrscheinlichkeit f\"ur dieses Ereignis geht.

F\"ur den Beweis machen wir die vereinfachende
Annahme, dass alle Mittelwerte verschwinden; dies 
l\"asst sich immer durch eine additive Verschiebung der
Zufallsvariablen erreichen. Damit verschwinden
nicht nur die Einzelmittelwerte $\bar{x}_i=0$
sondern auch der Gesamtmittelwert $\bar{m}=0$.
Weiterhin seien alle Varianzen 
$\sigma_i$ durch eine Konstante $C$
beschr\"ankt, also $\sigma_i \leq C$.
Wir betrachten nun die Varianz des Mittelwerts
der Zufallsvariablen:
\begin{equation}
  \sigma_m^2 = \left\langle \left( \frac{1}{N} \sum_i x_i
  \right)^2 \right\rangle = \frac{1}{N^2} 
  \left\langle \sum_{i,j} x_i x_j \right\rangle 
  = \frac{1}{N^2} \sum_{i,j} \langle x_i x_j \rangle \, .
\end{equation}
Nun nutzen wir die Unabh\"angigkeit der
Zufallsvariablen aus, d.h.\ f\"ur $i \neq j$ gilt
$\langle x_i x_j\rangle =0$. Damit erhalten wir:
\begin{equation}
  \sigma_m^2 = \frac{1}{N^2} \sum_i
  \langle  x_i^2 \rangle
   = \frac{1}{N^2} \sum_i \sigma_i^2 \leq  \frac{1}{N} C  \, .
\end{equation}
Im Grenzfall $N\rightarrow \infty$ geht die rechte
Seite gegen null, d.h., die Varianz zum Mittelwert
der Zufallsvariablen wird in diesem Grenzfall
kleiner als jedes vorgegebene $\epsilon$. 
F\"ur die Summe der Zufallsvariablen erhalten
wir entsprechend:
\begin{equation}
  \sigma_s^2 = \sum_i 
 \langle  x_i^2  \rangle
   =\sum_i  \sigma_i^2 \, .
\end{equation}

Hinter dieser Formulierung des Gesetzes
der gro\ss en Zahlen stecken mehrere wichtige
Anwendungen. Die Gleichsetzung von relativer
H\"aufigkeit und Wahrscheinlichkeit haben wir
schon erw\"ahnt. F\"ur die Standardabweichung
(Wurzel aus der Varianz) der Summe von 
Zufallsvariablen finden wir nach obiger
Herleitung:
\begin{equation}
     \sigma_s = \sqrt{ \sum_i \sigma_i^2} \, . 
\end{equation}
Dies ist das bekannte Fehlerfortpflanzungsgesetz:\index{Fehlerfortpflanzungsgesetz}
Setzt sich der Gesamtfehler bei einer Messung
als Summe von Einzelfehlern zusammen, die
als unkorreliert angenommen werden k\"onnen,
dann ergibt sich der Gesamtfehler
aus der Summe der Quadrate der Einzelfehler. 

\subsection{Der zentrale Grenzwertsatz}

W\"ahrend das Gesetz der gro\ss en Zahlen\index{Grenzwertsatz, zentraler}
eine Aussage \"uber die Varianz des Mittelwerts
bzw.\ der Summe von vielen Zufallszahlen
macht, geht der zentrale Grenzwertsatz
noch einen Schritt weiter: Er macht eine
Aussage \"uber die Verteilungsfunktion der
Summe bzw.\ des Mittelwerts von Zufallszahlen:\\[0.2cm]
F\"ur $N$ statistisch unabh\"angige Zufallszahlen 
$X_i$ mit Varianzen $\sigma_i$, die alle endlich
und durch eine Konstante beschr\"ankt sein 
sollen, wird die
Verteilungsfunktion des Mittelwerts (bzw.\
der Summe) zu einer Gau\ss verteilung
mit den durch das Gesetz der gro\ss en Zahlen
gegebenen Mittelwerten und Standardabweichungen.
\vspace{0.2cm}

Etwas pr\"aziser behauptet der zentrale
Grenzwertsatz: F\"ur die normierte Variable
\begin{equation}
       Z = \frac{S-Nm}{\sigma_m \sqrt{N}}  
\end{equation}
($S$, $m$ und $\sigma_m$ wie oben) wird
die Verteilungsfunktion im Grenzfall $N\rightarrow
\infty$ zur Verteilungsfunktion der
Standardnormalverteilung, also\index{Standardnormalverteilung}
\begin{equation}
    \lim_{N\rightarrow \infty} P(Z<z) = \Phi(z) 
\end{equation}
wobei 
\begin{equation}
     \Phi(z) =  \frac{1}{\sqrt{2\pi}} \int_{-\infty}^z
       {\rm e}^{- \frac{1}{2} t^2} {\rm d} t 
\end{equation}
die Verteilungsfunktion der Standardnormalverteilung
ist. (Die Standardnormalverteilung hat den
Mittelwert $0$ und die Standardabweichung $1$). 

Die herausragende Bedeutung der
Normalverteilung (also der Gau\ss-Verteilung)
ergibt sich unter anderem aus diesem Satz.
Unabh\"angig von den Einzelverteilungen 
irgendwelcher Zufallsvariaber wird die 
Verteilung ihrer Summe zu einer Gau\ss-Verteilung.
Die beiden getroffenen Annahmen -- die
statistische Unabh\"angigkeit der Zufallsvariablen
und die Beschr\"anktheit ihrer Varianzen -- sind
dabei wesentlich. Viele thermodynamische
Gr\"o\ss en sind die Summe mikroskopischer
Gr\"o\ss en. In manchen F\"allen (beispielsweise
an sogenannten Phasen\"uberg\"angen) 
sind diese mikroskopischen Gr\"o\ss en
derart korreliert, dass der zentrale Grenzwertsatz
nicht mehr g\"ultig ist. Tats\"achlich treten in 
diesen F\"allen oft Nicht-Gau\ss'sche 
Verteilungsfunktionen auf. 

Auch die Endlichkeit
der Varianzen ist wichtig: Es gibt normierbare
Wahrscheinlichkeitsdichten, beispielsweise\index{Lorentz-Verteilung}\index{Cauchy-Verteilung}
die Lorentz- bzw.\ Cauchy-Verteilung 
\begin{equation}
  w(x) = \frac{1}{\pi} \frac{a}{x^2 + a^2} \, ,
\end{equation}
mit einem endlichen Mittelwert, deren
Varianz aber unendlich ist. In diesen F\"allen
sind auch sehr gro\ss e Schwankungen
in den Realisationen der Zufallsvariablen
nicht selten. Bei Zufallsvariablen mit solchen
Verteilungen muss die Verteilung der Summe 
dieser Zufallsvariablen nicht gegen
eine Gau\ss-Funktion gehen. 

Eine Variante des zentralen Grenzwertsatzes
ist die Brown'sche Bewegung bzw.\ der
Random Walk. Die Zufallsvariablen $X_i$ sind
dabei Verteilungen f\"ur Einzelschritte, und
die Wahrscheinlichkeitsdichte der Summe $S$ 
beschreibt die Wahrscheinlichkeit, ein im
Ursprung gestartetes Teilchen nach 
$N$ Schritten an einem Punkt $x$ bzw.\
in einem Abstand $r$ zu finden.\index{Wahrscheinlichkeit|)} 

%\end{document}
