\documentclass[german,10pt]{book}      
\usepackage{makeidx}
\usepackage{babel}            % Sprachunterstuetzung
\usepackage{amsmath}          % AMS "Grundpaket"
\usepackage{amssymb,amsfonts,amsthm,amscd} 
\usepackage{mathrsfs}
\usepackage{rotating}
\usepackage{sidecap}
\usepackage{graphicx}
\usepackage{color}
\usepackage{fancybox}
\usepackage{tikz}
\usetikzlibrary{arrows,snakes,backgrounds}
\usepackage{hyperref}
\hypersetup{colorlinks=true,
                    linkcolor=blue,
                    filecolor=magenta,
                    urlcolor=cyan,
                    pdftitle={Overleaf Example},
                    pdfpagemode=FullScreen,}
%\newcommand{\hyperref}[1]{\ref{#1}}
%
\definecolor{Gray}{gray}{0.80}
\DeclareMathSymbol{,}{\mathord}{letters}{"3B}
%
\newcounter{num}
\renewcommand{\thenum}{\arabic{num}}
\newenvironment{anmerkungen}
   {\begin{list}{(\thenum)}{%
   \usecounter{num}%
   \leftmargin0pt
   \itemindent5pt
   \topsep0pt
   \labelwidth0pt}%
   }{\end{list}}
%
\renewcommand{\arraystretch}{1.15}                % in Formeln und Tabellen   
\renewcommand{\baselinestretch}{1.15}                 % 1.15 facher
                                                      % Zeilenabst.
\newcommand{\Anmerkung}[1]{{\begin{footnotesize}#1 \end{footnotesize}}\\[0.2cm]}
\newcommand{\comment}[1]{}
\setlength{\parindent}{0em}           % Nicht einruecken am Anfang der Zeile 

\setlength{\textwidth}{15.4cm}
\setlength{\textheight}{23.0cm}
\setlength{\oddsidemargin}{1.0mm} 
\setlength{\evensidemargin}{-6.5mm}
\setlength{\topmargin}{-10mm} 
\setlength{\headheight}{0mm}
\newcommand{\identity}{{\bf 1}}
%
\newcommand{\vs}{\vspace{0.3cm}}
\newcommand{\noi}{\noindent}
\newcommand{\leer}{}

\newcommand{\engl}[1]{[\textit{#1}]}
\parindent 1.2cm
\sloppy

         \begin{document}  \setcounter{chapter}{0}


\chapter{Axiomatischer Zugang zur Physik}
% Kap x
\label{chap_Axiom}

In der Mathematik ist ein Axiomensystem ein minimaler Satz von Aussagen,\footnote{Ich setze hier
voraus, dass man entscheiden kann, ob eine Folge von Symbolen eine wohl definierte Aussage
bildet oder nicht. In der Mathematik wird dies im Rahmen der Symbollogik gekl\"art.} die
als wahr definiert werden und aus denen durch festgelegte Schlussfolgerungsregeln
weitere Aussagen als wahr oder falsch bewiesen werden k\"onnen. Die wesentlichen
Bedingungen an ein Axiomensystem sind 
\begin{itemize}
\item[(a)]
die Widerspruchsfreiheit, d.h., es soll keine
Aussage geben, von der man beweisen kann, dass sie sowohl wahr als auch
falsch ist, und 
\item[(b)] 
die Unabh\"angigkeit der Axiome, d.h., es soll nicht m\"oglich sein, f\"ur eines
oder mehrere der Axiome unter Verwendung der anderen Axiome beweisen zu k\"onnen, 
dass sie richtig sind. 
\end{itemize}
Eine dritte Eigenschaft, die
Vollst\"andigkeit, kann nach den Unvollst\"andigkeitss\"atzen von G\"odel nicht immer garantiert werden.
\glqq Vollst\"andig\grqq\ bedeutet dabei, dass von jeder formulierbaren Aussage bewiesen
werden kann, ob sie richtig oder falsch ist. Dar\"uber hinaus spielt gelegentlich in der Praxis
eine Rolle, ob ein Axiomensystem reichhaltig bzw.\ interessant ist, d.h., ob sich viele
nicht selbstverst\"andliche Aussagen daraus ableiten lassen und ob diese Aussagen
zur Definition von interessanten Strukturen Anlass geben. 

Es gibt  viele Axiomensysteme in der Physik: die drei Newton'schen Postulate der Newton'schen
Mechanik, die Maxwell-Gleichungen zusammen mit der Lorentz-Kraft in der Elektrodynamik,
die Axiome der speziellen Relativit\"atstheorie (insbesondere die Konstanz der Lichtgeschwindigkeit
in allen Inertialsystemen unabh\"angig vom Bewegungszustand der Quelle) und der allgemeinen
Relativit\"atstheorie, die Haupts\"atze der Thermodynamik etc. Oft spricht man in der Physik
eher von Postulaten als von Axiomen, obwohl die Bedeutungen dieser beiden Terme kaum
zu trennen sind. 

Allgemein sind Postulate Aussagen, die man nicht mehr auf einfachere Aussagen
zur\"uckf\"uhren kann und die man als gegeben und richtig annimmt. In der Physik sollen
diese Aussagen und die aus ihnen ableitbaren Schlussfolgerungen nicht im Widerspruch
zum Experiment stehen. Oftmals handelt es sich sogar um Erfahrungstatsachen (z.B.\
die Konstanz der Lichtgeschwindigkeit in der SRT oder die Existenz einer Zustandsgr\"o\ss e
\glq Temperatur\grq\ in der Thermodynamik), die nicht unbedingt in jeder denkbaren Welt
gelten m\"ussten.   
Au\ss erdem sollen diese Aussagen untereinander nicht widerspr\"uchlich sein. 

In dieser Form k\"onnen Aussagen im Laufe der Zeit durchaus ihren Charakter als
Postulate verlieren: Aus heutiger Sicht k\"onnte man sagen, dass zu Keplers Zeit 
die Kepler'schen Gesetze den Charakter von Postulaten hatten - aus der Empirie gewonnenen 
Regelm\"a\ss igkeiten, die sich nicht weiter begr\"unden lassen. Nachdem Newton seine
Gesetze formuliert hatte (die nun den Charakter von Postulaten hatten), konnte man die
Kepler'schen Gesetze aus den Newton'schen Gesetzen sowie dem Newton'schen 
Gravitationsgesetz ableiten. Das Newton'sche Gravitationsgesetz l\"asst sich wiederum aus den
Grundgleichungen der Relativit\"atstheorie ableiten.

Ganz allgemein wird man in der Physik zwei Arten von Postulaten unterscheiden: Zum Einen
sind dies die Anforderungen, die man \"uberhaupt an eine physikalische Theorie bzw.\ ein
physikalisches Modell stellen muss, damit man diese Theorie als widerspruchsfreie Theorie
bezeichnen und in dieser Form akzeptieren kann. In Anlehnung an die Philosophie (und die
Kategorientheorie der Mathematik) k\"onnte man hier auch von Kategorien sprechen, die
erf\"ullt sein m\"ussen, damit wir \"uberhaupt von Physik sprechen k\"onnen. 
Zum anderen sind das Beobachtungstatsachen, die man nicht mehr weiter
erkl\"aren kann und daher einer Theorie als Postulat zugrunde legt. Zur ersten Gruppe kann
man beispielsweise das Postulat z\"ahlen, was \"uberhaupt (reine) physikalische
Zust\"ande sind, zur zweiten Klasse
kann man das Postulat der Speziellen Relativit\"atstheorie z\"ahlen, dass sich Licht f\"ur
jeden inertialen Beobachter unabh\"angig vom Bewegungszustand der Quelle mit Lichtgeschwindigkeit
ausbreitet. 

In diesem Kapitel geht es haupts\"achlich um die erste Klasse von Postulaten, die allen
Theorien in der Physik gemein sind bzw., die jede Theorie spezifizieren muss, damit man sie
als Theorie bezeichnen kann. 

\section{Allgemeiner Formalismus}

In jeder physikalischen Theorie gibt es zwei Konzepte, auf denen die Theorie beruht:
(1) das Konzept der Observablen und (2) das Konzept des Zustands. Neben der Spezifizierung
dieser Konzepte muss ein allgemeiner Formalismus angeben, wie man aus diesen beiden
Objekten zu physikalischen Vorhersagen gelangt, und wie man ein System in einem 
bestimmten Zustand pr\"apariert bzw.\ woher man wei\ss, durch welchen Zustand ein System
zu beschreiben ist. Schlie\ss lich muss eine Theorie noch angeben, wie die
zeitliche Entwicklung des Systems beschrieben werden kann.

\subsection{Observable}

\textbf{Definition:} \textit{Observable sind physikalische Gr\"o\ss en, 
die man an einem physikalischen System messen kann.}

 John von Neumann
definiert ein physikalisches System durch die Menge der Observablen, die an diesem
System beobachtet werden k\"onnen \cite{Neumann}. Welche Observablen das sind,
ist eine Erfahrungstatsache. Erst durch das Ergebnis des Stern-Gerlach-Experiments 1922
erkannte man, dass es neben Ort und Impuls noch einen weiteren Freiheitsgrad gibt, den man an einem
Elektron messen kann, n\"amlich seine Spin\-orien\-tierung entlang einer vorgegebenen Achse. 
Man kann daher nie behaupten, s\"amtliche an einem System messbaren Observablen zu kennen, bzw.\ es 
kann nie ausgeschlossen werden, dass in Zukunft noch weitere Observablen bekannt werden.
Ein allgemeiner physikalischer Formalismus muss f\"ur eine Theorie zun\"achst
kl\"aren, welche Observablen es gibt bzw.\ bekannt sind und wie diese Observablen mathematisch repr\"asentiert
werden sollen. Observablen bilden jedoch nicht einfach nur eine strukturlose Menge, sondern es gibt auch 
Beziehungen zwischen den Observablen, die bekannt sein sollten und sich in der mathematischen Darstellung
widerspiegeln m\"ussen.

Die physikalische Realisierung einer Observablen erfolgt durch die Angabe des Messprotokolls, wie
eine Messung dieser Observablen durchzuf\"uhren ist. Es ist eine Abfolge von Schritten, die angeben, in
welcher Form ein System mit einer Messapparatur in Wechselwirkung tritt und wie die 
Ver\"anderungen an der Messapparatur (z.B.\ die Zeigerstellung) abgelesen wird. Aus diesen
Ver\"anderungen muss sich eine Zahl bestimmen lassen, die man als das Ergebnis der Messung bezeichnet,
und die eine Aussage \"uber den Zustand eines Systems - wie es pr\"apariert wurde - macht.
In diesem Kapitel bezeichne ich die physikalische Observable
als Messvorschrift in Anlehnung an von Neumann \cite{Neumann} mit gotischen Buchstaben, z.B.\
${\cal R}$, die zugeh\"orige mathematische Darstellung dieser Observablen kennzeichne ich 
durch $R$. Sehr oft wird aber nicht zwischen diesen beiden Objekten unterschieden und es wird
erwartet, dass man im Einzelfall entscheiden kann, ob die physikalische Messvorschrift oder
die mathematische Darstellung gemeint ist.

Kennt man eine Observable ${\cal R}$, also die Vorschrift, wie man diese Observable an einem System 
messen kann, so kennt man auch die Observable $f({\cal R})$ f\"ur eine Funktion $f$: Man misst die
Observable ${\cal R}$ an einem physikalischen System und bildet von dem erhaltenen Messwert $r$ die
Funktion $f(r)$. Insbesondere ist die Observable $\alpha {\cal R}$ (f\"ur eine beliebige reelle Zahl $\alpha$)
durch die Messvorschrift von ${\cal R}$ gegeben, wobei jeder gemessene (an einer Zeigerstellung
abgelesene Messwert) mit $\alpha$ multipliziert wird. Lassen sich zwei Observable ${\cal R}$ und ${\cal S}$
gleichzeitig an einem System messen (das bedeutet, es gibt ein Protokoll, bei dem man gleichzeitig
einen Messwert sowohl f\"ur ${\cal R}$ als auch f\"ur ${\cal S}$ erh\"alt - diese Werte seien $r$ und $s$), 
dann ist 
die Observable $f({\cal R},{\cal S})$ definiert als die  Messvorschrift, bei der von den Messwerten die Funktion
$f(r,s)$ gebildet wird. 

Bei Observablen ${\cal R}$ und ${\cal S}$, die sich nicht gleichzeitig an einem System messen lassen - solche
Observable bezeichnet man als nicht kompatibel -, sind
allgemeine Funktionen $f({\cal R},{\cal S})$ zun\"achst nicht definiert. Insbesondere sind auch Ausdr\"ucke der Art
${\cal R}+{\cal S}$ oder ${\cal R}\cdot {\cal S}$ als Messvorschriften nicht definiert.   

Die mathematische Darstellung (Repr\"asentation) einer Observablen h\"angt von der Theorie
ab, die wir verwenden. In der klassischen Mechanik handelt es sich bei Observablen um 
Funktionen von Ort und Impuls (bzw.\ Geschwindigkeit), d.h.\ Funktionen auf dem
Phasenraum. In der Quantenmechanik
werden Observable durch sogenannte selbst-adjungierte bzw.\ hermitesche Operatoren auf einem
Hilbert-Raum - einem Vektorraum mit einem Skalarprodukt - dargestellt. 

An dieser Stelle sollte man betonen,
dass die mathematische Darstellung einer Observablen nicht den Messprozess repr\"asentiert, also
den dynamischen Vorgang der Messung, sondern eher die Informationen, die man durch solche
Messungen erlangen kann: die Menge der m\"oglichen Messwerte sowie die Zust\"ande, die mit
diesen Messwerten verbunden sind. In diesem Sinne (und in Anlehnung an einen Ausdruck von
Schr\"odinger \cite{Schroedinger} zum Begriff der Wellenfunktion) 
kann man von einer Observablen als einem \glqq Katalog von
m\"oglichen Ergebnissen\grqq\ sprechen. 


\subsection{Zustand}

Auch bei den Zust\"anden sollte man zwischen der physikalischen Realisierung und der
mathematischen Darstellung unterscheiden. Sehr oft definiert man einen Zustand als
ein \glqq Erwartungswertfunktional auf der Menge der Observablen\grqq. Diese zun\"achst
mathematische Definition kann man durchaus w\"ortlich interpretieren: Wenn wir ein System
durch einen Zustand beschreiben, erlaubt es dieser Zustand, jeder Observablen eine Zahl - ihren
Erwartungswert in diesem Zustand - zuzuordnen. Ein Zustand beschreibt also unsere Erwartungen
in Bezug auf die Ergebnisse, die bei einer Messung an einem System auftreten. 
Schr\"odinger bezeichnet einen Zustand als einen \glqq Katalog von Erwartungen\grqq\ \cite{Schroedinger}. 

Dieser Katalog von Erwartungen beruht auf unserem Wissen 
\"uber die Vergangenheit eines Systems bzw.\ eines Ensembles von Systemen. 
Im Wesentlichen bezieht sich dieses Wissen auf die Art, wie diese Systeme pr\"apariert wurden. 
Unsere Erwartungen in Bezug auf m\"ogliche Messergebnisse k\"onnen nur auf diesem Wissen
beruhen, und wenn uns bewusst ist, dass unser Wissen unvollst\"andig ist, verwenden
wir sogenannte gemischte Zust\"ande (in der klassischen Mechanik sind das Wahrscheinlichkeitsverteilungen
\"uber einem Zustandsraum, beispielsweise dem Phasenraum; in der Quantentheorie sind das
sogenannte Dichtematrizen). Unser Wissen \"uber die Vergangenheit
eines Sys\-tems muss nicht beliebig weit zur\"uckreichen: Es reicht das Wissen \"uber solche Aspekte, die
f\"ur die Vorhersagen zuk\"unftiger Messungen von Bedeutung sind. Das sind meist die letzten
Pr\"aparationen, die an einem System vorgenommen wurden.\footnote{Die Physik ist hier in einer
gl\"ucklichen Lage. M\"ochte man in der Psychologie den mentalen Zustand einer Person beschreiben,
k\"onnen beliebig weit zur\"uckliegende Ereignisse oder Erfahrungen f\"ur die Vorhersage des
zuk\"unftigen Verhaltens wesentlich sein.} Es gibt aber auch physikalische Systeme mit 
einem \glqq Ged\"achtnis\grqq, beispielsweise Spinglassysteme oder neuronale Netzwerke. 

Die physikalische Realisierung eines Zustands erfolgt durch ein im Prinzip beliebig gro\ss es
Ensemble gleichartig pr\"aparierter Systeme. Gew\"ohnlich f\"uhren wir ein Experiment nicht nur
einmal durch sondern nach M\"oglichkeit unter denselben Ausgangsbedingungen gen\"ugend oft,
sodass wir statistische Auswerteverfahren verwenden k\"onnen. Au\ss erdem k\"onnen wir auf diese
Weise an demselben Zustand verschiedene Observable messen, die sich an demselben System
nicht gleichzeitig bestimmen lassen (beispielsweise Ort und Impuls bei quantenmechanischen
Systemen oder zwei nicht kompatible Observable ${\cal R}$ und ${\cal S}$). 
Man unterteilt dazu das Ensemble gleichartig pr\"aparierter Systeme in zwei
ebenfalls gro\ss e Teilensembles und misst an dem einen Teilensemble die Observable
${\cal R}$ und an dem anderen Teilensemble die Observable ${\cal S}$. Auf diese Weise erh\"alt man f\"ur
einen Zustand - das Ensemble gleichartig pr\"aparierter Systeme - die Erwartungswerte von
nicht kompatiblen Observablen in diesem Zustand.

Man bezeichnet einen Zustand -
realisiert als Ensemble - als rein, wenn es keine Unterteilung dieses Ensembles in (f\"ur statistische Auswertungen
ebenfalls gen\"ugend gro\ss e) Subensembles gibt, die andere
Erwartungswerte liefert als das gesamte Ensemble. Diese Vorstellung von reinen Zust\"anden entspricht
auch unserer \"ublichen Vorstellung von reinen
Zust\"anden als maximales bzw.\ nicht mehr erweiterbares Wissen \"uber ein System. Entspricht
ein Ensemble von Systemen einem Gemisch, so l\"asst es sich in Subensembles aufteilen, die
weniger gemischten Zust\"anden (oder sogar reinen Zust\"anden) entsprechen. Reine Zust\"ande
lassen sich nicht mehr in noch reinere Zust\"ande aufteilen.
 
Mathematisch wird ein Zustand repr\"asentiert durch die Angabe einer Abbildung, die
jeder Observablen eine Zahl - ihren Erwartungswert in diesem Zustand - zuordnet. 
Ein Zustand ist somit eine mathematische Kodierung unseres Wissens \"uber die Art, wie ein
System pr\"apariert wurde, sodass wir dieses Wissen f\"ur die Vorhersage zuk\"unftiger
Messungen - die Vorhersage eines Erwartungswerts f\"ur beliebige Observable - an diesem 
System nutzen k\"onnen. 

Elementare Eigenschaften dieser Abbildung sind: 
\begin{enumerate}
\item 
Die Identit\"atsobservable, die immer nur
den Messwert 1 als Ergebnis liefert, soll den Erwartungswert 1 haben, 
\item 
eine positive Observable,
die immer nur positive Ergebnisse als Messwerte liefert, soll auch einen positiven Erwartungswert
haben, und 
\item
das $\lambda$-fache einer Observablen ${\cal R}$, die immer das $\lambda$-fache eines
Messwerts von einer Messung von ${\cal R}$ liefert, soll auch das $\lambda$-fache des Erwartungswerts
haben. 
\end{enumerate}
Weitere Eigenschaften h\"angen davon ab, welche Strukturen auf der Menge der
Observablen definiert sind. 

Oft unterscheidet man (insbesondere in der Quantentheorie) den Zustand von dem Erwartungswertfunktional, 
das dieser Zustand definiert. Der Zustand wird in der Quantentheorie z.B.\ durch einen normierten
Vektor $|\psi\rangle$ in einem Hilbert-Raum dargestellt, wohingegen die Abbildung 
${\cal R} \rightarrow {\rm Erw}({\cal R})$
durch ${\rm Erw}({\cal R})=\langle \psi|R|\psi\rangle$ gegeben ist. 

\subsection{Vorhersage und Pr\"aparation}

Der Begriff des Zustands als eine Abbildung, die jeder Observablen einen
Erwartungswert zuordnet, beinhaltet schon die Vorschrift, wie man aus einer Theorie
zu experimentell \"uberpr\"ufbaren Vorhersagen kommt. Wenn auf Seiten der Theorie
der Zustand eines Systems bekannt ist, kann man auch angeben, was man bei der
Messung einer bestimmten Observablen f\"ur einen Erwartungswert erh\"alt. Der
Erwartungswert ist dann der Mittelwert, den man bei der Messung einer Observablen an sehr vielen
Systemen erh\"alt. 
Hierbei ist wichtig, dass der Zustand physikalisch durch ein Ensemble von Systemen
repr\"asentiert wird und nicht nur durch ein Einzelsystem. 

Diese Bedingung ist andererseits aber auch problematisch: Der Kosmos als Ganzes l\"asst
 sich nicht als Ensemble realisieren, das Gleiche gilt auch f\"ur komplexere Systeme wie
 beispielsweise einen menschlichen Organismus oder ein menschliches Gehirn. Zumindest
 reine Zust\"ande sind bei solchen Systemen nicht mehr realisierbar (das gilt schon f\"ur
 einfache thermodynamische Systeme). 
 
 Etwas problematischer ist die Frage, woher wir wissen, durch welchen Zustand ein 
 physikalisches System zu repr\"asentieren ist. Wir haben oben gesagt, dass wir die
 Vergangenheit eines Systems - die Art wie es pr\"apariert wurde - kennen m\"ussen. 
 Wir ben\"otigen somit eine Vorschrift, wie wir aus der Kenntnis der Pr\"aparation eines
 Systems den Zustand erhalten, durch den wir das System beschreiben. Diese Vorschrift
 ist oftmals ein eigenes Postulat (in der Quantentheorie beispielsweise das sogenannte
 Kollaps- oder Projektionspostulat).

\subsection{Die zeitliche Entwicklung eines Systems}

Die bisherigen Axiome sind allgemeiner Natur und enthalten noch keinerlei Aussagen
dar\"uber, wie sich ein physikalisches System im Verlauf der Zeit ver\"andert. 
Zun\"achst einmal muss \"uberhaupt gekl\"art werden, was \glqq zeitliche Entwicklung\grqq\
bedeutet. In fast allen physikalischen Theorien (eine Ausnahme bilden manche 
Modelle der Quantenkosmologie) wird die Zeit durch einen Parameter $t$ dargestellt
und als eindimensionales geordnetes Kontinuum gedacht. In solchen F\"allen bedeutet
\glqq zeitliche Entwicklung\grqq\ meist die Angabe einer Bewegungsgleichung, entweder f\"ur
die Zust\"ande (in der Quantenmechanik spricht man dann vom Schr\"odinger-Bild) oder
f\"ur die Observablen (das Heisenberg-Bild) eines Systems. In vielen F\"allen hat die
Bewegungsgleichung die Form einer Differentialgleichung, es gibt aber auch Systeme,
bei denen es sich um eine Integralgleichung handelt (meist Systeme mit einem Ged\"achtnis).



\section{Die Postulate der Klassischen Mechanik}

In der klassischen Mechanik beschreiben wir einen (reinen Zustand) durch einen Punkt $(x,p)\in P$ im
Phasenraum $P$, also die Angabe eines Ortes $x$ und eines Impulses $p$. Eine Observable ist eine
Funktion $F:P\rightarrow \mathbb{R}$ \"uber dem Phasenraum, also eine Funktion von Ort und Impuls.
Der Erwartungswert einer Observablen $F$ in einem Zustand $(x,p)$ ist einfach der Wert $F(x,p)$ dieser
Observablen an dem Punkt im Phasenraum, der den Zustand beschreibt. Dieser Wert ist eindeutig
und sollte (innerhalb der Fehlertoleranz der Messger\"ate) immer derselbe sein.  

Haben wir umgekehrt an einem System eine Observable $F$ gemessen und einen Wert $f$ erhalten, 
wissen wir, dass der Zustand ein Punkt im Phasenraum sein muss, der der Bedingung $F(x,p)=f$
gen\"ugen muss. Ein vollst\"andiger Satz von Observablen $\{F_1,..., F_n\}$ legt einen Punkt
im Phasenraum eindeutig fest. Misst man also diese Observablen und erh\"alt Messwerte
$\{ f_1,...,f_n\}$, so ist der Punkt $(x,p)$ durch die Bedingungen $\{F_1(x,p)=f_1,...,F_n(x,p)=f_n\}$
festgelegt und somit der reine Zustand des Systems bekannt. In einem minimalen vollst\"andigen
Satz von Observablen l\"asst sich auch keine Observable als Funktion der anderen Observablen
ausdr\"ucken. Hat der Phasenraum die Dimension $6N$ (f\"ur $N$ Punktteilchen in 3 Raumdimensionen),
so ben\"otigt man auch $6N$ Observable, um einen Zustand eindeutig festzulegen. 


\section{Die Postulate der Quantenmechanik}

Observable werden in der Quantenmechanik durch selbst-adjungierte bzw.\ hermitesche
Operatoren auf einem Hilbert-Raum dargestellt. Ein Zustand wird mathematisch durch einen Strahl
(einen eindimensionalen Unterraum) in diesem Hilbert-Raum beschrieben. Meist w\"ahlen wir zur einfacheren
Beschreibung einen auf eins normierten Vektor $|\psi\rangle$ auf diesem Strahl als Repr\"asentanten, wir
k\"onnen einen Zustand aber auch durch die Angabe eines eindimensionalen Projektionsoperators $P_\psi$
(der jeden Vektor in dem Hilbert-Raum auf diesen eindimensionalen Unterraum projiziert)
darstellen. In der Quantenmechanik von Punktteilen verwenden wir zur Beschreibung eines
Zustands oft eine sogenannte Wellenfunktion $\psi(x)$, die man aber als Vektor eines unendlich dimensionalen
Hilbert-Raums (dem Raum der quadratintegrierbaren Funktionen) auffassen kann. 
Das Erwartungswertfunktional, d.h.\ die Vorschrift, nach der
wir einer Observablen $A$ eine Zahl $\langle A \rangle_\psi$ - ihren Erwartungswert in dem Zustand $\psi$ - zuordnen, ist
dann 
\begin{equation}
       \langle A \rangle_\psi = \langle \psi | A |  \psi \rangle  = {\rm Spur}\,(P_\psi A) =
                \int_V \psi(x)^* A \psi(x)\,{\rm d} x \, .  
\end{equation}
Dies sind drei Darstellungen des Erwartungswertfunktionals, je nachdem, ob man einen Zustand durch
einen normierten Vektor, einen Projektionsoperator oder eine normierte Wellenfunktion \"uber einem Volumen 
$V$ repr\"asentiert. 
%Da dieses Erwartungswertfunktional f\"ur alle selbst-adjungierten Operatoren ihren Erwartungswert

Das Kollapspostulat bzw.\ das von Neumann-L\"uders'sche Projektionspostulat gibt an, wie wir in der
Quantenmechanik ein System in einem bestimmten Zustand pr\"aparieren k\"onnen. Wurde eine Observable
${\cal R}$ an einem System gemessen und hat man den Messwert $r$ erhalten, so ist das System durch
einen Strahl zu beschreiben, der dem Eigenraum von $R$ zu dem Eigenwert $r$ entspricht. Ein 
vollst\"andiger Satz kompatibler Observabler legt durch ihre Messwerte diesen Eigenraum auf einen
eindimensionalen Strahl und damit einen reinen Zustand fest.


\section{Die Postulate in anderen Bereichen der Physik}

\begin{thebibliography}{99}
\bibitem{Neumann} von Neumann, John; \textit{Mathematische Grundlagen der
     Quantenmechanik}, Springer ??? 1935. 
\bibitem{Schroedinger} Schr\"odinger, Erwin; \textit{???};      
\end{thebibliography}

\end{document}

